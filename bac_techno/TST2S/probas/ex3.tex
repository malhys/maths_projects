\begin{myex}
	On lance un dé à 6 faces truqué. Une étude statistique donne le tableau suivant :
		\begin{center}
			
			\begin{tabular}{|@{\ }l@{\ }|@{\ }c@{\ }|@{\ }c@{\ }|@{\ }c@{\ }|@{\ }c@{\ }|@{\ }c@{\ }|@{\ }c@{\ }|}
				\hline
				Issue $x_i$ & 1 & 2 & 3 & 4 & 5 & 6 \\\hline
				Probabilité $p_i$ & $0,125$ & $0,125$ & $0,125$ & $0,125$ & $0,2$ & $0,3$ \\ \hline
			\end{tabular}
		
		\end{center}
	
	On s'intéresse à l'événement A : <<le nombre obtenu est pair>>.	On a : 
	
	\begin{align*}
		p(A) &= p_2 + p_4 + p_6 \\
			 &= 0,125 + 0,125 + 0,3 \\
			 &= 0,55
	\end{align*}
	
	La probabilité d'obtenir un nombre pair est de 0,55.
\end{myex}

