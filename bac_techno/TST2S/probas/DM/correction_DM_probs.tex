\documentclass[a4paper,11pt]{exam}
\printanswers % pour imprimer les réponses (corrigé)
%\noprintanswers % Pour ne pas imprimer les réponses (énoncé)
%\addpoints % Pour compter les points
 \noaddpoints % pour ne pas compter les points
%\qformat{\textbf{\thequestion ) } }
%\qformat{\textbf{\thequestion )} \textit{(\thepoints)} \\  } % Pour définir le style des questions (facultatif)
\usepackage{color} % définit une nouvelle couleur
\shadedsolutions % définit le style des réponses
% \framedsolutions % définit le style des réponses
\definecolor{SolutionColor}{rgb}{0.8,0.9,1} % bleu ciel
\renewcommand{\solutiontitle}{\noindent\textbf{Solution:}\par\noindent} % Définit le titre des solutions




\makeatletter

\def\maketitle{{\centering%
	\par{\huge\textbf{\@title}}%
	\par{\@date}%
	\par}}

\makeatother

\lhead{NOM Pr\'enom :}
\rhead{\textbf{Les r\'eponses doivent \^etre justifi\'ees}}
\cfoot{\thepage / \pageref{LastPage}}


%\usepackage{../../pas-math}
%\usepackage{../../moncours}


%\usepackage{pas-cours}
%-------------------------------------------------------------------------------
%          -Packages nécessaires pour écrire en Français et en UTF8-
%-------------------------------------------------------------------------------
\usepackage[utf8]{inputenc}
\usepackage[frenchb]{babel}
\usepackage[T1]{fontenc}
\usepackage{lmodern}
%-------------------------------------------------------------------------------

%-------------------------------------------------------------------------------
%                          -Outils de mise en forme-
%-------------------------------------------------------------------------------
\usepackage{hyperref}
\hypersetup{pdfstartview=XYZ}
\usepackage{enumerate}
\usepackage{graphicx}
\usepackage{multicol}

\usepackage{anysize} %%pour pouvoir mettre les marges qu'on veut
%\marginsize{2.5cm}{2.5cm}{2.5cm}{2.5cm}

\usepackage{indentfirst} %%pour que les premier paragraphes soient aussi indentés
%-------------------------------------------------------------------------------


%-------------------------------------------------------------------------------
%                  -Nécessaires pour écrire des mathématiques-
%-------------------------------------------------------------------------------
\usepackage{amsfonts}
\usepackage{amssymb}
\usepackage{amsmath}
\usepackage{amsthm}
\usepackage{tikz}
%-------------------------------------------------------------------------------

%-------------------------------------------------------------------------------
%                     -Mise en forme d'exercices-
%-------------------------------------------------------------------------------
\newtheoremstyle{exostyle}
{\topsep}% espace avant
{\topsep}% espace apres
{}% Police utilisee par le style de thm
{}% Indentation (vide = aucune, \parindent = indentation paragraphe)
{\bfseries}% Police du titre de thm
{.}% Signe de ponctuation apres le titre du thm
{ }% Espace apres le titre du thm (\newline = linebreak)
{\thmname{#1}\thmnumber{ #2}\thmnote{. \normalfont{\textit{#3}}}}% composants du titre du thm : \thmname = nom du thm, \thmnumber = numéro du thm, \thmnote = sous-titre du thm

\theoremstyle{exostyle}
\newtheorem{exercice}{Exercice}

\newenvironment{questions}{
\begin{enumerate}[\hspace{12pt}\bfseries\itshape a.]}{\end{enumerate}
} %mettre un 1 à la place du a si on veut des numéros au lieu de lettres pour les questions 
%-------------------------------------------------------------------------------



%-------------------------------------------------------------------------------
%                    - Racourcis d'écriture -
%-------------------------------------------------------------------------------

% Angles orientés (couples de vecteurs)
\newcommand{\aopp}[2]{(\vec{#1}, \vec{#2})} %Les deuc vecteurs sont positifs
\newcommand{\aopn}[2]{(\vec{#1}, -\vec{#2})} %Le second vecteur est négatif
\newcommand{\aonp}[2]{(-\vec{#1}, \vec{#2})} %Le premier vecteur est négatif
\newcommand{\aonn}[2]{(-\vec{#1}, -\vec{#2})} %Les deux vecteurs sont négatifs

%Ensembles mathématiques
\newcommand{\naturels}{\mathbb{N}} %Nombres naturels
\newcommand{\relatifs}{\mathbb{Z}} %Nombres relatifs
\newcommand{\rationnels}{\mathbb{Q}} %Nombres rationnels
\newcommand{\reels}{\mathbb{R}} %Nombres réels
\newcommand{\complexes}{\mathbb{C}} %Nombres complexes
%-------------------------------------------------------------------------------



\usepackage{tabu}

%\usepackage{fullpage}
\author{\ }
\date{13 Mars 2019}
\title{$T^{le}$ $ST_2S$ : Correction DM probabilités}


\begin{document}
	
\maketitle
	
\section{87 page 174}

\begin{questions}
	\question Parmi les \num{50000} visites de l'étude, \num{3000} concernent des filles de 0 à 12 ans.
	
	\begin{equation*}
		\frac{\num{3000} \times 100}{\num{50000}} = 6
	\end{equation*}
	
	Donc 6 \% des visites concernent des filles de 0 à 12 ans.
	
	\question \ \\
	
	\begin{tabular}{|l|c|c|c|}
		\hline
		& Femmes      & Hommes      & Total       \\ \hline
		0-12 ans       & \num{3000}  & \num{2500}  & \num{5500}  \\ \hline
		13-24 ans      & \num{2000}  & \num{2500}  & \num{4500}  \\ \hline
		25 - 44 ans    & \num{5000}  & \num{5000}  & \num{10000} \\ \hline
		45 - 69 ans    & \num{9500}  & \num{6500}  & \num{16000} \\ \hline
		70 ans ou plus & \num{8000}  & \num{6000}  & \num{14000} \\ \hline
		Total          & \num{27500} & \num{22500} & \num{50000} \\ \hline
	\end{tabular}

	\question
		\begin{parts}
			\part 
			
			\begin{multicols}{2}
				\begin{eqnarray*}
					P(A) &=& \frac{nb\; de\; femmes}{nb\; de \; consultations} \\
					P(A) &=& \frac{\num{27500}}{\num{50000}}\\
					P(A) &=& \num{0.55}\\				
				\end{eqnarray*}
				
				%\vspace*{-1cm}
				\begin{eqnarray*}
					P(B) &=& \frac{nb\; de\; 70\; ans\; ou\; plus}{nb\; de \; consultations} \\
					P(B) &=& \frac{\num{14000}}{\num{50000}}\\
					P(B) &=& \num{0.28}\\					
				\end{eqnarray*}
				
			\end{multicols}
			\vspace*{-0.5cm}				
			Les probabilités des événements $A$ et $B$ sont respectivement \num{0.55} et \num{0.28}.
			
			\part On a $A \cap B$ : "Le patient choisi est une femme et est âgé de 70 ans ou plus".
			
			\begin{eqnarray*}
				P(A \cap B) &=& \frac{Femmes\; de\; plus\; de\; 70\; ans}{nb\; de \; consultations} \\
				P(A \cap B) &=& \frac{\num{8000}}{\num{50000}}\\
				P(A \cap B) &=& \num{0.16}
			\end{eqnarray*}
			
			\part L'événement $C$ correspond à l'union des événements $A$ et $B$. On a donc :
			
				\begin{eqnarray*}
					P(C) &=& P(A \cup B) \\
					P(C) &=& P(A) + P(B) - P(A \cap B)\\
					P(C) &=& \num{0,55} + \num{0,28} - \num{0,16} \\
					P(C) &=& \num{0,67} 
				\end{eqnarray*}
			
			La probabilité de l'événement $C$ est \num{0,67}.
		\end{parts}
	
	\newpage
	\question
	
	\begin{parts}
	
	\part
	
	\begin{multicols}{2}
		\begin{eqnarray*}
			P_A(B) &=& \frac{P(A \cap B)}{P(A)}\\
			P_A(B) &=& \frac{\num{0.16}}{\num{0.55}}\\
			P_A(B) &=& \num{0.29}
		\end{eqnarray*}
		
		\begin{eqnarray*}
			P_B(\bar{A}) &=& \frac{nb\; d'\;hommes\;de\;plus\;de\;70\;ans}{nbde\;plus\;de\;70\;ans}\\
			P_B(\bar{A}) &=& \frac{\num{6000}}{\num{14000}}\\
			P_B(\bar{A}) &=& \num{0.43}
		\end{eqnarray*}
	\end{multicols}
%	\vspace*{-0.5cm}				
	Les probabilités conditionnelles $P_A(B)$ et $P_B(\bar{A})$ valent respectivement \num{0.29} et \num{0.43}.	
	
	\part Il y a \num{20000} cas de patients entre 0 et 44 ans (\num{5500} + \num{4500} + \num{10000}) et \num{10000} hommes de 0 à 44 ans (\num{2500} + \num{2500} + \num{5000}).
	
	La probabilité qu'un patient soit un homme sachant qu'il est âgé de 0 à 44 ans est \num{0.5} ($\num{10000} \div \num{20000}$)
\end{parts} 
\end{questions}

\newpage
\section{89 page 175}

\begin{questions}
	
	\question 
	\begin{equation*}
		\num{5900} \times\num{0.36} = \num{2124}
	\end{equation*}
	
	\num{2124} personnes bénéficient de l'APA en établissement.
	
	\begin{equation*}
		\num{5900} - \num{2124} = \num{3776}
	\end{equation*}
	
	\num{3776} personnes bénéficient de l'APA à domicile.
	
	\question \ \\
	\begin{tabular}{|@{\ }l@{\ }|@{\ }c@{\ }|@{\ }c@{\ }|@{\ }c@{\ }|@{\ }c@{\ }|@{\ }c@{\ }|}
		\hline
		Tranches d'âge          & $[60 , 75[$ & $[75, 85[$ & $[85, 95[$ & $[95, 100[$ & Total \\ \hline
		Domicile (en \%)        & 17         & 44         & 35         & 4           & 100   \\ \hline
		\'Etablissement (en \%) & 12         & 35         & 46         & 7           & 100   \\ \hline
	\end{tabular}
	
	\question 
	\begin{parts}
		\part 
		\begin{equation*}
			\num{0.44} \times \num{3776} = \num{1161}
		\end{equation*}
		
		\num{1161} personnes âgées de 75  à 85 ans bénéficient de l'APA à domicile.
		
		\part 
		\begin{equation*}
		\num{0.35} \times \num{2124} = \num{743}
		\end{equation*}
		
		\num{743} personnes âgées de 75  à 85 ans bénéficient de l'APA en établissement.
		
		\part \ \\
		
		\begin{tabular}{|@{\ }l@{\ }|@{\ }c@{\ }|@{\ }c@{\ }|@{\ }c@{\ }|@{\ }c@{\ }|@{\ }c@{\ }|}
			\hline
			Tranches d'âge  & $[60 , 75[$ & $[75, 85[$ & $[85, 95[$ & $[95, 100[$ & Total \\ \hline
			Domicile        & 642         & 1161       & 1322       & 151         & 3776  \\ \hline
			\'Etablissement & 255         & 743        & 977        & 149         & 2124  \\ \hline
			Total           & 897         & 2404       & 2299       & 300         & 5900  \\ \hline
		\end{tabular}
	\end{parts}

	\question 
	
	Calcul de la proportion de bénéficiaires de l'APA de 60 à 75 ans :
	
	\begin{equation*}
		\frac{897}{5900} \approx \num{0.152} \; soit \; \num{15.2} \%.
	\end{equation*} 
	
	Il y a environ \num{84.8} \% (\num{100} - \num{15.2}) de personnes de plus de 75 ans parmi les bénéficiaires de l'APA. L'affirmation 1 est donc vraie car on dépasse les 50 \% mais on reste en dessous des 85 \%, l'affirmation 2 est fausse.
	
	\question 
	
	\begin{parts}
		\part 
		%\begin{multicols}{2}
			\begin{eqnarray*}
				P(E) &=& \frac{Nb \; de \; 85-95\; ans}{nb\; total \; de \; personnes} \\
				P(E) &=& \frac{2299}{5900} \\
				P(E) &=& \num{0.39}\\
			\end{eqnarray*}
		
			\begin{eqnarray*}
				P(F) &=& \frac{Nb \; de \; personnes\; à\; domicile}{nb\; total \; de \; personnes} \\
				P(F) &=& \frac{3776}{5900} \\
				P(F) &=& \num{0.64}\\
			\end{eqnarray*}
		%\end{multicols}
	
		Les probabilités des événements $E$ et $F$ sont respectivement \num{0.39} et \num{0.64}.
		
		\part On a :
		
		\begin{itemize}
			\item $E \cup F$ : "La personne choisie est dans la tranche d'âge $[85; 95[$ ou bénéficie de l'APA à domicile".
			
			\item $E \cap F$ : "La personne choisie est dans la tranche d'âge $[85; 95[$ et bénéficie de l'APA à domicile".
		\end{itemize}
	
		\part 
		
			\begin{eqnarray*}
				P(E \cap F) &=& \frac{Nb \; de \; personnes \; de \; [85 ; 95[ \; à\; domicile}{nb\; total \; de \; personnes} \\
				P(E \cap F) &=& \frac{1322}{5900} \\
				P(E \cap F) &=& \num{0.22}\\
			\end{eqnarray*}
		
			La probabilité de l'événement $E \cap F$ est \num{0.22}.
			
			On a donc :
			
			\begin{eqnarray*}
				P(E \cup F) &=& P(E) + P(F) - P(E \cap F)\\
				P(E \cup F) &=& \num{0.39} + \num{0.64} - \num{0.22} \\
				P(E \cup F) &=& \num{0.81}\\
			\end{eqnarray*}
		
			La probabilité de l'événement $E \cup F$ est \num{0.81}.	
			
		\part 
			

				\begin{eqnarray*}
					P_F(E) &=& \frac{P(E \cap F)}{P(F)} \\
					P_F(E) &=& \frac{\num{0.22}}{\num{0.64}} \\
					P_F(E) &=& \num{0.34}
				\end{eqnarray*}
			
				\begin{eqnarray*}
					P_E(F) &=& \frac{P(E \cap F)}{P(F)} \\
					P_E(F) &=& \frac{\num{0.22}}{\num{0.39}} \\
					P_E(F) &=& \num{0.56}
				\end{eqnarray*}

			
		Les probabilités conditionnelles $P_F(E)$ et $P_E(F)$ valent respectivement \num{0.34} et \num{0.56}.		
	\end{parts}
\end{questions}

\label{LastPage}
\end{document}
