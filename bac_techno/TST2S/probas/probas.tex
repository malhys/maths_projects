\documentclass[12pt,a4paper]{article}

%\usepackage[left=1.5cm,right=1.5cm,top=1cm,bottom=2cm]{geometry}
\usepackage[in, plain]{fullpage}
\usepackage{array}
\usepackage{../../../pas-math}
\usepackage{../../../moncours}

\usepackage{multicol}
\usepackage{caption}
\usepackage{tikz}
\usetikzlibrary{trees}


%\usepackage{pas-cours}
%-------------------------------------------------------------------------------
%          -Packages nécessaires pour écrire en Français et en UTF8-
%-------------------------------------------------------------------------------
\usepackage[utf8]{inputenc}
\usepackage[frenchb]{babel}
\usepackage[T1]{fontenc}
\usepackage{lmodern}
\usepackage{textcomp}



%-------------------------------------------------------------------------------

%-------------------------------------------------------------------------------
%                          -Outils de mise en forme-
%-------------------------------------------------------------------------------
\usepackage{hyperref}
\hypersetup{pdfstartview=XYZ}
%\usepackage{enumerate}
\usepackage{graphicx}
\usepackage{multicol}
\usepackage{tabularx}
\usepackage{multirow}


\usepackage{anysize} %%pour pouvoir mettre les marges qu'on veut
%\marginsize{2.5cm}{2.5cm}{2.5cm}{2.5cm}

\usepackage{indentfirst} %%pour que les premier paragraphes soient aussi indentés
\usepackage{verbatim}
\usepackage{enumitem}
\usepackage[usenames,dvipsnames,svgnames,table]{xcolor}

\usepackage{variations}

%-------------------------------------------------------------------------------


%-------------------------------------------------------------------------------
%                  -Nécessaires pour écrire des mathématiques-
%-------------------------------------------------------------------------------
\usepackage{amsfonts}
\usepackage{amssymb}
\usepackage{amsmath}
\usepackage{amsthm}
\usepackage{tikz}
\usepackage{xlop}
%-------------------------------------------------------------------------------



%-------------------------------------------------------------------------------


%-------------------------------------------------------------------------------
%                    - Mise en forme avancée
%-------------------------------------------------------------------------------

\usepackage{ifthen}
\usepackage{ifmtarg}


\newcommand{\ifTrue}[2]{\ifthenelse{\equal{#1}{true}}{#2}{$\qquad \qquad$}}

%-------------------------------------------------------------------------------

%-------------------------------------------------------------------------------
%                     -Mise en forme d'exercices-
%-------------------------------------------------------------------------------
%\newtheoremstyle{exostyle}
%{\topsep}% espace avant
%{\topsep}% espace apres
%{}% Police utilisee par le style de thm
%{}% Indentation (vide = aucune, \parindent = indentation paragraphe)
%{\bfseries}% Police du titre de thm
%{.}% Signe de ponctuation apres le titre du thm
%{ }% Espace apres le titre du thm (\newline = linebreak)
%{\thmname{#1}\thmnumber{ #2}\thmnote{. \normalfont{\textit{#3}}}}% composants du titre du thm : \thmname = nom du thm, \thmnumber = numéro du thm, \thmnote = sous-titre du thm

%\theoremstyle{exostyle}
%\newtheorem{exercice}{Exercice}
%
%\newenvironment{questions}{
%\begin{enumerate}[\hspace{12pt}\bfseries\itshape a.]}{\end{enumerate}
%} %mettre un 1 à la place du a si on veut des numéros au lieu de lettres pour les questions 
%-------------------------------------------------------------------------------

%-------------------------------------------------------------------------------
%                    - Mise en forme de tableaux -
%-------------------------------------------------------------------------------

\renewcommand{\arraystretch}{1.7}

\setlength{\tabcolsep}{1.2cm}

%-------------------------------------------------------------------------------



%-------------------------------------------------------------------------------
%                    - Racourcis d'écriture -
%-------------------------------------------------------------------------------

% Angles orientés (couples de vecteurs)
\newcommand{\aopp}[2]{(\vec{#1}, \vec{#2})} %Les deuc vecteurs sont positifs
\newcommand{\aopn}[2]{(\vec{#1}, -\vec{#2})} %Le second vecteur est négatif
\newcommand{\aonp}[2]{(-\vec{#1}, \vec{#2})} %Le premier vecteur est négatif
\newcommand{\aonn}[2]{(-\vec{#1}, -\vec{#2})} %Les deux vecteurs sont négatifs

%Ensembles mathématiques
\newcommand{\naturels}{\mathbb{N}} %Nombres naturels
\newcommand{\relatifs}{\mathbb{Z}} %Nombres relatifs
\newcommand{\rationnels}{\mathbb{Q}} %Nombres rationnels
\newcommand{\reels}{\mathbb{R}} %Nombres réels
\newcommand{\complexes}{\mathbb{C}} %Nombres complexes


%Intégration des parenthèses aux cosinus
\newcommand{\cosP}[1]{\cos\left(#1\right)}
\newcommand{\sinP}[1]{\sin\left(#1\right)}


%Probas stats
\newcommand{\stat}{statistique}
\newcommand{\stats}{statistiques}
%-------------------------------------------------------------------------------

%-------------------------------------------------------------------------------
%                    - Mise en page -
%-------------------------------------------------------------------------------

\newcommand{\twoCol}[1]{\begin{multicols}{2}#1\end{multicols}}


\setenumerate[1]{font=\bfseries,label=\textit{\alph*})}
\setenumerate[2]{font=\bfseries,label=\arabic*)}


%-------------------------------------------------------------------------------
%                    - Elements cours -
%-------------------------------------------------------------------------------




\date{}
\title{}


\begin{document}
	%\maketitle
	\chap[num=4, color=red]{Probabilités}{}
	
	\begin{myobj}
	\begin{itemize}
		
		\item Construire le symétrique d’un point ou d'une figure par rapport à une droite à la main où à l’aide d’un logiciel;
		\item Construire le symétrique d’un point ou d'une figure par rapport à un point, à la main où à l’aide d’un logiciel;
		\item Utiliser les propriétés de la symétrie axiale ou centrale;
		\item Identifier des symétries dans des figures.		
	\end{itemize}
\end{myobj}

\begin{mycomp}
	\begin{itemize}
		\item \kw{Chercher (Ch2)} :  s’engager    dans    une    démarche    scientifique, observer, questionner, manipuler, expérimenter (sur une feuille de papier, avec des objets, à l’aide de logiciels), émettre des hypothèses, chercher des exemples ou des contre-exemples, simplifier ou particulariser une situation, émettre une conjecture ;
		\item \kw{Raisonner (Ra3)} :  démontrer : utiliser un raisonnement logique et des règles établies (propriétés, théorèmes, formules) pour parvenir à une conclusion ;
		\item \kw{Communiquer (Co2)} :  expliquer à l’oral ou à l’écrit (sa démarche, son raisonnement, un calcul, un protocole   de   construction   géométrique, un algorithme), comprendre les explications d’un autre et argumenter dans l’échange ; 
		
	\end{itemize}
\end{mycomp}



	
	
	\section{Vocabulaire}
		
		
		
		\begin{mybilan}
			
		
			\begin{itemize}
				\item Dans une \kw{expérience aléatoire}, l'\kw{univers $\Omega$} est l'ensemble des événements possibles.
				
				\item Un \kw{événement} est une partie de l'univers.
				
				\item Un événement qui possède un seul élément est un \kw{événement élémentaire}.
				
				\item Deux événements $A$ et $B$ sont \kw{disjoints ou incompatibles} si et seulement si \kw{$\mathbf{A \cap B = \emptyset}$}.
				
				\item L'\kw{événement contraire} d'un événement $A$ est l'\kw{événement $\bar{A}$} constitué des éléments de $\Omega$ qui ne sont pas dans $A$.
				
			\end{itemize}
		\end{mybilan}	
		

	\section{Calculs de probabilités}

\subsection{Defintion}

\begin{mydef}
	\begin{itemize}
		\item La \kw{probabilité} d'un événement $A$, notée $P(A)$ est la somme des probabilités des événements élémentaires qui le composent.
		\item Pour tout événement $A$, \kw{$0\leq P(A) \leq 1$}
		%\item Si $A$ est un événement \kw{certain}, alors \kw{$P(A)=1$}.
		%\item Si $A$ est un événement \kw{impossible}, alors \kw{$P(A)=0$}.
	\end{itemize}
	
\end{mydef}


\subsection{\'Equiprobailité}


\begin{mybilan}
	
	\begin{itemize}
		\item Il y a \kw{équiprobabilté} dans le cas où tous les évévnements élémentaires ont la même probabilité.
		Dans ce cas, la probabilité d'un événement élémentaire est :
		\begin{equation*}
		\dfrac{1}{nombre\; d'éléments\; de\; \Omega}
		\end{equation*}
		
		
		\item Dans une situation d'équiprobabilité, la probabilité d'un événement $A$ est :
		\begin{equation*}
		P(A)=\dfrac{nombre\;d'\;éléments\;de\;A}{nombre\; d'éléments\; de\; \Omega}=\dfrac{nombre\; de\; cas\;favorbales}{nombre\; de\; cas\;possibles}
		\end{equation*}
		
		
		\item Pour tout événement $A$, \kw{$P(\bar{A}) =  1- P(A)$}.
		\item Pour tous événements $A$ et $B$, \kw{$P(A \cup B) = P(A) + P(B) - P(A \cap B)$}.
		\item Pour tous événements $A$ et $B$ \kw{disjoints}, \kw{$P(A \cup B) = P(A) + P(B)$}.
	\end{itemize}
	
	
	
	
	
	
\end{mybilan}

\section{Probabilités conditionnelles et arbre pondéré}	



\subsection{Probabilités conditionnelles}

\begin{mybilan}
	\begin{itemize}
		\item \kw{$P_B(A)$} est la \kw{probabilité de A sachant que B} est réalisé.
		\begin{equation*}
			P_B(A)=\dfrac{P(A \cap B)}{P(B)}
		\end{equation*}
		\item Pour tous événements $A$ et $B$ de probabilités non nulles, 
		\begin{equation*}
			P(A\cap B)= P_A(B) \times P(A) = P_B(A) \times P(B)
		\end{equation*}
	\end{itemize}
\end{mybilan}

\subsection{\'Evénements indépendants}

\begin{mybilan}
	Deux événements $A$ et $B$ sont \kw{indépendants} si et seulement si
	\begin{equation*}
		P(A \cap B) = P(A) \times P(B)
	\end{equation*}

	\begin{center}
		ou
	\end{center}
	
	\begin{equation*}
		P_B(A) = P(A)
	\end{equation*}
\end{mybilan}

\subsection{Utilisation d'un arbre de probabilité pondéré}	
\begin{mymeth}
	On peut représenter une situation de probabilité par un arbre et l'utiliser pour faire des calculs.\\
	\begin{itemize}
		\item La somme des probabilités pour les branches d'un même n\oe ud est égale à $1$.
		\item Les branches de niveau 2 portent des probabilités conditionnelles.
		\item Un chemin est une intersection d'événements.
		\item La probabilité d'un chemin est le produit des probabilités portées par ses branches.
		\item La probabilité d'un événement est la somme des probabilité des chemins qui y aboutissent.
	\end{itemize}

	
\end{mymeth}
	

	
% Set the overall layout of the tree
\tikzstyle{level 1}=[level distance=3.5cm, sibling distance=3.5cm]
\tikzstyle{level 2}=[level distance=3.5cm, sibling distance=2cm]

% Define styles for bags and leafs
\tikzstyle{bag} = [text width=4em, text centered]
\tikzstyle{end} = [circle, minimum width=3pt,fill, inner sep=0pt]

% The sloped option gives rotated edge labels. Personally
% I find sloped labels a bit difficult to read. Remove the sloped options
% to get horizontal labels. 
\begin{tikzpicture}[grow=right, sloped]
\node[bag] {}
    child {
        node[bag] {$M_2$}        
            child {
                node[end, label=right:
                    {$\bar{C}$ }] {}
                edge from parent
                node[below] {...}
                %node[below]  {$\frac{4}{9}$}
            }
            child {
                node[end, label=right:
                    {$C$}] {}
                edge from parent
                node[above] {...}
                %node[below]  {$\frac{5}{9}$}
            }
            edge from parent 
            %node[above] {$W$}
            node[below]  {}
    }
    child {
        node[bag] {$M_1$}        
        child {
                node[end, label=right:
                    {$\bar{C} $}] {}
                edge from parent
                %node[above] {$B$}
                node[below]  {...}
            }
            child {
                node[end, label=right:
                    {$C$}] {}
                edge from parent
                node[above] {...}
%                node[below]  {$\frac{6}{9}$}
            }
        edge from parent         
            %node[above] {$B$}
            node[above]  {...}
    }
    child {
    	node[bag] {$A_0$}        
    	child {
    		node[end, label=right:
    		{$\bar{C}$}] {}
    		edge from parent
    		%node[above] {$B$}
    		node[below]  {...}
    	}
    	child {
    		node[end, label=right:
    		{$C$}] {}
    		edge from parent
    		node[above] {...}
    		%                node[below]  {$\frac{6}{9}$}
    	}
    	edge from parent         
    	%node[above] {$B$}
    	node[above]  {...}
    };
\end{tikzpicture}





\end{document}