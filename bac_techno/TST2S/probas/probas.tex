\documentclass[12pt,a4paper]{article}

%\usepackage[left=1.5cm,right=1.5cm,top=1cm,bottom=2cm]{geometry}
\usepackage[in, plain]{fullpage}
\usepackage{array}
\usepackage{../../../pas-math}
\usepackage{../../../moncours}

\usepackage{multicol}
\usepackage{caption}
\usepackage{tikz}
\usetikzlibrary{trees}


%\usepackage{pas-cours}
%-------------------------------------------------------------------------------
%          -Packages nécessaires pour écrire en Français et en UTF8-
%-------------------------------------------------------------------------------
\usepackage[utf8]{inputenc}
\usepackage[frenchb]{babel}
\usepackage[T1]{fontenc}
\usepackage{lmodern}
%-------------------------------------------------------------------------------

%-------------------------------------------------------------------------------
%                          -Outils de mise en forme-
%-------------------------------------------------------------------------------
\usepackage{hyperref}
\hypersetup{pdfstartview=XYZ}
\usepackage{enumerate}
\usepackage{graphicx}
\usepackage{multicol}

\usepackage{anysize} %%pour pouvoir mettre les marges qu'on veut
%\marginsize{2.5cm}{2.5cm}{2.5cm}{2.5cm}

\usepackage{indentfirst} %%pour que les premier paragraphes soient aussi indentés
%-------------------------------------------------------------------------------


%-------------------------------------------------------------------------------
%                  -Nécessaires pour écrire des mathématiques-
%-------------------------------------------------------------------------------
\usepackage{amsfonts}
\usepackage{amssymb}
\usepackage{amsmath}
\usepackage{amsthm}
\usepackage{tikz}
%-------------------------------------------------------------------------------

%-------------------------------------------------------------------------------
%                     -Mise en forme d'exercices-
%-------------------------------------------------------------------------------
\newtheoremstyle{exostyle}
{\topsep}% espace avant
{\topsep}% espace apres
{}% Police utilisee par le style de thm
{}% Indentation (vide = aucune, \parindent = indentation paragraphe)
{\bfseries}% Police du titre de thm
{.}% Signe de ponctuation apres le titre du thm
{ }% Espace apres le titre du thm (\newline = linebreak)
{\thmname{#1}\thmnumber{ #2}\thmnote{. \normalfont{\textit{#3}}}}% composants du titre du thm : \thmname = nom du thm, \thmnumber = numéro du thm, \thmnote = sous-titre du thm

\theoremstyle{exostyle}
\newtheorem{exercice}{Exercice}

\newenvironment{questions}{
\begin{enumerate}[\hspace{12pt}\bfseries\itshape a.]}{\end{enumerate}
} %mettre un 1 à la place du a si on veut des numéros au lieu de lettres pour les questions 
%-------------------------------------------------------------------------------



%-------------------------------------------------------------------------------
%                    - Racourcis d'écriture -
%-------------------------------------------------------------------------------

% Angles orientés (couples de vecteurs)
\newcommand{\aopp}[2]{(\vec{#1}, \vec{#2})} %Les deuc vecteurs sont positifs
\newcommand{\aopn}[2]{(\vec{#1}, -\vec{#2})} %Le second vecteur est négatif
\newcommand{\aonp}[2]{(-\vec{#1}, \vec{#2})} %Le premier vecteur est négatif
\newcommand{\aonn}[2]{(-\vec{#1}, -\vec{#2})} %Les deux vecteurs sont négatifs

%Ensembles mathématiques
\newcommand{\naturels}{\mathbb{N}} %Nombres naturels
\newcommand{\relatifs}{\mathbb{Z}} %Nombres relatifs
\newcommand{\rationnels}{\mathbb{Q}} %Nombres rationnels
\newcommand{\reels}{\mathbb{R}} %Nombres réels
\newcommand{\complexes}{\mathbb{C}} %Nombres complexes
%-------------------------------------------------------------------------------



\date{}
\title{}


\begin{document}
	%\maketitle
	\chap[num=4, color=red]{Probabilités}{}
	
	\section{Vocabulaire}
	

	\section{Calculs de probabilités}
\newpage

\section{Probabilités conditionnelles et arbre pondéré}	



\subsection{Probabilités conditionnelles}

\begin{mybilan}
	\begin{itemize}
		\item \kw{$P_B(A)$} est la \kw{probabilité de A sachant que B} est réalisé.
		\begin{equation*}
			P_B(A)=\dfrac{P(A \cap B)}{P(B)}
		\end{equation*}
		\item Pour tous événements $A$ et $B$ de probabilités non nulles, 
		\begin{equation*}
			P(A\cap B)= P_A(B) \times P(A) = P_B(A) \times P(B)
		\end{equation*}
	\end{itemize}
\end{mybilan}

\subsection{\'Evénements indépendants}

\begin{mybilan}
	Deux événements $A$ et $B$ sont \kw{indépendants} si et seulement si
	\begin{equation*}
		P(A \cap B) = P(A) \times P(B)
	\end{equation*}

	\begin{center}
		ou
	\end{center}
	
	\begin{equation*}
		P_B(A) = P(A)
	\end{equation*}
\end{mybilan}

\subsection{Utilisation d'un arbre de probabilité pondéré}	
\begin{mymeth}
	On peut représenter une situation de probabilité par un arbre et l'utiliser pour faire des calculs.\\
	\begin{itemize}
		\item La somme des probabilités pour les branches d'un même n\oe ud est égale à $1$.
		\item Les branches de niveau 2 portent des probabilités conditionnelles.
		\item Un chemin est une intersection d'événements.
		\item La probabilité d'un chemin est le produit des probabilités portées par ses branches.
		\item La probabilité d'un événement est la somme des probabilité des chemins qui y aboutissent.
	\end{itemize}

	
\end{mymeth}
	


\begin{myex}
	
Sur sa console de jeux, Dorian s'apprête à affronter en duel l'un des trois monstres Thor, Odin et Loki. Ces monstres sont de forces inégales : la probabilité pour que Dorian l'emporte contra Thor; Odin ou Loki est respectivement $\dfrac{1}{4}$, $\dfrac{1}{3}$ et $\dfrac{1}{2}$. De plus Dorian a une chance sur deux d'affronter Thor, et autant de chances de rencontrer Odin que Loki.\\

On considère un duel au hasard et les événements :
\begin{itemize}
	\item $T$ : Dorian combat Thor.
	\item $O$ : Dorian combat Odin.
	\item $L$ : Dorian combat Loki.
	\item $G$ : Dorian gagne son combat.
\end{itemize} 


% Set the overall layout of the tree
\tikzstyle{level 1}=[level distance=3.5cm, sibling distance=3.5cm]
\tikzstyle{level 2}=[level distance=3.5cm, sibling distance=2cm]

% Define styles for bags and leafs
\tikzstyle{bag} = [text width=4em, text centered]
\tikzstyle{end} = [circle, minimum width=3pt,fill, inner sep=0pt]

% The sloped option gives rotated edge labels. Personally
% I find sloped labels a bit difficult to read. Remove the sloped options
% to get horizontal labels. 
\begin{tikzpicture}[grow=right, sloped]
\node[bag] {}
    child {
        node[bag] {L}        
            child {
                node[end, label=right:
                    {$\bar{G} \qquad  \rightarrow p(L \cap \bar{G}) = \dfrac{1}{4} \times \dfrac{1}{2} = \dfrac{1}{8}$}] {}
                edge from parent
                node[below] {$P_L(\bar{G})=\dfrac{1}{2}$}
                %node[below]  {$\frac{4}{9}$}
            }
            child {
                node[end, label=right:
                    {$G \qquad  \rightarrow p(L \cap G) = \dfrac{1}{4} \times \dfrac{1}{2} = \dfrac{1}{8}$}] {}
                edge from parent
                node[above] {{\footnotesize $P_L(G)=\dfrac{1}{2}$}}
                %node[below]  {$\frac{5}{9}$}
            }
            edge from parent 
            %node[above] {$W$}
            node[below]  {$P(L)=\dfrac{1}{4}$}
    }
    child {
        node[bag] {O}        
        child {
                node[end, label=right:
                    {$\bar{G}  \qquad  \rightarrow p(O \cap \bar{G}) = \dfrac{1}{4} \times \dfrac{2}{3} = \dfrac{1}{6}$}] {}
                edge from parent
                %node[above] {$B$}
                node[below]  {{\footnotesize $P_O(\bar{G})=\dfrac{2}{3}$}}
            }
            child {
                node[end, label=right:
                    {$G  \qquad  \rightarrow p(O \cap G) = \dfrac{1}{4} \times \dfrac{1}{3} = \dfrac{1}{12}$}] {}
                edge from parent
                node[above] {{\footnotesize $P_O(G)=\dfrac{1}{3}$}}
%                node[below]  {$\frac{6}{9}$}
            }
        edge from parent         
            %node[above] {$B$}
            node[above]  {$P(O)=\dfrac{1}{4}$}
    }
    child {
    	node[bag] {T}        
    	child {
    		node[end, label=right:
    		{$\bar{G} \qquad  \rightarrow p(T \cap \bar{G}) = \dfrac{1}{2} \times \dfrac{3}{4} = \dfrac{3}{8}$}] {}
    		edge from parent
    		%node[above] {$B$}
    		node[below]  {{\footnotesize $P_T(\bar{G})=\dfrac{3}{4}$}}
    	}
    	child {
    		node[end, label=right:
    		{$G \qquad  \rightarrow p(T \cap G) = \dfrac{1}{2} \times \dfrac{1}{4} = \dfrac{1}{8}$}] {}
    		edge from parent
    		node[above] {$P_T(G)=\dfrac{1}{4}$}
    		%                node[below]  {$\frac{6}{9}$}
    	}
    	edge from parent         
    	%node[above] {$B$}
    	node[above]  {$P(T)=\dfrac{1}{2}$}
    };
\end{tikzpicture}

\vspace*{1cm}
\begin{align*}
p(G) &= p(T \cap G) + p(O \cap G) + p(L \cap G)\\
p(G) &= \dfrac{1}{8} + \dfrac{1}{12} + \dfrac{1}{8}\\
p(G) &= \dfrac{32}{96}\\
p(G) &= \dfrac{1}{3}
\end{align*}

Dorian a une chance sur trois de gagner son duel.
\end{myex}




\end{document}