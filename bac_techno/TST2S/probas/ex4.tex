\begin{myex}
	On lance un dé à 6 faces non truqué. Puisque le de n'est pas truqué, nous sommes dans une situation d'équiprobabilité.
	On s'intéresse à l'événement A : <<le nombre obtenu est pair>>.	On a : 
	
	\begin{align*}
		p(A) &= p_2 + p_4 + p_6 \\
		&= \dfrac{1}{6} + \dfrac{1}{6} + \dfrac{1}{6} \\
		&= \dfrac{3}{6} \\
		&= 0,5
	\end{align*}
	
	Dans ce cas, la probabilité d'obtenir un résultat pair est de 0,5.
\end{myex}