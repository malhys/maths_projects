\section{Cartables d'élèves de sixième}

\`A la rentrée scolaire, on fait une enquête dans une classe de sixième comprenant 25 élèves.

On sait que dans cette classe, 48 \% des élèves ont 11 ans, 20 \% ont 13 ans et les autres ont 12 ans.

Ces élèves utilisent deux types de sacs de cours : le sac à dos où le cartable classique.

15 élèves, dont les deux tiers ont 11 ans, ont acheté un cartable classique ; les autres, dont la moitié ont 12 ans ont acheté un sac à dos.

\begin{questions}
	\question[2] Compléter le tableau suivant à l'aide des données de l'énoncé.
	
	\begin{tabular}{|l|c|c|c|}
		\hline
		 	& Sac à dos & Cartable & Total \\
		 \hline
		 11 ans & & & \\	
		 \hline
		 12 ans & & & \\	
		 \hline
		 13 ans & & & \\	
		 \hline
		 Total & & & 25\\	
		 \hline
	\end{tabular}


	\question On interroge au hasard un élève de cette classe. On note :
	\begin{itemize}
		\item S l'événement << l'élève a un sac à dos >>;
		\item C l'événement << l'élève a un cartable >>;
		\item T l'événement << l'élève a treize ans >>;
	\end{itemize}

		\begin{parts}
			\part Calculer $P(S)$, $P(C)$ et $P(T)$.
			
			\part Calculer $P(C \cap T)$
		\end{parts}
\end{questions}