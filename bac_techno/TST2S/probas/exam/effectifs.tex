\section{Des paires de chaussures}

Dans une grande surface, il y a en stock, 500 paires de chaussures de sport dont 375 ont été fabriquées à l'étranger, 1000 paires de bottes dont 75 \% ont ont été fabriquées à l'étranger et 2500 paires de chaussons dont 625 ont été fabriquées en France.


\begin{questions}
	\question[2] Compléter le tableau suivant.
	
	\begin{tabular}{|@{\ }l@{\ }|@{\ }c@{\ }|@{\ }c@{\ }|@{$\qquad$ }c@{$\qquad$ }|}
	\hline
                       & Nombre de      & Nombre de         &       \\
                       & familles ayant & familles n'ayant  & Total \\
                       & un téléviseur  & pas de téléviseur &       \\ \hline
Nombre de familles     &                &                   &       \\
ayant une voiture      &                &                   &       \\ \hline
Nombre de familles     &                &                   &       \\
n'ayant pas de voiture &                &                   &       \\ \hline
Total                  &                &                   & 5000  \\ \hline
\end{tabular}
	
	\begin{solution}
		\begin{tabular}{|@{\ }l@{\ }|@{\ }c@{\ }|@{\ }c@{\ }|@{$\qquad$ }c@{$\qquad$ }|}
	\hline
                       & Nombre de      & Nombre de         &       \\
                       & familles ayant & familles n'ayant  & Total \\
                       & un téléviseur  & pas de téléviseur &       \\ \hline
Nombre de familles     & \num{2450}     &   \num{550}       &   \num{300}    \\
ayant une voiture      &                &                   &       \\ \hline
Nombre de familles     &  \num{800}     &   \num{1200}      &   \num{2000}    \\
n'ayant pas de voiture &                &                   &       \\ \hline
Total                  &  \num{3250}    &   \num{1750}      & 5000  \\ \hline
\end{tabular}
	\end{solution}
	\vspace*{0.2cm}
	\question[5] On prélève un article au hasard parmi les articles de ce stock. Tous les articles ont la même probabilité d'être choisis. On considère les événements suivants :
	
	\begin{itemize}
		\item $A$ : <<l'article prélevé a été fabriqué à l'étranger>>;
		\item $B$ : <<l'article prélevé est une paire de bottes>>;
		\item $C$ : <<l'article prélevé est une paire de chaussons>>.
	\end{itemize}

\emph{Dans cet exercice, sauf indication contraire, on demande les valeurs exactes des probabilités sous forme décimale.}

	\begin{parts}
		\part[1] Traduire par une phrase chacun des trois événements : $\bar{A}$, $A \cap B$ et $\bar{A} \cap C$.
		\begin{solution}
			Les événements demandés sont les suivants :\begin{itemize}
				\item $\bar{A}$ : <<l'article prélevé n'a pas été fabriqué à l'étranger>>;
				\item $A \cap B$ : <<l'article prélevé a été fabriqué à l'étranger et est une paire de bottes>>;
				\item $\bar{A} \cap C$ : <<l'article prélevé n'a pas été fabriqué à l'étranger et est une paire de chaussons>>.
			\end{itemize}
		\end{solution}
		
		\part[1] Calculer les probabilités $P(A)$, $P(B)$ et $P(A \cap B)$. 
		\begin{solution}
			Calcul de $P(A)$ :
			\begin{eqnarray*}
				P(A) &=& \dfrac{\num{3000}}{\num{4000}}\\
				P(A) &=& \num{0.75}
			\end{eqnarray*}
		
			Calcul de $P(B)$ :
			\begin{eqnarray*}
				P(B) &=& \dfrac{\num{1000}}{\num{4000}}\\
				P(B) &=& \num{0.25}
			\end{eqnarray*}
		
			Calcul de $P(A \cap B)$ :
			\begin{eqnarray*}
				P(A) &=& \dfrac{\num{750}}{\num{4000}}\\
				P(A) &=& \num{0.1875}
			\end{eqnarray*}
		
			La probabilité de l'événement A est \num{0.75}, celle de B est \num{0.25} et celle de $A \cap B$ est \num{0.1875}.
		\end{solution}
		
		\part[1] Calculer la probabilité que l'article prélevé ait été fabriqué à l'étranger ou que ce soit une paire de bottes.
		\begin{solution}
			Calcul de $P(A \cup B)$ :
			\begin{eqnarray*}
				P(A \cup B) &=& P(A) + P(B) - P(A \cap B)\\
				P(A \cup B) &=& \num{0.75} + \num{0.25} - \num{0.1875} \\
				P(A \cup B) &=& \num{0.8125}
			\end{eqnarray*}	
		
			La probabilité qu'un article pris au hasard ait été fabriqué à l'étranger ou soit une paire de bottes est \num{0.8125}.
		\end{solution}
		
		\part[1] Calculer les probabilités conditionnelles $P_B(A)$ et $P_C(A)$.
		\begin{solution}
			Calcul de $P_B(A)$ :
			\begin{eqnarray*}
				P_B(A) &=& \dfrac{\num{750}}{\num{1000}}\\
				P_B(A) &=& \num{0.75}
			\end{eqnarray*}	
		
			Calcul de $P_B(A)$ :
			\begin{eqnarray*}
				P_C(A) &=& \dfrac{\num{1875}}{\num{2500}}\\
				P_C(A) &=& \num{0.75}
			\end{eqnarray*}
			
			On a donc $P_B(A) = 0,75$ et $P_C(A)=0,75$.
		\end{solution}
		
		\part[1] Calculer la probabilité que sachant qu'il provient de France, l'article prélevé soit une paire de chaussures de sport. 
		\begin{solution}
			Calcul de la probabilité qu'un article soit une paire de  chaussures de sport sachant qu'il provient de France :
			
			Soit $S$ l'événement : <<l'article prélevé est une paire de chaussures de sport>>.
			
			\begin{eqnarray*}
				P_{\bar{A}}(S) &=& \dfrac{\num{125}}{\num{1000}}\\
				P_B(A) &=& \num{0.125}
			\end{eqnarray*}	
			
			La probabilité que sachant qu'il provient de France, l'article prélevé soit une paire de chaussures de sport est \num{0.125}.
		\end{solution}
	\end{parts}

	\question[1] Les événements $A$ et $B$ sont-ils indépendants ?
	\begin{solution}
		On a $P(A) = P_B{A}$ donc les événements $A$ et $B$ sont indépendants.
	\end{solution}
\end{questions}
