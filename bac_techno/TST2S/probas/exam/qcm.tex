\section{Un QCM (5 points)}

Cet exercice se présente sous la forme d'un questionnaire à choix multiple (QCM). Les cinq questions sont indépendantes. Pour chaque question, une seule réponse est exacte, on demande d'indiquer cette réponse sans la justifier. Chaque bonne réponse rapporte 1 point, chaque réponse incorrecte retire \num{0.25} point, une question sans réponse n'apporte ni ne retire aucun point.

Les 1200 élèves d'un lycée se répartissent de la façon suivante :

\begin{center}
	\begin{tabular}{|@{\ }c@{\ }|@{\ }c@{\ }|@{\ }c@{\ }|@{\ }c@{\ }|}
	\hline
	& \textbf{Fumeurs} & \textbf{Non-Fumeurs} & \textbf{Total} \\ \hline
	\textbf{Secondes}   & 231              & 189                  & 420            \\ \hline
	\textbf{Premières}  & 237              & 158                  & 395            \\ \hline
	\textbf{Terminales} & 192              & 193                  & 385            \\ \hline
	\textbf{Total}      & 660              & 540                  & 1200           \\ \hline
\end{tabular}

\end{center}

De plus, 60 \% des élèves de Seconde sont des filles, et parmi elles, 50 \% fument.

On choisit au hasard un élève parmi les 1200 du lycée. Chaque élève a la même probabilité d'être choisi.

On note :
\begin{itemize}
	\item $S$ l'événement : <<l'élève choisi est en Seconde>>;
	\item $F$ l'événement : <<l'élève choisi est fumeur>>;
	\item $T$ l'événement : <<l'élève choisi est en Terminale>>.
\end{itemize}

\begin{questions}
	\question[1] La probabilité $P(S)$ que l'élève choisi soit en seconde est égale à :
	
		\begin{oneparcheckboxes}
			\choice \num{0.66}
			\choice \num{0.55}
			\correctchoice \num{0.35}
		\end{oneparcheckboxes}
	
		
	\question[1] La probabilité $P_S(F)$ que l'élève soit fumeur sachant qu'il est en seconde est égale à :
	
	\begin{oneparcheckboxes}
		
		\correctchoice \num{0.55}
		\choice \num{0.1925}
		\choice \num{0.45}
	\end{oneparcheckboxes}	


	\question[1] Les événements $S$ et $F$ sont-ils indépendants ?
	
	\begin{oneparcheckboxes}
		
		\correctchoice oui
		\choice non
		\choice on ne peut pas répondre
	\end{oneparcheckboxes}

	\question[1] Les événements $S$ et $T$ sont :
	
	\begin{oneparcheckboxes}
		
		
		\choice contraires
		\correctchoice incompatibles
		\choice indépendants
	\end{oneparcheckboxes}

	\question[1] Quel est le pourcentage d'élèves de seconde qui sont des filles et qui fument ?
	
	\begin{oneparcheckboxes}
		
		\choice \num{10} \%
		\choice \num{45} \%
		\correctchoice \num{30} \%
	\end{oneparcheckboxes}	
\end{questions}