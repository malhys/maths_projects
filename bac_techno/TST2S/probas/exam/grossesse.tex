\section{Tabac pendant la grossesse (8 points)}

En 2007, une enquête réalisé sur le lien de cause à effet entre l'état tabagique de la mère pendant la grossesse  et les troubles respiratoires de l'enfant. Cette enquête est réalisée sur un échantillon de 1500 enfants de 10 ans.
Chaque enfant est classé dans un des trois groupe suivants :
\begin{itemize}
	\item les asthmatiques;
	\item ceux qui présentent des troubles asthmatiformes (considérés comme non asthmatiques);
	\item ceux sans troubles.

\end{itemize}

Le recueil des données est réalisé sous couvert de l'anonymat auprès des professionnels médicaux, 1500 fiches de renseignements anonymes ont été créées.
Ces fiches indiquent que :

\begin{itemize}
	\item 1223 enfants n'ont aucun troubles.
	\item \num{4.8} \% des enfants sont asthmatiques ; 75 \% d'entre eux ont une mère ayant fumé pendant la grossesse.
	\item 16 \% des mères ont fumé pendant la grossesse.
	\item 40 \% des enfants ayant des allergies asthmatiformes ont une mère n'ayant pas fumé pendant la grossesse.
\end{itemize}

\begin{questions}
	\question[2] Compléter le tableau suivant :
	
	\begin{center}
		\begin{tabular}{|@{\ }l@{\ }|@{\ }c@{\ }|@{\ }c@{\ }|@{\ \ \ \ \ }c@{\ \ \ \ \ }|}
	\hline
                     & Mère fumeuse & Mère non fumeuse &  \\
                     & pendant la                       & pendant la                           & Total                \\
                     & grossesse                        & grossesse                            &                      \\
	\hline
	& & &\\
Enfants asthmatiques &                                  &                                      & 72                   \\
	& & &\\
	\hline
Enfants présentant   &                                  &                                      &                      \\
un trouble           & 123                              &                                      &                      \\
asthmatiforme        &                                  &                                      &                      \\
	\hline
Enfant ne            &                                  &                                      &                      \\
présentant           &                                  &                                      & 1223                 \\
aucun trouble        &                                  &                                      &                      \\
	\hline
Total                & 240                              &                                      & 1500                \\
	\hline
\end{tabular}
	\end{center}

	\begin{solution}
		\begin{center}
			\begin{tabular}{|@{\ }l@{\ }|@{\ }c@{\ }|@{\ }c@{\ }|@{\ \ \ \ \ }c@{\ \ \ \ \ }|}
	\hline
                     & Mère fumeuse & Mère non fumeuse &  \\
                     & pendant la                       & pendant la                           & Total                \\
                     & grossesse                        & grossesse                            &                      \\
	\hline
	& & &\\
Enfants asthmatiques &        54                        &    18                                & 72                   \\
	& & &\\
	\hline
Enfants présentant   &                                  &                                      &                      \\
un trouble           & 123                              &   226                                &  349                 \\
asthmatiforme        &                                  &                                      &                      \\
	\hline
Enfant ne            &                                  &                                      &                      \\
présentant           &   63                             &    1160                              & 1223                 \\
aucun trouble        &                                  &                                      &                      \\
	\hline
Total                & 240                              &      1260                            & 1500                \\
	\hline
\end{tabular}
		\end{center}
	\end{solution}

	\question On prélève au hasard une fiche de renseignement d'un enfant. ON admet que chacun des choix est équiprobable. On considère les événements suivants :
	
	\begin{itemize}
		\item $A$ : << La fiche indique que l'élève est asthmatique>>; 
		\item $T$ : << La fiche indique que l'élève présente des troubles asthmatiformes>>;
		\item $F$ : << La fiche indique que la mère a fumé pendant la grossesse>>;
	\end{itemize}

	Les résultats approchés sont à arrondir au millième.
	
	\begin{parts}
		\part[1\half] Calculer la probabilité des événements $T$ et $F$.
		
		\begin{solution}
			\begin{eqnarray*}
				P(T) &=& \frac{205}{1500} \approx \num{0.137} \\
				P(F) &=& \frac{240}{1500} = \num{0.16} \\
			\end{eqnarray*}
		\end{solution}
	
		\part[1\half] Définir par une phrase l'événement $T \cap F$ puis calculer sa probabilité.
		\begin{solution}
			L'événement $T \cap F$ est <<La fiche indique que l'élève présente des troubles asthmatiformes et sa mère a fumé pendant la grossesse. Sa probabilité est \num{0.082} ($\frac{123}{1500}$).
			 
		\end{solution}
		
		\part[1] Calculer la probabilité que l'enfant ait des troubles asthmatiformes ou que sa mère soit fumeuse.
		\begin{solution}
			\begin{eqnarray*}
				P(T \cup F) &=& P(T) + P(F) - P(T \cap F) \\
				P(T \cup F) &=& \num{0.137} + \num{0.16} - \num{0.082} \\
				P(T \cup F) &=& \num{0.215}
			\end{eqnarray*}
		
			La probabilité que l'enfant ait des troubles asthmatiques ou que sa mère soit fumeuse est \num{0.311}.
		\end{solution}
		
		\part[1] Calculer la probabilité que l'enfant soit asthmatique sachant que sa mère est fumeuse.
		\begin{solution}
			\begin{eqnarray*}
				P_F(A) &=& \frac{54}{240} \\
				P_F(A) &=& \num{0.225}\\
			\end{eqnarray*}
		
		La probabilité que l'enfant soit asthmatique sachant que sa mère est fumeuse est \num{0.225}.
		\end{solution}
	\end{parts}

	\question[1] On prélève au hasard une fiche parmi celles indiquant que la mère a fumé pendant la grossesse. Calculer la probabilité que l'enfant n'ait aucun trouble. 
	
	\begin{solution}
		Soit $N$ l'événement : <<La fiche tirée indique que l'élève ne présente aucun trouble>>.
		
		\begin{eqnarray*}
			P_F(N) &=& \frac{63}{240} \\
			P_F(N) &\approx&  \num{0.263}\\
		\end{eqnarray*}
	
	La probabilité que l'enfant n'ait aucun trouble sachant que sa mère a fumé pendant la grossesse est \num{0.263}.
	\end{solution}
\end{questions}