\begin{center}
	%\centering
	
	{\scshape\LARGE \textbf{BACCALAUR\'EAT TECHNOLOGIQUE} \par}
	\vspace{1cm}
	{\scshape\Large \textbf{SESSION 2019}\par}
	\vspace{1.5cm}
	

	\begin{large}
		\begin{tabular}{|@{\ }c@{\ }|@{\ }c@{\ }|}
		\hline
		%\ & \ \\
		\'Epreuve : \textbf{MATH\'EMATIQUES} & Série : \textbf{Sciences et Technologies de}  \\ 
		\textit{\'Epreuve blanche}&  \textbf{la Santé et du Social (ST2S)} \\ \hline
		\ & \ \\
		Durée de l'épreuve : \textbf{2 heures} & Coefficient : \textbf{3} \\ 
		\ & \ \\
		\hline
	\end{tabular}
	\end{large}
		
	\vspace{1cm}
	{\large\bfseries \'EPREUVE DU MERCREDI 30 JANVIER 2019}
	
	\vspace{1cm}
	{\itshape L'usage de tout modèle de calculatrice, avec ou sans mode examen, est autorisé\par}
	%\vfill
	\vspace{1.5cm}
	{\bfseries Ce sujet comporte \pageref{LastPage} pages numérotées de 1/\pageref{LastPage} à \pageref{LastPage}/\pageref{LastPage} }
	
	\vspace{0.5cm}
	{\bfseries Ce sujet comporte 1 annexe située page \pageref{sec:nuage} /\pageref{LastPage} à remettre avec la copie.}
	
	\vspace{0.5cm}
	{\bfseries\itshape Le candidat doit s'assurer que le sujet distribué est complet. }
	
	\vfill	
	
	\fbox{
		\begin{minipage}{0.9\textwidth}
			\large
			Il  est  rappelé  que  la  qualité  de  la  rédaction,  la  clarté  et  la  précision  des  raisonnements entreront pour une part importante dans l'appréciation des copies. 
			
			Cependant, le candidat est invité à faire figurer sur la copie toute trace de recherche, même incomplète ou infructueuse, qu'il aura développée. 
		\end{minipage}
	}
	
	\vfill

% Bottom of the page
	%{\large \today\par}
\end{center}
\newpage