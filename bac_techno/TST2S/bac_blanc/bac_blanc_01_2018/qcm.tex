%QCM pourcentages / proportions / taux d'évolution / évolutions réciproques

\section{Questions à choix multiple \textit{(6 points)}}

Cet exercice se présente sous la forme d'un questionnaire à choix multiple (QCM). Les six questions sont indépendantes. Pour chaque question, une seule réponse est exacte, on demande d'indiquer cette réponse sans la justifier. Chaque bonne réponse rapporte 1 point, chaque réponse incorrecte retire \num{0.25} point, une question sans réponse n'apporte ni ne retire aucun point. Si le total est négatif, la note est ramenée à 0.

\begin{questions}
	\question[1] La population d'une ville est de \num{30000} habitants. Si elle augmente de 15 \% par an, quel sera le nombre d'habitants de cette ville dans deux ans ?
	
	\begin{oneparcheckboxes}
		\choice \num{30675};
		\choice \num{3900};
		\choice \num{35175};
		\CorrectChoice \num{39675}.
	\end{oneparcheckboxes} 

	\question[4] Une enquête est menée auprès de 250 personnes a donné les résultats suivants :
	
	%

\begin{tabular}{|@{\ }l@{\ }|@{\ }c@{\ }|@{\ }c@{\ }|@{\ }c@{\ }|@{\ }c@{\ }|@{\ }c@{\ }|@{\ }c@{\ }|@{\ }c@{\ }|}
	\hline
			& \multicolumn{3}{c|}{Soins} 			& \multicolumn{3}{c|}{Soins} 		&  Total\\
			& \multicolumn{3}{c|}{au dispensaire}   & \multicolumn{3}{c|}{à domicile} 	&  \\ \hline
%--------------------------------------------------------------------
Temps		&  10    	&  20      	&   60    	& 10    &  20     &   60			&  \\
des soins	&  min   	&  min      &     min  	& min   &  min    &     min  		&  \\ \hline
%--------------------------------------------------------------------
Femmes		&	     	&     		&       	&       &         &       			&  \\
(30 ans		&	     	&     		&       	&       &         &       			&   \\
et plus)	&	     	&     		&       	&       &         &       			&   \\ \hline
%--------------------------------------------------------------------
Femmes		&	     	&     		&       	&       &         &       			&  \\
(30 ans		&	     	&     		&       	&       &         &       			&   \\
et plus)	&	     	&     		&       	&       &         &       			&   \\ \hline%--------------------------------------------------------------------
Femmes		&	     	&     		&       	&       &         &       			&  \\
(30 ans		&	     	&     		&       	&       &         &       			&   \\
et plus)	&	     	&     		&       	&       &         &       			&   \\ \hline
%--------------------------------------------------------------------
Femmes		&	     	&     		&       	&       &         &       			&  \\
(30 ans		&	     	&     		&       	&       &         &       			&   \\
et plus)	&	     	&     		&       	&       &         &       			&   \\ \hline%--------------------------------------------------------------------
Total		&        &          &       			&       &         &       			&  \\ \hline
\end{tabular}
	\emph{Tous les pourcentages donnés ci-dessous sont 	arrondis à 1 \%.}
	\begin{parts}
		\part[1] Quel est le pourcentage des hommes ?
	
		\begin{checkboxes}
			\correctchoice 47 \% 
			\choice 34 \%
			\choice 14 \%
			\choice 79 \%
		\end{checkboxes}
	
		\part[1] Quel est le pourcentage des personnes qui reçoivent des soins de plus de 15 min ?
		
		\begin{checkboxes}
			\choice 25 \% 
			\choice 40 \%
			\correctchoice 48 \%
			\choice 53 \%
		\end{checkboxes}
	
		\part[1] Parmi les femmes, quel est le pourcentage de celles qui se font soigner à domicile ?
		
		\begin{checkboxes}
			\choice 58 \% 
			\correctchoice 62 \%
			\choice 65 \%
			\choice 70 \%
		\end{checkboxes}
	
		\part[1] Parmi les personnes qui se font soigner à domicile, quel est le pourcentage des hommes ?
		
		\begin{checkboxes}
			\choice 15 \% 
			\choice 31 \%
			\correctchoice 45 \%
			\choice 79 \%
		\end{checkboxes}
	\end{parts}

	\question[1] Dans les cas suivants, quels sont les taux d'évolution réciproque l'un de l'autre ?
	
	\begin{oneparcheckboxes}
		\choice 30 \% et - 30 \%
		\CorrectChoice 25 \% et - 20 \%
		\choice 150 \% et - 50 \%
		\choice 60 \% et - 40 \%
	\end{oneparcheckboxes}
\end{questions}  