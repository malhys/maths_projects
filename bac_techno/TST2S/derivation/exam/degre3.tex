\section{\'Etude des variations d'une fonction polynôme de degré 3}\label{ex:deg3}

On considère une fonction $f$ définie et dérivable sur l'intervalle $[0 ; 2,5]$.

On note $f'$ la fonction dérivée de la fonction $f$. 

On donne en annexe la courbe représentative de de la fonction $f$, appelée $\mathcal{C}$, dans un repère orthogonal.

La courbe $\mathcal{C}$ possède les propriétés suivantes :
\begin{itemize}
	\item La courbe $\mathcal{C}$ passe par le point $A (1 ; 5,5)$;
	\item La courbe $\mathcal{C}$ passe par le point $B (2 ; 2)$;
	\item La tangente en $B$ à la courbe $\mathcal{C}$ est horizontale ;
	\item La tangente en $A$ à la courbe $\mathcal{C}$ passe par le point $T (0; 8,5)$.
\end{itemize}

\subsection{Lectures graphiques}

\begin{questions}
	\question[2] Placer les points $A$, $B$ et $T$ sur la figure \ref{fig:c} présente en annexe \ref{app:fig} en tenant compte des informations ci-dessus, et tracer les tangentes à la courbe $\mathcal{C}$ en $A$ et en $B$.
	
	\question[2] Déterminer $f(1)$, $f(2)$ et $f'(1)$.
	
	\question[1] Donner par lecture graphique une valeur approchée des solutions de l'équation $f(x) = 3$.
	
	\question[2] Justifier que $f'(2) = 0$. Donner par lecture graphique une valeur approchée de la deuxième solution de $f'(x) = 0$.
\end{questions}

\subsection{\'Etude des variations}

La fonction dont on connait la courbe $\mathcal{C}$ est définie sur l'intervalle $[0 ; 2,5]$ par :

\begin{equation*}
	f(x) = 4x^3 - 16,5 x^2 + 18 x.
\end{equation*}


\begin{questions}
	\question[2] Compléter le tableau de valeurs suivant:
	
	\begin{center}
		\begin{tabular}{|@{\ \ \ \ }c@{\ \ \ \ }|@{\ \ \ \ }c@{\ \ \ \ }|@{\ \ \ \ }c@{\ \ \ \ }|@{\ \ \ \ }c@{\ \ \ \ }|@{\ \ \ \ }c@{\ \ \ \ }|@{\ \ \ \ }c@{\ \ \ \ }|@{\ \ \ \ }c@{\ \ \ \ }|}
		\hline
		$x$ & 0 & \num{0.5} & \num{1}  & \num{1.5} & \num{2} & \num{2.5} \\ \hline
		$f(x)$ & & & & & & \\ \hline
	\end{tabular}
	\end{center}

	\question[4\half]
		\begin{parts}
			\part[2] Calculer $f'(x)$.
			
			\part[1] Montrer que :
			
				\begin{equation}
					f'(x) = (12x - 24) (x - \num{0.75})
				\end{equation}
				
			\part[1\half] \'Etudier le signe de $f'(x)$ suivant les valeurs de $x$, sur l'intervalle $[0 ; \num{2.5}]$ à l'aide d'un tableau de signe.
			
			
		\end{parts}
	
	\question[1\half] En déduire le tableau de variation de la fonction $f$.
\end{questions}