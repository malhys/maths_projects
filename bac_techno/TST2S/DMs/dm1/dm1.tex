\documentclass[a4paper,11pt]{exam}
%\printanswers % pour imprimer les réponses (corrigé)
\noprintanswers % Pour ne pas imprimer les réponses (énoncé)
\addpoints % Pour compter les points
% \noaddpoints % pour ne pas compter les points
%\qformat{\textbf{\thequestion ) } }
\qformat{\textbf{\thequestion )}} % Pour définir le style des questions (facultatif)
\usepackage{color} % définit une nouvelle couleur
\shadedsolutions % définit le style des réponses
% \framedsolutions % définit le style des réponses
\definecolor{SolutionColor}{rgb}{0.8,0.9,1} % bleu ciel
\renewcommand{\solutiontitle}{\noindent\textbf{Solution:}\par\noindent} % Définit le titre des solutions

\usepackage{fullpage}



\makeatletter

\def\maketitle{{\centering%
	\par{\huge\textbf{\@title}}%
	\par{\@date}%
	\par}}

\makeatother

\lhead{NOM Pr\'enom :}
\rhead{\textbf{Les r\'eponses doivent \^etre justifi\'ees}}
\cfoot{\thepage / \pageref{LastPage}}


%\usepackage{../../pas-math}
%\usepackage{../../moncours}


%\usepackage{pas-cours}
%-------------------------------------------------------------------------------
%          -Packages nécessaires pour écrire en Français et en UTF8-
%-------------------------------------------------------------------------------
\usepackage[utf8]{inputenc}
\usepackage[frenchb]{babel}
\usepackage[T1]{fontenc}
\usepackage{lmodern}
%-------------------------------------------------------------------------------

%-------------------------------------------------------------------------------
%                          -Outils de mise en forme-
%-------------------------------------------------------------------------------
\usepackage{hyperref}
\hypersetup{pdfstartview=XYZ}
\usepackage{enumerate}
\usepackage{graphicx}
\usepackage{multicol}

\usepackage{anysize} %%pour pouvoir mettre les marges qu'on veut
%\marginsize{2.5cm}{2.5cm}{2.5cm}{2.5cm}

\usepackage{indentfirst} %%pour que les premier paragraphes soient aussi indentés
%-------------------------------------------------------------------------------


%-------------------------------------------------------------------------------
%                  -Nécessaires pour écrire des mathématiques-
%-------------------------------------------------------------------------------
\usepackage{amsfonts}
\usepackage{amssymb}
\usepackage{amsmath}
\usepackage{amsthm}
\usepackage{tikz}
%-------------------------------------------------------------------------------

%-------------------------------------------------------------------------------
%                     -Mise en forme d'exercices-
%-------------------------------------------------------------------------------
\newtheoremstyle{exostyle}
{\topsep}% espace avant
{\topsep}% espace apres
{}% Police utilisee par le style de thm
{}% Indentation (vide = aucune, \parindent = indentation paragraphe)
{\bfseries}% Police du titre de thm
{.}% Signe de ponctuation apres le titre du thm
{ }% Espace apres le titre du thm (\newline = linebreak)
{\thmname{#1}\thmnumber{ #2}\thmnote{. \normalfont{\textit{#3}}}}% composants du titre du thm : \thmname = nom du thm, \thmnumber = numéro du thm, \thmnote = sous-titre du thm

\theoremstyle{exostyle}
\newtheorem{exercice}{Exercice}

\newenvironment{questions}{
\begin{enumerate}[\hspace{12pt}\bfseries\itshape a.]}{\end{enumerate}
} %mettre un 1 à la place du a si on veut des numéros au lieu de lettres pour les questions 
%-------------------------------------------------------------------------------



%-------------------------------------------------------------------------------
%                    - Racourcis d'écriture -
%-------------------------------------------------------------------------------

% Angles orientés (couples de vecteurs)
\newcommand{\aopp}[2]{(\vec{#1}, \vec{#2})} %Les deuc vecteurs sont positifs
\newcommand{\aopn}[2]{(\vec{#1}, -\vec{#2})} %Le second vecteur est négatif
\newcommand{\aonp}[2]{(-\vec{#1}, \vec{#2})} %Le premier vecteur est négatif
\newcommand{\aonn}[2]{(-\vec{#1}, -\vec{#2})} %Les deux vecteurs sont négatifs

%Ensembles mathématiques
\newcommand{\naturels}{\mathbb{N}} %Nombres naturels
\newcommand{\relatifs}{\mathbb{Z}} %Nombres relatifs
\newcommand{\rationnels}{\mathbb{Q}} %Nombres rationnels
\newcommand{\reels}{\mathbb{R}} %Nombres réels
\newcommand{\complexes}{\mathbb{C}} %Nombres complexes
%-------------------------------------------------------------------------------




%\usepackage{fullpage}
\author{\ }
\date{A rendre pour le 12 Novembre 2018}
\title{$T^{le}$ $ST_2S$ : DM num\'ero 1}


\begin{document}
%	\usepackage{fancyhdr}
%	
%	\pagestyle{fancy}
%	\fancyhf{}
	%\rhead{Share\LaTeX}

	\maketitle

\section{Plan de redressement (4 points)}

Une entreprise soumet au vote de ses employés un plan de redressement, avec la menace : <<Si 10 \% des employés votent contre le projet, nous fermeront l'usine>>.

Le vote a eu lieu et on peut lire dans un journal : << l'entreprise ne fermera pas ; 2 \% seulement des votes sont contre le plan de la direction. Cependant, 25 \% des employés n'ont pas voté >>.

\begin{questions}

	\question Calculer, dans l'ensemble des employés, le pourcentage de ceux qui ont voté contre le plan ; constater qu'il est effectivement inférieur à 10 \%.
	
	\begin{solution}
		Calcul du taux de participation au vote :
		
		$100 - 25 = 75 $ \\
		Donc 75 \% des employés ont voté.
		
		Parmi les 75 \% de votants, 2 \% ont voté contre :
		
		\begin{equation*}
			\frac{2}{100} \times \dfrac{75}{100} = \num{0.015}
		\end{equation*}
		
		Donc \num{1.5} \% des employés ont voté contre le plan, c'est donc bien moins de 10 \%. 
		
	\end{solution}
	
	\question Dans un autre journal, il est écrit: <<la direction voulait que plus de 90 \% des employés votent en faveur du plan, faute de quoi elle fermerait l'usine. Ses v\oe ux ont été exaucés puisque 98 \% des votes sont en faveur de la direction. Certes 25 \% des employés n'ont pas voté, mais cela ne change rien.>>
	
	Calculer dans l'ensemble des employés, le pourcentage de ceux qui ont voté en faveur du plan. En déduire que l'auteur de l'article aurait dû réfléchir davantage avant de l'écrire.  
	
		\begin{solution}
			98 \% des votants se sont prononcés pour le plan proposé :
			
			\begin{equation*}
				\frac{98}{100} \times \dfrac{75}{100} = \num{0.735}
			\end{equation*}
			
			Sur l'ensemble des employés seuls \num{73.5} \% d'entre eux ont voté en faveur de la direction, et non 98 \%. L'auteur de l'article s'est trompé.
		\end{solution}
	
	
\end{questions} 


\section{Efficacité d'un médicament}

120 personnes atteintes d'une maladie ont accepté de tester l'efficacité d'un nouveau médicament. 

Pendant un mois 80 personnes parmi elles ont pris le médicament, les autres ont prit le placebo.

A l'issue de l'expérimentation :
\begin{itemize}
	\item parmi les personnes ayant pris le médicament, 75\% ont vu leur santé s'améliorer;
	\item parmi les personnes ayant prit le placebo, seulement 5 personnes ont vu leur santé s'améliorer.
\end{itemize} 
\begin{center}
	
	{\small \begin{tabular}{|@{\ }c@{\ }|@{\ }c@{\ }|@{\ }c@{\ }|@{\ }c@{\ }|}
			\hline
			& Ont vu leur santé s'améliorer & N'ont pas vu leur santé s'améliorer &  Total  \\
			\hline
			Ont pris le médicament &  &  &    \\
			\hline
			Ont prit le placebo &  &  &   \\
			\hline
			Total &  &  & 120   \\
			\hline
	\end{tabular}}
\end{center}


\begin{questions}
	\question Compléter le tableau.
	
	\begin{solution}
		{\small \begin{tabular}{|@{\ }c@{\ }|@{\ }c@{\ }|@{\ }c@{\ }|@{\ }c@{\ }|}
				\hline
				& Ont vu leur santé s'améliorer & N'ont pas vu leur santé s'améliorer &  Total  \\
				\hline
				Ont pris le médicament &  60 & 20 & 80   \\
				\hline
				Ont prit le placebo & 5 & 35 &  40 \\
				\hline
				Total &  65 & 55 & 120   \\
				\hline
		\end{tabular}}
		
	\end{solution}
	\question \begin{parts}
		\part Calculer le pourcentage de personnes qui ont vu leu santé s'améliorer (arrondir à \num{0.01}\% près).
		\begin{solution}
			$ \dfrac{65}{120} \approx \num{0.5417} $ Soit \num{54.17} \%.
		\end{solution}
		\part Parmi les personnes qui n'ont pas prit le médicament, calculer le pourcentage de celles qui ont vu leur santé s'améliorer.
		\begin{solution}
			$ \dfrac{60}{80} = \num{0.75} $ Soit \num{75} \%.
		\end{solution}
		\part Parmi les personnes qui ont vu leur santé s'améliorer, calculer le pourcentage de celles qui ont prit le médicament (arrondir à \num{0.1}\% près).
		\begin{solution}
			$ \dfrac{60}{65} \approx \num{0.923} $ Soit \num{92.3} \%.
		\end{solution}
	\end{parts}
\end{questions}

\section{Crise financière : des records historiques (6 points)}

Le tableau suivant indique les variations quotidiennes du CAC 40 dans la semaine du lundi 6 au vendredi 10 octobre 2008.

\begin{center}
	\begin{tabular}{|@{\ }c@{\ }|@{\ }c@{\ }|@{\ }c@{\ }|@{\ }c@{\ }|@{\ }c@{\ }|@{\ }c@{\ }|}
		\hline
		\textbf{Date}   & \textbf{6/10}   & \textbf{7/10}   & \textbf{8/10}   & \textbf{9/10}  & \textbf{10/10} \\ \hline
		\textbf{CAC 40} & - \num{9.04} \% & + \num{0.55} \% & - \num{6.31} \% & - \num{1.55} \% & - \num{7.73} \%  \\ \hline
	\end{tabular}
\end{center}

Donner tous les résultats demandés avec deux décimales.

\begin{questions}
	\question[1\half] Vérifier que le CAC 40 a perdu \num{22.16} \% cette semaine.
	
	\question[1\half] Quel pourcentage de hausse doit subir le CAC 40 pour retrouver son niveau d'avant cette semaine-là ?
	
	\question[1\half] Avant cette semaine de baisse, le CAC 40 était à \num{4080.75} points. \`A combien a-t-il clôturé le vendredi 10 octobre ?
	
	\question[1\half] Lundi 13 octobre 2008, le CAC 40 est passé de \num{3176.49} à \num{3531.50} points, signant ainsi la plus forte hausse de son histoire. 
	Calculer le pourcentage de hausse correspondant.  
\end{questions} 


\section{Peinture}

La peinture Net est vendue dans deux magasins A et B en bidons de 5 litres, au prix de 20 euros le bidon.

\begin{questions}
	\question Le magasin A fait l'offre publicitaire : <<réduction de 20 \% sur le prix du bidon>>.
	
	Calculer le nouveau prix du bidon de peinture et en déduire le prix d'un litre de peinture.
	
	\question Le magasin B fait l'offre publicitaire : <<20 \% de peinture en plus dans le bidon pour le même prix.>>.
	
	Calculer la nouvelle quantité de peinture dans le bidon et en déduire le prix d'un litre de peinture.
	
	\question Laquelle des deux offres est la plus intéressante ? 
\end{questions}


	\label{LastPage}
	

\end{document}