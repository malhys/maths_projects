\section{Crise financière : des records historiques (6 points)}

Le tableau suivant indique les variations quotidiennes du CAC 40 dans la semaine du lundi 6 au vendredi 10 octobre 2008.

\begin{center}
	\begin{tabular}{|@{\ }c@{\ }|@{\ }c@{\ }|@{\ }c@{\ }|@{\ }c@{\ }|@{\ }c@{\ }|@{\ }c@{\ }|}
		\hline
		\textbf{Date}   & \textbf{6/10}   & \textbf{7/10}   & \textbf{8/10}   & \textbf{9/10}  & \textbf{10/10} \\ \hline
		\textbf{CAC 40} & - \num{9.04} \% & + \num{0.55} \% & - \num{6.31} \% & - \num{1.55} \% & - \num{7.73} \%  \\ \hline
	\end{tabular}
\end{center}

Donner tous les résultats demandés avec deux décimales.

\begin{questions}
	\question[1\half] Vérifier que le CAC 40 a perdu \num{22.16} \% cette semaine.
	
	\question[1\half] Quel pourcentage de hausse doit subir le CAC 40 pour retrouver son niveau d'avant cette semaine-là ?
	
	\question[1\half] Avant cette semaine de baisse, le CAC 40 était à \num{4080.75} points. \`A combien a-t-il clôturé le vendredi 10 octobre ?
	
	\question[1\half] Lundi 13 octobre 2008, le CAC 40 est passé de \num{3176.49} à \num{3531.50} points, signant ainsi la plus forte hausse de son histoire. 
	Calculer le pourcentage de hausse correspondant.  
\end{questions} 
