\section{A propos des acides gras oméga-3}
On a récemment découvert que les acides gras oméga-3, présents dans des poissons comme la truite ou le saumon, ont un effet protecteur contre les maladies cardio-vasculaires.
\emph{Les pourcentages demandés seront arrondis à \num{e-2}. \%} 

\begin{questions}
	\question Une portion de 180 g de saumon d'élevage fournit environ \num{1.5}g d'oméga-3. Calculer le pourcentage d'oméga-3 dans le saumon d'élevage.
	\begin{solution}
		$\dfrac{\num{1.5}}{\num{180}} \approx \num{0.0083}$, soit \num{0.83} \% d'oméga-3 dans le saumon d'élevage.
	\end{solution}
	
	\question Le pourcentage d'omega-3 dans le saumon sauvage est de \num{0.78} \%. En déduire la quantité d'oméga-3 contenue dans une portion de 180g de saumon sauvage (arrondir à \num{0.1} g).
	\begin{solution}
		$180 \times \frac{\num{0.78}}{100} = \num{1.404}$, soit environ \num{1.4} g d'oméga-3 pour 180 g de saumon sauvage.
	\end{solution}
	
	\question Consigner les résultats précédents dans le tableau suivant et finir de le compléter. Le détail des calculs n'est pas demandé.
	
	{\footnotesize \begin{tabular}{@{\ }c@{\ }|@{\ }c@{\ }|@{\ }c@{\ }|@{\ }c@{\ }|@{\ }c@{\ }|}
		\cline{2-5}
		& \multicolumn{2}{c|}{Elevage}               & \multicolumn{2}{c|}{Sauvage}               \\ \cline{2-5} 
		& Pourcentage d'oméga-3 & Quantité d'oméga-3 & Pourcentage d'oméga-3 & Quantité d'oméga-3 \\ \hline
		\multicolumn{1}{|@{\ }c@{\ }|}{Saumon (180 g)} &                       & \num{1.5} g      & \num{0.78}\%        &                    \\ \hline
		\multicolumn{1}{|@{\ }c@{\ }|}{Truite (180 g)} &                       & \num{1.3} g      & \num{0.22}\%        &                    \\ \hline
	\end{tabular}}
	
	\begin{solution}
		{\footnotesize \begin{tabular}{@{\ }c@{\ }|@{\ }c@{\ }|@{\ }c@{\ }|@{\ }c@{\ }|@{\ }c@{\ }|}
				\cline{2-5}
				& \multicolumn{2}{c|}{Elevage}               & \multicolumn{2}{c|}{Sauvage}               \\ \cline{2-5} 
				& Pourcentage d'oméga-3 & Quantité d'oméga-3 & Pourcentage d'oméga-3 & Quantité d'oméga-3 \\ \hline
				\multicolumn{1}{|@{\ }c@{\ }|}{Saumon (180 g)} &  \num{0.83} \%   & \num{1.5} g      & \num{0.78}\%        &    \num{1.4} g                \\ \hline
				\multicolumn{1}{|@{\ }c@{\ }|}{Truite (180 g)} &  \num{0.72} \%  & \num{1.3} g      & \num{0.22}\%        &  \num{0.4} g\\ \hline
			\end{tabular}}
	\end{solution}
	
	\question La consommation d'une portion de 180 g de truite d'élevage couvre environ 37 \% des besoins hebdomadaires d'un être humain. Montrer que ces besoins, arrondis à \num{0.1} g, sont de \num{3.5} g.
	
	\begin{solution}
		Soit $x$ le besoin hebdomadaire en oméga-3 d'un être humain. On a $\dfrac{37 \times x}{100}=\num{1.3}$.
		
		Donc $x=\dfrac{\num{1.3}}{\num{0.37}} \approx \num{3.51}$ g.
	\end{solution} 
	
	\question Retrouver la réponse précédente sachant que ces besoins hebdomadaires sont exactement couverts si on consomme 450 g de saumon sauvage.
	\begin{solution}
		$\dfrac{450 \times \num{0.78}}{100} = \num{3.51}$ soit \num{3.5} g.
	\end{solution}
	
	\question Calculer la quantité de truite sauvage qu'il faudrait consommer pour couvrir la totalité de ces besoins hebdomadaires (arrondir à 10 g).
	\begin{solution}
		$\dfrac{\num{3.5}}{\num{0.4}}\times 180 = 1575$, soit 1570 g de truite sauvage pour couvrir les besoins hebdomadaires en oméga-3. 
	\end{solution}
\end{questions}