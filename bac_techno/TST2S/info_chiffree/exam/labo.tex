\section{Le laboratoire perd du terrain (4 points)}

Le chiffre d'affaires annuel d'un laboratoire pharmaceutique était en 2008 de \num{32860000} euros et en 2009 de \num{28947000} euros.
\begin{questions}
	\question[2] Calculer le pourcentage de baisse du chiffre d'affaire de l'entreprise entre 2008 et 2009. Arrondir à \num{0.01} \%.
	\begin{solution}
		$\dfrac{\num{28947000} - \num{32860000}}{\num{32860000}} \approx \num{-0.1191}$, soit une baisse de \num{11.91} \%.
	\end{solution}
	
	\question[2] Calculer le pourcentage de hausse qui ramènerait, en 2010, le chiffre d'affaires au niveau de 2008. Arrondir les coefficients multiplicateurs à \num{e-4}.
	\begin{solution}
		Le coefficient multiplicateur correspondant à une baisse de \num{11.91} \% est ($1-\num{0.1191} = \num{0.8809}$).
		
		$\dfrac{1}{\num{0.8809}} = \num{1.1352}$, soit une hausse de \num{13.52} \%.
	\end{solution}
\end{questions}
