\section{Plan de redressement}

Une entreprise soumet au vote de ses employés un plan de redressement, avec la menace : <<Si 10 \% des employés votent contre le projet, nous fermeront l'usine>>.

Le vote a eu lieu et on peut lire dans un journal : << l'entreprise ne fermera pas ; 2 \% seulement des votes sont contre le plan de la direction. Cependant, 25 \% des employés n'ont pas voté >>.

\begin{questions}
	\question Calculer, dans l'ensemble des employés, le pourcentage de ceux qui ont voté contre le plan ; constater qu'il est effectivement inférieur à 10 \%.
	
	\begin{solution}
		Calcul du taux de participation au vote :
		
		$100 - 25 = 75 $ \\
		Donc 75 \% des employés ont voté.
		
		Parmi les 75 \% de votants, 2 \% ont voté contre :
		
		\begin{equation*}
			\frac{2}{100} \times \dfrac{75}{100} = \num{0.015}
		\end{equation*}
		
		Donc \num{1.5} \% des employés ont voté contre le plan, c'est donc bien moins de 10 \%. 
		
	\end{solution}
	
	\question Dans un autre journal, il est écrit: <<la direction voulait que plus de 90 \% des employés votent en faveur du plan, faute de quoi elle fermerait l'usine. Ses v\oe ux ont été exaucés puisque 98 \% des votes sont en faveur de la direction. Certes 25 \% des employés n'ont pas voté, mais cela ne change rien.>>
	
	Calculer dans l'ensemble des employés, le pourcentage de ceux qui ont voté en faveur du plan. En déduire que l'auteur de l'article aurait dû réfléchir davantage avant de l'écrire.  
	
		\begin{solution}
			98 \% des votants se sont prononcés pour le plan proposé :
			
			\begin{equation*}
				\frac{98}{100} \times \dfrac{75}{100} = \num{0.735}
			\end{equation*}
			
			Sur l'ensemble des employés seuls \num{73.5} \% d'entre eux ont voté en faveur de la direction, et non 98 \%. L'auteur de l'article s'est trompé.
		\end{solution}
	
	
\end{questions} 