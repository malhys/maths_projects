\section{Réussite des filles (6 points)}

Lors d'un discours au cours duquel il a donné les résultats des examens de fin d'études des deux universités d'un pays, le dictateur dirigeant de ce pays a déclaré :

<< Dans l'Université du Nord, 82 \% des garçons et 80 \% des filles ont réussi. Dans l'université du Sud, 56 \% des garçons et 52 \% des filles ont réussi.

Je ne suis pas sexiste, mais il faut bien reconnaître que dans notre pays, les garçons réussissent mieux que les filles>>.

\begin{questions}
	\question[1\half] Dans l'Université du Nord, il y avait 500 candidats de sexe masculin et 500 candidats de sexe féminin.
	
	Calculer le nombre de garçons et le nombre de filles qui ont réussi dans cette université.
	

	\begin{solution}
		\begin{eqnarray*}
			500 \times \frac{82}{100} = 410 \\ 
			& & \\
			& & \\
			500 \times \frac{80}{100} = 400 \\
		\end{eqnarray*}
			

		
		Dans cette université 410 garçons et 400 filles ont réussi.
		
	\end{solution}
	
	\question Dans l'Université du Sud, il y avait 800 candidats de sexe masculin et 200 candidats de sexe féminin.
	
	Calculer le nombre de garçons et le nombre de filles qui ont réussi dans cette université.
	\begin{solution}
		%\begin{eqnarray*}
		\begin{eqnarray*}
			800 \times \frac{56}{100} & = & 448 \\ 
			& & \\
			& & \\
			200 \times \frac{52}{100} & = & 104 
		\end{eqnarray*}
			 
		%\end{eqnarray*}
		
		Dans cette université 448 garçons et 104 filles ont réussi.
		
	\end{solution}
	
	\question
		\begin{parts}

			\part Combien y avait-t-il de garçons candidats dans le pays ? Calculer le pourcentage de garçons qui ont réussi dans le pays.
			\begin{solution}
				\begin{eqnarray*}
					500 + 800 &=& 1300 \\
				\end{eqnarray*}
			
				Dans le pays, il y avait 1300 garçons candidats.
				
				\begin{eqnarray*}
					\frac{410 + 448}{1300} &= & \frac{858}{1300} \\
					& & \\
					\frac{410 + 448}{1300} &= & \num{0.66} \\
				\end{eqnarray*}
			
				Soit \num{66} \% de réussite chez les garçons dans le pays.
			\end{solution}
			\part Calculer le pourcentage de filles qui ont réussi dans le pays.
				\begin{solution}
					\begin{eqnarray*}
						\frac{400 + 104}{500 + 200} &= & \frac{504}{700} \\
						& & \\
						\frac{400 + 104}{500 + 200} &= & \num{0.72} \\
					\end{eqnarray*}
				
					Soit 72 \%  de réussite chez les filles dans le pays.
				\end{solution}
			
			\part La conclusion du dictateur est-elle exacte ?
				\begin{solution}
					La conclusion du dictateur n'est pas exacte, les filles du pays ont mieux réussi que les garçons (72 \% > 66 \%).
				\end{solution}

		\end{parts}
\end{questions}