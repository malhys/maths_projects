\section{Réussite des filles}

Lors d'un discours au cours duquel il a donné les résultats des examens de fin d'études des deux universités d'un pays, le dictateur dirigeant de ce pays a déclaré :

<< Dans l'Université du Nord, 82 \% des garçons et 80 \% des filles ont réussi. Dans l'université du Sud, 56 \% des garçons et 52 \% des filles ont réussi.

Je ne suis pas sexiste, mais il faut bien reconnaître que dans notre pays, les garçons réussissent mieux que les filles>>.

\begin{questions}
	\question Dans l'Université du Nord, il y avait 500 candidats de sexe masculin et 500 candidats de sexe féminin.
	
	Calculer le nombre de garçons et le nombre de filles qui ont réussi dans cette université.
	
	\question Dans l'Université du Sud, il y avait 800 candidats de sexe masculin et 200 candidats de sexe féminin.
	
	Calculer le nombre de garçons et le nombre de filles qui ont réussi dans cette université.
	
	\question
		\begin{parts}
			\part Combien y avait-t-il de garçons candidats dans le pays ? Calculer le pourcentage de garçons qui ont réussi dans le pays.
			
			\part Calculer le pourcentage de filles qui ont réussi dans le pays.
			
			\part La conclusion du dictateur est-elle exacte ?
		\end{parts}
\end{questions}