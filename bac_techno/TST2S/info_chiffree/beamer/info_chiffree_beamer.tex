\documentclass[xcolor={dvipsnames}]{beamer}
%\usepackage[utf8]{inputenc}
%\usetheme{Madrid}
\usetheme{CambridgeUS}
\usecolortheme{}

\input{../../../../utils_maths_beamer}


%\usepackage{../../../../pas-math}
%\usepackage{../../../moncours_beamer}

\usepackage{amssymb,amsmath}



\graphicspath{{../img/}}

\title{Quel espace un gaz peut-il occuper ?}
\author{O. FINOT}\institute{Collège S$^t$ Bernard}


\AtBeginSection[]
{
	\begin{frame}
		\frametitle{}
		\tableofcontents[currentsection, hideallsubsections]
	\end{frame} 

}


\AtBeginSubsection[]
{
	\begin{frame}
		\frametitle{Sommaire}
		\tableofcontents[currentsection, currentsubsection]
	\end{frame} 
}

\begin{document}

\begin{frame}
  \titlepage 
\end{frame}


	



\section{Effectifs et proportions}

\subsection{Expression d'une proportion à l'aide d'un pourcentage (TP1 p 6)}

\begin{frame}[allowframebreaks]{}

\begin{enumerate}%[label=\arabic*.]
	\item 
	\begin{enumerate} [a]
		\item Proportion de cyclomotoristes de 16 ans parmi les cyclomotoristes âgés de 12 à 18 ans victimes d'accidents de la route :
		\begin{eqnarray*}
			\dfrac{2549}{9493} \approx 0,2685 = \textcolor{red}{\underline{26,85\%}}
		\end{eqnarray*}
		
		\item Pourcentage des utilisateurs de "deux roues" parmi les victimes d'accidents de la route de 12 à 18 ans :
		
		\begin{equation*}
		\dfrac{923 + 9435 + 745}{17914} = \dfrac{11 161}{17914} \approx 0,6230 = \textcolor{red}{\underline{62,30 \%}}
		\end{equation*}
		
		\item Pourcentage de 12-16 ans parmi les victimes de "deux roues" :
		
		\begin{equation*}
		\dfrac{218 + 310 + 1180 + 1897 + 2796}{11161} = \dfrac{6401}{11161} \approx 0,5735 = \textcolor{red}{\underline{57,37 \%}}
		\end{equation*}
	\end{enumerate}
	
	\item Soit $N$ le nombre total de motocyclistes accidentés. On a :
	
	\begin{eqnarray*}
		N \times \dfrac{\num{4.48}}{100} &=& 745 \\
		N &=& \dfrac{745 \times 100}{\num{4.48}} \\
		N &=&  \mykw{\num{16629.46}}
	\end{eqnarray*}
	Soit environ \num{16629} motocyclistes accidentés.
	
	\item Pourcentage de <<porteurs de casque >> parmi les cyclistes de 12 à 18 ans accidentés :
	\begin{equation*}
	\num{8.6} + \num{14.6} = \num{13.2} \qquad soit \; \num{13.2}\%.
	\end{equation*}
\end{enumerate}
\end{frame}

\begin{frame}
%\begin{align}
%			N \times \dfrac{\num{4.48}}{100} &=& 745 \\
%			N &=& \dfrac{745 \times 100}{\num{4.48}} \\
%			N &=&  \myres{\num{16629.46}}
%\end{align}
\begin{eqnarray*}
N \times \dfrac{\num{4.48}}{100} &=& 745 \\
N &=& \dfrac{745 \times 100}{\num{4.48}} \\
N &=&  \mykw{\num{16629.46}}
\end{eqnarray*}
\end{frame}

\begin{frame}[allowframebreaks]{}
A\\ A\\ A\\ A\\ A\\ A\\ A\\ A\\ A\\ A\\ A\\ A\\ A\\
\framebreak
B\\ B\\ B\\ B\\ B\\ B\\ B\\\framebreak B\\ B\\ B\\ B\\ B\\ B\\
\end{frame}


%\begin{frame}
%
%
%%\begin{mybilan2}{Proportion}
%	La \kw{proportion ou fréquence} d'une partie $A$ d'une population $E$, est le rapport $p$ des effectifs de $A$ et de $E$ :
%	
%	\begin{eqnarray*}
%		p = \dfrac{n_A}{n_E} \; \left(\dfrac{Effectif de A}{Effectif de E}\right)
%	\end{eqnarray*}
%%\end{mybilan2}
%\end{frame}

\end{document}