\documentclass[xcolor={dvipsnames}]{beamer}
%\usepackage[utf8]{inputenc}
%\usetheme{Madrid}
\usetheme{Malmoe}
%\usecolortheme{beaver}
\usecolortheme{rose}

%-------------------------------------------------------------------------------
%          -Packages nécessaires pour écrire en Français et en UTF8-
%-------------------------------------------------------------------------------
\usepackage[utf8]{inputenc}
\usepackage[frenchb]{babel}
\usepackage[T1]{fontenc}
\usepackage{lmodern}
\usepackage{textcomp}

%-------------------------------------------------------------------------------

%-------------------------------------------------------------------------------
%                          -Outils de mise en forme-
%-------------------------------------------------------------------------------
\usepackage{hyperref}
\hypersetup{pdfstartview=XYZ}
\usepackage{enumerate}
\usepackage{graphicx}
%\usepackage{multicol}
%\usepackage{tabularx}

%\usepackage{anysize} %%pour pouvoir mettre les marges qu'on veut
%\marginsize{2.5cm}{2.5cm}{2.5cm}{2.5cm}

\usepackage{indentfirst} %%pour que les premier paragraphes soient aussi indentés
\usepackage{verbatim}
%\usepackage[table]{xcolor}  
%\usepackage{multirow}
\usepackage{ulem}
%-------------------------------------------------------------------------------


%-------------------------------------------------------------------------------
%                  -Nécessaires pour écrire des mathématiques-
%-------------------------------------------------------------------------------
\usepackage{amsfonts}
\usepackage{amssymb}
\usepackage{amsmath}
\usepackage{amsthm}
\usepackage{tikz}
\usepackage{xlop}
\usepackage[output-decimal-marker={,}]{siunitx}
%-------------------------------------------------------------------------------


%-------------------------------------------------------------------------------
%                    - Mise en forme 
%-------------------------------------------------------------------------------

\newcommand{\bu}[1]{\underline{\textbf{#1}}}


\usepackage{ifthen}


\newcommand{\ifTrue}[2]{\ifthenelse{\equal{#1}{true}}{#2}{$\qquad \qquad$}}

\newcommand{\kword}[1]{\textcolor{red}{\underline{#1}}}


%-------------------------------------------------------------------------------



%-------------------------------------------------------------------------------
%                    - Racourcis d'écriture -
%-------------------------------------------------------------------------------

% Angles orientés (couples de vecteurs)
\newcommand{\aopp}[2]{(\vec{#1}, \vec{#2})} %Les deuc vecteurs sont positifs
\newcommand{\aopn}[2]{(\vec{#1}, -\vec{#2})} %Le second vecteur est négatif
\newcommand{\aonp}[2]{(-\vec{#1}, \vec{#2})} %Le premier vecteur est négatif
\newcommand{\aonn}[2]{(-\vec{#1}, -\vec{#2})} %Les deux vecteurs sont négatifs

%Ensembles mathématiques
\newcommand{\naturels}{\mathbb{N}} %Nombres naturels
\newcommand{\relatifs}{\mathbb{Z}} %Nombres relatifs
\newcommand{\rationnels}{\mathbb{Q}} %Nombres rationnels
\newcommand{\reels}{\mathbb{R}} %Nombres réels
\newcommand{\complexes}{\mathbb{C}} %Nombres complexes


%Intégration des parenthèses aux cosinus
\newcommand{\cosP}[1]{\cos\left(#1\right)}
\newcommand{\sinP}[1]{\sin\left(#1\right)}

%Fractions
\newcommand{\myfrac}[2]{{\LARGE $\frac{#1}{#2}$}}

%Vocabulaire courrant
\newcommand{\cad}{c'est-à-dire}

%Droites
\newcommand{\dte}[1]{droite $(#1)$}
\newcommand{\fig}[1]{figure $#1$}
\newcommand{\sym}{symétrique}
\newcommand{\syms}{symétriques}
\newcommand{\asym}{axe de symétrie}
\newcommand{\asyms}{axes de symétrie}
\newcommand{\seg}[1]{$[#1]$}
\newcommand{\monAngle}[1]{$\widehat{#1}$}
\newcommand{\bissec}{bissectrice}
\newcommand{\mediat}{médiatrice}
\newcommand{\ddte}[1]{$[#1)$}

%Figures
\newcommand{\para}{parallélogramme}
\newcommand{\paras}{parallélogrammes}
\newcommand{\myquad}{quadrilatère}
\newcommand{\myquads}{quadrilatères}
\newcommand{\co}{côtés opposés}
\newcommand{\diag}{diagonale}
\newcommand{\diags}{diagonales}
\newcommand{\supp}{supplémentaires}
\newcommand{\car}{carré}
\newcommand{\cars}{carrés}
\newcommand{\rect}{rectangle}
\newcommand{\rects}{rectangles}
\newcommand{\los}{losange}
\newcommand{\loss}{losanges}


%----------------------------------------------------


%\usepackage{../../../../pas-math}
%\usepackage{../../../moncours_beamer}

\usepackage{amssymb,amsmath}



\graphicspath{{../img/}}

\title{Information chiffrée (révisions)}
%\author{O. FINOT}\institute{Collège S$^t$ Bernard}

%
%\AtBeginSection[]
%{
%	\begin{frame}
%		\frametitle{}
%		\tableofcontents[currentsection, hideallsubsections]
%	\end{frame} 
%
%}
%
%
%\AtBeginSubsection[]
%{
%	\begin{frame}
%		\frametitle{Sommaire}
%		\tableofcontents[currentsection, currentsubsection]
%	\end{frame} 
%}

\begin{document}



\begin{frame}
  \titlepage 
\end{frame}


	



\section{Effectifs et proportions}

\subsection{Expression d'une proportion à l'aide d'un pourcentage }


\begin{frame}
\

\begin{Large}
	\textcolor{Red}{\underline{I. Effectifs et proportions }}
\end{Large}\pause
\

\vspace*{1cm}

\textcolor{Green}{\underline{1) Expression d'une proportion à l'aide d'un pourcentage (TP1 p 6)}}
\end{frame}

\begin{frame}{}
	
\begin{enumerate}%[label=\arabic*.]
	\item 
	\begin{enumerate} [a)]
		\item Proportion de cyclomotoristes de 16 ans parmi les cyclomotoristes âgés de 12 à 18 ans victimes d'accidents de la route :\pause
		\begin{eqnarray*}
			\dfrac{2549}{9493} \approx 0,2685 = \textcolor{red}{\underline{26,85\%}}
		\end{eqnarray*}\pause
		
		\item Pourcentage des utilisateurs de "deux roues" parmi les victimes d'accidents de la route de 12 à 18 ans :\pause
		
		\begin{equation*}
		\dfrac{923 + 9493 + 745}{17914} = \dfrac{11 161}{17914} \approx 0,6230 = \textcolor{red}{\underline{62,30 \%}}
		\end{equation*}\pause
		
		\item Pourcentage de 12-16 ans parmi les victimes de "deux roues" :\pause
		
		\begin{equation*}
		\dfrac{218 + 310 + 1180 + 1897 + 2796}{11161} = \dfrac{6401}{11161} \approx 0,5735 = \textcolor{red}{\underline{57,35 \%}}
		\end{equation*}\pause
	\end{enumerate}
\end{enumerate}
\end{frame}


\begin{frame}{}
	\begin{enumerate}%[label=\arabic*.]
		\setcounter{enumi}{1}
		\item Soit $N$ le nombre total de motocyclistes accidentés. On a :\pause
		
		\begin{eqnarray*}
			N \times \dfrac{\num{4.48}}{100} &=& 745 \\
			N &=& \dfrac{745 \times 100}{\num{4.48}} \\
			N &=&  \mykw{\num{16629.46}}
		\end{eqnarray*}
		Soit environ \num{16629} motocyclistes accidentés.
			
	\end{enumerate}
\end{frame}

\begin{frame}{}
\begin{enumerate}%[label=\arabic*.]
	\setcounter{enumi}{2}
		\item Pourcentage de <<porteurs de casque >> parmi les cyclistes de 12 à 18 ans accidentés : \pause 
		\begin{equation*}
			\num{697} \times \num{0.086} + \num{226} \times \num{0.146} \approx 93 \; porteurs \: de \: casque.
		\end{equation*}
		
		\begin{equation*}
			\dfrac{93}{923} \approx \num{0.1007} \: soit \; \num{10.07}\%. 
		\end{equation*}\pause
		
	\begin{alertblock}{A retenir : Proportion}
		La \mykw{proportion ou fréquence} d'une partie $A$ d'une population $E$, est le rapport $p$ des effectifs de $A$ et de $E$ :
		
		\begin{eqnarray*}
			p = \dfrac{n_A}{n_E} \; \left(\dfrac{Effectif\;de\; A}{Effectif\; de\; E}\right)
		\end{eqnarray*}
	\end{alertblock}
\end{enumerate}
\end{frame}		


\subsection{Comparaison de pourcentages, pourcentage de pourcentages}


\begin{frame}
\


\textcolor{Green}{\underline{2) Comparaison de pourcentages, pourcentage de pourcentages (TP2 p 6)}}
\end{frame}

\begin{frame}
	\begin{enumerate}
		\item \begin{enumerate}[a)]
			\item Pourcentage d'hommes parmi les personnes décédées d'une tumeur :\pause
			\begin{equation*}
				\dfrac{\num{28259}}{\num{43875}} \approx \num{0.6441} \: = \: \mykw{\num{64.41} \%}.
			\end{equation*} \pause
			
			\item Pourcentage de décès par une tumeur parmi l'ensemble des personnes décédées :\pause
			\begin{equation*}
				\dfrac{\num{43875}}{\num{113537}} \approx \num{0.3864}\: = \: \mykw{\num{38.64} \%}.
			\end{equation*}\pause
			
			\item Proportion de femmes décédées d'une tumeur parmi l'ensemble des femmes décédées :\pause
			\begin{equation*}
				\dfrac{\num{15616}}{\num{35101}} \approx \num{0.4449} \: = \: \mykw{\num{44.49} \%}.
			\end{equation*}\pause
			
			\item Proportion d'hommes parmi les <<décès prématurés>> :\pause
			\begin{equation*}
				\dfrac{\num{78436}}{\num{113537}} \approx \num{0.6908} \: = \: \mykw{\num{69.08} \%}.
			\end{equation*}
		\end{enumerate}
	\end{enumerate}
\end{frame}

\begin{frame}{}
\begin{enumerate}%[label=\arabic*.]
	\setcounter{enumi}{1}
		\item \begin{enumerate}[a)]
			\item Proportion d'hommes décédés  d'une maladie du système nerveux parmi les hommes décédés avant 65 ans :\pause
			
			\begin{equation*}
				\dfrac{\num{2011}}{\num{78436}} \approx \num{0.0256} \: = \: \mykw{\num{2.56} \%}.
			\end{equation*}\pause
			
			\item Proportions de femmes décédées d'une maladie du système nerveux parmi les femmes décédées avant 65 ans :
			\begin{equation*}
				\dfrac{\num{1217}}{\num{35101}} \approx \num{0.0347} 	\: = \:  \mykw{\num{3.47} \%}.
			\end{equation*}\pause
			
			\item En terme d'effectifs il y a plus d'hommes que de femmes qui décèdent d'une maladie du système nerveux, mais en pourcentage il y a plus de femmes. Il y a moins de femmes que d'hommes qui décèdent prématurément mais en proportion elles meurent plus de maladie du système nerveux.
		\end{enumerate}
\end{enumerate}
\end{frame}

	
\begin{frame}{}
\begin{enumerate}%[label=\arabic*.]
	\setcounter{enumi}{2}
	
	\item \begin{enumerate}[a)]
		\item Proportion $p_1$ de femmes décédées d'une maladie infectieuse ou parasitaire parmi l'ensemble des des personnes décédées d'une maladie infectieuse ou parasitaire :\pause
		\begin{equation*}
			p_1 = \dfrac{\num{747}}{\num{2568}} \approx \num{0.2909} = \mykw{\num{29.09} \%}.
		\end{equation*}\pause
		
		\item Proportion $p_2$ de personnes décédées d'une maladie infectieuse ou parasitaire parmi l'ensemble des personnes décédées avant 65 ans :\pause
		\begin{equation*}
			p_2 = \dfrac{\num{2568}}{\num{113537}} \approx \num{0.0226} = \mykw{\num{2.26} \%}.
		\end{equation*}\pause
		
		\item Proportion $p_2$ de femmes décédées d'une maladie infectieuse ou parasitaire parmi l'ensemble des personnes décédées avant 65 ans :\pause
		\begin{equation*}
			p_3 = \dfrac{\num{747}}{\num{113537}} \approx \num{0.0066} = \mykw{\num{0.66} \%}.
		\end{equation*}\pause
		
		\item On a $p_1 \times p_2 = p_3$.
		
	\end{enumerate}
\end{enumerate}
\end{frame}	
	
	
\begin{frame}
	\begin{block}{Remarque}
		\begin{equation*}
			\dfrac{\num{747}}{\num{2568}} \times \dfrac{\num{2568}}{\num{113537}} = \dfrac{\num{747}}{\num{113537}}, \: donc \: p_1\times p_2 = p_3.
		\end{equation*}
		
		On peut aussi calculer $\num{2.26}$ \% de $\num{29.09}$ \% :
		\begin{equation*}
			\dfrac{\num{2.26}}{\num{100}} \times \dfrac{\num{29.09}}{\num{100}} = \num{0.00657434} \qquad soit \: environ \: \num{0.66} \%.
		\end{equation*}
	\end{block}
\end{frame}


\section{Pourcentage d'évolution, coefficient multiplicateur}

\subsection{Variation relative, taux d'évolution}




\begin{frame}
\

\begin{Large}
	\textcolor{Red}{\underline{II. Pourcentage d'évolution, coefficient multiplicateur }}
\end{Large}\pause
\

\vspace*{1cm}

\textcolor{Green}{\underline{1) Variation relative, taux d'évolution}}
\end{frame}



\begin{frame}{}
\begin{enumerate}%[label=\arabic*.]
	%\setcounter{enumi}{}
	\item Entre 1990 et 2005, le nombre de médecins généralistes en France à \mykw{augmenté} de \mykw{\num{8.45}} \%. ($\frac{\num{1012067} - \num{93380}}{\num{93380}}$)\pause
	
	\item Entre 2005 et 2015, le nombre de médecins généralistes en France devrait \mykw{diminuer} de \mykw{\num{1.58}} \%.($\frac{\num{99670} - \num{1012067}}{\num{1012067}}$)\pause
	
	\item Nombre de médecins des spécialités chirurgicales en 2005 : 
	
	\begin{table}[h!]
		\centering
		\begin{tabular}{|ccc|}
			\hline
			en \num{1990} & + \num{14.21} \%  & en \num{2005} \\
			& {\LARGE $\rightarrow$} &			\\
			\num{21390} médecins&   $ \times \qquad \quad $  & ? médecins \\
			\hline
		\end{tabular}
	\end{table}\pause
	
	D'où : $\num{21390} \times \num{1.1421} = \num{24429.519}$, soit environ \num{24430} médecins
	
\end{enumerate}
\end{frame}

\begin{frame}{}
\begin{enumerate}%[label=\arabic*.]
	\setcounter{enumi}{3}
	
	\item Nombre de médecins des spécialités médicales en 2015 \pause
	
	\begin{table}[h!]
		\centering
		\begin{tabular}{|ccc|}
			\hline
			en \num{2005} & - \num{6.90} \%  & en \num{2015} \\
			& {\LARGE $\rightarrow$} &			\\
			\num{58489} médecins& $\times \qquad \quad $ & ? médecins \\
			\hline
		\end{tabular}
	\end{table}\pause
	
	D'où : $\num{58489} \times \num{0.931} = \num{54453.259}$, soit environ \num{54453} médecins.\pause
	
	
	\item Nombre de médecins des spécialités médicales en 1990 \pause
	
	\begin{table}[h!]
		\centering
		\begin{tabular}{|ccc|}
			\hline
			en \num{1990} & + \num{21.77} \%  & en \num{2005} \\
			& {\LARGE $\rightarrow$} &			\\
			? médecins& $\times \qquad \quad $ & \num{58489} médecins \\
			\hline
			& {\LARGE $\leftarrow$} & \\
			& $\div \qquad  \quad $ & \\
			\hline
		\end{tabular}
	\end{table}\pause
	
	D'où : $\num{58489} \div \num{1.2177} = \num{48032.35}...$, soit environ \num{48032} médecins.
\end{enumerate}
\end{frame}

\begin{frame}

\begin{alertblock}{\`A retenir : Taux d'évolution et coefficient multiplicateur}
	Le taux d'évolution $t$ (ou variation relative) d'une quantité passant de la valeur $y_1$ à une valeur $y_2$ est égal à :
	\begin{equation*}
	t = \dfrac{y_2 - y_1}{y_1} \left(\dfrac{V_{arrivée} - V_{départ}}{V_{départ}}\right)
	\end{equation*}
	
	\underline{Remarque} : Un taux d'évolution positif traduit une hausse, un taux d'évolution négatif traduit une baisse.\\
	
	
	
	
\end{alertblock}

\end{frame}

\begin{frame}

\begin{alertblock}{\`A retenir : Taux d'évolution et coefficient multiplicateur (suite)}
\underline{Coefficients multiplicateurs :} 
\begin{itemize}
	
	\item \mykw{Augmenter} une grandeur de $t \%$ revient à multiplier cette grandeur par $\left(1 + \dfrac{t}{100}\right)$.
	
	\item \underline{Exemple :} $+ 5 \% = \times \num{1.05}$ ; $+ 20 \% = \times \num{1.20}$ \\
	
	\item \mykw{Diminuer} une grandeur de $t \%$ revient à multiplier cette grandeur par $\qquad \left(1 - \dfrac{t}{100}\right)$.
	\item \underline{Exemple :} $- 12 \% = \times \num{0.88}$ ; $- 3 \% = \times \num{0.97}$ \\
	
	\item Dans le cas d'une \mykw{hausse}, le coefficient multiplicateur est \mykw{supérieur à 1}.
	
	\item Dans le cas d'une \mykw{baisse}, le coefficient multiplicateur est \mykw{inférieur à 1}.
\end{itemize}



\end{alertblock}

\end{frame}

\subsection{\'Evolutions successives, évolution réciproque}


\begin{frame}
\

\textcolor{Green}{\underline{2) \'Evolutions successives, évolution réciproque)}}\\

\vspace*{1cm}

A. \'Evolutions successives
\end{frame}

\begin{frame}{}

\begin{enumerate}[1)]
\item \

\setlength{\tabcolsep}{4pt}
\begin{table}[h!]
	\centering
	\begin{tabular}{|ccc|c|}
		\hline
		$P_1$ & + \num{25} \%  & $P_2$ &\\
		& {\LARGE $\rightarrow$} &	&	$P_2 = \num{16} \times \num{1.25} = 20$, soit 20 \$ .	\\
		\num{16} \$ & $\times \num{1.25}$ & ? \$ & \\
		\hline
	\end{tabular}
	
\end{table}

\pause

\item \ 

\begin{table}[h!]
	\centering
	\begin{tabular}{|ccc|c|}
		\hline
		$P_2$ & + \num{30} \%  & $P_3$ &\\
		& {\LARGE$\rightarrow$} &	&	$P_2 = \num{20} \times \num{1.30} = 26$, soit 26 \$ .	\\
		\num{20} \$ & $\times \num{1.30}$ & ? \$ & \\
		\hline
	\end{tabular}
	
\end{table} \pause


\item \ 
\begin{table}[h!]
	\centering
	\begin{tabular}{|ccc|c|}
		\hline
		$P_1$ & + ... \%  & $P_3$ & \\
		& {\LARGE$\rightarrow$} &	&	 $\dfrac{\num{26} - \num{16}}{\num{16}} = \num{0.625}$	\\
		\num{16} \$ & $\times ...$ & 26 \$ & Soit une hausse globale de \num{62.5} \% \\
		\hline
	\end{tabular}
	
\end{table}


\end{enumerate}

\end{frame}


\begin{frame}{}

\underline{Calcul du coefficient multiplicateur :}
\begin{equation*}
k = \dfrac{26}{16} = \num{1.625}
\end{equation*}\pause

On peut aussi calculer indépendamment des prix : $\num{1.25} \times \num{1.30} = \num{1.625}$, soit une hausse globale de \num{62.5} \%. \pause

\begin{block}{Remarque}
Le pourcentage de hausse globale \num{62.5} \% n'est pas égal à la somme des deux pourcentages de hausse successives \num{25} \% et \num{30} \%, car ces deux pourcentages ne s'appliquent pas sur le même prix, donc ne s'additionnent pas.
\end{block}\pause

\begin{alertblock}{\`A retenir : \'Evolutions successives}
Deux évolutions (hausse ou baisse) successives de coefficients multiplicateurs $c$ et $c'$  correspondent  une évolution globale (hausse ou baisse) de $c \times c'$ (on multiplie).
\end{alertblock}



\end{frame}


\begin{frame}
\

B. \'Evolution réciproque \pause
%\end{frame}


%\begin{frame}
\begin{enumerate}[1.]
\item \begin{eqnarray*}
	P_2 &=& P_1 \times \num{1.25} \\
	P_2 &=& 16 \times  \num{1.25} \\
	P_2 &=& 20\\
\end{eqnarray*}\pause

\vspace*{-0.5cm}

\item 
\begin{enumerate}[a.]
\item 

\begin{eqnarray*}
P_3 &=& P_2 \times \num{0.75} \\
P_3 &=& 20 \times  \num{0.75} \\
P_3 &=& 15\\
\end{eqnarray*}\pause

\vspace*{-0.5cm} 
%\begin{table}[h!]
%\centering{\ }
%\begin{tabular}{|@{\ \ }c@{\ \ }c@{\ \ }c@{\ \ }c@{\ \ }c@{\ \ }|}
%	\hline
%	$P_1$ & +\num{25} \%  & $P_2$ & -\num{25} \%  & $P'_3$ \\
%	& {\LARGE$\rightarrow$} &	&	 {\LARGE$\rightarrow$} &	\\
%	\num{16} \$ & $\times \num{1.25} $ & 20 \$ &  $\times \num{0.75}$ & \num{15} \$ \\
%	\hline
%\end{tabular}

%\end{table}\pause

\item On constate que la baisse de \num{25} \% n'annule pas la hausse de \num{25} \%.\pause
\end{enumerate}
\end{enumerate}

\end{frame}

\begin{frame}
	\begin{block}{Remarque}
		\begin{eqnarray*}
			P'_3 &=& \num{16} \times \num{1.25} \times \num{0.75} \\
			P'_3 &=& \num{16} \times \num{0.9375} \\
			On\;a\; & &\num{0.9375} \neq 1
		\end{eqnarray*}
	\end{block}
	
\end{frame}



%\end{frame}


\begin{frame}{}

\begin{enumerate}[1.]
\setcounter{enumi}{2}
\item 

On recherche le coefficient multiplicateur $c$ qui annule l'augmentation de 25 \% :

\begin{eqnarray*}
	\num{1.25} \times c &=& 1 \\
	c &=& \frac{1}{\num{1.25}} \\
	c &=& \num{0.8}
\end{eqnarray*}  
	
\begin{equation*}
	20 \times \num{0.8} = 16
\end{equation*}

\end{enumerate}

\begin{alertblock}{\`A retenir : \'Evolution réciproque}
Deux évolutions (hausse et baisse) successives sont réciproques si et seulement si leur \mykw{coefficients multiplicateurs $c$ et $c'$ sont inverses} : $c \times c' = 1$
\end{alertblock}

\end{frame}


\begin{frame}
	\begin{enumerate}[1.]
		\setcounter{enumi}{3}
		\item 
		
		On recherche le coefficient multiplicateur $c$ qui annule l'augmentation de 50 \% :
		
		\begin{eqnarray*}
			\num{1.5} \times c &=& 1 \\
			c &=& \frac{1}{\num{1.5}} \\
			c &\approx& \num{0.6667}
		\end{eqnarray*}  
		
		Une baisse de \num{66.67} \% compense une hausse de 50 \%. 
		
	\end{enumerate}
\end{frame}
\end{document}