\documentclass[xcolor={dvipsnames}]{beamer}
%\usepackage[utf8]{inputenc}
%\usetheme{Madrid}
\usetheme{Malmoe}
%\usecolortheme{beaver}
\usecolortheme{rose}

\input{../../../../utils_maths_beamer}


%\usepackage{../../../../pas-math}
%\usepackage{../../../moncours_beamer}

\usepackage{amssymb,amsmath}



\graphicspath{{../img/}}

\title{Information chiffrée (révisions)}
%\author{O. FINOT}\institute{Collège S$^t$ Bernard}

%
%\AtBeginSection[]
%{
%	\begin{frame}
%		\frametitle{}
%		\tableofcontents[currentsection, hideallsubsections]
%	\end{frame} 
%
%}
%
%
%\AtBeginSubsection[]
%{
%	\begin{frame}
%		\frametitle{Sommaire}
%		\tableofcontents[currentsection, currentsubsection]
%	\end{frame} 
%}

\begin{document}



\begin{frame}
  \titlepage 
\end{frame}


	



\section{Effectifs et proportions}

\subsection{Expression d'une proportion à l'aide d'un pourcentage }


\begin{frame}
\

\begin{Large}
	\textcolor{Red}{\underline{I. Effectifs et proportions }}
\end{Large}\pause
\

\vspace*{1cm}

\textcolor{Green}{\underline{1) Expression d'une proportion à l'aide d'un pourcentage (TP1 p 6)}}
\end{frame}

\begin{frame}{}
	
\begin{enumerate}%[label=\arabic*.]
	\item 
	\begin{enumerate} [a)]
		\item Proportion de cyclomotoristes de 16 ans parmi les cyclomotoristes âgés de 12 à 18 ans victimes d'accidents de la route :\pause
		\begin{eqnarray*}
			\dfrac{2549}{9493} \approx 0,2685 = \textcolor{red}{\underline{26,85\%}}
		\end{eqnarray*}\pause
		
		\item Pourcentage des utilisateurs de "deux roues" parmi les victimes d'accidents de la route de 12 à 18 ans :\pause
		
		\begin{equation*}
		\dfrac{923 + 9435 + 745}{17914} = \dfrac{11 161}{17914} \approx 0,6230 = \textcolor{red}{\underline{62,30 \%}}
		\end{equation*}\pause
		
		\item Pourcentage de 12-16 ans parmi les victimes de "deux roues" :\pause
		
		\begin{equation*}
		\dfrac{218 + 310 + 1180 + 1897 + 2796}{11161} = \dfrac{6401}{11161} \approx 0,5735 = \textcolor{red}{\underline{57,37 \%}}
		\end{equation*}\pause
	\end{enumerate}
\end{enumerate}
\end{frame}


\begin{frame}{}
	\begin{enumerate}%[label=\arabic*.]
		\setcounter{enumi}{1}
		\item Soit $N$ le nombre total de motocyclistes accidentés. On a :\pause
		
		\begin{eqnarray*}
			N \times \dfrac{\num{4.48}}{100} &=& 745 \\
			N &=& \dfrac{745 \times 100}{\num{4.48}} \\
			N &=&  \mykw{\num{16629.46}}
		\end{eqnarray*}
		Soit environ \num{16629} motocyclistes accidentés.
			
	\end{enumerate}
\end{frame}

\begin{frame}{}
\begin{enumerate}%[label=\arabic*.]
	\setcounter{enumi}{2}
		\item Pourcentage de <<porteurs de casque >> parmi les cyclistes de 12 à 18 ans accidentés :
		\begin{equation*}
		\num{8.6} + \num{14.6} = \num{13.2} \qquad soit \; \num{13.2}\%.
		\end{equation*}\pause
		
	\begin{alertblock}{A retenir : Proportion}
		La \mykw{proportion ou fréquence} d'une partie $A$ d'une population $E$, est le rapport $p$ des effectifs de $A$ et de $E$ :
		
		\begin{eqnarray*}
			p = \dfrac{n_A}{n_E} \; \left(\dfrac{Effectif\;de\; A}{Effectif\; de\; E}\right)
		\end{eqnarray*}
	\end{alertblock}
\end{enumerate}
\end{frame}		


\subsection{Comparaison de pourcentages, pourcentage de pourcentages}


\begin{frame}
\


\textcolor{Green}{\underline{2) Comparaison de pourcentages, pourcentage de pourcentages (TP2 p 6)}}
\end{frame}

\begin{frame}
	\begin{enumerate}
		\item \begin{enumerate}[a)]
			\item Pourcentage d'hommes parmi les personnes décédées d'une tumeur :\pause
			\begin{equation*}
				\dfrac{\num{28259}}{\num{43875}} \approx \num{0.6441} \: = \: \mykw{\num{64.41} \%}.
			\end{equation*} \pause
			
			\item Pourcentage de décès par une tumeur parmi l'ensemble des personnes décédées :\pause
			\begin{equation*}
				\dfrac{\num{43875}}{\num{113537}} \approx \num{0.3864}\: = \: \mykw{\num{38.64} \%}.
			\end{equation*}\pause
			
			\item Proportion de femmes décédées d'une tumeur parmi l'ensemble des femmes décédées :\pause
			\begin{equation*}
				\dfrac{\num{15616}}{\num{35101}} \approx \num{0.4449} \: = \: \mykw{\num{44.49} \%}.
			\end{equation*}\pause
			
			\item Proportion d'hommes parmi les <<décès prématurés>> :\pause
			\begin{equation*}
				\dfrac{\num{78436}}{\num{113537}} \approx \num{0.6908} \: = \: \mykw{\num{69.08} \%}.
			\end{equation*}
		\end{enumerate}
	\end{enumerate}
\end{frame}

\begin{frame}{}
\begin{enumerate}%[label=\arabic*.]
	\setcounter{enumi}{1}
		\item \begin{enumerate}[a)]
			\item Proportion d'hommes décédés  d'une maladie du système nerveux parmi les hommes décédés avant 65 ans :\pause
			
			\begin{equation*}
				\dfrac{\num{2011}}{\num{78436}} \approx \num{0.0256} \: = \: \mykw{\num{2.56} \%}.
			\end{equation*}\pause
			
			\item Proportions de femmes décédées d'une maladie du système nerveux parmi les femmes décédées avant 65 ans :
			\begin{equation*}
				\dfrac{\num{1217}}{\num{35101}} \approx \num{0.0347} 	\: = \:  \mykw{\num{3.47} \%}.
			\end{equation*}\pause
			
			\item En terme d'effectifs il y a plus d'hommes que de femmes qui décèdent d'une maladie du système nerveux, mais en pourcentage il y a plus de femmes. Il y a moins de femmes que d'hommes qui décèdent prématurément mais en proportion elles meurent plus de maladie du système nerveux.
		\end{enumerate}
\end{enumerate}
\end{frame}

	
\begin{frame}{}
\begin{enumerate}%[label=\arabic*.]
	\setcounter{enumi}{2}
	
	\item \begin{enumerate}[a)]
		\item Proportion $p_1$ de femmes décédées d'une maladie infectieuse ou parasitaire parmi l'ensemble des des personnes décédées d'une maladie infectieuse ou parasitaire :\pause
		\begin{equation*}
			p_1 = \dfrac{\num{747}}{\num{2568}} \approx \num{0.2909} = \mykw{\num{29.09} \%}.
		\end{equation*}\pause
		
		\item Proportion $p_2$ de personnes décédées d'une maladie infectieuse ou parasitaire parmi l'ensemble des personnes décédées avant 65 ans :\pause
		\begin{equation*}
			p_2 = \dfrac{\num{2568}}{\num{113537}} \approx \num{0.0226} = \mykw{\num{2.26} \%}.
		\end{equation*}\pause
		
		\item Proportion $p_2$ de femmes décédées d'une maladie infectieuse ou parasitaire parmi l'ensemble des personnes décédées avant 65 ans :\pause
		\begin{equation*}
			p_3 = \dfrac{\num{747}}{\num{113537}} \approx \num{0.0066} = \mykw{\num{0.66} \%}.
		\end{equation*}\pause
		
		\item On a $p_1 \times p_2 = p_3$.
		
	\end{enumerate}
\end{enumerate}
\end{frame}	
	
	
\begin{frame}
	\begin{block}{Remarque}
		\begin{equation*}
			\dfrac{\num{747}}{\num{2568}} \times \dfrac{\num{2568}}{\num{113537}} = \dfrac{\num{747}}{\num{113537}}, \: donc \: p_1\times p_2 = p_3.
		\end{equation*}
		
		On peut aussi calculer $\num{2.26}$ \% de $\num{29.09}$ \% :
		\begin{equation*}
			\dfrac{\num{2.26}}{\num{100}} \times \dfrac{\num{29.09}}{\num{100}} = \num{0.00657434} \qquad soit \: environ \: \num{0.66} \%.
		\end{equation*}
	\end{block}
\end{frame}







%\begin{frame}{}
%\begin{enumerate}%[label=\arabic*.]
%	\setcounter{enumi}{1}
%	
%\end{enumerate}
%\end{frame}


%\begin{frame}
%
%
%%\begin{mybilan2}{Proportion}
%	La \kw{proportion ou fréquence} d'une partie $A$ d'une population $E$, est le rapport $p$ des effectifs de $A$ et de $E$ :
%	
%	\begin{eqnarray*}
%		p = \dfrac{n_A}{n_E} \; \left(\dfrac{Effectif de A}{Effectif de E}\right)
%	\end{eqnarray*}
%%\end{mybilan2}
%\end{frame}

\end{document}