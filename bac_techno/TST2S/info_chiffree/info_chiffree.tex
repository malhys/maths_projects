\documentclass[12pt,a4paper]{article}

%\usepackage[left=1.5cm,right=1.5cm,top=1cm,bottom=2cm]{geometry}
\usepackage[in, plain]{fullpage}
\usepackage{array}
\usepackage{../../../pas-math}
\usepackage{../../../moncours}


%\usepackage{pas-cours}
%-------------------------------------------------------------------------------
%          -Packages nécessaires pour écrire en Français et en UTF8-
%-------------------------------------------------------------------------------
\usepackage[utf8]{inputenc}
\usepackage[frenchb]{babel}
\usepackage[T1]{fontenc}
\usepackage{lmodern}
\usepackage{textcomp}



%-------------------------------------------------------------------------------

%-------------------------------------------------------------------------------
%                          -Outils de mise en forme-
%-------------------------------------------------------------------------------
\usepackage{hyperref}
\hypersetup{pdfstartview=XYZ}
%\usepackage{enumerate}
\usepackage{graphicx}
\usepackage{multicol}
\usepackage{tabularx}
\usepackage{multirow}


\usepackage{anysize} %%pour pouvoir mettre les marges qu'on veut
%\marginsize{2.5cm}{2.5cm}{2.5cm}{2.5cm}

\usepackage{indentfirst} %%pour que les premier paragraphes soient aussi indentés
\usepackage{verbatim}
\usepackage{enumitem}
\usepackage[usenames,dvipsnames,svgnames,table]{xcolor}

\usepackage{variations}

%-------------------------------------------------------------------------------


%-------------------------------------------------------------------------------
%                  -Nécessaires pour écrire des mathématiques-
%-------------------------------------------------------------------------------
\usepackage{amsfonts}
\usepackage{amssymb}
\usepackage{amsmath}
\usepackage{amsthm}
\usepackage{tikz}
\usepackage{xlop}
%-------------------------------------------------------------------------------



%-------------------------------------------------------------------------------


%-------------------------------------------------------------------------------
%                    - Mise en forme avancée
%-------------------------------------------------------------------------------

\usepackage{ifthen}
\usepackage{ifmtarg}


\newcommand{\ifTrue}[2]{\ifthenelse{\equal{#1}{true}}{#2}{$\qquad \qquad$}}

%-------------------------------------------------------------------------------

%-------------------------------------------------------------------------------
%                     -Mise en forme d'exercices-
%-------------------------------------------------------------------------------
%\newtheoremstyle{exostyle}
%{\topsep}% espace avant
%{\topsep}% espace apres
%{}% Police utilisee par le style de thm
%{}% Indentation (vide = aucune, \parindent = indentation paragraphe)
%{\bfseries}% Police du titre de thm
%{.}% Signe de ponctuation apres le titre du thm
%{ }% Espace apres le titre du thm (\newline = linebreak)
%{\thmname{#1}\thmnumber{ #2}\thmnote{. \normalfont{\textit{#3}}}}% composants du titre du thm : \thmname = nom du thm, \thmnumber = numéro du thm, \thmnote = sous-titre du thm

%\theoremstyle{exostyle}
%\newtheorem{exercice}{Exercice}
%
%\newenvironment{questions}{
%\begin{enumerate}[\hspace{12pt}\bfseries\itshape a.]}{\end{enumerate}
%} %mettre un 1 à la place du a si on veut des numéros au lieu de lettres pour les questions 
%-------------------------------------------------------------------------------

%-------------------------------------------------------------------------------
%                    - Mise en forme de tableaux -
%-------------------------------------------------------------------------------

\renewcommand{\arraystretch}{1.7}

\setlength{\tabcolsep}{1.2cm}

%-------------------------------------------------------------------------------



%-------------------------------------------------------------------------------
%                    - Racourcis d'écriture -
%-------------------------------------------------------------------------------

% Angles orientés (couples de vecteurs)
\newcommand{\aopp}[2]{(\vec{#1}, \vec{#2})} %Les deuc vecteurs sont positifs
\newcommand{\aopn}[2]{(\vec{#1}, -\vec{#2})} %Le second vecteur est négatif
\newcommand{\aonp}[2]{(-\vec{#1}, \vec{#2})} %Le premier vecteur est négatif
\newcommand{\aonn}[2]{(-\vec{#1}, -\vec{#2})} %Les deux vecteurs sont négatifs

%Ensembles mathématiques
\newcommand{\naturels}{\mathbb{N}} %Nombres naturels
\newcommand{\relatifs}{\mathbb{Z}} %Nombres relatifs
\newcommand{\rationnels}{\mathbb{Q}} %Nombres rationnels
\newcommand{\reels}{\mathbb{R}} %Nombres réels
\newcommand{\complexes}{\mathbb{C}} %Nombres complexes


%Intégration des parenthèses aux cosinus
\newcommand{\cosP}[1]{\cos\left(#1\right)}
\newcommand{\sinP}[1]{\sin\left(#1\right)}


%Probas stats
\newcommand{\stat}{statistique}
\newcommand{\stats}{statistiques}
%-------------------------------------------------------------------------------

%-------------------------------------------------------------------------------
%                    - Mise en page -
%-------------------------------------------------------------------------------

\newcommand{\twoCol}[1]{\begin{multicols}{2}#1\end{multicols}}


\setenumerate[1]{font=\bfseries,label=\textit{\alph*})}
\setenumerate[2]{font=\bfseries,label=\arabic*)}


%-------------------------------------------------------------------------------
%                    - Elements cours -
%-------------------------------------------------------------------------------





%\makeatletter
%\renewcommand*{\@seccntformat}[1]{\csname the#1\endcsname\hspace{0.1cm}}
%\makeatother


%\author{Olivier FINOT}
\date{}
\title{Information chiffrée }

%\newcommand{\disp}{false}

\lhead{CH1 : Info chiffrée}
\rhead{O. FINOT}
%
%\rfoot{Page \thepage}
\begin{document}
%\maketitle

\setlength{\tabcolsep}{5pt}

\chap[num=1, color=red]{Information chiffrée (révisions)}{Olivier FINOT, \today }

\begin{myobj}
	\begin{itemize}
		
		\item Construire le symétrique d’un point ou d'une figure par rapport à une droite à la main où à l’aide d’un logiciel;
		\item Construire le symétrique d’un point ou d'une figure par rapport à un point, à la main où à l’aide d’un logiciel;
		\item Utiliser les propriétés de la symétrie axiale ou centrale;
		\item Identifier des symétries dans des figures.		
	\end{itemize}
\end{myobj}

\begin{mycomp}
	\begin{itemize}
		\item \kw{Chercher (Ch2)} :  s’engager    dans    une    démarche    scientifique, observer, questionner, manipuler, expérimenter (sur une feuille de papier, avec des objets, à l’aide de logiciels), émettre des hypothèses, chercher des exemples ou des contre-exemples, simplifier ou particulariser une situation, émettre une conjecture ;
		\item \kw{Raisonner (Ra3)} :  démontrer : utiliser un raisonnement logique et des règles établies (propriétés, théorèmes, formules) pour parvenir à une conclusion ;
		\item \kw{Communiquer (Co2)} :  expliquer à l’oral ou à l’écrit (sa démarche, son raisonnement, un calcul, un protocole   de   construction   géométrique, un algorithme), comprendre les explications d’un autre et argumenter dans l’échange ; 
		
	\end{itemize}
\end{mycomp}





\section{Effectifs et proportions}

\subsection{Expression d'une proportion à l'aide d'un pourcentage}

TP1 p 6

\begin{enumerate}[label=\arabic*.]
	\item 
	\begin{enumerate} [label=\alph*) ]
		\item Proportion de cyclomotoristes de 16 ans parmi les cyclomotoristes âgés de 12 à 18 ans victimes d'accidents de la route :
			\begin{eqnarray*}
				\dfrac{2549}{9493} \approx 0,2685 = \textcolor{red}{\underline{26,85\%}}
			\end{eqnarray*}
			
		\item Pourcentage des utilisateurs de "deux roues" parmi les victimes d'accidents de la route de 12 à 18 ans :
		
			\begin{equation*}
				\dfrac{923 + 9435 + 745}{17914} = \dfrac{11 161}{17914} \approx 0,6230 = \textcolor{red}{\underline{62,30 \%}}
			\end{equation*}
			
		\item Pourcentage de 12-16 ans parmi les victimes de "deux roues" :
		
			\begin{equation*}
				\dfrac{218 + 310 + 1180 + 1897 + 2796}{11161} = \dfrac{6401}{11161} \approx 0,5735 = \textcolor{red}{\underline{57,37 \%}}
			\end{equation*}
	\end{enumerate}

	\item Soit $N$ le nombre total de motocyclistes accidentés. On a :
	
		\begin{eqnarray*}
			N \times \dfrac{\num{4.48}}{100} &=& 745 \\
			N &=& \dfrac{745 \times 100}{\num{4.48}} \\
			N &=&  \myres{\num{16629.46}}
		\end{eqnarray*}
	Soit environ \num{16629} motocyclistes accidentés.
	
	\item Pourcentage de <<porteurs de casque >> parmi les cyclistes de 12 à 18 ans accidentés :
	\begin{equation*}
		\num{8.6} + \num{14.6} = \num{13.2} \qquad soit \; \num{13.2}\%.
	\end{equation*}
\end{enumerate}

 \begin{mybilan2}{Proportion}
		La \kw{proportion ou fréquence} d'une partie $A$ d'une population $E$, est le rapport $p$ des effectifs de $A$ et de $E$ :
		
		\begin{eqnarray*}
			p = \dfrac{n_A}{n_E} \; \left(\dfrac{Effectif de A}{Effectif de E}\right)
		\end{eqnarray*}
	\end{mybilan2}

\begin{myexs}
	\twoCol{
	\begin{itemize}
		\item \exo{1}{14}
		\item \exo{2}{14}
		\item \exo{3}{14}
		\item \exo{29}{22}
		\item \exo{30}{23}
	\end{itemize}}
\end{myexs}

\subsection{Comparaison de pourcentages, pourcentage de pourcentages}

TP2 Page 6.

\begin{enumerate}[label=\arabic*.]
	\item \begin{enumerate}[label=\alph*)]
		\item Pourcentage d'hommes parmi les personnes décédées d'une tumeur :
		\begin{equation*}
			\dfrac{\num{28259}}{\num{43875}} \approx \num{0.6441} \: = \: \mykw{\num{64.41} \%}.
		\end{equation*} 
		
		\item Pourcentage de décès par une tumeur parmi l'ensemble des personnes décédées :
		\begin{equation*}
			\dfrac{\num{43875}}{\num{113537}} \approx \num{0.3864}\: = \: \mykw{\num{38.64} \%}.
		\end{equation*}
		
		\item Proportion de femmes décédées d'une tumeur parmi l'ensemble des femmes décédées :
		\begin{equation*}
			\dfrac{\num{15616}}{\num{35101}} \approx \num{0.4449} \: = \: \mykw{\num{44.49} \%}.
		\end{equation*}
		
		\item Proportion d'hommes parmi les <<décès prématurés>> :
		\begin{equation*}
			\dfrac{\num{78436}}{\num{113537}} \approx \num{0.6908} \: = \: \mykw{\num{69.08} \%}.
		\end{equation*}
	\end{enumerate}

	\item \begin{enumerate}[label=\alph*)]
		\item Proportion d'hommes décédés  d'une maladie du système nerveux parmi les hommes décédés avant 65 ans :
		
		\begin{equation*}
			\dfrac{\num{2011}}{\num{78436}} \approx \num{0.0256} \: = \: \mykw{\num{2.56} \%}.
		\end{equation*}
		
		\item Proportions de femmes décédées d'une maladie du système nerveux parmi les femmes décédées avant 65 ans :
		\begin{equation*}
			\dfrac{\num{1217}}{\num{35101}} \approx \num{0.0347} 	\: = \:  \mykw{\num{3.47} \%}.
		\end{equation*}
		
		\item En terme d'effectifs il y a plus d'hommes que de femmes qui décèdent d'une maladie du système nerveux, mais en pourcentage il y a plus de femmes. Il y a moins de femmes que d'hommes qui décèdent prématurément mais en proportion elles meurent plus de maladie du système nerveux.
	\end{enumerate}

	\item \begin{enumerate}[label=\alph*)]
		\item Proportion $p_1$ de femmes décédées d'une maladie infectieuse ou parasitaire parmi l'ensemble des des personnes décédées d'une maladie infectieuse ou parasitaire :
		\begin{equation*}
			p_1 = \dfrac{\num{747}}{\num{2568}} \approx \num{0.2909} = \mykw{\num{29.09} \%}.
		\end{equation*}
		
		\item Proportion $p_2$ de personnes décédées d'une maladie infectieuse ou parasitaire parmi l'ensemble des personnes décédées avant 65 ans :
		\begin{equation*}
			p_2 = \dfrac{\num{2568}}{\num{113537}} \approx \num{0.0226} = \mykw{\num{2.26} \%}.
		\end{equation*}
		
		\item Proportion $p_2$ de femmes décédées d'une maladie infectieuse ou parasitaire parmi l'ensemble des personnes décédées avant 65 ans :
		\begin{equation*}
			p_3 = \dfrac{\num{747}}{\num{113537}} \approx \num{0.0066} = \mykw{\num{0.66} \%}.
		\end{equation*}
		
		\item On a $p_1 \times p_2 = p_3$.
		
	\end{enumerate}
\end{enumerate}

\begin{myrem}
	\begin{equation*}
		\dfrac{\num{747}}{\num{2568}} \times \dfrac{\num{2568}}{\num{113537}} = \dfrac{\num{747}}{\num{113537}}, \: donc \: p_1\times p_2 = p_3.
	\end{equation*}
	
	On peut aussi calculer $\num{2.26}$ \% de $\num{29.09}$ \% :
	\begin{equation*}
		\dfrac{\num{2.26}}{\num{100}} \times \dfrac{\num{29.09}}{\num{100}} = \num{0.00657434} \qquad soit \: environ \: \num{0.66} \%.
	\end{equation*}
\end{myrem}

\begin{myexs}
	\twoCol{
	\begin{itemize}
		\item \exo{4}{19}
		\item \exo{17}{19}
		\item \exo{33}{24}
		\item \exo{34}{24}
	\end{itemize}}

\end{myexs}

\subsection{Additionner et comparer des pourcentages}

\begin{enumerate}[label=\arabic*.]
	\item Pourcentage d'enfants en surpoids dans les zones rurales :
	\begin{equation*}
		\num{100} - \num{87.2} = \mykw{\num{12.8}  \%}
	\end{equation*}
	
	\item Pourcentage d'enfants obèses dans les zones rurales :
	\begin{equation*}
		\num{12.8} - \num{9.2} = \mykw{\num{3.6}  \%}
	\end{equation*}
	
	\item \begin{enumerate}[label=\alph*)]
		\item Dans l'agglomération parisienne, il y a \num{5} \% d'enfants obèses et \num{16} \% en surpoids; la proportion d'enfants obèses parmi ceux en surpoids est donc égale à $\dfrac{5}{16}=\num{0.301} \%,$ soit environ un peu plus de 3 enfants souffrant d'obésité pour 10 en surpoids. L'affirmation est donc juste.
		
		\item Les effectifs pour les différents types d'agglomération ne sont pas connus. On ne peut donc rien affirmer concernant le nombre d'enfants en surpoids.
	\end{enumerate}
\end{enumerate}

\begin{myexs}
	\twoCol{
	\begin{itemize}
		\item \exo{4}{15}
		\item \exo{35}{24}
		\item \exo{36}{25}
		\item \exo{37}{25}
	\end{itemize}}
\end{myexs}

\section{Pourcentage d'évolution, coefficient multiplicateur}

\subsection{Variation relative, taux d'évolution}

TP4 page 8

\begin{enumerate}[label=\arabic*)]
	\item Entre 1990 et 2005, le nombre de médecins généralistes en France à \mykw{augmenté} de \mykw{\num{8.45}} \%. ($\frac{\num{1012067} - \num{93380}}{\num{93380}}$)
	\item Entre 2005 et 2015, le nombre de médecins généralistes en France devrait \mykw{diminuer} de \mykw{\num{1.58}} \%.($\frac{\num{99670} - \num{1012067}}{\num{1012067}}$)
	
	\item Nombre de médecins des spécialités chirurgicales en 2005 : 
	
	\begin{table}[h!]
		\centering
		\begin{tabular}{|ccc|}
			\hline
			en \num{1990} & + \num{14.21} \%  & en \num{2005} \\
			& {\LARGE $\rightarrow$} &			\\
			\num{21390} médecins& $\times \num{1.1421}$ & ? médecins \\
			\hline
		\end{tabular}
	\end{table}
	
	D'où : $\num{21390} \times \num{1.1421} = \num{24429.519}$, soit environ \num{24430} médecins.
	
	\item Nombre de médecins des spécialités médicales en 2015 
	
	\begin{table}[h!]
		\centering
		\begin{tabular}{|ccc|}
			\hline
			en \num{2005} & - \num{6.90} \%  & en \num{2015} \\
			& {\LARGE $\rightarrow$} &			\\
			\num{58489} médecins& $\times \num{0.931}$ & ? médecins \\
			\hline
		\end{tabular}
	\end{table}
	
	D'où : $\num{58489} \times \num{0.931} = \num{54453.259}$, soit environ \num{54453} médecins.
	
	\newpage
	
	\item Nombre de médecins des spécialités médicales en 1990 
	
	\begin{table}[h!]
		\centering
		\begin{tabular}{|ccc|}
			\hline
			en \num{1990} & + \num{21.77} \%  & en \num{2005} \\
			& {\LARGE $\rightarrow$} &			\\
			? médecins& $\times \num{1.2177}$ & \num{58489} médecins \\
			\hline
			& {\LARGE $\leftarrow$} & \\
			& $\div \num{1.2177}$ & \\
			\hline
		\end{tabular}
	\end{table}
	
	D'où : $\num{58489} \div \num{1.2177} = \num{48032.35}...$, soit environ \num{48032} médecins.
\end{enumerate}

\begin{mybilan2}{Taux d'évolution et coefficient multiplicateur}
	Le taux d'évolution $t$ (ou variation relative) d'une quantité passant de la valeur $y_1$ à une valeur $y_2$ est égal à :
	\begin{equation*}
	t = \dfrac{y_2 - y_1}{y_1} \left(\dfrac{V_{arrivée} - V_{départ}}{V_{départ}}\right)
	\end{equation*}
	
	\underline{Remarque} : Un taux d'évolution positif traduit une hausse, un taux d'évolution négatif traduit une baisse.\\
	
	\underline{Coefficients multiplicateurs :} 
	\begin{itemize}
		\item \kw{Augmenter} une grandeur de $t \%$ revient à multiplier cette grandeur par $(1 + \dfrac{t}{100})$.
		
		\item \underline{Exemple :} $+ 5 \% = \times \num{1.05}$ ; $+ 20 \% = \times \num{1.20}$ \\
		
		\item \kw{Diminuer} une grandeur de $t \%$ revient à multiplier cette grandeur par $\qquad (1 - \dfrac{t}{100})$.
		\item \underline{Exemple :} $- 12 \% = \times \num{0.88}$ ; $- 3 \% = \times \num{0.97}$ \\
		
		\item Dans le cas d'une \kw{hausse}, le coefficient multiplicateur est \kw{supérieur à 1}.
		
		\item Dans le cas d'une \kw{baisse}, le coefficient multiplicateur est \kw{inférieur à 1}.
	\end{itemize}	
	
\end{mybilan2}

\begin{myexs}
	\twoCol{
	\begin{itemize}
		\item \exo{10}{17}
		\item \exo{11}{17}
		\item \exo{40}{25-26}
		\item \exo{42}{26}
		\item \exo{43}{26}
		\item \exo{45}{26}
	\end{itemize}
	}
\end{myexs}

\newpage

\subsection{\'Evolutions successives, évolution réciproque}

TP5 page 8

\begin{enumerate}[label=\Alph*.]
	\item \'Evolutions successives
	
	\begin{enumerate}[label=\arabic*)]
		\item \ 

		\begin{table}[h!]
			\centering
			\begin{tabular}{|ccc|c|}
				\hline
				$P_1$ & + \num{25} \%  & $P_2$ &\\
				& {\LARGE $\rightarrow$} &	&	$P_2 = \num{16} \times \num{1.25} = 20$, soit 20 \$ .	\\
				\num{16} \$ & $\times \num{1.25}$ & ? \$ \\
				\hline
			\end{tabular}
			
		\end{table}
		
		\item \ 
		
		\begin{table}[h!]
			\centering
			\begin{tabular}{|ccc|c|}
				\hline
				$P_2$ & + \num{30} \%  & $P_3$ &\\
				& {\LARGE$\rightarrow$} &	&	$P_2 = \num{20} \times \num{1.30} = 26$, soit 26 \$ .	\\
				\num{20} \$ & $\times \num{1.30}$ & ? \$ & \\
				\hline
			\end{tabular}
			
		\end{table}
		\item \ 
		
		\begin{table}[h!]
			\centering
			\begin{tabular}{|ccc|c|}
				\hline
				$P_1$ & + ... \%  & $P_3$ & \\
				& {\LARGE$\rightarrow$} &	&	 $\dfrac{\num{26} - \num{16}}{\num{16}} = \num{0.625}$	\\
				\num{16} \$ & $\times ...$ & 26 \$ & Soit une hausse globale de \num{62.5} \% \\
				\hline
			\end{tabular}
			
		\end{table}
		
		\underline{Calcul du coefficient multiplicateur :}
		\begin{equation*}
		k = \dfrac{26}{16} = \num{1.625}
		\end{equation*}
		
		On peut aussi calculer indépendamment des prix : $\num{1.25} \times \num{1.30} = \num{1.625}$, soit une hausse globale de \num{62.5} \%.\\
		
		\begin{myrem}
			Le pourcentage de hausse globale \num{62.5} \% n'est pas égal à la somme des deux pourcentages de hausse successives \num{25} \% et \num{30} \%, car ces deux pourcentages ne s'appliquent pas sur le même prix, donc ne s'additionnent pas.
		\end{myrem}
		
		\begin{mybilan2}{\'Evolutions successives}
			Deux évolutions (hausse ou baisse) successives de coefficients multiplicateurs $c$ et $c'$  correspondent  une évolution globale (hausse ou baisse) de $c \times c'$ (on multiplie).
		\end{mybilan2}
	\end{enumerate}
	
	\item \'Evolution réciproque
	\begin{enumerate}[label=\arabic*)]
		\item 
		\begin{enumerate}[label=\alph*)]
			\item \ 
			\begin{table}[h!]
				\centering{\ }
				\begin{tabular}{|@{\ \ }c@{\ \ }c@{\ \ }c@{\ \ }c@{\ \ }c@{\ \ }|}
					\hline
					$P_1$ & +\num{25} \%  & $P_2$ & -\num{25} \%  & $P'_3$ \\
					& {\LARGE$\rightarrow$} &	&	 {\LARGE$\rightarrow$} &	\\
					\num{16} \$ & $\times \num{1.25} $ & 20 \$ &  $\times \num{0.75}$ & \num{15} \$ \\
					\hline
				\end{tabular}
				
			\end{table}
			
			\item On constate que la baisse de \num{25} \% n'annule pas la hausse de \num{25} \%.
		\end{enumerate}
		
		\begin{myrem}
			\begin{eqnarray*}
				P'_3 &=& \num{16} \times \num{1.25} \times \num{0.75} \\
				P'_3 &=& \num{16} \times \num{0.9375} \\
				On\;a\; & &\num{0.9375} \neq 1
			\end{eqnarray*}
		\end{myrem}
		
		\item \ 
		\begin{table}[h!]
			\centering{\ }
			\begin{tabular}{|@{\ \ }c@{\ \ }c@{\ \ }c@{\ \ }c@{\ \ }c@{\ \ }|}
				\hline
				$P_1$ & +\num{25} \%  & $P_2$ & -\num{25} \%  & $P_1$ \\
				& {\LARGE$\rightarrow$} &	&	 {\LARGE$\rightarrow$} &	\\
				\num{16} \$ & $\times \num{1.25} $ & 20 \$ &  $\times \num{0.75}$ & \num{15} \$ \\
				\hline
			\end{tabular}
			
		\end{table}
	\end{enumerate}
	
\end{enumerate}

\begin{mybilan2}{\'Evolution réciproque}
	Deux évolutions (hausse et baisse) successives sont réciproques si et seulement si leur \kw{coefficients multiplicateurs $c$ et $c'$ sont inverses} : $c \times c' = 1$
\end{mybilan2}

\begin{myexs}
	\twoCol{
	\begin{itemize}
		\item \exo{14}{18}
		\item \exo{19}{19-20}
		\item  \exo{55}{28}
		\item \exo{56}{29}
	\end{itemize}
	}
\end{myexs}
\end{document}