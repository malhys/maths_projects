	\section{Calculer un pourcentage}
	Calculer :
	\begin{questions}
		
	
		\question[2] \ 
		
		\begin{enumerate}[label=\alph*)]
			
			\item $25 \%$ de $\num{300}$ :  ......................................................................................
			\item $33 \%$ de $\num{660}$ : .....................................................................................
			\item $\num{0.5} \%$ de $\num{2496000}$ : ..................................................................................
			\item $300 \%$ de $\num{12}$ : ......................................................................................
		\end{enumerate}
	\end{questions}
	
	
\section{Relation entre effectif et proportion}

On s'intéresse à la proportion $p$ d'une sous-population $A$ (effectif $n_A$) dans une population globale $E$ (effectif $n_E$).

Calculer :
\begin{questions}

	
	\question[6] \
	
	\begin{enumerate}[label=\alph*)]
		
		\item p lorsque $n_A=\num{14.6}$ et $n_E= \num{59.6} $ :  ...............................................................
		\item p lorsque $n_A=\num{18}$ et $n_E= \num{2400}$ : ...............................................................
		\item $n_A$ lorsque $p=\num{0.098}$ $n_E= \num{250000}$ : ...............................................................
		\item $n_E$ lorsque $p=\num{0.315} $ et $n_A=\num{7875}.$ : ...............................................................
		\item Sachant que 30 \% d'une somme $S$ vaut 330 euros, calculer $S$ ::\\ ..............................................................................................................................
		\item Calculer la proportion de garçons dans la classe ::\\ ..............................................................................................................................
	\end{enumerate}
\end{questions}


%\section{Taux d'évolution}
%On s'intéresse à l'évolution d'une grandeur $y_1$ vers une grandeur $y_2$, $t$ est le taux d'évolution.
%
%\`A chaque fois, calculer l'un des ces trois nombres en connaissant les deux autres.
%
%\begin{questions}
%	\question[4] \ 
%	\begin{enumerate}[label=\alph*)]
%		
%		\item $y_1 = \num{2.7} \quad ; \quad y_2=\num{2.9} $:  ..............................................................................................\\
%		...................................................................................................................................
%		\item $y_1 = \num{3.5} \quad ; \quad y_2=\num{3.3} $:  ..............................................................................................\\
%		...................................................................................................................................
%		\item $y_1 = \num{4.5} \quad ; \quad t=\num{-0.20} $:   ..............................................................................................\\
%		...................................................................................................................................
%		\item $y_2 = \num{1.03} \quad ; \quad t=\num{0.1} $:  ..............................................................................................\\
%		...................................................................................................................................
%		\end{enumerate}
%\end{questions}