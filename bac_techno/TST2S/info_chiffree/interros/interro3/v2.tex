
\section{Taux d'évolution et coefficient multiplicateur}

\begin{questions}
	
	\question[3] Donner le pourcentage d'évolution correspondant au coefficient multiplicateur donné en précisant si c'est une hausse ou une baisse :
	
	\begin{parts}
		\part c=\num{1.337} :
		\fillwithdottedlines{1.5cm}
		
		\part c=\num{3.24} :
		\fillwithdottedlines{1.5cm}
		
		\part c=\num{1} :
		\fillwithdottedlines{1.5cm}
	\end{parts}
\end{questions}


\section{\'Evolutions successives}

\begin{questions}
	
	
	\question[3] Donner le pourcentage d'évolution globale correspondant aux évolutions successives suivantes
	
	\begin{parts}
		
		
		%\part Une hausse de $\num{20}$ \% et une autre de $\num{10}$ \%:   
		%\fillwithdottedlines{2cm}		
		
		\part Une baisse de $\num{20}$ \% et une hausse de $\num{15}$ \%:
		\fillwithdottedlines{2cm}
		
		
		\part Une baisse de $\num{25}$ \% et une autre de $\num{35}$ \%:
		\fillwithdottedlines{2cm}
		
		
		\part Une hausse de $\num{20}$ \% et une baisse de $\num{15}$ \%:
		\fillwithdottedlines{2cm}		
		
		
	\end{parts}
\end{questions}


\newpage

\section{Taux d'évolution}
On s'intéresse à l'évolution d'une grandeur $y_1$ vers une grandeur $y_2$, $t$ est le taux d'évolution.

\`A chaque fois, calculer l'un des ces trois nombres en connaissant les deux autres.
Arrondir à $10^{-2}$.

\begin{questions}
	\question[4] 
	\begin{parts}
		
		
		\part $y_2 = \num{2.2} \quad ; \quad t=\num{0.45} $:  
		\fillwithdottedlines{2cm}
		
		\part $y_1 = \num{2.3} \quad ; \quad t=\num{0.52} $:   
		\fillwithdottedlines{2cm}		
		
		
		\part $y_2 = \num{3.3} \quad ; \quad t=\num{-0.3} $:  
		\fillwithdottedlines{2cm}
		
		
		\part $y_1 = \num{2.2} \quad ; \quad t=\num{-0.3} $:   
		\fillwithdottedlines{2cm}		
		
			
	\end{parts}
\end{questions}


