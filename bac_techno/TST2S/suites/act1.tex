\begin{myact}{La suite des nombres impairs}
	On considère la suite des nombres impairs, 1, 3, 5, 7, ..., que l'on note successivement $u_1$, $u_2$, $u_3$, $u_4$...
	Donc $u_1=1$, $u_2=3$, $u_3=5$...\\
	
	\begin{enumerate}[label=\arabic*)]
		\item \begin{enumerate}[label=\alph*.]
			\item Compléter : $u_4=.....$, $u_? =15$, $u_{10}=......$.
			\item Quel est le premier terme de la suite ?
			\item Comment passe-t-on d'un terme au suivant ?
			\item $n$ est est nombre entier positif non nul, on s'intéresse au terme de rang $n$ (donc le $n^{ième}$ nombre impair). Exprimer $u_{n+1}$ en fonction de $u_n$.
			\item Exprimer $u_n$ en fonction de $n$.
			\item Calculer $u_{100}$, $u_{150}$, $u_{1000}$.
		\end{enumerate} 
		
		\item Somme de nombres impairs. 
		
		On note $S_1=u_1=1$; $S_2=u_1+u_2=1+3=4$; puis, plus généralement $S_n=u_1+u_2+u_3+...+u_n$.
		
		\begin{enumerate}
			\item Compléter le tableau suivant :\\
			\begin{tabular}{|@{\ \ }l@{\ \ }|@{\ \ }c@{\ \ }|@{\ \ }c@{\ \ }|@{\ \ }c@{\ \ }|@{\ \ }c@{\ \ }|@{\ \ }c@{\ \ }|@{\ \ }c@{\ \ }|@{\ \ }c@{\ \ }|@{\ \ }c@{\ \ }|}
				\hline
				$n$   & 1 & 2 & 3 & 4 & 5 & 6 & 7 & 8 \\ \hline
				$u_n$ & 1 & 3 & 5 &   &   &   &   &   \\ \hline
				$S_n$ & 1 & 4 &   &   &   &   &   &   \\ \hline
			\end{tabular}
			
			\item En déduire une relation entre $S_{n+1}$, $S_{n}$, et $u_{n+1}$.
			
			\item En observant les résultats du tableau conjecturer une expression de $S_n$ en fonction de $n$.
		\end{enumerate}
	\end{enumerate}
	
	
\end{myact}