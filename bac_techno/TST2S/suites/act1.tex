\begin{myact}{La suite des nombres impairs}
	On considère la suite des nombres impairs, 1, 3, 5, 7, ..., que l'on note successivement $u_1$, $u_2$, $u_3$, $u_4$...
	Donc $u_1=1$, $u_2=3$, $u_3=5$...\\
	
	
		\begin{enumerate}
			\item Compléter : $u_4=.....$, $u_? =15$, $u_{10}=......$.
			\item Quel est le premier terme de la suite ?
			\item Comment passe-t-on d'un terme au suivant ?
			\item $n$ est est nombre entier positif non nul, on s'intéresse au terme de rang $n$ (donc le $n^{ième}$ nombre impair). Exprimer $u_{n+1}$ en fonction de $u_n$.
			\item Exprimer $u_n$ en fonction de $n$.
			\item Calculer $u_{100}$, $u_{150}$, $u_{1000}$.
		\end{enumerate} 
		
			
	
\end{myact}