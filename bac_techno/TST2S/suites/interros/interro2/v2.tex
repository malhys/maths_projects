%	\section{Suite géométrique}
%Les séries de termes suivants forment-ils une suite géométrique ? Donner le premier terme et la raison si c'est le cas.
%\begin{questions}
%	
%	
%	\question[2] $\num{480} \: ; \: \num{120} \: ; \: \num{30}  \: ; \:  \num{-7.5} \: ; \: \num{1.875}$ 
%	\fillwithdottedlines{3cm}
%	
%	\question[2] $\num{2} \: ; \:  \num{6} \: ; \: \num{18} \: ; \: \num{54} \: ; \: \num{152}$ 
%	\fillwithdottedlines{3cm}
%	
%	
%\end{questions}


\section{Terme général, raison et somme}

Pour chacune des suites arithmétiques ou géométriques définies ci-dessous par leur premier terme et leur raison :
\begin{itemize}
	%	\item $u_{n+1}$ en fonction de  $u_n$;
	\item exprimer $u_n$ en fonction de $n$;
	\item calculer la valeur de $u_{15}$ terme;
	\item calculer la somme $S_{15}$ des premiers termes.
	
\end{itemize} 
\begin{questions}
	
	
	\question[2\half] $u_1 = 2$, $r= 3$
	
	\fillwithdottedlines{8cm}
	
	%\vspace*{1.5cm}
	
	\question[2\half] $u_0 = 15$, $r= -5$
	
	\fillwithdottedlines{8cm}
	
	\question[2\half] $u_1 = 4096$, $q=\num{0.5}$
	
	\fillwithdottedlines{9cm}
	
	\question[2\half] $u_0 = 10$, $q=\num{-2}$
	
	\fillwithdottedlines{9cm}
\end{questions}

