\section{Des normes antipollution \textit{(7 points)}}

Un grand groupe industriel fait le bilan de sa quantité de rejets polluants. En 2001, sa quantité de rejets était de \num{49000} tonnes. Elle est passée à \num{68000} tonnes en 2004.


De nouvelles normes antipollution ont été mises en place à partir de 2001. Le groupe, pour être aux normes ne doit pas dépasser \num{42000} tonnes de rejets par an.

\subsection{}
Chaque année, si ses rejets dépassent la quantité autorisée, le groupe doit payer une amende.

Tant que le groupe ne prend pas de mesure pour faire baisser sa quantité de rejets, l'amande augmente de \num{6000} € tous les ans. En 2001, le groupe a payé une amende de \num{83000} €.

\textit{Dans toute cette partie, on fait l'hypothèse que le groupe ne prend aucune mesure pour diminuer sa quantité de rejets.}

On appelle $C_1$ l'amende payée en 2001 et $C_n$ l'amende payée l'année $2000 + n$. On a alors $C_1 = \num{83000}$ €.\\

\begin{questions}
	\question[1] Calculer la valeur de l'amende payée par l'entreprise en 2002 et en 2003.
	\question[1] Quelle est la nature de la suite $(C_n)$ ?
	\question[1] Calculer l'amende que le groupe devra payer en 2015.
\end{questions}

\subsection{}

Au vu des résultats précédents, le groupe a décidé en 2004 de mettre en place un dispositif lui permettant de se mettre aux normes progressivement, l'objectif étant de ramener sa quantité de déchets à une valeur inférieure ou égale à \num{42000} tonnes en 2014.

Le groupe s'est engagé à réduire sa production de déchets de 4 \% à partir de 2004.\\

\begin{questions}
	\question[1] Si le groupe a rejeté \num{66000} tonnes en 2005, a-t-il respecté son engagement ? \\
	
	\question On appelle $Q_n$ la quantité de rejets prévue pour l'année $2004 + n$. Ainsi, $Q_0 = \num{68000}$.
	
	\begin{parts}
		\part[1] Quelle est la nature de la suite $(Q_n)$ ?
		
		\part[1] Exprimer $Q_n$ en fonction de $n$.
		
		\part[1] Calculer à la tonne près, la quantité de rejets prévue pour l'année 2014. L'entreprise aura-t-elle atteint son objectif ?
	\end{parts}
\end{questions}