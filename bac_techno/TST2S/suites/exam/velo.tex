\section{Entrainement à vélo (8 points)}

Aline et Blandine décident de reprendre l'entrainement à vélo, chaque samedi pendant 15 semaines.

Chacune a établi son programme d'entrainement. Elles parcourent 20 km la première semaine et souhaitent effectuer une sortie ensemble la quinzième semaine.


\begin{questions}
	\question Programme d'entrainement d'Aline
	
		Après la première semaine, Aline décide  d'augmenter chaque semaine la distance parcourue de 7 km.
		
		On note $u_n$ la distance parcourue la n-ième semaine. Ainsi $u_1 = 20$ et $u{15}=118$.
		\begin{parts}
			\part[1] Montrer que la suite $(u_n)$ correspondante est une suite arithmétique de terme initial $u_1=10$ et de raison 7.
			
			\part[1] Exprimer $u_n$ en fonction de $n$.
			
			\part[1] Calculer la distance parcourue par Aline le samedi de la dixième semaine.
			
			\part[1] Calculer la distance totale parcourue par Aline au cours de ses entrainements, quinzième semaine inclue.
		\end{parts}
	
		\question Programme d'entrainement de Blandine
		
		Chaque semaine, Blandine augmente de \num{13.5} \% la distance parcourue, de telle sorte que la distance parcourue la quinzième semaine soit aussi de 118 km, à l'unité près.
		
		On note $v_n$ la distance parcourue la n-ième semaine. Ainsi $v_1 = 20$ et $v{15}=118$.
		\begin{parts}
			\part[1] Montrer que la suite $(v_n)$ correspondante est une suite géométrique et déterminer sa raison.
			
			\part[1] Exprimer $v_n$ en fonction de $n$.
			
			\part[1] Calculer la distance parcourue par Blandine le samedi de la dixième semaine.
			
			\part[1] Calculer la distance totale parcourue par Blandine au cours de ses entrainements, quinzième semaine inclue.
		\end{parts}
\end{questions}