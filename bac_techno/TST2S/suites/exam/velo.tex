\section{Entrainement à vélo (8 points)}

Aline et Blandine décident de reprendre l'entrainement à vélo, chaque samedi pendant 15 semaines.

Chacune a établi son programme d'entrainement. Elles parcourent 20 km la première semaine et souhaitent effectuer une sortie ensemble la quinzième semaine.


\begin{questions}
	\question Programme d'entrainement d'Aline
	
		Après la première semaine, Aline décide  d'augmenter chaque semaine la distance parcourue de 7 km.
		
		On note $u_n$ la distance parcourue la n-ième semaine. Ainsi $u_1 = 20$ et $u{15}=118$.
		\begin{parts}
			\part[1] Montrer que la suite $(u_n)$ correspondante est une suite arithmétique de terme initial $u_1=20$ dont on précisera la raison.
			\begin{solution}
				La distance parcourue augmente chaque semaine de 7 km, c'est donc une suite arithmétique de terme initial $u_1=20$ et de raison $r=7$.
			\end{solution}
			
			\part[1] Exprimer $u_n$ en fonction de $n$.
			\begin{solution}
				On a :
				\begin{eqnarray*}
					u_n &=& u_1 + n \times r \\
					u_n &=& 20 + n \times 7 
				\end{eqnarray*}
			\end{solution}
			
			\part[1] Calculer la distance parcourue par Aline le samedi de la dixième semaine.
			\begin{solution}
				La distance parcourue la dixième semaine correspond à $u_{10}.$ 
				
				\begin{eqnarray*}
					u_{10} &=& 20 + 10 \times 7 \\
					u_{10} &=& 90.
				\end{eqnarray*}	
				
				Aline parcourt 80 km la dixième semaine.
			\end{solution}
			
			\part[1] Calculer la distance totale parcourue par Aline au cours de ses entrainements, quinzième semaine inclue.
			
			\begin{solution}
				\begin{eqnarray*}
					S_15 &=& \dfrac{15 \times (u_1 + u_15)}{2} \\
					S_15 &=& \dfrac{15 \times (20 + 118)}{2} \\
					S_15 &=& \num{1035}
				\end{eqnarray*}
			
			Au total, Aline a parcouru \num{1035} km au cours de ses entrainements.
			\end{solution}
		\end{parts}
	
		\question Programme d'entrainement de Blandine
		
		Chaque semaine, Blandine augmente de \num{13.5} \% la distance parcourue, de telle sorte que la distance parcourue la quinzième semaine soit aussi de 118 km, à l'unité près.
		
		On note $v_n$ la distance parcourue la n-ième semaine. Ainsi $v_1 = 20$ et $v{15}=118$.
		\begin{parts}
			\part[1] Montrer que la suite $(v_n)$ correspondante est une suite géométrique et déterminer sa raison.
			\begin{solution}
				La distance parcourue augmente chaque semaine de \num{13.5} \%, elle est donc multipliée par \num{1.135} ($1+\frac{\num{13.5}}{100}$).
				J'en déduis que c'est une suite géométrique de premier terme $v_1 = 20$, et de raison $q=\num{1.135}$.
			\end{solution}
			
			\part[1] Exprimer $v_n$ en fonction de $n$.
			\begin{solution}
				
				On a :				
				\begin{eqnarray*}
					v_{n} &=& v_1 \times q^(n-1) \\
					v_{n} &=& 20 \times \num{1.135}^(n-1).
				\end{eqnarray*}	
				
				
			\end{solution}
			
			\part[1] Calculer la distance parcourue par Blandine le samedi de la dixième semaine.
			\begin{solution}
				La distance parcourue la dixième semaine correspond à $v_{10}.$ 
				
				\begin{eqnarray*}
					v_{10} &=& 20 \times \num{1.135}^9 \\
					u_{10} &\approx& \num{62.52}
				\end{eqnarray*}	
				
				Blandine parcourt 63 km la dixième semaine.
			\end{solution}
			
			\part[1] Calculer la distance totale parcourue par Blandine au cours de ses entrainements, quinzième semaine inclue.
			\begin{solution}
				\begin{eqnarray*}
					S_15 &=&  u_1 \times \dfrac{1 - \num{1.135}^15}{1-\num{1.135}} \\
					S_15 &=& 20 \times \dfrac{1 - \num{1.135}^15}{1-\num{1.135}} \\
					S_15 &\approx& \num{841.82}
				\end{eqnarray*}
				
				Au total, Blandine a parcouru \num{842} km au cours de ses entrainements.
			\end{solution}
		\end{parts}
\end{questions}

\vspace*{3cm}


\fbox{%
	
	
	\begin{minipage}{0.9\textwidth}
		\textbf{\underline{Formulaire :}}\\
		Somme des $n$ premiers termes d'une suite arithmétique :
		
		\begin{equation*}
			S_n = \dfrac{n \times (u_1 + u_n)}{2}
		\end{equation*}
		
		
		Somme des $n$ premiers termes d'une suite géométrique :
		
		\begin{equation*}
		S_n = u_1 \times \dfrac{1 - q^n}{1-q}
		\end{equation*}
	\end{minipage}%
}