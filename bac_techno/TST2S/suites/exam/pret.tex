\section{Offres de prêt}

Une infirmière souhaitant créer un cabinet a besoin de d'un prêt de \num{100000} €. Elle contacte deux banques, A et B.

\begin{questions}
	\question La banque A lui propose un prêt remboursable en 7 annuités. Les annuités sont les termes consécutifs d'une suite arithmétique de premier terme $u_0 = \num{15000}$ (le premier remboursement est de \num{15000} €), et de raison \num{1800}.
	
	\begin{parts}
		\part Calculer le montant des versements $u_1$, $u_2$, $u_3$, $u_4$, $u_5$ et $u_6$.
		\part Quelle serait la somme totale remboursée si l'infirmière souscrivait le prêt auprès de la banque A ?
	\end{parts}

	\question La banque B propose également de rembourser le prêt sur 7 ans, en 7 versements. Le premier remboursement noté $v_0$ serait de \num{20000} €. Les remboursements suivants, notés $v_1$, $v_2$, $v_3$, $v_4$, $v_5$ et $v_6$ seraient chacun en augmentation de 2 \% par rapport au remboursement précédent.
	
	\begin{parts}
		\part Calculer $v_1$ en précisant par quel calcul on passe de $v_0$ à $v_1$. Calculer $v_2$.
		
		\part Donner pour tout entier $n$, $0 \leq n \leq 5$, l'expression de $v_{n+1}$ en fonction de $v_n$.
		
		\part En déduire que les nombres $v_0$, ..., $v_6$ sont des termes consécutifs d'une suite géométrique de premier terme $v_0$ dont on précisera la raison $q$.
		
		\part Quelle serait la somme totale remboursée si l'infirmière souscrivait le prêt auprès de la banque B ? Arrondir à l'euro.
		
	\end{parts}

	\question Quelle banque offre la solution la plus avantageuse ?
\end{questions}