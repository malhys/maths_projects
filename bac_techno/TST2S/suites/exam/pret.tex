\section{Offres de prêt \textit{(8 points)}}

Une infirmière souhaitant créer un cabinet a besoin de d'un prêt de \num{100000} €. Elle contacte deux banques, A et B.\\


\begin{questions}
	\question La banque A lui propose un prêt remboursable en 7 annuités. Les annuités sont les termes consécutifs d'une suite arithmétique de premier terme $u_0 = \num{15000}$ (le premier remboursement est de \num{15000} €), et de raison \num{1800}.
	
	\begin{parts}
		\part[1] Calculer le montant des versements $u_1$, $u_2$, $u_3$, $u_4$, $u_5$ et $u_6$.
			\begin{solution}
				\begin{itemize}
					\item $u_1 = u_0 + \num{1800} = \num{15000} + \num{1800} = \num{16800}$;
					\item $u_2 = u_1 + \num{1800} = \num{16800} + \num{1800} = \num{18600}$;
					\item $u_3 = u_2 + \num{1800} = \num{18600} + \num{1800} = \num{20400}$;
					\item $u_4 = u_3 + \num{1800} = \num{20400} + \num{1800} = \num{22200}$;
					\item $u_5 = u_4 + \num{1800} = \num{22200} + \num{1800} = \num{24000}$;
					\item $u_6 = u_5 + \num{1800} = \num{24000} + \num{1800} = \num{25800}$.
				\end{itemize}
			\end{solution}
		\part[1\half] Quelle serait la somme totale remboursée si l'infirmière souscrivait le prêt auprès de la banque A ?\\
		
			\begin{solution}
				Je calcule la somme des 7 annuités :
				\begin{eqnarray*}
					S_n &=& \frac{(u_0 + u_n) \times (n +1)}{2} \\
					S_6 &=& \frac{(u_0 + u_6) \times 7}{2} \\
					S_6 &=& \frac{(\num{15000} + \num{25800}) \times 7}{2} \\
					S_6 &=& \frac{\num{40800} \times 7}{2} \\
					S_6 &=& \frac{\num{285600}}{2} \\
					S_6 &=& \num{142800} \\
				\end{eqnarray*}
				
				Si elle souscrivait le prêt auprès de la banque A, elle devrait rembourser \num{142800} €.
			\end{solution}
	\end{parts}

	\question La banque B propose également de rembourser le prêt sur 7 ans, en 7 versements. Le premier remboursement noté $v_0$ serait de \num{20000} €. Les remboursements suivants, notés $v_1$, $v_2$, $v_3$, $v_4$, $v_5$ et $v_6$ seraient chacun en augmentation de 2 \% par rapport au remboursement précédent.
	
	\begin{parts}
		\part[1] Calculer $v_1$ en précisant par quel calcul on passe de $v_0$ à $v_1$. Calculer $v_2$.
			\begin{solution}
				Les annuités augmentent de 2 \% chaque année, donc on multiplie par \num{1.02}.
				\begin{itemize}
					\item $ v_1 = v_0 \times \num{1.02} = \num{20000} \times \num{1.02} = \num{20400} $;
					\item $ v_2 = v_1 \times \num{1.02} = \num{20400} \times \num{1.02} = \num{20808} $.
				\end{itemize}
			\end{solution}
		
		\part[1] Donner pour tout entier $n$, $0 \leq n \leq 5$, l'expression de $v_{n+1}$ en fonction de $v_n$.
			\begin{solution}
				$v_{n+1} = v_n \times \num{1.02}$.
			\end{solution}
		
		\part[1] En déduire que les nombres $v_0$, ..., $v_6$ sont des termes consécutifs d'une suite géométrique de premier terme $v_0$ dont on précisera la raison $q$.
			\begin{solution}
				Pour passer d'un terme à l'autre, on multiplie toujours par \num{1.02}, donc $ (v_n)$ est une suite géométrique de raison $q = \num{1.02}$.
			\end{solution}
		
		\part[1\half] Quelle serait la somme totale remboursée si l'infirmière souscrivait le prêt auprès de la banque B ? Arrondir à l'euro.\\
			\begin{solution}
				\begin{eqnarray*}
					S_n &=& v_0  \times \frac{1 - q^{n+1}}{1 - q} \\
					S_6 &=& \num{20000}  \times \frac{1 - \num{1.02}^{7}}{1 - \num{1.02}} \\
					S_6 &=& \num{148686} 
				\end{eqnarray*}
				
				Si elle souscrivait le prêt auprès de la banque A, elle devrait rembourser \num{148686} €.
			\end{solution}
	\end{parts}

	\question[1] Quelle banque offre la solution la plus avantageuse ?
	
		\begin{solution}
			\num{142800} < \num{148686}, donc la banque A est plus avantageuse que la banque B.
		\end{solution}
\end{questions}