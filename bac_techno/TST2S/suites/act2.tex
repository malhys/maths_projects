\begin{myact}{Somme de nombres impairs}
	On note $S_1=u_1=1$; $S_2=u_1+u_2=1+3=4$; puis, plus généralement $S_n=u_1+u_2+u_3+...+u_n$.
	
	\begin{enumerate}
		\item Compléter le tableau suivant :\\
		\begin{tabular}{|@{\ \ }l@{\ \ }|@{\ \ }c@{\ \ }|@{\ \ }c@{\ \ }|@{\ \ }c@{\ \ }|@{\ \ }c@{\ \ }|@{\ \ }c@{\ \ }|@{\ \ }c@{\ \ }|@{\ \ }c@{\ \ }|@{\ \ }c@{\ \ }|}
			\hline
			$n$   & 1 & 2 & 3 & 4 & 5 & 6 & 7 & 8 \\ \hline
			$u_n$ & 1 & 3 & 5 &   &   &   &   &   \\ \hline
			$S_n$ & 1 & 4 &   &   &   &   &   &   \\ \hline
		\end{tabular}
		
		\item En déduire une relation entre $S_{n+1}$, $S_{n}$, et $u_{n+1}$.
		
		\item En observant les résultats du tableau conjecturer une expression de $S_n$ en fonction de $n$.
	\end{enumerate}
\end{myact}


		
		
	
