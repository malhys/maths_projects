\documentclass[xcolor={dvipsnames}]{beamer}
%\usepackage[utf8]{inputenc}
%\usetheme{Madrid}
\usetheme{CambridgeUS}
\usecolortheme{}
\usefonttheme[onlymath]{serif}

%-------------------------------------------------------------------------------
%          -Packages nécessaires pour écrire en Français et en UTF8-
%-------------------------------------------------------------------------------
\usepackage[utf8]{inputenc}
\usepackage[frenchb]{babel}
\usepackage[T1]{fontenc}
\usepackage{lmodern}
\usepackage{textcomp}

%-------------------------------------------------------------------------------

%-------------------------------------------------------------------------------
%                          -Outils de mise en forme-
%-------------------------------------------------------------------------------
\usepackage{hyperref}
\hypersetup{pdfstartview=XYZ}
\usepackage{enumerate}
\usepackage{graphicx}
%\usepackage{multicol}
%\usepackage{tabularx}

%\usepackage{anysize} %%pour pouvoir mettre les marges qu'on veut
%\marginsize{2.5cm}{2.5cm}{2.5cm}{2.5cm}

\usepackage{indentfirst} %%pour que les premier paragraphes soient aussi indentés
\usepackage{verbatim}
%\usepackage[table]{xcolor}  
%\usepackage{multirow}
\usepackage{ulem}
%-------------------------------------------------------------------------------


%-------------------------------------------------------------------------------
%                  -Nécessaires pour écrire des mathématiques-
%-------------------------------------------------------------------------------
\usepackage{amsfonts}
\usepackage{amssymb}
\usepackage{amsmath}
\usepackage{amsthm}
\usepackage{tikz}
\usepackage{xlop}
\usepackage[output-decimal-marker={,}]{siunitx}
%-------------------------------------------------------------------------------


%-------------------------------------------------------------------------------
%                    - Mise en forme 
%-------------------------------------------------------------------------------

\newcommand{\bu}[1]{\underline{\textbf{#1}}}


\usepackage{ifthen}


\newcommand{\ifTrue}[2]{\ifthenelse{\equal{#1}{true}}{#2}{$\qquad \qquad$}}

\newcommand{\kword}[1]{\textcolor{red}{\underline{#1}}}


%-------------------------------------------------------------------------------



%-------------------------------------------------------------------------------
%                    - Racourcis d'écriture -
%-------------------------------------------------------------------------------

% Angles orientés (couples de vecteurs)
\newcommand{\aopp}[2]{(\vec{#1}, \vec{#2})} %Les deuc vecteurs sont positifs
\newcommand{\aopn}[2]{(\vec{#1}, -\vec{#2})} %Le second vecteur est négatif
\newcommand{\aonp}[2]{(-\vec{#1}, \vec{#2})} %Le premier vecteur est négatif
\newcommand{\aonn}[2]{(-\vec{#1}, -\vec{#2})} %Les deux vecteurs sont négatifs

%Ensembles mathématiques
\newcommand{\naturels}{\mathbb{N}} %Nombres naturels
\newcommand{\relatifs}{\mathbb{Z}} %Nombres relatifs
\newcommand{\rationnels}{\mathbb{Q}} %Nombres rationnels
\newcommand{\reels}{\mathbb{R}} %Nombres réels
\newcommand{\complexes}{\mathbb{C}} %Nombres complexes


%Intégration des parenthèses aux cosinus
\newcommand{\cosP}[1]{\cos\left(#1\right)}
\newcommand{\sinP}[1]{\sin\left(#1\right)}

%Fractions
\newcommand{\myfrac}[2]{{\LARGE $\frac{#1}{#2}$}}

%Vocabulaire courrant
\newcommand{\cad}{c'est-à-dire}

%Droites
\newcommand{\dte}[1]{droite $(#1)$}
\newcommand{\fig}[1]{figure $#1$}
\newcommand{\sym}{symétrique}
\newcommand{\syms}{symétriques}
\newcommand{\asym}{axe de symétrie}
\newcommand{\asyms}{axes de symétrie}
\newcommand{\seg}[1]{$[#1]$}
\newcommand{\monAngle}[1]{$\widehat{#1}$}
\newcommand{\bissec}{bissectrice}
\newcommand{\mediat}{médiatrice}
\newcommand{\ddte}[1]{$[#1)$}

%Figures
\newcommand{\para}{parallélogramme}
\newcommand{\paras}{parallélogrammes}
\newcommand{\myquad}{quadrilatère}
\newcommand{\myquads}{quadrilatères}
\newcommand{\co}{côtés opposés}
\newcommand{\diag}{diagonale}
\newcommand{\diags}{diagonales}
\newcommand{\supp}{supplémentaires}
\newcommand{\car}{carré}
\newcommand{\cars}{carrés}
\newcommand{\rect}{rectangle}
\newcommand{\rects}{rectangles}
\newcommand{\los}{losange}
\newcommand{\loss}{losanges}


%----------------------------------------------------


%\usepackage{../../../../pas-math}
%\usepackage{../../../moncours_beamer}

\usepackage{amssymb,amsmath}



\graphicspath{{../img/}}

\title{Statistiques à 2 variables}
\author{O. FINOT}\institute{Lycée S$^t$ Vincent}


\AtBeginSection[]
{
	\begin{frame}
		\frametitle{}
		\tableofcontents[currentsection, hideallsubsections]
	\end{frame} 

}


\AtBeginSubsection[]
{
	\begin{frame}
		\frametitle{Sommaire}
		\tableofcontents[currentsection, currentsubsection]
	\end{frame} 
}

\begin{document}

\begin{frame}
  \titlepage 
\end{frame}

\begin{frame}{A retenir}
	
	$A$ et $B$ sont deux sous-populations d'une population $E$.
	
	\begin{itemize}
		\item $ f(B) $ est la \textcolor{red}{fréquence marginale} de $B$ :
					
			\begin{eqnarray*}
				f(B) &=& \dfrac{Effectif\; de\; B }{Effectif\; de\; E}
			\end{eqnarray*}\pause
					
		\item $f(A \cap B)$ est la \textcolor{red}{fréquence conjointe} de $A$ et $B$ :
					
			\begin{eqnarray*}
				f(A \cap B) &=& \dfrac{Effectif\; du\; croisement \; de\;A\; et\; de\; B }{Effectif\; de\; E}
			\end{eqnarray*}
		

			
	\end{itemize}
\end{frame}



\begin{frame}{A retenir}

\begin{itemize}
	
	
	\item $f(A \cup B)$ est la \textcolor{red}{fréquence de la réunion} de $A$ et $B$ :
	
	\begin{eqnarray*}
		f(A \cup B) &=& \dfrac{Ef.\, de\, A + Ef.\, de\, B  - Ef.\, du\, croisement \, de\,A\, et\, de\, B}{Effectif\; de\; E}
	\end{eqnarray*}
	
	\centering{ou}
	\begin{eqnarray*}
		f(A \cup B) &=& f(A) + f(B) - f(A \cap B)
	\end{eqnarray*}

	
\end{itemize}
\end{frame}


\begin{frame}{A retenir}

\begin{itemize}	
	
	\item $f_B(A)$ est la \textcolor{red}{fréquence conditionnelle} de $A$ \textcolor{red}{sachant} $B$ :
	
	\begin{eqnarray*}
		f_B(A) &=& \dfrac{Effectif\; du\; croisement \; de\;A\; et\; de\; B }{Effectif\; de\; B}
	\end{eqnarray*}
	
	\centering{ou} 				
	\begin{eqnarray*}
		f_B(A) &=& \dfrac{f(A \cap B)}{f(B)}
	\end{eqnarray*}
	
\end{itemize}
\end{frame}
%
%\begin{frame}{A retenir}
%	%\begin{alertblock}{}
%		\begin{itemize}
%			\item $\mathbf{f_B}$ est la \kw{fréquence marginale} de $B$ :
%			
%			\begin{eqnarray*}
%				f_B &=& \dfrac{Effectif\; de\; B }{Effectif\; de\; E}
%			\end{eqnarray*}
%			
%			\item $\mathbf{f(A \cap B)}$ est la \kw{fréquence conjointe} de $A$ et $B$ :
%			
%			\begin{eqnarray*}
%				f(A \cap B) &=& \dfrac{Effectif\; du\; croisement \; de\;A\; et\; de\; B }{Effectif\; de\; E}
%			\end{eqnarray*}
%		\end{itemize}
%	%\end{alertblock}
%	
%\end{frame}

	






%\begin{frame}
%
%
%%\begin{mybilan2}{Proportion}
%	La \kw{proportion ou fréquence} d'une partie $A$ d'une population $E$, est le rapport $p$ des effectifs de $A$ et de $E$ :
%	
%	\begin{eqnarray*}
%		p = \dfrac{n_A}{n_E} \; \left(\dfrac{Effectif de A}{Effectif de E}\right)
%	\end{eqnarray*}
%%\end{mybilan2}
%\end{frame}

\end{document}