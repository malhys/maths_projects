\documentclass[xcolor={dvipsnames}]{beamer}
%\usepackage[utf8]{inputenc}
%\usetheme{Madrid}
\usetheme{CambridgeUS}
\usecolortheme{}
\usefonttheme[onlymath]{serif}

\input{../../../../utils_maths_beamer}


%\usepackage{../../../../pas-math}
%\usepackage{../../../moncours_beamer}

\usepackage{amssymb,amsmath}



\graphicspath{{../img/}}

\title{Statistiques à 2 variables}
\author{O. FINOT}\institute{Lycée S$^t$ Vincent}


\AtBeginSection[]
{
	\begin{frame}
		\frametitle{}
		\tableofcontents[currentsection, hideallsubsections]
	\end{frame} 

}


\AtBeginSubsection[]
{
	\begin{frame}
		\frametitle{Sommaire}
		\tableofcontents[currentsection, currentsubsection]
	\end{frame} 
}

\begin{document}

\begin{frame}
  \titlepage 
\end{frame}

\begin{frame}{A retenir}
	
	$A$ et $B$ sont deux sous-populations d'une population $E$.
	
	\begin{itemize}
		\item $ f(B) $ est la \textcolor{red}{fréquence marginale} de $B$ :
					
			\begin{eqnarray*}
				f(B) &=& \dfrac{Effectif\; de\; B }{Effectif\; de\; E}
			\end{eqnarray*}\pause
					
		\item $f(A \cap B)$ est la \textcolor{red}{fréquence conjointe} de $A$ et $B$ :
					
			\begin{eqnarray*}
				f(A \cap B) &=& \dfrac{Effectif\; du\; croisement \; de\;A\; et\; de\; B }{Effectif\; de\; E}
			\end{eqnarray*}
		

			
	\end{itemize}
\end{frame}



\begin{frame}{A retenir}

\begin{itemize}
	
	
	\item $f(A \cup B)$ est la \textcolor{red}{fréquence de la réunion} de $A$ et $B$ :
	
	\begin{eqnarray*}
		f(A \cup B) &=& \dfrac{Ef.\, de\, A + Ef.\, de\, B  - Ef.\, du\, croisement \, de\,A\, et\, de\, B}{Effectif\; de\; E}
	\end{eqnarray*}
	
	\centering{ou}
	\begin{eqnarray*}
		f(A \cup B) &=& f(A) + f(B) - f(A \cap B)
	\end{eqnarray*}

	
\end{itemize}
\end{frame}


\begin{frame}{A retenir}

\begin{itemize}	
	
	\item $f_B(A)$ est la \textcolor{red}{fréquence conditionnelle} de $A$ \textcolor{red}{sachant} $B$ :
	
	\begin{eqnarray*}
		f_B(A) &=& \dfrac{Effectif\; du\; croisement \; de\;A\; et\; de\; B }{Effectif\; de\; B}
	\end{eqnarray*}
	
	\centering{ou} 				
	\begin{eqnarray*}
		f_B(A) &=& \dfrac{f(A \cap B)}{f(B)}
	\end{eqnarray*}
	
\end{itemize}
\end{frame}
%
%\begin{frame}{A retenir}
%	%\begin{alertblock}{}
%		\begin{itemize}
%			\item $\mathbf{f_B}$ est la \kw{fréquence marginale} de $B$ :
%			
%			\begin{eqnarray*}
%				f_B &=& \dfrac{Effectif\; de\; B }{Effectif\; de\; E}
%			\end{eqnarray*}
%			
%			\item $\mathbf{f(A \cap B)}$ est la \kw{fréquence conjointe} de $A$ et $B$ :
%			
%			\begin{eqnarray*}
%				f(A \cap B) &=& \dfrac{Effectif\; du\; croisement \; de\;A\; et\; de\; B }{Effectif\; de\; E}
%			\end{eqnarray*}
%		\end{itemize}
%	%\end{alertblock}
%	
%\end{frame}

	






%\begin{frame}
%
%
%%\begin{mybilan2}{Proportion}
%	La \kw{proportion ou fréquence} d'une partie $A$ d'une population $E$, est le rapport $p$ des effectifs de $A$ et de $E$ :
%	
%	\begin{eqnarray*}
%		p = \dfrac{n_A}{n_E} \; \left(\dfrac{Effectif de A}{Effectif de E}\right)
%	\end{eqnarray*}
%%\end{mybilan2}
%\end{frame}

\end{document}