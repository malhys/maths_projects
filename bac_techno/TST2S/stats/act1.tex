\begin{myact}{Adhérents d'un club de sport}
	Parmi les 360 adhérents d'un club de sport, une enquête à donné les résultats suivants :
	\begin{itemize}
		\item 5 \% des adhérents sont fumeurs et pratiquent la compétition;
		\item 54 sont des fumeurs;
		\item Les non-fumeurs ne pratiquant pas la compétition sont cinq fois plus nombreux que les fumeurs qui pratiquent la compétition.		
	\end{itemize}

	\begin{enumerate}
		\item Compléter le tableau suivant :
		
		\begin{tabular}{|@{\ }l@{\ }|@{\ }c@{\ }|@{\ }c@{\ }|@{$\quad$}c@{$\quad$}|}
			\hline
			& Compétition $(C)$ & Pas compétition $(\bar{C})$ & Total \\ \hline
			Fumeurs  $(F)$   &             &                 &       \\ \hline
			Non fumeurs $(\bar{F})$ &             &                 &       \\ \hline
			Total       &             &                 &       \\ \hline
		\end{tabular}
	
		\item \begin{enumerate}[label=\alph*)]
			\item Quelle est la proportion, notée $f(C)$ de personnes pratiquant la compétition ?
			\item Déterminer la proportion $f(F)$ de fumeurs.
			\item Quelle est la proportion, notée $f(F \cap C)$ de personnes qui fument \textbf{et} pratiquent la compétition ? (On l'appelle fréquence conjointe de $F$ et $C$)
			\item Déterminer la proportion, notée $f_c(F)$ de fumeurs parmi les personnes pratiquant la compétition ? (On l'appelle fréquence conditionnelle de F sachant C).
			\item Quelle est la proportion, notée $f(F\cup C)$, des personnes qui fument \textbf{ou} qui pratiquent la compétition ? (On l'appelle fréquence de la réunion de F et C).			
		\end{enumerate}
	
		%\item Définir à l'aide d'une phrase puis calculer : $f(\bar{C})$, $f(\bar{F})$, $f(\bar{C} \cap \bar{F})$, $f_{\bar{F}}(\bar{C})$, $f(\bar{F} \cup \bar{C})$.
	\end{enumerate}
\end{myact}