\documentclass[12pt,a4paper]{article}

%\usepackage[left=1.5cm,right=1.5cm,top=1cm,bottom=2cm]{geometry}
\usepackage[in, plain]{fullpage}
\usepackage{array}
\usepackage{../../../pas-math}
\usepackage{../../../moncours}


%\usepackage{pas-cours}
%-------------------------------------------------------------------------------
%          -Packages nécessaires pour écrire en Français et en UTF8-
%-------------------------------------------------------------------------------
\usepackage[utf8]{inputenc}
\usepackage[frenchb]{babel}
\usepackage[T1]{fontenc}
\usepackage{lmodern}
%-------------------------------------------------------------------------------

%-------------------------------------------------------------------------------
%                          -Outils de mise en forme-
%-------------------------------------------------------------------------------
\usepackage{hyperref}
\hypersetup{pdfstartview=XYZ}
\usepackage{enumerate}
\usepackage{graphicx}
\usepackage{multicol}

\usepackage{anysize} %%pour pouvoir mettre les marges qu'on veut
%\marginsize{2.5cm}{2.5cm}{2.5cm}{2.5cm}

\usepackage{indentfirst} %%pour que les premier paragraphes soient aussi indentés
%-------------------------------------------------------------------------------


%-------------------------------------------------------------------------------
%                  -Nécessaires pour écrire des mathématiques-
%-------------------------------------------------------------------------------
\usepackage{amsfonts}
\usepackage{amssymb}
\usepackage{amsmath}
\usepackage{amsthm}
\usepackage{tikz}
%-------------------------------------------------------------------------------

%-------------------------------------------------------------------------------
%                     -Mise en forme d'exercices-
%-------------------------------------------------------------------------------
\newtheoremstyle{exostyle}
{\topsep}% espace avant
{\topsep}% espace apres
{}% Police utilisee par le style de thm
{}% Indentation (vide = aucune, \parindent = indentation paragraphe)
{\bfseries}% Police du titre de thm
{.}% Signe de ponctuation apres le titre du thm
{ }% Espace apres le titre du thm (\newline = linebreak)
{\thmname{#1}\thmnumber{ #2}\thmnote{. \normalfont{\textit{#3}}}}% composants du titre du thm : \thmname = nom du thm, \thmnumber = numéro du thm, \thmnote = sous-titre du thm

\theoremstyle{exostyle}
\newtheorem{exercice}{Exercice}

\newenvironment{questions}{
\begin{enumerate}[\hspace{12pt}\bfseries\itshape a.]}{\end{enumerate}
} %mettre un 1 à la place du a si on veut des numéros au lieu de lettres pour les questions 
%-------------------------------------------------------------------------------



%-------------------------------------------------------------------------------
%                    - Racourcis d'écriture -
%-------------------------------------------------------------------------------

% Angles orientés (couples de vecteurs)
\newcommand{\aopp}[2]{(\vec{#1}, \vec{#2})} %Les deuc vecteurs sont positifs
\newcommand{\aopn}[2]{(\vec{#1}, -\vec{#2})} %Le second vecteur est négatif
\newcommand{\aonp}[2]{(-\vec{#1}, \vec{#2})} %Le premier vecteur est négatif
\newcommand{\aonn}[2]{(-\vec{#1}, -\vec{#2})} %Les deux vecteurs sont négatifs

%Ensembles mathématiques
\newcommand{\naturels}{\mathbb{N}} %Nombres naturels
\newcommand{\relatifs}{\mathbb{Z}} %Nombres relatifs
\newcommand{\rationnels}{\mathbb{Q}} %Nombres rationnels
\newcommand{\reels}{\mathbb{R}} %Nombres réels
\newcommand{\complexes}{\mathbb{C}} %Nombres complexes
%-------------------------------------------------------------------------------




%\makeatletter
%\renewcommand*{\@seccntformat}[1]{\csname the#1\endcsname\hspace{0.1cm}}
%\makeatother


%\author{Olivier FINOT}
\date{}
\title{Information chiffrée }

%\newcommand{\disp}{false}

\lhead{CH2 : Suites numériques}
\rhead{O. FINOT}
%
%\rfoot{Page \thepage}
\begin{document}
%\maketitle

\chap[num=3, color=red]{Séries statistiques à deux variables}{Olivier FINOT, \today }

\begin{myobj}
	\begin{itemize}
		\item Reconnaître un segment, une demie-droite, une droite et savoir les tracer;
		\item Tracer avec l’équerre la droite perpendiculaire à une droite donnée passant par un point donné;
		\item Tracer avec la règle et l’équerre la droite parallèle à une droite donnée passant par un point donné;
		\item Déterminer la distance entre deux points, entre un point et une droite;
		\item Savoir coder et lire une figure.
	\end{itemize}
\end{myobj}

\begin{mycomp}
	\begin{itemize}
		\item \kw{Modéliser} 
		\item \kw{Représenter} 
		\item \kw{Raisonner} 
		\item \kw{Communiquer}
		
	\end{itemize}
\end{mycomp}

\section{Statistiques à une variable (révisions)}

\subsection{Médiane et moyenne}
\begin{mydef}
	La \kw{médiane $Me$} d'une série statistique est le nombre qui \kw{partage la série en deux} séries ayant \kw{le même effectif}.
	
	La moitié (ou 50 \%)  des valeurs de la série sont inférieures ou égales à la médiane et l'autre moitié (50 \%) lui sont supérieures ou égales.
\end{mydef}

\begin{mydef}
	On note $x_1, x_2, ..., x_p$ les valeurs du caractère étudié et $n_1, n_2, ...,n_p$ les effectifs correspondants.
	
	La \kw{moyenne $\bar{x}$} de la série statistique est $\bar{x} = \dfrac{n_1x_1 + n_2x_2 + ... + n_px_p}{N} = \dfrac{\Sigma n_ix_i}{N} $
	
\end{mydef}

\subsection{\'Etendue}

\begin{mydef}
	L'\kw{étendue $e$} d'une série statistique est la différence entre la plus grande et la plus petite valeur de la série.
\end{mydef}	



\subsection{Quartiles}

\begin{mydef}
	\begin{itemize}
		\item Le \kw{premier quartile $Q_1$}, est la plus petite valeur à laquelle un quart (ou 25 \%) des valeurs sont inférieures ou égales.
		\item Le \kw{troisième quartile $Q_3$}, est la plus petite valeur à laquelle trois quarts (ou 75 \%) des valeurs sont inférieures ou égales.
		\item L'\kw{écart interquartile $Q_3-Q_1$} est la différence entre les 3$^e$ et 1$^{er}$ quartiles : $Q_3 - Q_1$. Il regroupe au moins 50 \% des effectifs de la série avec un nombre égal de valeurs réparties de part et d'autre de la médiane $Me$.
	\end{itemize}
	
\end{mydef}	

\subsection{\'Ecart type}

\begin{mydef}
	L'\kw{écart type $\sigma$} (sigma), fourni par la calculatrice ou le tableur, mesure la dispersion de la série autour de la moyenne $\bar{x}$. 
	
	Plus l'écart type $\sigma$ est grand, plus les valeurs sont <<\kw{dispersées}>> autour de la moyenne. 
	
	Inversement, plus l'écart type $\sigma$ est grand, plus les valeurs sont <<\kw{resserrées}>> autour de la moyenne.
\end{mydef}	

\end{document}