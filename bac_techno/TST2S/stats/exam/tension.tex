\section{La tension artérielle en fonction de l'âge}

Le tableau suivant donne, dans une population féminine, la moyenne de la tension artérielle maximale en fonction de l'âge.

\begin{center}
	
\begin{tabular}{|@{\ }l@{\ }|@{\ }c@{\ }|@{\ }c@{\ }|@{\ }c@{\ }|@{\ }c@{\ }|@{\ }c@{\ }|@{\ }c@{\ }|}
	\hline
	\^Age en années : $x$ & 36            & 42           & 48 & 54           & 60           & 66           \\ \hline
	Tension max :$y$    & \num{11.18} & \num{13.32} & 14 & \num{14.4} & \num{15.5} & \num{15.1}	\\ \hline
\end{tabular}
	
\end{center}	
\begin{questions}
	\question[] Représenter graphiquement le nuage de points de coordonnées $(x,y)$ de cette série dans un repère orthogonal. 
	On graduera l'axe des abscisses à partir de 36 et l'axe des ordonnées à partir de 11.
	De plus on prendra pour unités graphiques : \num{0.5} cm pour une année et 2 cm pour une unité de tension.
	
	\question[] H désigne le point moyen des 3 premiers points du nuage et K celui des 3 derniers points.
		\begin{parts}
			\part[] Déterminer les coordonnées des points H et K.
			\part[] Tracer la droite (HK).
			\part[] Vérifier que la droite (HK) a pour équation :
				\begin{equation*}
					y = \frac{1}{9}x + \frac{25}{3}.
				\end{equation*}
		\end{parts}
	
	\question[] On admet que la droite (HK) constitue un ajustement convenable du nuage de points précédent.
	\begin{parts}
		\part[] Déterminer graphiquement, en faisant apparaître les traits de construction utiles, la tension artérielle maximale prévisible pour une personne de 70 ans.
		\part[] Vérifier le résultat précédent par le calcul en utilisant l'équation de la droite (HK).
	\end{parts}
\end{questions}