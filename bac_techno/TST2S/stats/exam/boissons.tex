\section{Boissons rafraichissantes (13 points)}

M. Ka vend des boissons rafraichissantes. Il note, six jours de suite, la température maximale de la journée et les ventes réalisées au cours de la journée.
Les résultats sont donnés dans le tableau suivant :

\vspace*{0.5cm}
\begin{center}
	
\begin{tabular}{|@{\ }l@{\ }|@{\ }c@{\ }|@{\ }c@{\ }|@{\ }c@{\ }|@{\ }c@{\ }|@{\ }c@{\ }|@{\ }c@{\ }|}
	\hline
	Jour                              & $1^{er}$ & $2^e$ & $3^e$ & $4^e$ & $5^e$ & $6^e$ \\ \hline
	Température (en °C), $x_i$        & 18       & 20    & 22    & 26    & 28    & 30    \\ \hline
	Nombre de boissons vendues, $y_i$ & 24       & 44    & 62    & 100   & 132   & 148   \\ \hline
\end{tabular}

\end{center}

\begin{questions}
	\question 
		\begin{parts}
			\part[2] Représenter le nuage de points de la série statistique (Axes orthogonaux ; unités : 1 cm pour 1°C en abscisse, en commençant à l'abscisse 17 ; 1 cm pour 10 boissons en ordonnée).
			
			\part[1] Indiquer pourquoi un ajustement affine est envisageable.
		\end{parts}
	
	\question[2] La droite $\Delta$ passe par les points du nuage de coordonnées $(20\; ; \;44)$ et $(30\; ; \;148)$, correspondant aux $2^e$ et $6^e$ jours. Donner son équation, et la tracer sur le graphique.
	
	\question On choisit la droite d'équation $y=\num{10.4} x - 164$ comme droite d'ajustement du nuage de points. 
	
	Estimer par le calcul en utilisant l'équation de cette droite :
	
		\begin{parts}
			\part[1\half] le nombre de boissons vendues pour une température de supérieure de 5°C à celle du $6^e$ jour;
			
			\part[1] le nombre de boissons que vendrait  M. Ka pour une température de 25 °C;
			
			\part[1\half] à partir de quelle température M. Ka vendrait au moins 160 boissons.
		\end{parts}
	
	\question[2] Contrôler graphiquement les résultats de la question précédente en faisant apparaitre les tracés utiles.
	
	\question En fait, le $7^e$ jour, la température a augmenté de $20 \%$ par rapport au $6^e$ jour.
		\begin{parts}
			\part[1] Calculer la température du $7^e$ jour.
			
			\part[1] En déduire une estimation du nombre de boissons vendues le $7^e$ jour à l'aide de l'équation de la droite d'ajustement.
		\end{parts}
		
\end{questions}