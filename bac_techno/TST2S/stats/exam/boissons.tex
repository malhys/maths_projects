\section{Boissons rafraichissantes (13 points)}

M. Ka vend des boissons rafraichissantes. Il note, six jours de suite, la température maximale de la journée et les ventes réalisées au cours de la journée.
Les résultats sont donnés dans le tableau suivant :

\vspace*{0.5cm}
\begin{center}
	
\begin{tabular}{|@{\ }l@{\ }|@{\ }c@{\ }|@{\ }c@{\ }|@{\ }c@{\ }|@{\ }c@{\ }|@{\ }c@{\ }|@{\ }c@{\ }|}
	\hline
	Jour                              & $1^{er}$ & $2^e$ & $3^e$ & $4^e$ & $5^e$ & $6^e$ \\ \hline
	Température (en °C), $x_i$        & 18       & 20    & 22    & 26    & 28    & 30    \\ \hline
	Nombre de boissons vendues, $y_i$ & 24       & 44    & 62    & 100   & 132   & 148   \\ \hline
\end{tabular}

\end{center}

\begin{questions}
	\question 
		\begin{parts}
			\part[2] Représenter le nuage de points de la série statistique (Axes orthogonaux ; unités : 1 cm pour 1°C en abscisse, en commençant à l'abscisse 17 ; 1 cm pour 10 boissons en ordonnée).
			
			\part[1] Indiquer pourquoi un ajustement affine est envisageable.
			\begin{solution}
				Un ajustement affine est envisageable car le nuage de point est allongé.
			\end{solution}
		\end{parts}
	
	\question[2] La droite $\Delta$ passe par les points du nuage de coordonnées $(20\; ; \;44)$ et $(30\; ; \;148)$, correspondant aux $2^e$ et $6^e$ jours. Donner son équation, et la tracer sur le graphique.
	\begin{solution}
		La droite $\Delta$ passe par les points $A(20\; ; \;44)$ et $B(30\; ; \;148)$.
		
		Calcul du coefficient directeur de la droite :
		
		\begin{eqnarray*}
			a &=& \frac{Y_B - Y_A}{X_B-X_A} \\
			a &=& \frac{148 - 44}{30 - 20} \\
			a &=& \frac{104}{10} \\
			a &=& \num{10.4}
		\end{eqnarray*}
		
		Donc la droite $\Delta$ a pour équation $y= \num{10.4} x + b$, calcul de l'ordonnée à l'origine, je remplace $x$ et $y$ par les coordonnées du point A :
		
		\begin{eqnarray*}
			y &=& \num{10.4} x + b \\
			44 &=& \num{10.4} \times 20 + b \\
			44 &=& 208 + b \\
			44 - 208 &=& b \\
			-164 &=& b
 		\end{eqnarray*}
 	
 		L'équation de la droite $\Delta$ est $y=\num{10.4} x - 164$.
	\end{solution}
	
	
	\question On choisit la droite d'équation  comme droite d'ajustement du nuage de points. 
	
	Estimer par le calcul en utilisant l'équation de cette droite :
	
		\begin{parts}
			\part[1\half] le nombre de boissons vendues pour une température de supérieure de 5°C à celle du $6^e$ jour;
			\begin{solution}
				Le $6^e$ jour la température était de 30°C, je recherche le nombre de boissons vendues pour une température de 35°C :
				
				\begin{eqnarray*}
					y &=& \num{10.4} x - 164 \\
					y &=& \num{10.4} \times 35 - 164\\
					y &=& 364 - 164 \\
					y &=& 200
				\end{eqnarray*}
			
				Pour une température de 35°C, il vendrait 200 boissons.
			\end{solution}
			
			\part[1] le nombre de boissons que vendrait  M. Ka pour une température de 25 °C;
			
			\begin{solution}
				Je recherche le nombre de boissons vendues pour une température de 25°C :
				
				\begin{eqnarray*}
					y &=& \num{10.4} x - 164 \\
					y &=& \num{10.4} \times 25 - 164\\
					y &=& 260 - 164 \\
					y &=& 96
				\end{eqnarray*}
				
				Pour une température de 25°C, il vendrait 96 boissons.
			\end{solution}
			
			\part[1\half] à partir de quelle température M. Ka vendrait au moins 160 boissons.
			
			\begin{solution}
				Je recherche la température correspondant à la vente de 160 boissons :
				
				\begin{eqnarray*}
					y &=& \num{10.4} x - 164 \\
					160 &=& \num{10.4} x - 164\\
					160 + 164 &=& \num{10.4} x \\
					324 &=& \num{10.4} x \\
					\frac{324}{\num{10.4}} &=& x \\
					\num{31.2} &\approx& x 
				\end{eqnarray*}
				
				Pour vendre au moins 160 boissons, il faudra que la température atteigne les 32°C.
			\end{solution}
		\end{parts}
	
	\question[2] Contrôler graphiquement les résultats de la question précédente en faisant apparaitre les tracés utiles.
	
	\question En fait, le $7^e$ jour, la température a augmenté de $20 \%$ par rapport au $6^e$ jour.
		\begin{parts}
			\part[1] Calculer la température du $7^e$ jour.
			\begin{solution}
				Le coefficient multiplicateur correspondant à une hausse de 20 \% est \num{1.2}.  Donc la température le  $7^e$ jour est de 36°C ($30 \times \num{1.2} = 36$).
			\end{solution}
			
			\part[1] En déduire une estimation du nombre de boissons vendues le $7^e$ jour à l'aide de l'équation de la droite d'ajustement.
			\begin{solution}
				Je recherche le nombre de boissons vendues pour une température de 36°C :
				
				\begin{eqnarray*}
					y &=& \num{10.4} x - 164 \\
					y &=& \num{10.4} \times 36 - 164\\
					y &=& \num{374.4} - 164 \\
					y &=& \num{210.4}
				\end{eqnarray*}
				
				Pour une température de 36°C, il vendrait 210 boissons.
			\end{solution}
		\end{parts}
		
\end{questions}