\section{Salariés d'une entreprise pharmaceutique}

Le tableau suivant donne la répartition des \num{1300} salariés d'une entreprise du secteur pharmaceutique en fonction de leur salaire moyen (exprimé en euros) et de leur sexe.

\begin{center}
	\begin{tabular}{|c|c|c|}
		\hline
		& Hommes & Femmes \\ \hline
		{[}1000 ; 1500{[}  & 440    & 400    \\ \hline
		{[}1500 ; 2000{[}  & 200    & 180    \\ \hline
		{[}2000  ; 2500{[} & 50     & 15     \\ \hline
		{[}2500 ; 3000{[}  & 10     & 5      \\ \hline
	\end{tabular}
\end{center}

\emph{Dans cet exercice, tous les résultats seront arrondis à $10^{-2} $. Dans chaque classe, on admet que la populations est au centre.}


\begin{questions}
	\question[3]\label{q:1} Déterminer :
		\begin{parts}
			\part[1] le salaire moyen de hommes ;
			\part[1] le salaire moyen des femmes ;
			\part[1] le salaire moyen de l'ensemble des salariés de l'entreprise.
		\end{parts} 
	
	\question[3] On note : 
	$H$ la sous population des hommes parmi les salariés, et $C$ la sous-population des cadres (les salariés ayant un salaire compris entre \num{2000} et \num{2500} euros).
	
	\begin{parts}
		\part[1] Calculer les fréquences respectives des, notées $f(H)$, $f(C)$, $f(H \cap C)$, des sous-populations $H$, $C$ et $H \cap C$ dans l'ensemble des salariés de l'entreprise.
		
		\part[1] \`A l'aide du tableau et des résultats obtenus au \ref{q:1} calculer la fréquence de la sous-population des cadres dans la sous-population des hommes. Cette fréquence appelée <<fréquence de $C$ sachant $H$>> est notée $f_H(C)$.
		
		\part[1] vérifier que $f_H(C) = \dfrac{f(H \cap C)}{f(H)}$
	\end{parts}
\end{questions}