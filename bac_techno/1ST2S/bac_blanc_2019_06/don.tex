\section{Don du sang (7 points)}

Dans un lycée, lors d'une campagne de don du sang, on a demandé aux quatre-vingt-dix élèves des classes de Terminale ST2S d'indiquer leur groupe sanguin et leur rhésus.

On a obtenu les renseignements suivants :
\begin{itemize}
	\item un tiers des élèves est du groupe O,
	\item 30 \% des élèves du groupe O ont un rhésus négatif,
	\item 50 \% des élèves sont du groupe A dont six ont un rhésus négatif,
	\item quatre élèves sont du groupe AB ; ils ont tous un rhésus positif,
	\item 20 \% des élèves ont un rhésus négatif.
\end{itemize}

\begin{questions}
	\question[2] En utilsant ces renseignements compléter le tableau des effectifs donné ci-dessous. 
	
	\emph{Dans les questions suivantes, les résultats seront donnés sous forme fractionnaire.}
	
	\begin{tabular}{|l|@{\ \ \ \ }c@{\ \ \ \ }|@{\ \ \ \ }c@{\ \ \ \ }|@{\ \ \ \ }c@{\ \ \ \ }|@{\ \ \ \ }c@{\ \ \ \ }|@{\ \ }c@{\ \ }|}
\hline
\diagbox{Rhésus}{Groupe}        & A & B & AB & O & Total \\ \hline
Positif &   &   &    &   &       \\ \hline
Négatif &   &   &    &   &       \\ \hline
Total   &   &   &    &   & 90    \\ \hline
\end{tabular}
	
	\question On choisit au hasard un élève parmi les quatre-vingt-dix interrogés. On considère les événements suivants :
	
	\begin{itemize}
		\item $A$ : <<L'élève est du groupe A>>;
		\item $B$ : <<L'élève est du groupe B>>;
		\item $C$ : <<L'élève a un rhésus positif>>;
		\item $D$ : <<L'élève est du groupe A et a un rhésus positif>>.
	\end{itemize}

	\begin{parts}
		\part[1] \'Ecrire l'événement $D$ à l'aide des événements $A$ et $C$.
		
		\part[2] Calculer la probabilité de chacun des événements $A$, $B$, $C$ et $D$.
		
		\part[1] $\bar{C}$ est l'événement contraire de $C$. Définir à l'aide d'une phrase l'événement $A \cup B$, puis calculer sa probabilité.
		
	\end{parts}

	\question[1] On choisit au hasard un élève de rhésus positif. Quelle est la probabilité qu'il soit du groupe $B$ ?
\end{questions}

