\section{l'évaporation de deux liquides (7 points)}

Dans un laboratoire, un observateur étudie l'évaporation de deux liquides $A$ et$B$ en notant chaque jour la hauteur des liquides dans leurs tubes respectifs. il déduit de ces relevés que les hauteurs $h_A(t)$ et $h_B(t)$ s'expriment en fonction du temps $t$, en jours par les formules :
$h_A(t) = -\num{0.5}t + \num{6.5}$ et $h_B(t) = -\frac{2}{3}t + 8 $


\begin{questions}
	\question[1] Pour chaque tube, quelle hauteur de chaque liquide y avait-t-il au départ, c'est à dire à l'instant $t=0$ ?
	
	\question[2] Au bout de combien de jours n'y a-t-il plus de liquide $A$ dans le tube ? Même question pour le liquide $B$.
	
	\question
		\begin{parts}
			\part[2] Dans le repère fourni, tracer les représentations graphiques de $h_A(t)$ et $h_B(t)$.
			\part[1] Lire sur le graphique les coordonnées de leur point commun.
			\part[1] Interpréter en une phrase ce résultat à propos  de l'évaporation des liquides $A$ et $B$ dans leurs tubes respectifs.
		\end{parts}
\end{questions}