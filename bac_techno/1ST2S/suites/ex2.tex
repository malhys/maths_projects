\begin{myex}
	$(u_n)$ est la suite des entiers naturels impairs :
	
	On a $u_0$ = 1; $u_1$ = 3; $u_2$ = 5; $u_3$ = 7 ...
	
	C'est une suite arithmétique de premier terme $u_0$ = 1 et de raison 2.
	
	Calcul du centième nombre impair :
	 On calcule donc $u_{99}$
	 
	 \begin{eqnarray*}
	 	u_{99} & = & u_0 + 99 \times r \\
				& = & 1 + 99 \times 2 \\
				& = & 1 + 198 \\
		u_{99}	& = & 199	 	
	 \end{eqnarray*}
 
 Le centième nombre impair est égal à 199.
 
 
 Pour cette suite on a :
	 \begin{eqnarray*}
	 	 u_{n} & = & u_0 + n \times r \\
		soit \quad u_n & = & 1 + n \times 2 \\
	 	 u_n & = & 1 + 2n \\
	 	 u_{n}	& = & 2n + 1	 	
	 \end{eqnarray*}
 
\end{myex}