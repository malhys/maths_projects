\section{Scolarisation (8 points)}

Un pays en voie de développement comptait, en l'an 2000, trois millions d'enfants d'age compris entre six et onze ans. Seuls \num{700000} d'entre eux étaient scolarisés.

Dans tout cet exercice on comparera la <<population d'âge scolaire>>, c'est-à-dire le nombre des enfants compris entre six et onze ans, et la <<population scolarisée>>, c'est-à-dire le nombre des enfants d'âge scolaire qui sont inscrits à l'école.

La population d'âge scolaire de ce pays augmente de 2 \% par an et la population scolarisée augmente de \num{150000} par an.

\begin{questions}
	\question[2] Compléter le tableau suivant :
	
	\begin{center}
		\begin{tabular}{|@{\ }c@{\ }|@{\ }c@{\ }|@{\ }c@{\ }|}
			\hline
			\textbf{Année} & \textbf{Population d'âge scolaire} & \textbf{Population scolarisée} \\ \hline
			\num{2000}     & \num{3000000}                      & \num{700000}                   \\ \hline
			\num{2001}     &                                    &                                \\ \hline
			\num{2002}     &                                    &                                \\ \hline
			\num{2003}     &                                    &                                \\ \hline
		\end{tabular}
	\end{center}

	\begin{solution}
		\begin{center}
			\begin{tabular}{|@{\ }c@{\ }|@{\ }c@{\ }|@{\ }c@{\ }|}
				\hline
				\textbf{Année} & \textbf{Population d'âge scolaire} & \textbf{Population scolarisée} \\ \hline
				\num{2000}     & \num{3000000}                      & \num{700000}                   \\ \hline
				\num{2001}     &  \num{3060000}                     &  \num{850000}                    \\ \hline
				\num{2002}     &  \num{3121200}                      &  \num{1000000}                  \\ \hline
				\num{2003}     & \num{3183624}                      &   \num{1150000}               \\ \hline
			\end{tabular}
		\end{center}
	\end{solution}

%	\question Quelle est la proportion de la population scolarisée dans la population d'âge scolaire en 2000 ? en 2003 ?
	
	\question $n$ est un entier positif ou nul. On note $u_n$ la population d'âge scolaire de ce pays en l'an $2000 + n$ et $v_n$ la population scolarisée la même année.
	
		\begin{parts}
			\part[1] Quelles sont les valeurs de $u_0$ et $v_0$ ?
			\begin{solution}
				On a $u_0 = \num{3000000}$ et $v_0 = \num{700000}$.
			\end{solution}
			
			\part[1\half] Montrer que que la suite $(u_n)$ est géométrique, exprimer $u_n$ \\en fonction de $n$. 
			
			\begin{solution}
				Chaque année, la population d'âge scolaire augmente de 2 \%, elle est donc multipliée par \num{1.02} ($1 + \frac{2}{100}$).
				J'en déduis que $(v_n)$ est une suite géométrique de raison \num{1.02}. 
				
				On a donc :				
				\begin{eqnarray*}
					u_n &=& u_0 \times q^n \\
					u_n &=& \num{3000000} \times \num{1.02}^n
				\end{eqnarray*}
			\end{solution}
			
			%\part[1] Exprimer $u_{n+1}$ en fonction de $u_n$, puis $u_n$ en fonction de $n$.
			
			\part[1\half] Montrer que que la suite $(v_n)$ est arithmétique, exprimer $v_n$ \\en fonction de $n$. 
			
			\begin{solution}
				Chaque année, la population scolarisée augmente de \num{150000}.
				J'en déduis que $(u_n)$ est une suite arithmétique de raison \num{150000}. 
				
				On a donc :				
				\begin{eqnarray*}
					v_n &=& v_0 + n \times r \\
					u_n &=& \num{700000} + n \times \num{150000}
				\end{eqnarray*}
			\end{solution}
			
			%\part[1] Exprimer $v_{n+1}$ en fonction de $v_n$, puis $v_n$ en fonction de $n$.
			
			%\part Calculer une valeur approchée du pourcentage  de la population scolarisée dans la population d'âge scolaire en 2005.
		\end{parts}
	
	\question[2] \`A l'aide de la calculatrice, déterminer en quelle année on peut espérer que pour la première fois, plus de la moitié de la population d'âge scolaire sera scolarisée.
	
	\begin{solution}
		On a :
		$\dfrac{v_6}{u_6} = \dfrac{\num{3378487}}{1600000} \approx \num{0.473}$\\
		
		et :\\
		$\dfrac{v_7}{u_7} = \dfrac{\num{3446057}}{1750000} \approx \num{0.507}$
		
		C'est donc à partir de l'année 2007 que la moitié de la population d'âge scolaire sera scolarisée.
	\end{solution}
\end{questions}

