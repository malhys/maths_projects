\section{Chiffre d'affaires (6 points)}

Le chiffre d'affaires d'un laboratoire pharmaceutique augmente tous les ans de \num{50000} €.
En 2007, le chiffre d'affaires était de \num{500000} €. On note $C_0 = \num{500000}$ et $C_n$ le chiffre d'affaires au cours de l'année $2007 + n$.

\begin{questions}
	\question[1] Donner pour tout entier $n$, l'expression de $C_{n+1}$ en fonction de $C_n$.
	\begin{solution}
		$C_{n+1} = C_n + \num{50000}$
	\end{solution}

	\question[3] 
	\begin{parts}
		\part[1] En déduire que les nombres $C_0$, $C_1$, $C_2$, ... $C_n$ sont des termes successifs d'une suite arithmétique de premier terme $C_0$ dont on précisera la raison.
		\begin{solution}
			Tous les ans, le chiffre d'affaires augmente de \num{50000} €, donc pour passer d'un terme à l'autre an ajoute \num{50000}. J'en déduis qu'il s'agit d'une suite arithmétique de raison $r=\num{50000}$.
		\end{solution}
		
		\part[1] Calculer $C_5$. Que représente $C_5$ ?
		\begin{solution}
			$C_5$ représente le chiffre d'affaires du laboratoire en 2012 (2007 + 5 = 2012).
			
			On a $C_5 = \num{500000} + 5 \times \num{50000} = \num{750000}$ .
		\end{solution}
		
		\part[1] Calculer le chiffre d'affaires  prévisible pour 2013.
		\begin{solution}
			D'après la question $(a)$, je sais que $C_5$ correspond au chiffre d'affaires de l'année 2012. Donc pour 2013, il faut calculer $C_6$.
			
			On a : $C_6 = C_5 + \num{50000} =  \num{750000} + \num{50000} = \num{800000}$
		\end{solution}
	\end{parts}	
	
	\question[2] Déterminer en quelle année on peut prévoir un chiffre d'affaire de \num{1050000} €.
	\begin{solution}
		Je sais que $C_n = \num{500000} + n \times \num{50000}$.
		Je cherche la valeur de $n$ pour laquelle $C_n = \num{1050000}$ :
		
		\begin{eqnarray*}
			\num{1050000} &  = & \num{500000} + n \times \num{50000} \\
			\num{1050000} - \num{500000} & = & n \times \num{50000} \\
			\num{550000} & = & n \times \num{50000} \\
			\frac{\num{550000}}{\num{50000}} & = & n \\
			11 &=& n			
		\end{eqnarray*}
	
		Donc le chiffre d'affaires de \num{1050000} est atteint la 11$^e$ année, c'est à dire 2018.
	\end{solution}
	
\end{questions}