	\section{Suite géométrique}
	Les séries de termes suivants forment-ils une suite géométrique ? Donner le premier terme et la raison si c'est le cas.
	\begin{questions}
		
	
		\question[2] $\num{480} \: ; \: \num{-120} \: ; \: \num{30}  \: ; \:  \num{-7.5} \: ; \: \num{1.875}$ 
		\fillwithdottedlines{3cm}
		
		\question[2] $\num{2} \: ; \:  \num{6} \: ; \: \num{18} \: ; \: \num{54} \: ; \: \num{152}$%152 
		\fillwithdottedlines{3cm}
		
	\end{questions}
	
	
\section{Terme général et raison}

Pour chacune des suites géométriques définies ci-dessous par leur premier terme et leur raison :
\begin{itemize}
%	\item $u_{n+1}$ en fonction de  $u_n$;
	\item Exprimer $u_n$ en fonction de $n$;
	\item Calculer la valeur du $15^{ème}$ terme.
\end{itemize} 
\begin{questions}

	
	\question[2] $u_0 = 2$, $q= 3$
	
	\fillwithdottedlines{6cm}
	
	%\vspace*{1.5cm}
	
	\question[2] $u_1 = 15$, $q= -5$
	
	\fillwithdottedlines{6cm}
	
	\question[2] $u_0 = 4096$, $q=\num{0.5}$
	
	\fillwithdottedlines{6cm}
\end{questions}

