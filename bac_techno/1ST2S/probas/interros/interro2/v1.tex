
Le sang humain est classé en quatre groupes distincts : A; B, AB et O.

Indépendamment du groupe, le sang peut posséder le facteur rhésus. Si le sang d'un individu possède ce facteur, il est dit de rhésus positif (noté $Rh^+$). Dans le cas contraire l'individu est de rhésus négatif ($Rh^-$) .

Pour \num{10000} personnes, on a relevé que :

\begin{itemize}
	\item 40 \% des personnes sont de groupe A ;
	\item 10 \% des personnes sont de groupe B ;
	\item 5 \% des personnes sont de groupe AB ;
\end{itemize}

\begin{questions}
	
	\question[5] Compléter le tableau suivant qui donne la répartition pour \num{10000} personnes
	
	\begin{center}
		\begin{tabular}{|@{\ }l@{\ }|c|c|c|c|@{\ }@{\ }c@{\ }@{\ }|}
			\hline
			\textbf{Groupe} & \textbf{A}   & \textbf{B}   & \textbf{AB} & \textbf{O}        & \textbf{Total}       \\ \hline
			\textbf{$Rh^+$} &     & 810 &    &  & \num{8105}  \\ \hline
			\textbf{$Rh^-$} & 720 &     & 85 &          &             \\ \hline
			\textbf{Total}  &     &     &    &          & \num{10000} \\ \hline
		\end{tabular}
		
	\end{center}
	
	
	\question[5]
	
	On choisit au hasard une personne parmi les \num{10000}. Toutes les personnes ont la même probabilité d'être choisies.
	
	Déterminer la probabilité de chacun des événements suivants :
	
	\begin{itemize}
		\item $E_1$ : <<La personne est du groupe O >>;
		\item $E_2$ : <<La personne est de rhésus positif >>;
		\item $E_3$ : <<La personne est du groupe 0 et de rhésus positif >>;
		\item $E_4$ : <<La personne est du groupe 0 ou de rhésus positif >>;
	\end{itemize}
	\textit{Donner les résultats sous forme décimales arrondie à $10^{-2}$. }
	
	
	\fillwithdottedlines{9cm}
\end{questions}