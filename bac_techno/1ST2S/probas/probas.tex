\documentclass[12pt,a4paper]{article}

%\usepackage[left=1.5cm,right=1.5cm,top=1cm,bottom=2cm]{geometry}
\usepackage[in, plain]{fullpage}
\usepackage{array}
\usepackage{../../../pas-math}
\usepackage{../../../moncours}


%\usepackage{pas-cours}
%-------------------------------------------------------------------------------
%          -Packages nécessaires pour écrire en Français et en UTF8-
%-------------------------------------------------------------------------------
\usepackage[utf8]{inputenc}
\usepackage[frenchb]{babel}
\usepackage[T1]{fontenc}
\usepackage{lmodern}
\usepackage{textcomp}



%-------------------------------------------------------------------------------

%-------------------------------------------------------------------------------
%                          -Outils de mise en forme-
%-------------------------------------------------------------------------------
\usepackage{hyperref}
\hypersetup{pdfstartview=XYZ}
%\usepackage{enumerate}
\usepackage{graphicx}
\usepackage{multicol}
\usepackage{tabularx}
\usepackage{multirow}


\usepackage{anysize} %%pour pouvoir mettre les marges qu'on veut
%\marginsize{2.5cm}{2.5cm}{2.5cm}{2.5cm}

\usepackage{indentfirst} %%pour que les premier paragraphes soient aussi indentés
\usepackage{verbatim}
\usepackage{enumitem}
\usepackage[usenames,dvipsnames,svgnames,table]{xcolor}

\usepackage{variations}

%-------------------------------------------------------------------------------


%-------------------------------------------------------------------------------
%                  -Nécessaires pour écrire des mathématiques-
%-------------------------------------------------------------------------------
\usepackage{amsfonts}
\usepackage{amssymb}
\usepackage{amsmath}
\usepackage{amsthm}
\usepackage{tikz}
\usepackage{xlop}
%-------------------------------------------------------------------------------



%-------------------------------------------------------------------------------


%-------------------------------------------------------------------------------
%                    - Mise en forme avancée
%-------------------------------------------------------------------------------

\usepackage{ifthen}
\usepackage{ifmtarg}


\newcommand{\ifTrue}[2]{\ifthenelse{\equal{#1}{true}}{#2}{$\qquad \qquad$}}

%-------------------------------------------------------------------------------

%-------------------------------------------------------------------------------
%                     -Mise en forme d'exercices-
%-------------------------------------------------------------------------------
%\newtheoremstyle{exostyle}
%{\topsep}% espace avant
%{\topsep}% espace apres
%{}% Police utilisee par le style de thm
%{}% Indentation (vide = aucune, \parindent = indentation paragraphe)
%{\bfseries}% Police du titre de thm
%{.}% Signe de ponctuation apres le titre du thm
%{ }% Espace apres le titre du thm (\newline = linebreak)
%{\thmname{#1}\thmnumber{ #2}\thmnote{. \normalfont{\textit{#3}}}}% composants du titre du thm : \thmname = nom du thm, \thmnumber = numéro du thm, \thmnote = sous-titre du thm

%\theoremstyle{exostyle}
%\newtheorem{exercice}{Exercice}
%
%\newenvironment{questions}{
%\begin{enumerate}[\hspace{12pt}\bfseries\itshape a.]}{\end{enumerate}
%} %mettre un 1 à la place du a si on veut des numéros au lieu de lettres pour les questions 
%-------------------------------------------------------------------------------

%-------------------------------------------------------------------------------
%                    - Mise en forme de tableaux -
%-------------------------------------------------------------------------------

\renewcommand{\arraystretch}{1.7}

\setlength{\tabcolsep}{1.2cm}

%-------------------------------------------------------------------------------



%-------------------------------------------------------------------------------
%                    - Racourcis d'écriture -
%-------------------------------------------------------------------------------

% Angles orientés (couples de vecteurs)
\newcommand{\aopp}[2]{(\vec{#1}, \vec{#2})} %Les deuc vecteurs sont positifs
\newcommand{\aopn}[2]{(\vec{#1}, -\vec{#2})} %Le second vecteur est négatif
\newcommand{\aonp}[2]{(-\vec{#1}, \vec{#2})} %Le premier vecteur est négatif
\newcommand{\aonn}[2]{(-\vec{#1}, -\vec{#2})} %Les deux vecteurs sont négatifs

%Ensembles mathématiques
\newcommand{\naturels}{\mathbb{N}} %Nombres naturels
\newcommand{\relatifs}{\mathbb{Z}} %Nombres relatifs
\newcommand{\rationnels}{\mathbb{Q}} %Nombres rationnels
\newcommand{\reels}{\mathbb{R}} %Nombres réels
\newcommand{\complexes}{\mathbb{C}} %Nombres complexes


%Intégration des parenthèses aux cosinus
\newcommand{\cosP}[1]{\cos\left(#1\right)}
\newcommand{\sinP}[1]{\sin\left(#1\right)}


%Probas stats
\newcommand{\stat}{statistique}
\newcommand{\stats}{statistiques}
%-------------------------------------------------------------------------------

%-------------------------------------------------------------------------------
%                    - Mise en page -
%-------------------------------------------------------------------------------

\newcommand{\twoCol}[1]{\begin{multicols}{2}#1\end{multicols}}


\setenumerate[1]{font=\bfseries,label=\textit{\alph*})}
\setenumerate[2]{font=\bfseries,label=\arabic*)}


%-------------------------------------------------------------------------------
%                    - Elements cours -
%-------------------------------------------------------------------------------





%\makeatletter
%\renewcommand*{\@seccntformat}[1]{\csname the#1\endcsname\hspace{0.1cm}}
%\makeatother


%\author{Olivier FINOT}
\date{}
\title{Information chiffrée }

%\newcommand{\disp}{false}

\lhead{CH3 : Statistiques}
\rhead{O. FINOT}
%
%\rfoot{Page \thepage}
\begin{document}
%\maketitle

\chap[num=4, color=red]{Probabilités}{Olivier FINOT, \today }

\begin{myobj}
	\begin{itemize}
		
		\item Construire le symétrique d’un point ou d'une figure par rapport à une droite à la main où à l’aide d’un logiciel;
		\item Construire le symétrique d’un point ou d'une figure par rapport à un point, à la main où à l’aide d’un logiciel;
		\item Utiliser les propriétés de la symétrie axiale ou centrale;
		\item Identifier des symétries dans des figures.		
	\end{itemize}
\end{myobj}

\begin{mycomp}
	\begin{itemize}
		\item \kw{Chercher (Ch2)} :  s’engager    dans    une    démarche    scientifique, observer, questionner, manipuler, expérimenter (sur une feuille de papier, avec des objets, à l’aide de logiciels), émettre des hypothèses, chercher des exemples ou des contre-exemples, simplifier ou particulariser une situation, émettre une conjecture ;
		\item \kw{Raisonner (Ra3)} :  démontrer : utiliser un raisonnement logique et des règles établies (propriétés, théorèmes, formules) pour parvenir à une conclusion ;
		\item \kw{Communiquer (Co2)} :  expliquer à l’oral ou à l’écrit (sa démarche, son raisonnement, un calcul, un protocole   de   construction   géométrique, un algorithme), comprendre les explications d’un autre et argumenter dans l’échange ; 
		
	\end{itemize}
\end{mycomp}




\section{Vocabulaire}

\subsection{Expérience aléatoire et univers}

\begin{mydefs}
 Une expérience dont on ne peut pas prévoir en avance le résultat est une \kw{expérience aléatoire}. Le résultat obtenu est l'\kw{issue} de l'expérience. L'ensemble de toutes les issues possibles de l'expérience est l'\kw{univers}, noté $\Omega$ (omega)
\end{mydefs}

\begin{myex}
	Soit l'expérience suivante : lancé d'un dé cubique \textit{non pipé} :
	\begin{itemize}
		\item $2$ est une issue de l'expérience.
		\item L'univers est $\Omega = \{1 ; 2 ; 3 ; 4 ; 5 ; 6\}$.
	\end{itemize}
\end{myex}

\subsection{\'Evénements}

\begin{mydefs}
	Un \kw{événement} est une partie de l'univers. 
	\begin{itemize}
		\item Un événement qui contient toutes les issues de l'univers est un événement \kw{certain}.
		\item Un événement qui contient une seule issue est un \kw{événement élémentaire}.
		\item Un événement qui ne contient aucune issue est \kw{impossible}
		\item L'événement contraire d'un événement $A$ est noté $\bar{A}$, il contient toutes les issues qui ne correspondent pas à $A$.
	\end{itemize}
\end{mydefs}

\begin{myex}
	Dans le cas du lancé d'un dé à 6 faces non truqué, on appelle $A$ l'événement <<obtenir un 5>>, $B$ l'événement <<obtenir un nombre pair>> et $C$ <<obtenir un résultat inférieur à 6>>.
	\begin{itemize}
		\item L'événement $A$ correspond à l'ensemble $\{5\}$, c'est un événement élémentaire.
		\item On a $B$ = $\{2;4;6\}$.
		\item L'événement contraire de $B$ est <<obtenir un résultat impair>>, on a $\bar{B} $ = $\{1;3;5\}$.
		\item L'événement  $C$ est certain, on a $C=\Omega$.
		\item L'événement $\bar{C}$ est impossible, on a $\bar{C} = \emptyset$.
	\end{itemize}
\end{myex}


\end{document}