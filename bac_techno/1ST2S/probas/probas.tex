\documentclass[12pt,a4paper]{article}

%\usepackage[left=1.5cm,right=1.5cm,top=1cm,bottom=2cm]{geometry}
\usepackage[in, plain]{fullpage}
\usepackage{array}
\usepackage{../../../pas-math}
\usepackage{../../../moncours}


%\usepackage{pas-cours}
%-------------------------------------------------------------------------------
%          -Packages nécessaires pour écrire en Français et en UTF8-
%-------------------------------------------------------------------------------
\usepackage[utf8]{inputenc}
\usepackage[frenchb]{babel}
\usepackage[T1]{fontenc}
\usepackage{lmodern}
\usepackage{textcomp}



%-------------------------------------------------------------------------------

%-------------------------------------------------------------------------------
%                          -Outils de mise en forme-
%-------------------------------------------------------------------------------
\usepackage{hyperref}
\hypersetup{pdfstartview=XYZ}
%\usepackage{enumerate}
\usepackage{graphicx}
\usepackage{multicol}
\usepackage{tabularx}
\usepackage{multirow}


\usepackage{anysize} %%pour pouvoir mettre les marges qu'on veut
%\marginsize{2.5cm}{2.5cm}{2.5cm}{2.5cm}

\usepackage{indentfirst} %%pour que les premier paragraphes soient aussi indentés
\usepackage{verbatim}
\usepackage{enumitem}
\usepackage[usenames,dvipsnames,svgnames,table]{xcolor}

\usepackage{variations}

%-------------------------------------------------------------------------------


%-------------------------------------------------------------------------------
%                  -Nécessaires pour écrire des mathématiques-
%-------------------------------------------------------------------------------
\usepackage{amsfonts}
\usepackage{amssymb}
\usepackage{amsmath}
\usepackage{amsthm}
\usepackage{tikz}
\usepackage{xlop}
%-------------------------------------------------------------------------------



%-------------------------------------------------------------------------------


%-------------------------------------------------------------------------------
%                    - Mise en forme avancée
%-------------------------------------------------------------------------------

\usepackage{ifthen}
\usepackage{ifmtarg}


\newcommand{\ifTrue}[2]{\ifthenelse{\equal{#1}{true}}{#2}{$\qquad \qquad$}}

%-------------------------------------------------------------------------------

%-------------------------------------------------------------------------------
%                     -Mise en forme d'exercices-
%-------------------------------------------------------------------------------
%\newtheoremstyle{exostyle}
%{\topsep}% espace avant
%{\topsep}% espace apres
%{}% Police utilisee par le style de thm
%{}% Indentation (vide = aucune, \parindent = indentation paragraphe)
%{\bfseries}% Police du titre de thm
%{.}% Signe de ponctuation apres le titre du thm
%{ }% Espace apres le titre du thm (\newline = linebreak)
%{\thmname{#1}\thmnumber{ #2}\thmnote{. \normalfont{\textit{#3}}}}% composants du titre du thm : \thmname = nom du thm, \thmnumber = numéro du thm, \thmnote = sous-titre du thm

%\theoremstyle{exostyle}
%\newtheorem{exercice}{Exercice}
%
%\newenvironment{questions}{
%\begin{enumerate}[\hspace{12pt}\bfseries\itshape a.]}{\end{enumerate}
%} %mettre un 1 à la place du a si on veut des numéros au lieu de lettres pour les questions 
%-------------------------------------------------------------------------------

%-------------------------------------------------------------------------------
%                    - Mise en forme de tableaux -
%-------------------------------------------------------------------------------

\renewcommand{\arraystretch}{1.7}

\setlength{\tabcolsep}{1.2cm}

%-------------------------------------------------------------------------------



%-------------------------------------------------------------------------------
%                    - Racourcis d'écriture -
%-------------------------------------------------------------------------------

% Angles orientés (couples de vecteurs)
\newcommand{\aopp}[2]{(\vec{#1}, \vec{#2})} %Les deuc vecteurs sont positifs
\newcommand{\aopn}[2]{(\vec{#1}, -\vec{#2})} %Le second vecteur est négatif
\newcommand{\aonp}[2]{(-\vec{#1}, \vec{#2})} %Le premier vecteur est négatif
\newcommand{\aonn}[2]{(-\vec{#1}, -\vec{#2})} %Les deux vecteurs sont négatifs

%Ensembles mathématiques
\newcommand{\naturels}{\mathbb{N}} %Nombres naturels
\newcommand{\relatifs}{\mathbb{Z}} %Nombres relatifs
\newcommand{\rationnels}{\mathbb{Q}} %Nombres rationnels
\newcommand{\reels}{\mathbb{R}} %Nombres réels
\newcommand{\complexes}{\mathbb{C}} %Nombres complexes


%Intégration des parenthèses aux cosinus
\newcommand{\cosP}[1]{\cos\left(#1\right)}
\newcommand{\sinP}[1]{\sin\left(#1\right)}


%Probas stats
\newcommand{\stat}{statistique}
\newcommand{\stats}{statistiques}
%-------------------------------------------------------------------------------

%-------------------------------------------------------------------------------
%                    - Mise en page -
%-------------------------------------------------------------------------------

\newcommand{\twoCol}[1]{\begin{multicols}{2}#1\end{multicols}}


\setenumerate[1]{font=\bfseries,label=\textit{\alph*})}
\setenumerate[2]{font=\bfseries,label=\arabic*)}


%-------------------------------------------------------------------------------
%                    - Elements cours -
%-------------------------------------------------------------------------------





%\makeatletter
%\renewcommand*{\@seccntformat}[1]{\csname the#1\endcsname\hspace{0.1cm}}
%\makeatother


%\author{Olivier FINOT}
\date{}
\title{Information chiffrée }

%\newcommand{\disp}{false}

\lhead{CH3 : Statistiques}
\rhead{O. FINOT}
%
%\rfoot{Page \thepage}
\begin{document}
%\maketitle

\chap[num=4, color=red]{Probabilités}{}

\begin{myobj}
	\begin{itemize}
		
		\item Construire le symétrique d’un point ou d'une figure par rapport à une droite à la main où à l’aide d’un logiciel;
		\item Construire le symétrique d’un point ou d'une figure par rapport à un point, à la main où à l’aide d’un logiciel;
		\item Utiliser les propriétés de la symétrie axiale ou centrale;
		\item Identifier des symétries dans des figures.		
	\end{itemize}
\end{myobj}

\begin{mycomp}
	\begin{itemize}
		\item \kw{Chercher (Ch2)} :  s’engager    dans    une    démarche    scientifique, observer, questionner, manipuler, expérimenter (sur une feuille de papier, avec des objets, à l’aide de logiciels), émettre des hypothèses, chercher des exemples ou des contre-exemples, simplifier ou particulariser une situation, émettre une conjecture ;
		\item \kw{Raisonner (Ra3)} :  démontrer : utiliser un raisonnement logique et des règles établies (propriétés, théorèmes, formules) pour parvenir à une conclusion ;
		\item \kw{Communiquer (Co2)} :  expliquer à l’oral ou à l’écrit (sa démarche, son raisonnement, un calcul, un protocole   de   construction   géométrique, un algorithme), comprendre les explications d’un autre et argumenter dans l’échange ; 
		
	\end{itemize}
\end{mycomp}




\section{Vocabulaire}

\subsection{Expérience aléatoire et univers}

\begin{mydefs}
 Une expérience dont on ne peut pas prévoir en avance le résultat est une \kw{expérience aléatoire}. Le résultat obtenu est l'\kw{issue} de l'expérience. L'ensemble de toutes les issues possibles de l'expérience est l'\kw{univers}, noté $\Omega$ (omega)
\end{mydefs}

\begin{myex}
	Soit l'expérience suivante : lancé d'un dé cubique \textit{non pipé} :
	\begin{itemize}
		\item $2$ est une issue de l'expérience.
		\item L'univers est $\Omega = \{1 ; 2 ; 3 ; 4 ; 5 ; 6\}$.
	\end{itemize}
\end{myex}

\subsection{\'Evénements}

\begin{mydefs}
	Un \kw{événement} est une partie de l'univers. 
	\begin{itemize}
		\item Un événement qui contient toutes les issues de l'univers est un événement \kw{certain}.
		\item Un événement qui contient une seule issue est un \kw{événement élémentaire}.
		\item Un événement qui ne contient aucune issue est \kw{impossible}
		\item L'\kw{événement contraire} d'un événement $A$ est noté \kw{$\bar{A}$}, il contient toutes les issues qui ne correspondent pas à $A$.
	\end{itemize}
\end{mydefs}

\begin{myex}
	Dans le cas du lancé d'un dé à 6 faces non truqué, on appelle $A$ l'événement <<obtenir un nombre pair>>, $B$ l'événement <<obtenir un nombre supérieur ou égal à 5>>, $C$ l'événement <<obtenir un 3>>, et $D$ <<obtenir un nombre inférieur ou égal à 6>>.
	\begin{itemize}
		\item L'événement $C$ correspond à l'ensemble $\{3\}$, c'est un événement élémentaire.
		\item On a $A$ = $\{2;4;6\}$.
		\item L'événement contraire de $A$ est <<obtenir un résultat impair>>, on a $\bar{A} $ = $\{1;3;5\}$.
		\item L'événement  $D$ est certain, on a $D=\Omega$.
		\item L'événement $\bar{D}$ est impossible, on a $\bar{D} = \emptyset$.
	\end{itemize}
\end{myex}

\section{Calcul de probabilités}

\subsection{Définition}

\begin{mydef}
	\begin{itemize}
		\item La \kw{probabilité} d'un événement $A$, notée $P(A)$ est la somme des probabilités des événements élémentaires qui le composent.
		\item Pour tout événement $A$, \kw{$0\leq P(A) \leq 1$}
		\item Si $A$ est un événement \kw{certain}, alors \kw{$P(A)=1$}.
		\item Si $A$ est un événement \kw{impossible}, alors \kw{$P(A)=0$}.
	\end{itemize}
	
\end{mydef}


\subsection{\'Equiprobailité}


\begin{mybilan}
	
		Il y a \kw{équiprobabilté} dans le cas où tous les évévnements élémentaires ont la même probabilité.
		Dans ce cas, la probabilité d'un événement élémentaire est :
		\begin{equation*}
			\dfrac{1}{nombre\; d'éléments\; de\; \Omega}
		\end{equation*}


\end{mybilan}

\begin{myex}
	Il y a équiprobabilité dans le cas où l'on lance un dé à 6 faces non truqué. Ici l'univers $\Omega$ contient 6 éléments ($\Omega=\{1;2;3;4;5;6\}$). Donc la probabilité de chaque événement élémentaire est :
	\begin{equation*}
		\dfrac{1}{6}
	\end{equation*}
\end{myex}

\begin{mybilan}
	Dans une situation d'équiprobabilité, la probabilité d'un événement $A$ est :
	\begin{equation*}
		P(A)=\dfrac{nombre\;d'\;éléments\;de\;A}{nombre\; d'éléments\; de\; \Omega}=\dfrac{nombre\; de\; cas\;favorbales}{nombre\; de\; cas\;possibles}
	\end{equation*}
\end{mybilan}

\begin{myex}
	On lance un dé à 6 faces. 
	\begin{itemize}
		\item Soit $A$ l'événement <<obtenir un nombre pair>>. $A$ est constitué de 3 événements élémentaires, $A={2;4;6}$ . On a donc :
		\begin{eqnarray*}
			P(A)&=&\dfrac{1}{6}+\dfrac{1}{6}+\dfrac{1}{6}\\
			P(A)&=&\dfrac{3}{6}\\
			P(A)&=&\dfrac{1}{2}
		\end{eqnarray*}
	
		\item Soit $B$ l'événement <<obtenir un nombre supérieur à 5>>, $B$ est constitué des 2 événements élémentaires, $B=\{5; 6\}$. On a donc : 
		\begin{eqnarray*}
			P(B)&=&\dfrac{2}{6}\\
			P(B)&=&\dfrac{1}{3}
		\end{eqnarray*}
	\end{itemize}
\end{myex}

\subsection{Propriétés}

%\subsubsection{Unions et intersection d'événements}

\begin{mydef}
	Soient $A$ et $B$ deux événements : 
	\begin{itemize}
		
		\item l'événement $\bar{A}$ est \kw{l'événement contraire} de l'événement $A$, il est constitué de tous les éléments de $\Omega$ qui ne sont \kw{pas inclus dans $A$}.
		
		\item l'événement \kw{$A\cap B$} ($A$ inter $B$), est l'événement constitué de tous les événements élémentaires de $\Omega$ qui sont \kw{inclus à la fois dans $A$ et dans $B$}.
		
		\item l'événement \kw{$A\cup B$} ($A$ union $B$), est l'événement constitué de tous les événements élémentaires de $\Omega$ qui sont \kw{inclus dans $A$ ou dans $B$}.
		
		\item $A$ et $B$ sont \kw{disjoints ou incompatibles} si et seulement si \kw{$A \cap B = \emptyset$}
	\end{itemize} 
\end{mydef}

\begin{myexs}
	On lance un dé à 6 faces. Soient les événements $A$ et $B$ définis ci-dessus, et l'événement C <<obtenir 3>> on a:
	\begin{itemize}
		\item L'événement $\bar{A}$ est <<ne pas obtenir un nombre pair>>, $\bar{A}=\{1;3;5\}$.
		\item L'événement $\bar{C}$ est <<ne pas obtenir un nombre pair>>, $\bar{C}=\{1;2;4;5;6\}$.
		\item L'événement $A\cap B$ est <<obtenir un nombre pair et supérieur ou égal à 5 >> , $A\cap B = \{6\}$, $A$ et $B$ ne sont pas disjoints.
		\item L'événement $A\cup B$ est <<obtenir un nombre pair ou supérieur ou égal à 5 >> , $A\cap B = \{2;4;5;6\}$.
		\item $A \cap C = \emptyset$, donc $A$ et $C$ sont disjoints.
	\end{itemize}
	
\end{myexs}

\begin{mybilan}
	Soient $A$ et $B$ deux événements : 
	\begin{itemize}
		\item $\bar{A} \cup A = \Omega$, donc $P(\bar{A} \cup A)=1$.
		\item $P(\bar{A}) = 1 - P(A)$.
		\item Si les événements $A$ et $B$ sont disjoints : $P(A \cup B) = P(A) + P(B)$.
		\item Cas général, pour tous événements $A$ et $B$, \\ $P(A \cup B) = P(A) + P(B) - P(A \cap B)$.
	\end{itemize}
\end{mybilan}

\begin{myex}
	On lance un dé à 6 faces, soient les événements $A$, $B$ et $C$ définis ci-dessus.
	\begin{itemize}
		\item $C$ est un événement élémentaire, $P(C)=\dfrac{1}{6}$, donc
		\begin{equation*}
			P(\bar{C})= 1- P(C) = 1 - \dfrac{1}{6} = \dfrac{5}{6}
		\end{equation*}
	
		\item $A$ et $C$ sont disjoints donc :
			\begin{equation*}
				P(A \cup C) = P(A) + P(C) = \dfrac{1}{2} + \dfrac{1}{6} = \dfrac{4}{6} = \dfrac{2}{3}
			\end{equation*}
		
		\item $A$ et $B$ ne sont pas disjoints donc :
		\begin{equation*}
			P(A \cup B) = P(A) + P(B) - P(A \cap B) = \dfrac{1}{2} + \dfrac{1}{3} - \dfrac{1}{6} = \dfrac{5}{6}  - \dfrac{1}{6} = \dfrac{4}{6} = \dfrac{2}{3}
		\end{equation*}
	\end{itemize}
\end{myex}
\end{document}