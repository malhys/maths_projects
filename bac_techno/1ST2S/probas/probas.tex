\documentclass[12pt,a4paper]{article}

%\usepackage[left=1.5cm,right=1.5cm,top=1cm,bottom=2cm]{geometry}
\usepackage[in, plain]{fullpage}
\usepackage{array}
\usepackage{../../../pas-math}
\usepackage{../../../moncours}


%\usepackage{pas-cours}
%-------------------------------------------------------------------------------
%          -Packages nécessaires pour écrire en Français et en UTF8-
%-------------------------------------------------------------------------------
\usepackage[utf8]{inputenc}
\usepackage[frenchb]{babel}
\usepackage[T1]{fontenc}
\usepackage{lmodern}
%-------------------------------------------------------------------------------

%-------------------------------------------------------------------------------
%                          -Outils de mise en forme-
%-------------------------------------------------------------------------------
\usepackage{hyperref}
\hypersetup{pdfstartview=XYZ}
\usepackage{enumerate}
\usepackage{graphicx}
\usepackage{multicol}

\usepackage{anysize} %%pour pouvoir mettre les marges qu'on veut
%\marginsize{2.5cm}{2.5cm}{2.5cm}{2.5cm}

\usepackage{indentfirst} %%pour que les premier paragraphes soient aussi indentés
%-------------------------------------------------------------------------------


%-------------------------------------------------------------------------------
%                  -Nécessaires pour écrire des mathématiques-
%-------------------------------------------------------------------------------
\usepackage{amsfonts}
\usepackage{amssymb}
\usepackage{amsmath}
\usepackage{amsthm}
\usepackage{tikz}
%-------------------------------------------------------------------------------

%-------------------------------------------------------------------------------
%                     -Mise en forme d'exercices-
%-------------------------------------------------------------------------------
\newtheoremstyle{exostyle}
{\topsep}% espace avant
{\topsep}% espace apres
{}% Police utilisee par le style de thm
{}% Indentation (vide = aucune, \parindent = indentation paragraphe)
{\bfseries}% Police du titre de thm
{.}% Signe de ponctuation apres le titre du thm
{ }% Espace apres le titre du thm (\newline = linebreak)
{\thmname{#1}\thmnumber{ #2}\thmnote{. \normalfont{\textit{#3}}}}% composants du titre du thm : \thmname = nom du thm, \thmnumber = numéro du thm, \thmnote = sous-titre du thm

\theoremstyle{exostyle}
\newtheorem{exercice}{Exercice}

\newenvironment{questions}{
\begin{enumerate}[\hspace{12pt}\bfseries\itshape a.]}{\end{enumerate}
} %mettre un 1 à la place du a si on veut des numéros au lieu de lettres pour les questions 
%-------------------------------------------------------------------------------



%-------------------------------------------------------------------------------
%                    - Racourcis d'écriture -
%-------------------------------------------------------------------------------

% Angles orientés (couples de vecteurs)
\newcommand{\aopp}[2]{(\vec{#1}, \vec{#2})} %Les deuc vecteurs sont positifs
\newcommand{\aopn}[2]{(\vec{#1}, -\vec{#2})} %Le second vecteur est négatif
\newcommand{\aonp}[2]{(-\vec{#1}, \vec{#2})} %Le premier vecteur est négatif
\newcommand{\aonn}[2]{(-\vec{#1}, -\vec{#2})} %Les deux vecteurs sont négatifs

%Ensembles mathématiques
\newcommand{\naturels}{\mathbb{N}} %Nombres naturels
\newcommand{\relatifs}{\mathbb{Z}} %Nombres relatifs
\newcommand{\rationnels}{\mathbb{Q}} %Nombres rationnels
\newcommand{\reels}{\mathbb{R}} %Nombres réels
\newcommand{\complexes}{\mathbb{C}} %Nombres complexes
%-------------------------------------------------------------------------------




%\makeatletter
%\renewcommand*{\@seccntformat}[1]{\csname the#1\endcsname\hspace{0.1cm}}
%\makeatother


%\author{Olivier FINOT}
\date{}
\title{Information chiffrée }

%\newcommand{\disp}{false}

\lhead{CH3 : Statistiques}
\rhead{O. FINOT}
%
%\rfoot{Page \thepage}
\begin{document}
%\maketitle

\chap[num=4, color=red]{Probabilités}{Olivier FINOT, \today }

\begin{myobj}
	\begin{itemize}
		\item Reconnaître un segment, une demie-droite, une droite et savoir les tracer;
		\item Tracer avec l’équerre la droite perpendiculaire à une droite donnée passant par un point donné;
		\item Tracer avec la règle et l’équerre la droite parallèle à une droite donnée passant par un point donné;
		\item Déterminer la distance entre deux points, entre un point et une droite;
		\item Savoir coder et lire une figure.
	\end{itemize}
\end{myobj}

\begin{mycomp}
	\begin{itemize}
		\item \kw{Modéliser} 
		\item \kw{Représenter} 
		\item \kw{Raisonner} 
		\item \kw{Communiquer}
		
	\end{itemize}
\end{mycomp}

\section{Vocabulaire}

\subsection{Expérience aléatoire et univers}

\begin{mydefs}
 Une expérience dont on ne peut pas prévoir en avance le résultat est une \kw{expérience aléatoire}. Le résultat obtenu est l'\kw{issue} de l'expérience. L'ensemble de toutes les issues possibles de l'expérience est l'\kw{univers}, noté $\Omega$ (omega)
\end{mydefs}

\begin{myex}
	Soit l'expérience suivante : lancé d'un dé cubique \textit{non pipé} :
	\begin{itemize}
		\item $2$ est une issue de l'expérience.
		\item L'univers est $\Omega = \{1 ; 2 ; 3 ; 4 ; 5 ; 6\}$.
	\end{itemize}
\end{myex}

\subsection{\'Evénements}

\begin{mydefs}
	Un \kw{événement} est une partie de l'univers. 
	\begin{itemize}
		\item Un événement qui contient toutes les issues de l'univers est un événement \kw{certain}.
		\item Un événement qui contient une seule issue est un \kw{événement élémentaire}.
		\item Un événement qui ne contient aucune issue est \kw{impossible}
		\item L'événement contraire d'un événement $A$ est noté $\bar{A}$, il contient toutes les issues qui ne correspondent pas à $A$.
	\end{itemize}
\end{mydefs}

\begin{myex}
	Dans le cas du lancé d'un dé à 6 faces non truqué, on appelle $A$ l'événement <<obtenir un 5>>, $B$ l'événement <<obtenir un nombre pair>> et $C$ <<obtenir un résultat inférieur à 6>>.
	\begin{itemize}
		\item L'événement $A$ correspond à l'ensemble $\{5\}$, c'est un événement élémentaire.
		\item On a $B$ = $\{2;4;6\}$.
		\item L'événement contraire de $B$ est <<obtenir un résultat impair>>, on a $\bar{B} $ = $\{1;3;5\}$.
		\item L'événement  $C$ est certain, on a $C=\Omega$.
		\item L'événement $\bar{C}$ est impossible, on a $\bar{C} = \emptyset$.
	\end{itemize}
\end{myex}

\section{Calcul de probabilités}

\subsection{Définition}

\begin{mydef}
	\begin{itemize}
		\item La \kw{probabilité} d'un événement $A$, notée $P(A)$ est la somme des probabilités des événements élémentaires qui le composent.
		\item Pour tout événement $A$, $0\leq P(A) \leq 1$
	\end{itemize}
	
\end{mydef}


\subsection{\'Equiprobailité}


\begin{mybilan}
	
		Il y a \kw{équiprobabilté} dans le cas où tous les évévnements élémentaires ont la même probabilité.
		Dans ce cas, la probabilité d'un événement élémentaire est :
		\begin{equation*}
			\dfrac{1}{nombre\; d'éléments\; de\; \Omega}
		\end{equation*}


\end{mybilan}

\begin{myex}
	Il y a équiprobabilité dans le cas où l'on tire une carte au hasard dans un jeu de 32 cartes standard. Ici l'univers $\Omega$ contient 32 éléments, 1 pour chacune des cartes. Donc la probabilité de chaque événement élémentaire est :
	\begin{equation*}
		\dfrac{1}{32}
	\end{equation*}
\end{myex}

\begin{mybilan}
	Dans une situation d'équiprobabilité, la probabilité d'un événement $A$ est :
	\begin{equation*}
		P(A)=\dfrac{nombre\;d'\;éléments\;de\;A}{nombre\; d'éléments\; de\; \Omega}=\dfrac{nombre\; de\; cas\;favorbales}{nombre\; de\; cas\;possibles}
	\end{equation*}
\end{mybilan}

\begin{myex}
	Dans la situation où on tire une carte au hasard parmi 32. 
	\begin{itemize}
		\item Soit $A$ l'événement <<tirer un c\oe ur>>. $A$ est constitué des 8 événements élémentaires qui correspondent aux 8 cartes de c\oe ur. On a donc :
		\begin{eqnarray*}
			P(A)&=&\dfrac{1}{32}+\dfrac{1}{32}+...+\dfrac{1}{32}\\
			P(A)&=&\dfrac{8}{32}\\
			P(A)&=&\dfrac{1}{4}
		\end{eqnarray*}
	
		\item Soit $B$ l'événement <<tirer un as>>, $B$ est constitué des 4 événements élémentaires qui correspondent aux 4 as du jeu. On a donc : 
		\begin{eqnarray*}
			P(B)&=&\dfrac{4}{32}\\
			P(B)&=&\dfrac{1}{8}
		\end{eqnarray*}
	\end{itemize}
\end{myex}
\end{document}