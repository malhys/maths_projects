\section{Le self de la cantine (8 points)}

Une cantine propose en self-service un choix de deux hors-d'\oe uvre, trois plats chauds et trois desserts. Deux plateaux repas sont dit identiques lorsqu'ils sont composés du même hors d'\oe uvre, du même plat chaud et du même dessert.

\begin{questions}
	\question[4] Combien de plateaux repas différents peut-on constituer dans cette cantine ? Donner tous les choix possibles. 
	
	
	On pourra noter les hors d'\oe uvre $H$, les plats chauds $P$ et les desserts $D$. Par exemple le choix $H_2P_1D_1$ correspond à un plateau composé du hors d'\oe uvre n°2, du plat chaud n°1 et du dessert n°1. 
	
	\question 
		\begin{parts}
			\part[1] Un(e) camarade compose au hasard un plateau repas pour vous un jour où un seul plateau vous fait envie. Quelle est, sous forme de fraction, la probabilité que ce choix vous convienne, en supposant que tous les plateaux repas sont équiprobables ?
			
			\part[1] Même question un jour où vous aimez tout sauf un des plats chauds.
			
			\part[1] Même question un jour où seul un des desserts ne vous convient pas mais tout le reste vous plait. 
		\end{parts}
	
	\question[1] \'A la demande des élèves, il est décidé qu'un plat supplémentaire sera préparé. Ce plat doit-il être un hors d'\oe uvre, un plat chaud ou un dessert pour que les élèves aient un maximum de choix pour leur plateaux repas. 
\end{questions}