\section{Débouchés post bac (8 points)}

Un lycée lance une enquête pour connaître les poursuites d'études suivies par les élèves reçu au bac ST2S en 2006.
Pour les 54 élèves lauréats, on a obtenu les seuls renseignements suivants :
\begin{itemize}
	\item 14 filles et 1 garçon sont en instituts de soins en soin infirmiers (IFSI);
	\item 18 filles et 3 garçons sont en BTS ESF ;
	\item 12 filles sont entrées dans la vie active ;
	\item aucun garçon n'est entré dans la vie active ;
	\item tous les garçons ont répondu.
\end{itemize}

\begin{questions}
	\question[2] Compléter le tableau suivant :
	
	\begin{center}		
		\begin{tabular}{|@{\ \ \ }l@{\ \ \ }|@{\ \ \ }c@{\ \ \ }|@{\ \ \ }c@{\ \ \ }|@{\ \ \ }c@{\ \ \ }|}
\hline
                            & \textbf{Fille(s)} & \textbf{Garçon(s)} & \textbf{Total} \\ \hline
\textbf{En IFSI}            &                   &                    & 15             \\ \hline
\textbf{En BTS ESF}         &                   &                    &                \\ \hline
\textbf{Dans la vie active} &                   &                    &                \\ \hline
\textbf{Pas de réponse}     &                   &                    &                \\ \hline
\textbf{Total}              &                   &                    & 54             \\ \hline
\end{tabular}
	\end{center}
	
	%\question Calculer le pourcentage de lauréats ayant répondu à l'enquête. Arrondir le résultat à \num{0.1} \% près.
	
	\textit{Dans les questions suivantes, les résultats seront donnés sous forme décimale à \num{0.01} près.}
	\question On choisit un élève au hasard parmi les 54 lauréats et on considère les événements suivants :
	\begin{itemize}
		\item $A$ : << Le lauréat est un garçon >>;
		\item $B$ : << Le lauréat a répondu qu'il est en institut de formation en soins infirmiers >>;
		\item $C$ : << Le lauréat est un garçon  en BTS ESF>>;
		\item $B$ : << Le lauréat est une filles qui a répondu être en institut de formation en soins infirmiers >>.
	\end{itemize}

		\begin{parts}
			\part[1] \'Ecrire l'événement $D$ à l'aide des événements $A$ et $B$.
			\part[2] Calculer la probabilité de chacun des événements $A$, $B$, $C$ et $D$.
			\part[1] Décrire l'événement $\bar{A} \cup B$ à l'aide d'une phrase. Calculer la probabilité de cet événement.
		\end{parts}
	
	\question[2] On suit au hasard un lauréat qui a répondu être en institut de formations en soins infirmiers. Calculer la probabilité que ce soit une fille. 
\end{questions}