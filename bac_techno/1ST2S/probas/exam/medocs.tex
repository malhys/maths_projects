\section{Différents conditionnements pour des médicaments (11 points)}

Trois médicaments sont proposés sous différents conditionnements :


Le premier médicament $M_1$ est proposé en ampoules $(A)$, en comprimés $(C)$ ou en gélules $(G)$.

Le deuxième médicament $M_2$ est proposé en ampoules $(A)$ ou en comprimés $(C)$.

Le troisième médicament $M_3$ est proposé en comprimés $(C)$ ou en gélules $(G)$.\\

Une personne achète d'abord $M_1$ puis $M_2$ puis $M_3$ en laissant le hasard décider du conditionnement.\\


On note dans l'ordre les choix respectifs pour $M_1$, $M_2$ et $M_3$.
Par exemple le choix $CAG$ signifie que :
\begin{itemize}
	\item $M_1$ est sous forme de Comprimés;
	\item $M_2$ est sous forme d'Ampoules;	
	\item $M_3$ est sous forme de Gélules.
\end{itemize}

\subsection{}
\begin{questions}
	\question[2] Donner les 12 choix possibles. On pourra s'aider d'un arbre.
	
	\question[5] Donner les choix correspondants aux événements suivants :
	
	\begin{itemize}
		\item[$E_1$ :]  << Les trois médicaments sont délivrés sous forme de comprimés>>;
		\item[$E_2$ :] << Deux médicaments exactement sont délivrés sous forme de comprimés>>;
		\item[$E_3$ :] << Les trois médicaments sont délivrés sous trois conditionnements différents>>;
		\item[$E_4$ :] << $M_1$ est délivré sous forme de comprimés et $M_2$ sous forme de gélules>>;
		\item[$E_5$ :] << $M_1$ est délivré sous forme de comprimés ou $M_3$ sous forme de gélules>>;
	\end{itemize}
\end{questions}

\subsection{}

On suppose que tous les choix sont équiprobables. On donnera les résultats sous forme de fractions irréductibles.

\begin{questions}
	\question[1] Calculer la probabilité $P(E_1)$ de l'événement $E_1$.
	
	\question[1] Montrer que $P(E_2)=\dfrac{1}{3}$.
	
	\question[2] Calculer de même $P(E_3)$ ; $P(E_4)$ ; $P(E_5)$.
\end{questions} 