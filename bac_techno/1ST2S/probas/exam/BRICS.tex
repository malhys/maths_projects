\section{Voiture et télévision chez les BRICS (9 points)}

Dans un pays des BRICS\footnote{Brésil, Russie, Inde, Chine, Afrique du Sud}, une enquête a été réalisée auprès d'un échantillon  de 5000 familles ne possédant pas plus d'une voiture et pas plus d'un téléviseur.

Lors de cette enquête, 65 \% des familles déclarent posséder un téléviseur et 40 \% déclarent ne pas posséder de voiture ; parmi celles-ci 60 \% ne possèdent pas de télévision. 

\begin{questions}
	\question[2] Justifier que \num{1200} familles de l'échantillon ne possèdent ni voiture, ni téléviseur.
	\begin{solution}
		\begin{equation*}
			\dfrac{40}{100} \times \num{5000} = \num{2000}			
		\end{equation*}
	
	\num{2000} personnes ne possèdent pas de voiture.
	
	
		\begin{equation*}
			\dfrac{60}{100} \times \num{2000} = \num{1200}
		\end{equation*}
	
	Donc \num{1200} personnes ne possèdent ni voiture, ni télévision.
	\end{solution}
	
	\question[3] Compléter le tableau suivant :
	
		\begin{center}
			\begin{tabular}{|@{\ }l@{\ }|@{\ }c@{\ }|@{\ }c@{\ }|@{$\qquad$ }c@{$\qquad$ }|}
	\hline
                       & Nombre de      & Nombre de         &       \\
                       & familles ayant & familles n'ayant  & Total \\
                       & un téléviseur  & pas de téléviseur &       \\ \hline
Nombre de familles     &                &                   &       \\
ayant une voiture      &                &                   &       \\ \hline
Nombre de familles     &                &                   &       \\
n'ayant pas de voiture &                &                   &       \\ \hline
Total                  &                &                   & 5000  \\ \hline
\end{tabular}
		\end{center}
	
		\begin{solution}
			\begin{center}
				\begin{tabular}{|@{\ }l@{\ }|@{\ }c@{\ }|@{\ }c@{\ }|@{$\qquad$ }c@{$\qquad$ }|}
	\hline
                       & Nombre de      & Nombre de         &       \\
                       & familles ayant & familles n'ayant  & Total \\
                       & un téléviseur  & pas de téléviseur &       \\ \hline
Nombre de familles     & \num{2450}     &   \num{550}       &   \num{300}    \\
ayant une voiture      &                &                   &       \\ \hline
Nombre de familles     &  \num{800}     &   \num{1200}      &   \num{2000}    \\
n'ayant pas de voiture &                &                   &       \\ \hline
Total                  &  \num{3250}    &   \num{1750}      & 5000  \\ \hline
\end{tabular}
			\end{center}
		\end{solution}
	
	\question[3] On choisit une famille au hasard parmi cet échantillon. 
	On pourra noter :
	
	\begin{itemize}
		\item $T$ : l'événement <<la famille choisie possède un téléviseur>> et $\bar{T}$ son événement contraire.
		
		\item $V$ : l'événement <<la famille choisie possède une voiture>> et $\bar{V}$ son événement contraire.
	\end{itemize}

	\begin{parts}
		\part[1] Déterminer la probabilité que la famille choisie possède une voiture.
		\begin{solution}
			\begin{equation*}
				P(V) = \dfrac{3000}{5000}  = \dfrac{3}{5} = \num{0.6}
			\end{equation*}
		
		La probabilité que la famille choisie possède une voiture est \num{0.6}.
		\end{solution}
		\part[1] Déterminer la probabilité que la famille choisie possède une voiture et un téléviseur.
		\begin{solution}
			\begin{equation*}
				P(V \cap T) = \dfrac{2450}{5000}  = \dfrac{49}{100} = \num{0.49}
			\end{equation*}
			
			La probabilité que la famille choisie possède une voiture et un téléviseur est \num{0.49}.
		\end{solution}
		\part[1] Déterminer la probabilité que la famille choisie possède une voiture ou un téléviseur.
		\begin{solution}
			\begin{eqnarray*}
				P(V \cup T) &=&  P(V)  + P(T) - P(V \cap T) \\
				P(V \cup T) &=& \dfrac{3250}{5000} + \dfrac{3000}{5000} - \dfrac{2450}{5000} \\
				P(V \cup T) &=& \dfrac{3800}{5000} \\
				P(V \cup T) &=& \dfrac{19}{25} \\
				P(V \cup T) &=& \num{0.76}
			\end{eqnarray*}
			
			La probabilité que la famille choisie possède une voiture ou un téléviseur est \num{0.76}.
		\end{solution}
	\end{parts}

	\question[1] On choisit une famille au hasard parmi celles qui ne possèdent pas de voiture. Déterminer la probabilité que la famille choisie n'ait pas de télévision.
	\begin{solution}
		\begin{equation*}
			\dfrac{1200}{2000} = \dfrac{3}{5} = \num{0.6}
		\end{equation*}
	
	La probabilité qu'une famille choisie au hasard parmi celles qui ne possèdent pas de voiture, n'ait pas de télévision est \num{0.6}.
	\end{solution}
	
\end{questions}