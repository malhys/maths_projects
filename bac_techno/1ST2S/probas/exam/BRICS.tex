\section{Voiture et télévision chez les BRICS (9 points)}

Dans un pays des BRICS\footnote{Brésil, Russie, Inde, Chine, Afrique du Sud}, une enquête a été réalisée auprès d'un échantillon  de 5000 familles ne possédant pas plus d'une voiture et pas plus d'un téléviseur.

Lors de cette enquête, 65 \% des familles déclarent posséder un téléviseur et 40 \% déclarent ne pas posséder de voiture ; parmi celles-ci 60 \% ne possèdent pas de télévision. 

\begin{questions}
	\question[1] Justifier que \num{1200} familles de l'échantillon ne possèdent ni voiture, ni téléviseur.
	
	\question[3] Compléter le tableau suivant :
	
		\begin{center}
			\begin{tabular}{|@{\ }l@{\ }|@{\ }c@{\ }|@{\ }c@{\ }|@{$\qquad$ }c@{$\qquad$ }|}
	\hline
                       & Nombre de      & Nombre de         &       \\
                       & familles ayant & familles n'ayant  & Total \\
                       & un téléviseur  & pas de téléviseur &       \\ \hline
Nombre de familles     &                &                   &       \\
ayant une voiture      &                &                   &       \\ \hline
Nombre de familles     &                &                   &       \\
n'ayant pas de voiture &                &                   &       \\ \hline
Total                  &                &                   & 5000  \\ \hline
\end{tabular}
		\end{center}
	
	\question[3] On choisit une famille au hasard parmi cet échantillon. 
	On pourra noter :
	
	\begin{itemize}
		\item $T$ : l'événement <<la famille choisie possède un téléviseur>> et $\bar{T}$ son événement contraire.
		
		\item $V$ : l'événement <<la famille choisie possède une voiture>> et $\bar{V}$ son événement contraire.
	\end{itemize}

	\begin{parts}
		\part[1] Déterminer la probabilité que la famille choisie possède une voiture.
		\part[1] Déterminer la probabilité que la famille choisie possède une voiture et un téléviseur.
		\part[1] Déterminer la probabilité que la famille choisie possède une voiture ou un téléviseur.
	\end{parts}

	\question[2] On choisit une famille au hasard parmi celles qui ne possèdent pas de voiture. Déterminer la probabilité que la famille choisie n'ait pas de télévision.
	
\end{questions}