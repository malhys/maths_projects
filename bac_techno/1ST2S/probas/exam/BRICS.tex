\section{Voiture et télévision chez les BRICS (9 points)}

Dans un pays des BRICS\footnote{Brésil, Russie, Inde, Chine, Afrique du Sud}, une enquête a été réalisée auprès d'un échantillon  de 5000 familles ne possédant pas plus d'une voiture et pas plus d'un téléviseur.

Lors de cette enquête, 65 \% des familles déclarent posséder un téléviseur et 40 \% déclarent ne pas posséder de voiture ; parmi celles-ci 60 \% ne possèdent pas de télévision. 

\begin{questions}
	\question[1] Justifier que \num{1200} familles de l'échantillon ne possèdent ni voiture, ni téléviseur.
	
	\question[3] Compléter le tableau suivant :
	
		\begin{center}
			\section{Calculs (5 points)}

\begin{questions}


\question Compléter ce tableau (le détail des calculs n'est pas demandé) :

\begin{center}
	{\Large \begin{tabular}{|@{\ }c@{\ }|@{\ }c@{\ }|@{\ }c@{\ }|@{\ }c@{\ }|@{\ }c@{\ }|@{\ }c@{\ }|}
	\hline
	\textbf{a} & \textbf{b} & \textbf{c} & \textbf{$a-b \times c$} & \textbf{$(a-b) \times c$} & \textbf{$a + b - c$} \\ \hline
	12         & 3          & 4          &                         &                           &                      \\ \hline
	48         & 8          & 6          &                         &                           &                      \\ \hline
	\num{7.5}  & \num{2.5}  & 2          &                         &                           &                      \\ \hline
	8          & 3          & \num{1.5}  &                         &                           &                      \\ \hline
	\num{7.7}  & \num{3.9}  & 0          &                         &                           &                      \\ \hline
\end{tabular}}
\end{center}


\begin{solution}
	\begin{center}
		{\Large \begin{tabular}{|@{\ }c@{\ }|@{\ }c@{\ }|@{\ }c@{\ }|@{\ }c@{\ }|@{\ }c@{\ }|@{\ }c@{\ }|}
				\hline
				\textbf{a} & \textbf{b} & \textbf{c} & \textbf{$a-b \times c$} & \textbf{$(a-b) \times c$} & \textbf{$a + b - c$} \\ \hline
				12         & 3          & 4          & 0                       &   36                      &   11                 \\ \hline
				48         & 8          & 6          & 0                       &   240                     &       50             \\ \hline
				\num{7.5}  & \num{2.5}  & 2          &  \num{2.5}              &   10                      &        8             \\ \hline
				8          & 3          & \num{1.5}  &  \num{3.5}              &  \num{7.5}                &   \num{9.5}          \\ \hline
				\num{7.7}  & \num{3.9}  & 0          &  \num{7.7}              & 0                         &  \num{11.6}          \\ \hline
		\end{tabular}}
	\end{center}
\end{solution}
\end{questions}
		\end{center}
	
	\question[3] On choisit une famille au hasard parmi cet échantillon. 
	On pourra noter :
	
	\begin{itemize}
		\item $T$ : l'événement <<la famille choisie possède un téléviseur>> et $\bar{T}$ son événement contraire.
		
		\item $V$ : l'événement <<la famille choisie possède une voiture>> et $\bar{V}$ son événement contraire.
	\end{itemize}

	\begin{parts}
		\part[1] Déterminer la probabilité que la famille choisie possède une voiture.
		\part[1] Déterminer la probabilité que la famille choisie possède une voiture et un téléviseur.
		\part[1] Déterminer la probabilité que la famille choisie possède une voiture ou un téléviseur.
	\end{parts}

	\question[2] On choisit une famille au hasard parmi celles qui ne possèdent pas de voiture. Déterminer la probabilité que la famille choisie n'ait pas de télévision.
	
\end{questions}