\documentclass[12pt,a4paper]{article}

%\usepackage[left=1.5cm,right=1.5cm,top=1cm,bottom=2cm]{geometry}
\usepackage[in, plain]{fullpage}
\usepackage{array}
\usepackage{../../../pas-math}
\usepackage{../../../moncours}


%\usepackage{pas-cours}
%-------------------------------------------------------------------------------
%          -Packages nécessaires pour écrire en Français et en UTF8-
%-------------------------------------------------------------------------------
\usepackage[utf8]{inputenc}
\usepackage[frenchb]{babel}
\usepackage[T1]{fontenc}
\usepackage{lmodern}
%-------------------------------------------------------------------------------

%-------------------------------------------------------------------------------
%                          -Outils de mise en forme-
%-------------------------------------------------------------------------------
\usepackage{hyperref}
\hypersetup{pdfstartview=XYZ}
\usepackage{enumerate}
\usepackage{graphicx}
\usepackage{multicol}

\usepackage{anysize} %%pour pouvoir mettre les marges qu'on veut
%\marginsize{2.5cm}{2.5cm}{2.5cm}{2.5cm}

\usepackage{indentfirst} %%pour que les premier paragraphes soient aussi indentés
%-------------------------------------------------------------------------------


%-------------------------------------------------------------------------------
%                  -Nécessaires pour écrire des mathématiques-
%-------------------------------------------------------------------------------
\usepackage{amsfonts}
\usepackage{amssymb}
\usepackage{amsmath}
\usepackage{amsthm}
\usepackage{tikz}
%-------------------------------------------------------------------------------

%-------------------------------------------------------------------------------
%                     -Mise en forme d'exercices-
%-------------------------------------------------------------------------------
\newtheoremstyle{exostyle}
{\topsep}% espace avant
{\topsep}% espace apres
{}% Police utilisee par le style de thm
{}% Indentation (vide = aucune, \parindent = indentation paragraphe)
{\bfseries}% Police du titre de thm
{.}% Signe de ponctuation apres le titre du thm
{ }% Espace apres le titre du thm (\newline = linebreak)
{\thmname{#1}\thmnumber{ #2}\thmnote{. \normalfont{\textit{#3}}}}% composants du titre du thm : \thmname = nom du thm, \thmnumber = numéro du thm, \thmnote = sous-titre du thm

\theoremstyle{exostyle}
\newtheorem{exercice}{Exercice}

\newenvironment{questions}{
\begin{enumerate}[\hspace{12pt}\bfseries\itshape a.]}{\end{enumerate}
} %mettre un 1 à la place du a si on veut des numéros au lieu de lettres pour les questions 
%-------------------------------------------------------------------------------



%-------------------------------------------------------------------------------
%                    - Racourcis d'écriture -
%-------------------------------------------------------------------------------

% Angles orientés (couples de vecteurs)
\newcommand{\aopp}[2]{(\vec{#1}, \vec{#2})} %Les deuc vecteurs sont positifs
\newcommand{\aopn}[2]{(\vec{#1}, -\vec{#2})} %Le second vecteur est négatif
\newcommand{\aonp}[2]{(-\vec{#1}, \vec{#2})} %Le premier vecteur est négatif
\newcommand{\aonn}[2]{(-\vec{#1}, -\vec{#2})} %Les deux vecteurs sont négatifs

%Ensembles mathématiques
\newcommand{\naturels}{\mathbb{N}} %Nombres naturels
\newcommand{\relatifs}{\mathbb{Z}} %Nombres relatifs
\newcommand{\rationnels}{\mathbb{Q}} %Nombres rationnels
\newcommand{\reels}{\mathbb{R}} %Nombres réels
\newcommand{\complexes}{\mathbb{C}} %Nombres complexes
%-------------------------------------------------------------------------------




%\makeatletter
%\renewcommand*{\@seccntformat}[1]{\csname the#1\endcsname\hspace{0.1cm}}
%\makeatother


%\author{Olivier FINOT}
\date{}
\title{}

%\newcommand{\disp}{false}

%
%\rfoot{Page \thepage}
\begin{document}
%\maketitle

%\begin{myobj}
	\begin{itemize}
		\item Reconnaître un segment, une demie-droite, une droite et savoir les tracer;
		\item Tracer avec l’équerre la droite perpendiculaire à une droite donnée passant par un point donné;
		\item Tracer avec la règle et l’équerre la droite parallèle à une droite donnée passant par un point donné;
		\item Déterminer la distance entre deux points, entre un point et une droite;
		\item Savoir coder et lire une figure.
	\end{itemize}
\end{myobj}

\begin{mycomp}
	\begin{itemize}
		\item \kw{Modéliser} 
		\item \kw{Représenter} 
		\item \kw{Raisonner} 
		\item \kw{Communiquer}
		
	\end{itemize}
\end{mycomp}
%
%\begin{myobj}
	\begin{itemize}
		\item Reconnaître un segment, une demie-droite, une droite et savoir les tracer;
		\item Tracer avec l’équerre la droite perpendiculaire à une droite donnée passant par un point donné;
		\item Tracer avec la règle et l’équerre la droite parallèle à une droite donnée passant par un point donné;
		\item Déterminer la distance entre deux points, entre un point et une droite;
		\item Savoir coder et lire une figure.
	\end{itemize}
\end{myobj}

\begin{mycomp}
	\begin{itemize}
		\item \kw{Modéliser} 
		\item \kw{Représenter} 
		\item \kw{Raisonner} 
		\item \kw{Communiquer}
		
	\end{itemize}
\end{mycomp}
%
%\begin{myobj}
	\begin{itemize}
		\item Reconnaître un segment, une demie-droite, une droite et savoir les tracer;
		\item Tracer avec l’équerre la droite perpendiculaire à une droite donnée passant par un point donné;
		\item Tracer avec la règle et l’équerre la droite parallèle à une droite donnée passant par un point donné;
		\item Déterminer la distance entre deux points, entre un point et une droite;
		\item Savoir coder et lire une figure.
	\end{itemize}
\end{myobj}

\begin{mycomp}
	\begin{itemize}
		\item \kw{Modéliser} 
		\item \kw{Représenter} 
		\item \kw{Raisonner} 
		\item \kw{Communiquer}
		
	\end{itemize}
\end{mycomp}
%
%\begin{myobj}
	\begin{itemize}
		\item Reconnaître un segment, une demie-droite, une droite et savoir les tracer;
		\item Tracer avec l’équerre la droite perpendiculaire à une droite donnée passant par un point donné;
		\item Tracer avec la règle et l’équerre la droite parallèle à une droite donnée passant par un point donné;
		\item Déterminer la distance entre deux points, entre un point et une droite;
		\item Savoir coder et lire une figure.
	\end{itemize}
\end{myobj}

\begin{mycomp}
	\begin{itemize}
		\item \kw{Modéliser} 
		\item \kw{Représenter} 
		\item \kw{Raisonner} 
		\item \kw{Communiquer}
		
	\end{itemize}
\end{mycomp}
%
%\begin{myobj}
	\begin{itemize}
		\item Reconnaître un segment, une demie-droite, une droite et savoir les tracer;
		\item Tracer avec l’équerre la droite perpendiculaire à une droite donnée passant par un point donné;
		\item Tracer avec la règle et l’équerre la droite parallèle à une droite donnée passant par un point donné;
		\item Déterminer la distance entre deux points, entre un point et une droite;
		\item Savoir coder et lire une figure.
	\end{itemize}
\end{myobj}

\begin{mycomp}
	\begin{itemize}
		\item \kw{Modéliser} 
		\item \kw{Représenter} 
		\item \kw{Raisonner} 
		\item \kw{Communiquer}
		
	\end{itemize}
\end{mycomp}



\begin{myex}
	On a relevé la taille en cm de 20 personnes :
	
	Dans ce cas, il faut déterminer le centre de la classe.
	
	\begin{tabular}{|@{\ }l@{\ }|@{\ }c@{\ }|@{\ }c@{\ }|@{\ }c@{\ }|@{\ }c@{\ }|@{\ }c@{\ }|}
		\hline
		Classe           & $[145 ; 155[$ & $[155 ; 165[$ & $[165 ; 175[$ & $[175 ; 185[$ & $[185 ; 195[$ \\ \hline
		Centre de classe & 150           & 160           & 170           & 180           & 190           \\ \hline
		Effectif         & 2             & 5             & 8             & 4             & 1            \\ \hline
	\end{tabular}
	
	\vspace*{0.5cm}
	
	On remarque que l'effectif total est 20, la moyenne des tailles est :
	
	\begin{eqnarray*}
		\bar{x} &=& \frac{150 \times 2 + 160 \times 5 + 170 \times 8 + 180 \times 4 + 190 \times 1}{20}\\
		\bar{x} &=& \num{168.5}
	\end{eqnarray*}
\end{myex}

\begin{myex}
	On a relevé la taille en cm de 20 personnes :
	
	Dans ce cas, il faut déterminer le centre de la classe.
	
	\begin{tabular}{|@{\ }l@{\ }|@{\ }c@{\ }|@{\ }c@{\ }|@{\ }c@{\ }|@{\ }c@{\ }|@{\ }c@{\ }|}
		\hline
		Classe           & $[145 ; 155[$ & $[155 ; 165[$ & $[165 ; 175[$ & $[175 ; 185[$ & $[185 ; 195[$ \\ \hline
		Centre de classe & 150           & 160           & 170           & 180           & 190           \\ \hline
		Effectif         & 2             & 5             & 8             & 4             & 1            \\ \hline
	\end{tabular}
	
	\vspace*{0.5cm}
	
	On remarque que l'effectif total est 20, la moyenne des tailles est :
	
	\begin{eqnarray*}
		\bar{x} &=& \frac{150 \times 2 + 160 \times 5 + 170 \times 8 + 180 \times 4 + 190 \times 1}{20}\\
		\bar{x} &=& \num{168.5}
	\end{eqnarray*}
\end{myex}
\newpage

\begin{myexs}
	\begin{enumerate}
		\item On considère la série des notes suivantes : 
		
		\num{10} ; \num{12} ; \num{15} ; \num{17} ; \num{12.5} ; \num{9} ; \num{13} ; \num{18.5} ; \num{16.5}
		
		\begin{itemize}
			\item Je range, ces notes par ordre croissant :
			\num{9} ; \num{10} ; \num{12} ; \num{12.5} ; \num{13} ; \num{15} ; \num{16.5} ; \num{17} ; \num{18.5};
			
			\item Il y a neuf notes, donc $N = 9$, c'est un nombre impair;
			\item $\frac{9+1}{2} = 5$, donc la médiane est la $5^{eme}$ note;
			\item $Me = 13$.
		\end{itemize}
		
	\end{enumerate}
\end{myexs}

\begin{myexs}
	\begin{enumerate}
		\item On considère la série des notes suivantes : 
		
		\num{10} ; \num{12} ; \num{15} ; \num{17} ; \num{12.5} ; \num{9} ; \num{13} ; \num{18.5} ; \num{16.5}
		
		\begin{itemize}
			\item Je range, ces notes par ordre croissant :
			\num{9} ; \num{10} ; \num{12} ; \num{12.5} ; \num{13} ; \num{15} ; \num{16.5} ; \num{17} ; \num{18.5};
			
			\item Il y a neuf notes, donc $N = 9$, c'est un nombre impair;
			\item $\frac{9+1}{2} = 5$, donc la médiane est la $5^{eme}$ note;
			\item $Me = 13$.
		\end{itemize}
		
	\end{enumerate}
\end{myexs}

\begin{myexs}
	\begin{enumerate}
		\item On considère la série des notes suivantes : 
		
		\num{10} ; \num{12} ; \num{15} ; \num{17} ; \num{12.5} ; \num{9} ; \num{13} ; \num{18.5} ; \num{16.5}
		
		\begin{itemize}
			\item Je range, ces notes par ordre croissant :
			\num{9} ; \num{10} ; \num{12} ; \num{12.5} ; \num{13} ; \num{15} ; \num{16.5} ; \num{17} ; \num{18.5};
			
			\item Il y a neuf notes, donc $N = 9$, c'est un nombre impair;
			\item $\frac{9+1}{2} = 5$, donc la médiane est la $5^{eme}$ note;
			\item $Me = 13$.
		\end{itemize}
		
	\end{enumerate}
\end{myexs}

\begin{myexs}
	\begin{enumerate}
		\item On considère la série des notes suivantes : 
		
		\num{10} ; \num{12} ; \num{15} ; \num{17} ; \num{12.5} ; \num{9} ; \num{13} ; \num{18.5} ; \num{16.5}
		
		\begin{itemize}
			\item Je range, ces notes par ordre croissant :
			\num{9} ; \num{10} ; \num{12} ; \num{12.5} ; \num{13} ; \num{15} ; \num{16.5} ; \num{17} ; \num{18.5};
			
			\item Il y a neuf notes, donc $N = 9$, c'est un nombre impair;
			\item $\frac{9+1}{2} = 5$, donc la médiane est la $5^{eme}$ note;
			\item $Me = 13$.
		\end{itemize}
		
	\end{enumerate}
\end{myexs}


\begin{myex}
	On considère la série des notes suivantes : 
	
	\num{10} ; \num{12} ; \num{15} ; \num{17} ; \num{12.5} ; \num{9} ; \num{13} ; \num{18.5} ; \num{16.5}
	
	\begin{itemize}
		\item Je range, ces notes par ordre croissant :
		\num{9} ; \num{10} ; \num{12} ; \num{12.5} ; \num{13} ; \num{15} ; \num{16.5} ; \num{17} ; \num{18.5};
		
		\item Il y a neuf notes, donc $N = 9$;
		\item $\num{0.25} \times 9 = \num{2.25}$ et $\num{0.75} \times 9 = \num{6.75}$, donc le premier quartile est la $3^{eme}$ note et le troisième quartile est la $7^{eme}$ note;
		\item $Q_1 = 15$ et $Q_3 = \num{16.5}$.
	\end{itemize}
	
	
\end{myex}


\begin{myex}
	On considère la série des notes suivantes : 
	
	\num{10} ; \num{12} ; \num{15} ; \num{17} ; \num{12.5} ; \num{9} ; \num{13} ; \num{18.5} ; \num{16.5}
	
	\begin{itemize}
		\item Je range, ces notes par ordre croissant :
		\num{9} ; \num{10} ; \num{12} ; \num{12.5} ; \num{13} ; \num{15} ; \num{16.5} ; \num{17} ; \num{18.5};
		
		\item Il y a neuf notes, donc $N = 9$;
		\item $\num{0.25} \times 9 = \num{2.25}$ et $\num{0.75} \times 9 = \num{6.75}$, donc le premier quartile est la $3^{eme}$ note et le troisième quartile est la $7^{eme}$ note;
		\item $Q_1 = 15$ et $Q_3 = \num{16.5}$.
	\end{itemize}
	
	
\end{myex}


\begin{myex}
	On considère la série des notes suivantes : 
	
	\num{10} ; \num{12} ; \num{15} ; \num{17} ; \num{12.5} ; \num{9} ; \num{13} ; \num{18.5} ; \num{16.5}
	
	\begin{itemize}
		\item Je range, ces notes par ordre croissant :
		\num{9} ; \num{10} ; \num{12} ; \num{12.5} ; \num{13} ; \num{15} ; \num{16.5} ; \num{17} ; \num{18.5};
		
		\item Il y a neuf notes, donc $N = 9$;
		\item $\num{0.25} \times 9 = \num{2.25}$ et $\num{0.75} \times 9 = \num{6.75}$, donc le premier quartile est la $3^{eme}$ note et le troisième quartile est la $7^{eme}$ note;
		\item $Q_1 = 15$ et $Q_3 = \num{16.5}$.
	\end{itemize}
	
	
\end{myex}


%\newpage
%
%\begin{center}
%	\includegraphics[scale=0.7]{moustache}
%\end{center}
%
% \vspace*{2cm}
%
%\begin{center}
%	\includegraphics[scale=0.7]{moustache}
%\end{center}


\end{document}