\section{La répartition des médecins \textit{(7,5 points)}}

\subsection{Actifs et retraités actifs \textit{(4 points)}}

Le diagramme de la figure \ref{fig:repartition} donne la répartition des médecins actifs inscrits au conseil de l'ordre en 2011.

\begin{figure}[h]
	\begin{center}
		\includegraphics[scale=1]{repartition}
		\caption{Répartition des médecins actifs en 2011}
		\label{fig:repartition}
	\end{center}
\end{figure}


\begin{questions}
	\question[1] Quel pourcentage de l'ensemble des médecins actifs représentent les médecins retraités en 2011 ? Arrondir à \num{0.1} \%.
	
	\question[1] Déterminer le nombre de médecins actifs non retraités en 2010.
	
	\question[1] Déterminer le nombre de médecins actifs retraités en 2010.
	
	\question[1] Déterminer l'augmentation en pourcentage du nombre de médecins actifs entre 2010 et 2011. Arrondir à \num{0.1} \%.
	
\end{questions}

\subsection{Les modes d'exercice \textit{(3,5 points)}}

\begin{questions}
	
	\question[1] Le diagramme de la figure \ref{fig:mode_exercice1} donne la répartition des modes d'exercice des médecins actifs inscrits au conseil de l'ordre en 2011. Déterminer l'effectif pour chacun des modes d'exercice.
	
	\begin{figure}[h]
		\begin{center}
			\includegraphics[scale=0.35]{mode_exercice1}
			\caption{Modes d'exercice des médecins actifs en 2011}
			\label{fig:mode_exercice1}
		\end{center}
	\end{figure}
	
	\question[1] Le diagramme de la figure \ref{fig:mode_exercice2} donne la répartition des modes d'exercice des \num{27774} médecins de moins de 40 ans en 2011. Donner l'effectif pour chacun des modes d'exercice.
	
	\begin{figure}[h]
		\begin{center}
			\includegraphics[scale=0.35]{mode_exercice2}
			\caption{Modes d'exercice des moins de 40 ans en 2011}
			\label{fig:mode_exercice2}
		\end{center}
	\end{figure}
	
	
	\question[1\half] Quel commentaire peut-on faire sur les modes d'exercice des médecins actifs ?
\end{questions}