\section{Fabrication de vêtements (4 points)}

La taille des français étant en augmentation, un fabricant de vêtements décide de faire une enquête pour aligner sa production avec les besoins du marché. Les résultats sur un échantillon de 200 personnes sont donnés dans le tableau ci-dessous. On suppose que toutes les valeurs d'une même classe sont égales au centre de cette classe.

%\vspace*{-0.4cm}

\begin{center}
	\begin{tabular}{|@{\ \ }c@{\ \ }|@{\ \ }c@{\ \ }|}
	\hline
	Taille (en cm)     & Effectifs $n_i$ \\ \hline
	164 $\leq$ t < 168 & 5               \\ \hline
	168 $\leq$ t < 172 & 15              \\ \hline
	172 $\leq$ t < 176 & 25              \\ \hline
	176 $\leq$ t < 180 & 45              \\ \hline
	180 $\leq$ t < 184 & 60              \\ \hline
	184 $\leq$ t < 188 & 30              \\ \hline
	188 $\leq$ t < 192 & 20              \\ \hline
	Total              &                 \\ \hline
\end{tabular}
\end{center}

%\vspace*{-0.7cm}

\begin{questions}
	\question[1] Calculer la moyenne $\bar{x}$ et l'écart-type $\sigma$ de cette série statistique.
	\begin{solution}
		\begin{eqnarray*}
			\bar{x} &=& \frac{168 \times 5 + 170 \times 15 + 174 \times 25 + 178 \times 45 + 182 \times 60 + 186 \times 30 + 190 \times 20}{200}\\
			\bar{x}&=& \num{180.2}
		\end{eqnarray*}
	
		La moyenne de cette série est donc \num{180.2} et son écart type vaut \num{5.9}.
	\end{solution}
	
	\question[1] Calculer la médiane et les quartiles $Q_1$ et $Q_3$ de cette série statistique
	
	\begin{solution}
		Il y a 200 valeurs, donc la médiane est entre la $100^e$ et la $101^e$ valeur. Le premier quartile est la $50^e$ valeur et le troisième quartile la $150^e$. On a :
		
		\begin{itemize}
			\item $Q_1$ = 178;
			\item $Me$ = 182 ;
			\item $Q_3$ = 182.
		\end{itemize}
	\end{solution}
	
	\question[2] 
	
		\begin{parts}
			\part[1] Déterminer le nombre de personnes ayant une taille comprise entre $\bar{x} - 2\sigma$ et $\bar{x} + 2\sigma$.
			
			\begin{solution}
				\begin{eqnarray*}
					\bar{x} - 2\sigma &=& \num{180.2} - 2 \times \num{5.9} = \num{168.4} \\
					\bar{x} + 2\sigma &=& \num{180.2} + 2 \times \num{5.9} = \num{195} \\
				\end{eqnarray*}
			
				La taille maximale relevée est 190 cm et seules 5 personnes sur les 200 font moins de \num{168.4} cm. Donc 195 personnes ont une taille comprise entre ces deux valeurs.
			\end{solution}
			
			\part[1] Calculer le pourcentage. correspondant.
				\begin{solution}
					\begin{equation*}
						\frac{195 \times 100}{200} = \num{97.5} 
					\end{equation*}
					
					\num{97.5} \% des personnes ont une taille comprise entre $\bar{x} - 2\sigma$ et $\bar{x} + 2\sigma$.
				\end{solution}
		\end{parts} 
		
\end{questions}