\section{Fabrication de vêtements (4 points)}

La taille des français étant en augmentation, un fabricant de vêtements décide de faire une enquête pour aligner sa production avec les besoins du marché. Les résultats sur un échantillon de 200 personnes sont donnés dans le tableau ci-dessous. On suppose que toutes les valeurs d'une même classe sont égales au centre de cette classe.

%\vspace*{-0.4cm}

\begin{center}
	\begin{tabular}{|@{\ \ }c@{\ \ }|@{\ \ }c@{\ \ }|}
	\hline
	Taille (en cm)     & Effectifs $n_i$ \\ \hline
	164 $\leq$ t < 168 & 5               \\ \hline
	168 $\leq$ t < 172 & 15              \\ \hline
	172 $\leq$ t < 176 & 25              \\ \hline
	176 $\leq$ t < 180 & 45              \\ \hline
	180 $\leq$ t < 184 & 60              \\ \hline
	184 $\leq$ t < 188 & 30              \\ \hline
	188 $\leq$ t < 192 & 20              \\ \hline
	Total              &                 \\ \hline
\end{tabular}
\end{center}

%\vspace*{-0.7cm}

\begin{questions}
	\question[1] Calculer la moyenne $\bar{x}$ et l'écart-type $\sigma$ de cette série statistique.
	
	\question[1] Calculer la médiane et les quartiles $Q_1$ et $Q_3$ de cette série statistique
	
	\question[2] 
	
		\begin{parts}
			\part[1] Déterminer le nombre de personnes ayant une taille comprise entre $\bar{x} - 2\sigma$ et $\bar{x} + 2\sigma$.
			
			\part[1] Calculer le pourcentage. correspondant.
		\end{parts} 
		
\end{questions}