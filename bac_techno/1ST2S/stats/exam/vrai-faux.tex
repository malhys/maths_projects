\section{Un vrai-faux (5 points)}

Répondez par VRAI ou FAUX aux affirmations suivantes. Une justification est demandée lorsque la réponse est FAUX, aucune justification n'est demandée lorsque la réponse est VRAI.

\begin{questions}
	\question[1] Pour une série ordonnée comptant 512 nombres, la médiane n'existe pas car 512 est pair.
	\begin{solution}
		Faux, s'il y a un nombre pair de valeurs dans la série, la médiane n'est pas forcément une valeur de la série, mais elle existe.
	\end{solution}
	
	\question[1] En France, le salaire mensuel moyen s'élève à \num{2500} € et le salaire mensuel médian s'élève à \num{1600} €. Plus de 50 \% des salariés gagnent moins de \num{2500} € par mois.
	\begin{solution}
		Vrai, la moitié des salariés gagne moins de \num{1600} €, donc plus de la moitié gagne moins de \num{2500} €.
	\end{solution}
	
	\question[1] Le couple médiane et écart interquartile est peu sensible aux valeurs extrêmes de la série statistique.
	\begin{solution}
		Vrai, la moyenne est sensible aux valeurs minimales et maximales, mais pas la médiane et les quartiles.
	\end{solution}
	
	\question[1] La moyenne rend compte de la dispersion de la série statistique.
	\begin{solution}
		Faux, l'étendue, l'écart type et les quartiles sont des indicateurs de dispersion et la moyenne un indicateur de tendance centrale.
	\end{solution}
	
	\question[1] Si une série statistique compte 10 valeurs, les quartiles sont toujours des valeurs de la série.
	\begin{solution}
		Vrai, les quartiles sont toujours des valeurs de la série.
	\end{solution}
	
%	\question[1] On donne la série : 1 ; 2 ; 3 ; 4 ; 4 ; 4 ; 5 ; 8 ; 9 ; 10. L'écart interquartille est 5.
%	\begin{solution}
%		Vrai, on a $Q_1 = 3$ et $Q_3 = 8$, donc l'écart interquartile est bien 5.
%	\end{solution}
\end{questions}