\section{Moyennes trimestrielles (11 points)}

Dans un lycée, on étudie les moyennes trimestrielles du premier trimestre de deux classes, la Première A et la Première B.

\subsection{}
Les 28 élèves de la classe de première A ont obtenu les moyennes trimestrielles suivantes au premier trimestre :

\begin{equation*}
1 ; 3 ; 4 ; 5 ; 7 ; 7 ; 9 ; 10 ; 10 ; 10 ; 10 ; 10 ; 11 ;  11 ; 12 ; 12 ; 12 ; 12 ; 12 ; 13 ; 13 ; 13 ; 14 ; 15 ; 15 ; 16 ; 18 ; 19
\end{equation*}

La moyenne trimestrielle de la classe s'obtient à partir des notes moyennes de chaque élève.

\begin{questions}
	\question[2] Déterminer la médiane $Me$, le premier quartile $Q_1$ et le troisième quartile $Q_3$ de cette série statistique de moyennes trimestrielles.
	\begin{solution}
		Il y a 28 valeurs dans la série, donc la médiane sera entre la 14$^e$ et la 15$^e$, le premier quartile la $6^e$ ($24 \times \num{0.25} = 7$) et le troisième quartile la $18^e$ ($24 \times \num{0.75} = 21$). On a donc :
		
		\begin{itemize}
			\item $Q_1 $ = 9 ;
			\item $Me$ = \num{11.5};
			\item $Q_3$ = 13.
		\end{itemize} 
	\end{solution}
	
	\question[1] Représenter le diagramme en boite correspondant en faisant apparaître les valeurs extrêmes.
	
	\question[1] Calculer la moyenne trimestrielle et l'écart type de la première A. Arrondir à \num{0.1}.
	\begin{solution}
		\begin{equation*}
			\bar{x} = \dfrac{1 + 3 + 4 + 5 + 2 \times 7 + 9 + 5 \times 10 + 2 \times 11 + 5 \times 12 + 3 \times 13 + 14 + 2 \times 15 + 16 + 18 + 19 }{28} = \dfrac{304}{28} = \num{10.83}
		\end{equation*}
		
		La moyenne trimestrielle de la classe est \num{10.86}.
	\end{solution}
\end{questions}

\subsection{}

Les indicateurs de la première B sont les suivants :

\begin{itemize}
	\item Minimum = 3;
	\item premier quartile $Q'_1$ = 8;
	\item médiane $Me'$ = 10;
	\item troisième quartile $Q'_3$ = 12;
	\item Maximum = 17.
\end{itemize}

\begin{questions}
	\question[1] Représenter le diagramme en boite correspondant.
	
	\question[6] Parmi les informations suivantes, lesquelles sont vraies, fausses ou indécidables (Indécidable signifie que l'on ne peut pas conclure avec les éléments connus). Justifier votre réponse dans chacun des cas.
	
	\begin{parts}
		\part[2] 50\% des élèves de la première B ont une note comprise entre 10 et 12.
		\begin{solution}
			Faux, 10 est la médiane et 12 le troisième quartile donc 25 \% des élèves de la classe ont une note comprise entre 10 et 12.
		\end{solution}
		\part[2] 75\% des élèves de la Première B ont une note inférieure ou égale à 12.
		\begin{solution}
			Vrai, 12 est le troisième quartile.
		\end{solution}
		\part[2] Au moins 50\% des élèves de la classe de Première B ont une note inférieure ou égale à la note médiane de la Première A.
		\begin{solution}
			Vrai, a médiane de la classe rouge est inférieure à celle de la classe jaune.
		\end{solution}
	\end{parts}
\end{questions}