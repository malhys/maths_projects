\section{Intervalle de confiance (4 points)}

Un automobiliste est souvent confronté aux embouteillages de l'heure de pointe. Il a relevé pendant un trimestre la durée de son trajet habituel pour se rendre au travail. \emph{Pour chaque classe on considérera que l'ensemble de l'effectif se trouve au centre.}

\vspace*{0.5cm}

\begin{center}
	\begin{tabular}{|c|c|}
	\hline
	Durée en minutes & Nombre de trajets \\ \hline
	$[15 ; 20[$        & 10                \\ \hline
	$[20 ; 25[$        & 17                \\ \hline
	$[25 ; 30[$        & 24                \\ \hline
	$[30 ; 35[$        & 7                 \\ \hline
	$[35 ; 40[$        & 4                 \\ \hline
	$[40 ; 45[$        & 2                 \\ \hline
	$[45 ; 50[$        & 1                 \\ \hline
\end{tabular}
\end{center}

\begin{questions}
	\question[2] Calculer la moyenne et l'écart type de la série (arrondis à $10^{-1}$).
	\begin{solution}
		En rentrant les valeurs dans la calculatrice, on obtient $\bar{x} = \num{26.6}$, et $\sigma = \num{6.6}$.
	\end{solution}
	
	\question[2] L'automobiliste considère qu'il doit prévoir pour son trajet la durée moyenne plus une marge de deux fois l'écart type : <<ainsi, dit-il, je serai à l'heure au travail, au moins dans 95 \% des cas>>.
	
	Vérifier ses prévisions. (On arrondira à la minute la durée à prévoir pour son trajet.)
	
	\begin{solution}
		\begin{equation*}
			\bar{x} + 2 \times \sigma = \num{26.6} + 2 \times \num{6.6} = \num{39.8}
		\end{equation*}
		
		Il prévoit donc un temps de trajet d'environ 40 minutes. 
		Calcul du nombre de trajets inférieurs à 40 minutes :
		
		\begin{equation*}
			10 + 17 +24 +7 + 4 = 62
		\end{equation*}
		
		Calcul du nombre total de trajets :
		
		\begin{equation*}
			10 + 17 +24 +7 + 4  + 2 + 1 = 65
		\end{equation*}
		
		62 des 65 trajets se sont déroulés en moins de 40 minutes, soit \num{95.4} \% ($62 \div 65 \times 100$). Il a donc raison. 
	\end{solution}
\end{questions}