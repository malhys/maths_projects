\section{Moyennes trimestrielles (11 points)}

Dans un lycée, on étudie les moyennes trimestrielles du premier trimestre de deux classes appelées respectivement Jaune et Rouge.

\subsection{}
Les 24 élèves de la classe Jaune ont obtenu les moyennes trimestrielles suivantes au premier trimestre :

\begin{equation*}
3 ; 4 ; 5 ; 7 ; 7 ; 10 ; 10 ; 10 ; 10 ; 10 ; 11 ;  11 ; 12 ; 12 ; 12 ; 12 ; 12 ; 13 ; 13 ; 13 ; 14 ; 15 ; 16 ; 18
\end{equation*}

La moyenne trimestrielle de la classe s'obtient à partie des notes moyennes de chaque élève.

\begin{questions}
	\question[2] Déterminer la médiane $Me$, le premier quartile $Q_1$ et le troisième quartile $Q_3$ de cette série statistique de moyennes trimestrielles.
	\begin{solution}
		Il y a 24 valeurs dans la série, donc la médiane sera entre la 12$^e$ et la 13$^e$, le premier quartile la $6^e$ ($24 \times \num{0.25} = 6$) et le troisième quartile la $18^e$ ($24 \times \num{0.75} = 18$). On a donc :
		
		\begin{itemize}
			\item $Q_1 $ = 10 ;
			\item $Me$ = \num{11.5};
			\item $Q_3$ = 13.
		\end{itemize} 
	\end{solution}
	
	\question[1] Représenter le diagramme en boite correspondant en faisant apparaître les valeurs extrêmes.
	
	\question[1] Calculer la moyenne trimestrielle de la classe Jaune.
	\begin{solution}
		\begin{equation*}
			\bar{x} = \dfrac{3 + 4 + 5 + 2 \times 7 + 5 \times 10 + 2 \times 11 + 5 \times 12 + 3 \times 13 + 14 + 15 + 16 + 18 }{24} = \dfrac{260}{24} = \num{10.83}
		\end{equation*}
		
		La moyenne trimestrielle de la classe est \num{10.83}.
	\end{solution}
\end{questions}

\subsection{}

Les indicateurs de la classe Rouge permettant de de résumer la série statistique du premier trimestre sont les suivants :

\begin{itemize}
	\item Minimum = 3;
	\item premier quartile $Q'_1$ = 8;
	\item médiane $Me'$ = 10;
	\item troisième quartile $Q'_3$ = 12;
	\item Maximum = 17.
\end{itemize}

\begin{questions}
	\question[1] Représenter le diagramme en boite correspondant.
	
	\question[6] Parmi les informations suivantes, lesquelles sont vraies, fausses ou indécidables (Indécidable signifie que l'on ne peut pas conclure avec les éléments connus). Justifier votre réponse dans chacun des cas.
	
	\begin{parts}
		\part[2] 50\% des élèves de la classe Rouge ont une note comprise entre 10 et 12.
		\begin{solution}
			Faux, 10 est la médiane et 12 le troisième quartile donc 25 \% des élèves de la classe ont une note comprise entre 10 et 12.
		\end{solution}
		\part[2] 75\% des élèves de la classe Rouge ont une note inférieure ou égale à 12.
		\begin{solution}
			Vrai, 12 est le troisième quartile.
		\end{solution}
		\part[2] Au moins 50\% des élèves de la classe Rouge ont une note inférieure ou égale à la note médiane de la série Jaune.
		\begin{solution}
			Vrai, a médiane de la classe rouge est inférieure à celle de la classe jaune.
		\end{solution}
	\end{parts}
\end{questions}