\section{Moyennes trimestrielles (12 points)}

Dans un lycée, on étudie les moyennes trimestrielles du premier trimestre de deux classes appelées respectivement Jaune et Rouge.

\subsection{}
Les 25 élèves de la classe Jaune ont obtenu les moyennes trimestrielles suivantes au premier trimestre :

\begin{equation*}
3 ; 4 ; 5 ; 7 ; 7 ; 10 ; 10 ; 10 ; 10 ; 10 ; 11 ;  11 ; 12 ; 12 ; 12 ; 12 ; 12 ; 13 ; 13 ; 13 ; 14 ; 15 ; 16 ; 18
\end{equation*}

La moyenne trimestrielle de la classe s'obtient à partie des notes moyennes de chaque élève.

\begin{questions}
	\question[2] Déterminer la médiane $Me$, le premier quartile $Q_1$ et le troisième quartile $Q_3$ de cette série statistique de moyennes trimestrielles.
	
	\question[1\half] Représenter le diagramme en boite correspondant en faisant apparaître les valeurs extrêmes.
	
	\question[1] Calculer la moyenne trimestrielle de la classe Jaune.
\end{questions}

\subsection{}

Les indicateurs de la classe Rouge permettant de de résumer la série statistique du premier trimestre sont les suivants :

\begin{itemize}
	\item Minimum = 3;
	\item premier quartile $Q'_1$ = 8;
	\item médiane $Me'$ = 10;
	\item troisième quartile $Q'_3$ = 12;
	\item Maximum = 17.
\end{itemize}

\begin{questions}
	\question[1\half] Représenter le diagramme en boite correspondant.
	
	\question[6] Parmi les informations suivantes, lesquelles sont vraies, fausses ou indécidables (Indécidable signifie que l'on ne peut pas conclure avec les éléments connus). Justifier votre réponse dans chacun des cas.
	
	\begin{parts}
		\part[2] 50\% des élèves de la classe Rouge ont une note comprise entre 10 et 12.
		\part[2] 75\% des élèves de la classe Rouge ont une note inférieure ou égale à 12.
		\part[2] Au moins 50\% des élèves de la classe Rouge ont une note inférieure ou égale à la note médiane de la série Jaune.
	\end{parts}
\end{questions}