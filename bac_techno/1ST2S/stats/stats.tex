\documentclass[12pt,a4paper]{article}

%\usepackage[left=1.5cm,right=1.5cm,top=1cm,bottom=2cm]{geometry}
\usepackage[in, plain]{fullpage}
\usepackage{array}
\usepackage{../../../pas-math}
\usepackage{../../../moncours}


%\usepackage{pas-cours}
%-------------------------------------------------------------------------------
%          -Packages nécessaires pour écrire en Français et en UTF8-
%-------------------------------------------------------------------------------
\usepackage[utf8]{inputenc}
\usepackage[frenchb]{babel}
\usepackage[T1]{fontenc}
\usepackage{lmodern}
%-------------------------------------------------------------------------------

%-------------------------------------------------------------------------------
%                          -Outils de mise en forme-
%-------------------------------------------------------------------------------
\usepackage{hyperref}
\hypersetup{pdfstartview=XYZ}
\usepackage{enumerate}
\usepackage{graphicx}
\usepackage{multicol}

\usepackage{anysize} %%pour pouvoir mettre les marges qu'on veut
%\marginsize{2.5cm}{2.5cm}{2.5cm}{2.5cm}

\usepackage{indentfirst} %%pour que les premier paragraphes soient aussi indentés
%-------------------------------------------------------------------------------


%-------------------------------------------------------------------------------
%                  -Nécessaires pour écrire des mathématiques-
%-------------------------------------------------------------------------------
\usepackage{amsfonts}
\usepackage{amssymb}
\usepackage{amsmath}
\usepackage{amsthm}
\usepackage{tikz}
%-------------------------------------------------------------------------------

%-------------------------------------------------------------------------------
%                     -Mise en forme d'exercices-
%-------------------------------------------------------------------------------
\newtheoremstyle{exostyle}
{\topsep}% espace avant
{\topsep}% espace apres
{}% Police utilisee par le style de thm
{}% Indentation (vide = aucune, \parindent = indentation paragraphe)
{\bfseries}% Police du titre de thm
{.}% Signe de ponctuation apres le titre du thm
{ }% Espace apres le titre du thm (\newline = linebreak)
{\thmname{#1}\thmnumber{ #2}\thmnote{. \normalfont{\textit{#3}}}}% composants du titre du thm : \thmname = nom du thm, \thmnumber = numéro du thm, \thmnote = sous-titre du thm

\theoremstyle{exostyle}
\newtheorem{exercice}{Exercice}

\newenvironment{questions}{
\begin{enumerate}[\hspace{12pt}\bfseries\itshape a.]}{\end{enumerate}
} %mettre un 1 à la place du a si on veut des numéros au lieu de lettres pour les questions 
%-------------------------------------------------------------------------------



%-------------------------------------------------------------------------------
%                    - Racourcis d'écriture -
%-------------------------------------------------------------------------------

% Angles orientés (couples de vecteurs)
\newcommand{\aopp}[2]{(\vec{#1}, \vec{#2})} %Les deuc vecteurs sont positifs
\newcommand{\aopn}[2]{(\vec{#1}, -\vec{#2})} %Le second vecteur est négatif
\newcommand{\aonp}[2]{(-\vec{#1}, \vec{#2})} %Le premier vecteur est négatif
\newcommand{\aonn}[2]{(-\vec{#1}, -\vec{#2})} %Les deux vecteurs sont négatifs

%Ensembles mathématiques
\newcommand{\naturels}{\mathbb{N}} %Nombres naturels
\newcommand{\relatifs}{\mathbb{Z}} %Nombres relatifs
\newcommand{\rationnels}{\mathbb{Q}} %Nombres rationnels
\newcommand{\reels}{\mathbb{R}} %Nombres réels
\newcommand{\complexes}{\mathbb{C}} %Nombres complexes
%-------------------------------------------------------------------------------




%\makeatletter
%\renewcommand*{\@seccntformat}[1]{\csname the#1\endcsname\hspace{0.1cm}}
%\makeatother


%\author{Olivier FINOT}
\date{}
\title{Information chiffrée }

%\newcommand{\disp}{false}

\lhead{CH3 : Statistiques}
\rhead{O. FINOT}
%
%\rfoot{Page \thepage}
\begin{document}
%\maketitle

\chap[num=3, color=red]{Statistiques}{Olivier FINOT, \today }

\begin{myobj}
	\begin{itemize}
		\item Reconnaître un segment, une demie-droite, une droite et savoir les tracer;
		\item Tracer avec l’équerre la droite perpendiculaire à une droite donnée passant par un point donné;
		\item Tracer avec la règle et l’équerre la droite parallèle à une droite donnée passant par un point donné;
		\item Déterminer la distance entre deux points, entre un point et une droite;
		\item Savoir coder et lire une figure.
	\end{itemize}
\end{myobj}

\begin{mycomp}
	\begin{itemize}
		\item \kw{Modéliser} 
		\item \kw{Représenter} 
		\item \kw{Raisonner} 
		\item \kw{Communiquer}
		
	\end{itemize}
\end{mycomp}

\section{Vocabulaire et représentations graphiques}

\subsection{Vocabulaire}

\begin{mydefs}
	Une \kw{population} est un ensemble de personnes ou d'objets, appelés \kw{individus}, définis par une propriété commune. 
	Pour une population choisie, on peut étudier un caractère de ses individus, il est :
	
	\begin{itemize}
		\item \kw{quantitatif} quand il est mesurable :
		\begin{itemize}
			\item \kw{discret} si les valeurs sont des nombres isolés ;
			\item \kw{continu} si les valeurs ne sont pas isolées. Les valeurs sont regroupées en \kw{classes} ou \kw{intervalles}  $\intervFO{a}{b}$ %; l'\kw{amplitude} de l'intervalle est $b -a$.
		\end{itemize}
		\item \kw{qualitatif} quand il n'est pas mesurable. %, les valeurs s'appellent alors $\ll$ modalités $\gg$ .
	\end{itemize}

	L'\kw{effectif} $n_i$ est le nombre d'individus correspondant à une valeur du caractère. L'\kw{effectif total} $N$ est le nombre total d'individus de la population étudiée.
	Pour chaque valeur du caractère la \kw{fréquence} $f_i$ est calculée en divisant l'effectif correspondant à la valeur par l'effectif total  ($\frac{n_i}{N}$).
\end{mydefs}

\subsection{Représentation graphique}

\begin{mybilan}
	\begin{itemize}
		\item Le \kw{diagramme en secteurs (ou circulaire)} est une représentation adaptée une série à \kw{caractère qualitatif}.
		\item Le \kw{diagramme en bâtons (ou en barres)} est une représentation adaptée pour une série à \kw{caractère quantitatif discret}.
		\item L'\kw{histogramme} est utilisé pour représenter les séries à \kw{caractère quantitatif continu}.
	\end{itemize}
\end{mybilan}

\section{Indicateurs de tendance centrale}

\subsection{Moyenne}

\subsubsection*{Activité 1 page 76}

\begin{enumerate}[label=\arabic*°]
	\item Calcul de la distance moyenne à la piscine pour cet ensemble de neuf lycées :
	
	\begin{eqnarray*}
		\bar{x} & = & \dfrac{\num{1.8} + \num{1.0} + \num{20.2} + \num{0} + \num{0.6} + \num{0} + \num{0.8} + \num{2.6} + \num{0}}{9} \\
		\bar{x} & = & \frac{27}{9} \\
		\bar{x} & = & 3 \\
	\end{eqnarray*}

	La distance moyenne à la piscine pour ces neuf lycées est de 3 km, il faut donc les classer dans la troisième catégorie, distance supérieure à \num{2.5} km.
	
	\item Calcul de la distance moyenne à la piscine pour cet ensemble de neuf lycées en prenant en compte le nombre d'élèves :
	
	\begin{small}
		\begin{eqnarray*}
			\bar{x} & = & \dfrac{\num{930} \times \num{1.8} + \num{1130} \times \num{1.0} + \num{420} \times \num{20.2} + \num{1710} \times \num{0} + \num{1450} \times \num{0.6} + \num{1430} \times \num{0} + \num{1920} \times \num{0.8} + \num{530} \times \num{2.6} + \num{1250} \times \num{0}}{\num{930} + \num{1130} + \num{420}  + \num{1710} + \num{1450} + \num{1430} + \num{1920} + \num{530} + \num{1250} } \\
			\bar{x} & = & \frac{15072}{10770} \\
			\bar{x} & \approx & \num{1.4} \\
		\end{eqnarray*}
	\end{small}

	En tenant compte du nombre d'élèves de chaque lycée, on obtient une distance moyenne à la piscine d'environ \num{1.4} km.
	
	\item Pour estimer les frais supplémentaires créés par les déplacement entre les lycées et les piscines il faut tenir compte du nombre d'élèves donc la deuxième distance moyenne est la plus appropriée.
	
	  
\end{enumerate}
\end{document}