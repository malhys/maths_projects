
\section{Taux d'évolution et coefficient multiplicateur}

\begin{questions}
	\question[3] Donner le coefficient multiplicateur correspondant à l'évolution donnée :
	
		\begin{parts}
			\part Une hausse de 30 \% :
			\fillwithdottedlines{1.5cm}
			
			\part Une baisse de 15 \% :
			\fillwithdottedlines{1.5cm}
			
			\part Une hausse de 200 \% :
			\fillwithdottedlines{1.5cm}
		\end{parts}
	
	\question[3] Donner le pourcentage d'évolution correspondant au coefficient multiplicateur donné en précisant si c'est une hausse ou une baisse :
	
	\begin{parts}
		\part c=\num{0.89} :
		\fillwithdottedlines{1.5cm}
		
		\part c=\num{1.12} :
		\fillwithdottedlines{1.5cm}
		
		\part c=5 :
		\fillwithdottedlines{1.5cm}
	\end{parts}
\end{questions}

\newpage

\section{Taux d'évolution}
On s'intéresse à l'évolution d'une grandeur $y_1$ vers une grandeur $y_2$, $t$ est le taux d'évolution.

\`A chaque fois, calculer l'un des ces trois nombres en connaissant les deux autres.

\begin{questions}
	\question[4] 
	\begin{parts}
		
		\part $y_1 = \num{2.7} \quad ; \quad y_2=\num{2.9} $: 
		\fillwithdottedlines{2cm}
		 
		\part $y_1 = \num{3.5} \quad ; \quad y_2=\num{3.3} $:  
		\fillwithdottedlines{2cm}
		
		\part $y_1 = \num{4.5} \quad ; \quad t=\num{-0.20} $:   
		\fillwithdottedlines{2cm}		
		
		\part $y_2 = \num{1.03} \quad ; \quad t=\num{0.1} $:  
		\fillwithdottedlines{2cm}
				
		\end{parts}
\end{questions}