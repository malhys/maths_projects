\documentclass[xcolor={dvipsnames}]{beamer}
%\usepackage[utf8]{inputenc}
%\usetheme{Madrid}
\usetheme{Boadilla}
\usecolortheme{beaver}

\input{../../../../utils_maths_beamer}


%\usepackage{../../../../pas-math}
%\usepackage{../../../moncours_beamer}

\usepackage{amssymb,amsmath}



\graphicspath{{../img/}}

\title{Information chiffrée}
\author{O. FINOT}\institute{Lycée S$^t$ Vincent}


\AtBeginSection[]
{
	\begin{frame}
		\frametitle{}
		\tableofcontents[currentsection, hideallsubsections]
	\end{frame} 

}


\AtBeginSubsection[]
{
	\begin{frame}
		\frametitle{Sommaire}
		\tableofcontents[currentsection, currentsubsection]
	\end{frame} 
}

\begin{document}

\begin{frame}
  \titlepage 
\end{frame}


	

\begin{frame}{Objectifs}

\begin{block}{Être capable :}
	
	\begin{enumerate}
		\item de reconnaître des pourcentages d'évolution : augmentations et baisses successives;
		\item d'additionner et de comparer des pourcentages : pourcentages relatifs à un même ensemble, comparaison de deux pourcentages relatifs à deux ensembles de référence distincts;
		\item de déterminer et d'analyser des pourcentages de pourcentages;
		\item d'analyser des des variations d'un pourcentage;
		\item d'apprendre à distinguer les pourcentages décrivant le rapport d'une partie au tout des pourcentages d'évolution (augmentation ou baisse).
	\end{enumerate}
\end{block}
\end{frame}

\section{Effectifs et proportions (Activité : TP 1 page 8)}


\subsection{Expression d'une proportion à l'aide d'un pourcentage}

\begin{frame}{}

\begin{enumerate}%[label=\arabic*.]
	\item 
	\begin{enumerate} [a]
		\item Proportion des "pratiquants de roller" parmi les personnes interrogées:\pause
		\begin{itemize}
			\item Sous forme de fraction : $\dfrac{1192}{13685}$\pause
			\item Sous forme d'un nombre décimal arrondi à $10^{-4}$ : $\approx 0,0871$ ($10^{-4} = 0,0001=\frac{1}{10000}=\frac{1}{10^4}$)\pause
			\item Sous la forme d'un pourcentage arrondi à $10^{-2} \approx 8,71 \% $\pause
		\end{itemize}
		
		\vspace*{0.2cm}
		
		\begin{alertblock}{A retenir : Proportion}
			La \mykw{proportion ou fréquence} d'une partie $A$ d'une population $E$, est le rapport $p$ des effectifs de $A$ et de $E$ :
			\vspace*{-0.2cm}
			\begin{eqnarray*}
				p = \dfrac{n_A}{n_E} \; \left(\dfrac{Effectif de A}{Effectif de E}\right)
			\end{eqnarray*}
		\end{alertblock}
	
		\vspace*{0.2cm}
		
		\item Pourcentage de femmes parmi ces "pratiquants du roller" :\pause
		
		\begin{align*}
		\dfrac{657}{1192} \times 100 = 55,117, soit\; environ\; 55,12 \%
		\end{align*}
	\end{enumerate}
\end{enumerate}
\end{frame}




\begin{frame}{}

\begin{enumerate}%[label=\arabic*.]
	\setcounter{enumi}{1}
	\item 
	\begin{enumerate} [a]
		\item Nombre des 16-25 ans interrogés qui pratiquent le roller :\pause
		
		\begin{eqnarray*}
			\dfrac{521 \times 19}{100} = 521 \times 0,19 = 98,99
		\end{eqnarray*}
		
		Soit environ 99 "16-25 ans".\pause
		
		
		\item Soit $N$ le nombre des "12-24" ans interrogées. On a :\pause
		
		\begin{eqnarray*}
			N \times \dfrac{43,15}{100} &= 356 \\
			N &= \dfrac{356 \times 100}{43,15} = 825,02
		\end{eqnarray*}
		Soit environ 825 "12-24 ans".
	\end{enumerate}
\end{enumerate}
\end{frame}

\begin{frame}{}

\begin{enumerate}%[label=\arabic*.]
	\setcounter{enumi}{3}
	\item Pourcentage de "porteurs de casque" parmi les "pratiquants de roller" :\pause
	
	\begin{eqnarray*}
		657 \times 0,088 + 535 \times 0,144 = 134,856 = 135\; porteurs\; de\; casque. \pause
	\end{eqnarray*}
	
	\begin{eqnarray*}
		\dfrac{135}{1192}=0,11325 \approx 11,33 \%
	\end{eqnarray*}
	
\end{enumerate}

\end{frame}

%\begin{align}
%			N \times \dfrac{\num{4.48}}{100} &=& 745 \\
%			N &=& \dfrac{745 \times 100}{\num{4.48}} \\
%			N &=&  \myres{\num{16629.46}}
%\end{align}

%\begin{frame}
%
%
%%\begin{mybilan2}{Proportion}
%	La \kw{proportion ou fréquence} d'une partie $A$ d'une population $E$, est le rapport $p$ des effectifs de $A$ et de $E$ :
%	
%	\begin{eqnarray*}
%		p = \dfrac{n_A}{n_E} \; \left(\dfrac{Effectif de A}{Effectif de E}\right)
%	\end{eqnarray*}
%%\end{mybilan2}
%\end{frame}

\end{document}