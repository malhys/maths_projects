\documentclass[xcolor={dvipsnames}]{beamer}
%\usepackage[utf8]{inputenc}
%\usetheme{Madrid}
\usetheme{Malmoe}
%\usecolortheme{beaver}
\usecolortheme{rose}

%-------------------------------------------------------------------------------
%          -Packages nécessaires pour écrire en Français et en UTF8-
%-------------------------------------------------------------------------------
\usepackage[utf8]{inputenc}
\usepackage[frenchb]{babel}
\usepackage[T1]{fontenc}
\usepackage{lmodern}
\usepackage{textcomp}

%-------------------------------------------------------------------------------

%-------------------------------------------------------------------------------
%                          -Outils de mise en forme-
%-------------------------------------------------------------------------------
\usepackage{hyperref}
\hypersetup{pdfstartview=XYZ}
\usepackage{enumerate}
\usepackage{graphicx}
%\usepackage{multicol}
%\usepackage{tabularx}

%\usepackage{anysize} %%pour pouvoir mettre les marges qu'on veut
%\marginsize{2.5cm}{2.5cm}{2.5cm}{2.5cm}

\usepackage{indentfirst} %%pour que les premier paragraphes soient aussi indentés
\usepackage{verbatim}
%\usepackage[table]{xcolor}  
%\usepackage{multirow}
\usepackage{ulem}
%-------------------------------------------------------------------------------


%-------------------------------------------------------------------------------
%                  -Nécessaires pour écrire des mathématiques-
%-------------------------------------------------------------------------------
\usepackage{amsfonts}
\usepackage{amssymb}
\usepackage{amsmath}
\usepackage{amsthm}
\usepackage{tikz}
\usepackage{xlop}
\usepackage[output-decimal-marker={,}]{siunitx}
%-------------------------------------------------------------------------------


%-------------------------------------------------------------------------------
%                    - Mise en forme 
%-------------------------------------------------------------------------------

\newcommand{\bu}[1]{\underline{\textbf{#1}}}


\usepackage{ifthen}


\newcommand{\ifTrue}[2]{\ifthenelse{\equal{#1}{true}}{#2}{$\qquad \qquad$}}

\newcommand{\kword}[1]{\textcolor{red}{\underline{#1}}}


%-------------------------------------------------------------------------------



%-------------------------------------------------------------------------------
%                    - Racourcis d'écriture -
%-------------------------------------------------------------------------------

% Angles orientés (couples de vecteurs)
\newcommand{\aopp}[2]{(\vec{#1}, \vec{#2})} %Les deuc vecteurs sont positifs
\newcommand{\aopn}[2]{(\vec{#1}, -\vec{#2})} %Le second vecteur est négatif
\newcommand{\aonp}[2]{(-\vec{#1}, \vec{#2})} %Le premier vecteur est négatif
\newcommand{\aonn}[2]{(-\vec{#1}, -\vec{#2})} %Les deux vecteurs sont négatifs

%Ensembles mathématiques
\newcommand{\naturels}{\mathbb{N}} %Nombres naturels
\newcommand{\relatifs}{\mathbb{Z}} %Nombres relatifs
\newcommand{\rationnels}{\mathbb{Q}} %Nombres rationnels
\newcommand{\reels}{\mathbb{R}} %Nombres réels
\newcommand{\complexes}{\mathbb{C}} %Nombres complexes


%Intégration des parenthèses aux cosinus
\newcommand{\cosP}[1]{\cos\left(#1\right)}
\newcommand{\sinP}[1]{\sin\left(#1\right)}

%Fractions
\newcommand{\myfrac}[2]{{\LARGE $\frac{#1}{#2}$}}

%Vocabulaire courrant
\newcommand{\cad}{c'est-à-dire}

%Droites
\newcommand{\dte}[1]{droite $(#1)$}
\newcommand{\fig}[1]{figure $#1$}
\newcommand{\sym}{symétrique}
\newcommand{\syms}{symétriques}
\newcommand{\asym}{axe de symétrie}
\newcommand{\asyms}{axes de symétrie}
\newcommand{\seg}[1]{$[#1]$}
\newcommand{\monAngle}[1]{$\widehat{#1}$}
\newcommand{\bissec}{bissectrice}
\newcommand{\mediat}{médiatrice}
\newcommand{\ddte}[1]{$[#1)$}

%Figures
\newcommand{\para}{parallélogramme}
\newcommand{\paras}{parallélogrammes}
\newcommand{\myquad}{quadrilatère}
\newcommand{\myquads}{quadrilatères}
\newcommand{\co}{côtés opposés}
\newcommand{\diag}{diagonale}
\newcommand{\diags}{diagonales}
\newcommand{\supp}{supplémentaires}
\newcommand{\car}{carré}
\newcommand{\cars}{carrés}
\newcommand{\rect}{rectangle}
\newcommand{\rects}{rectangles}
\newcommand{\los}{losange}
\newcommand{\loss}{losanges}


%----------------------------------------------------

%\makeatletter
%\def\insertsectionheadnumber{\@Roman\c@section}
%\makeatother

%\usepackage{../../../../pas-math}
%\usepackage{../../../moncours_beamer}

\usepackage{amssymb,amsmath}



\graphicspath{{../img/}}

\title{Information chiffrée}
\author{O. FINOT}\institute{Lycée S$^t$ Vincent}



%\AtBeginSection[]
%{
%	\begin{frame}
%		\frametitle{}
%		\tableofcontents[currentsection, hideallsubsections]
%	\end{frame} 
%
%}
%
%
%\AtBeginSubsection[]
%{
%	\begin{frame}
%		\frametitle{}
%		\tableofcontents[currentsection, currentsubsection]
%	\end{frame} 
%}

\begin{document}

\begin{frame}
  \titlepage 
\end{frame}


	

\begin{frame}{Objectifs}

\begin{block}{Être capable :}
	
	\begin{enumerate}
		\item de reconnaître des pourcentages d'évolution : augmentations et baisses successives;
		\item d'additionner et de comparer des pourcentages : pourcentages relatifs à un même ensemble, comparaison de deux pourcentages relatifs à deux ensembles de référence distincts;
		\item de déterminer et d'analyser des pourcentages de pourcentages;
		\item d'analyser des des variations d'un pourcentage;
		\item d'apprendre à distinguer les pourcentages décrivant le rapport d'une partie au tout des pourcentages d'évolution (augmentation ou baisse).
	\end{enumerate}
\end{block}
\end{frame}

\section{Effectifs et proportions (Activité : TP 1 page 8)}



\subsection{Expression d'une proportion à l'aide d'un pourcentage}


\begin{frame}
\

\begin{Large}
	\textcolor{Red}{\underline{I. Effectifs et proportions (Activité : TP 1 page 8)}}
\end{Large}\pause
\

\vspace*{1cm}

\textcolor{Green}{\underline{1) Expression d'une proportion à l'aide d'un pourcentage}}
\end{frame}

\begin{frame}{}

\begin{enumerate}%[label=\arabic*.]
	\item 
	\begin{enumerate} [a]
		\item Proportion des "pratiquants de roller" parmi les personnes interrogées:\pause
		\begin{itemize}
			\item Sous forme de fraction : $\dfrac{1192}{13685}$\pause
			\item Sous forme d'un nombre décimal arrondi à $10^{-4}$ : $\approx 0,0871$ ($10^{-4} = 0,0001=\frac{1}{10000}=\frac{1}{10^4}$)\pause
			\item Sous la forme d'un pourcentage arrondi à $10^{-2} \approx 8,71 \% $\pause
		\end{itemize}
		
		\vspace*{0.2cm}
		
		\begin{alertblock}{A retenir : Proportion}
			La \mykw{proportion ou fréquence} d'une partie $A$ d'une population $E$, est le rapport $p$ des effectifs de $A$ et de $E$ :
			\vspace*{-0.2cm}
			\begin{eqnarray*}
				p = \dfrac{n_A}{n_E} \; \left(\dfrac{Effectif de A}{Effectif de E}\right)
			\end{eqnarray*}
		\end{alertblock}\pause
	
		\vspace*{0.2cm}
		
		\item Pourcentage de femmes parmi ces "pratiquants du roller" :\pause
		
		\begin{align*}
		\dfrac{657}{1192} \times 100 = 55,117, soit\; environ\; 55,12 \%
		\end{align*}
	\end{enumerate}
\end{enumerate}
\end{frame}




\begin{frame}{}

\begin{enumerate}%[label=\arabic*.]
	\setcounter{enumi}{1}
	\item 
	\begin{enumerate} [a]
		\item Nombre des 16-25 ans interrogés qui pratiquent le roller :\pause
		
		\begin{eqnarray*}
			\dfrac{521 \times 19}{100} = 521 \times 0,19 = 98,99
		\end{eqnarray*}
		
		Soit environ 99 "16-25 ans".\pause
		
		
		\item Soit $N$ le nombre des "12-24" ans interrogées. On a :\pause
		
		\begin{eqnarray*}
			N \times \dfrac{43,15}{100} &= 356 \\
			N &= \dfrac{356 \times 100}{43,15} = 825,02
		\end{eqnarray*}
		Soit environ 825 "12-24 ans".
	\end{enumerate}
\end{enumerate}
\end{frame}

\begin{frame}{}

\begin{enumerate}%[label=\arabic*.]
	\setcounter{enumi}{2}
	\item Pourcentage de "porteurs de casque" parmi les "pratiquants de roller" :\pause
	
	\begin{eqnarray*}
		657 \times 0,088 + 535 \times 0,144 = 134,856 = 135\; porteurs\; de\; casque. \pause
	\end{eqnarray*}
	
	\begin{eqnarray*}
		\dfrac{135}{1192}=0,11325 \approx 11,33 \%
	\end{eqnarray*}
	
\end{enumerate}

\end{frame}


\subsection{Comparaison de deux pourcentages, pourcentages de pourcentages}

\begin{frame}
\

%\begin{Large}
%	\textcolor{Red}{\underline{I. Effectifs et proportions (Activité : TP 1 page 8)}}
%\end{Large}\pause
%\
%
%\vspace*{1cm}

\textcolor{Green}{\underline{2) Comparaison de deux pourcentages, pourcentages de pourcentages}}
\end{frame}



\begin{frame}{}

\begin{enumerate}%[label=\arabic*.]
	\item 
	\begin{enumerate} [a]
		\item \begin{equation*}
		\dfrac{73}{149} \approx \num{0.4899},\; soit\; \num{48.99}\; \%.
		\end{equation*}
		Il y a \num{48.99} \% d'hommes parmi les victimes d'accidents de roller de "35 ans et plus".\pause
		
		\item \begin{equation*}
		\dfrac{343}{2075} \approx \num{0.1653},\; soit\; \num{16.53}\; \%.
		\end{equation*}
		\num{16.53} \% des victimes d'accidents de roller ont "9 ans et moins".\pause
		
		\item \begin{equation*}
		\dfrac{312}{745} \approx \num{0.4188},\; soit\; \num{41.88}\; \%.
		\end{equation*}
		Il y a \num{41.88} \% de "10 à 14 ans" parmi l'ensemble des femmes victimes d'un accident de roller.\pause
		
		\item \begin{equation*}
		\dfrac{1330}{2075} \approx \num{0.6410},\; soit\; \num{64.10}\; \%.
		\end{equation*}
		\num{64.10} \% des accidents de roller concernent des hommes.
	\end{enumerate}
\end{enumerate}
\end{frame}



\begin{frame}{}

\begin{enumerate}%[label=\arabic*.]
	\setcounter{enumi}{1}
	\item \begin{enumerate}[a]
		\item \begin{equation*}
		\dfrac{174}{1330} \approx \num{0.1308},\; soit \; \num{13.08}\; \%.
		\end{equation*}
		\num{13.08} \% des hommes victimes d'accidents de roller ont "de 20 à 34 ans".\pause
		
		\item \begin{equation*}
		\dfrac{127}{745} \approx \num{0.1705},\; soit \; \num{17.05}\; \%.
		\end{equation*}
		\num{17.05} \% des femmes victimes d'accidents de la route ont "de 20 à 34 ans".\pause
		
		\item Dans les effectifs, il y a plus d'hommes que de femmes de 20 à 34 ans, mais en pourcentage il y a plus de femmes. Il y a moins de femmes pratiquantes du roller que d'hommes mais en proportion elles ont plus d'accidents.
	\end{enumerate}
\end{enumerate}
\end{frame}


\begin{frame}{}

\begin{enumerate}%[label=\arabic*.]
	\setcounter{enumi}{2}
	\item \begin{enumerate}[a]
		\item Proportion $p_1$ de femmes parmi les accidentés de "15 à 19 ans" :
		\begin{equation*}
		p_1=\dfrac{47}{276}\approx \num{0.1703}, \; soit \; \num{17.03}\; \%.
		\end{equation*}\pause
		
		\item Proportion $p_2$ des "15 à 19 ans" parmi les accidentés :
		\begin{equation*}
		p_2=\dfrac{276}{2075} \approx \num{0.1330}, \; soit \; \num{13.30}\; \%.
		\end{equation*}\pause
		
		\item Proportion $p_3$ des femmes de "15 à 19 ans" parmi les accidentés :
		\begin{equation*}
		p_3 = \dfrac{47}{2075} \approx \num{0.0227}, \; soit \; \num{2.27} \; \%.
		\end{equation*}\pause
		
	
	
	
	\end{enumerate}
\end{enumerate}
\end{frame}

\begin{frame}{}
		\begin{block}{Remarque :}
		
		\begin{equation*}
		\dfrac{47}{276}  \times \dfrac{276}{2075} = \dfrac{47}{2075}, donc \; p_1 \times p_2 = p_3.
		\end{equation*}
		
		
		On peut faire : \num{17.03} \% de \num{13.30} \%
		\begin{equation*}
		\dfrac{\num{17.03}}{100} \times \dfrac{\num{13.30}}{100} = \num{0.022649}, \; soit \; environ \; \num{2.26}\; \%.
		\end{equation*}	
	\end{block}
\end{frame}

\subsection{Additionner et comparer des pourcentages}

\begin{frame}
\

%\begin{Large}
%	\textcolor{Red}{\underline{I. Effectifs et proportions (Activité : TP 1 page 8)}}
%\end{Large}\pause
%\
%
%\vspace*{1cm}

\textcolor{Green}{\underline{3) Additionner et comparer des pourcentages}}
\end{frame}

\begin{frame}{}
\begin{enumerate}
	\item  Pourcentage d'enfants en surpoids dans les zones rurales :\pause
	\begin{equation*}
	\num{100} - \num{87.2} = \num{12.8} \quad soit \; \num{12.8} \%.
	\end{equation*}\pause
	
	 
	
	\item  Pourcentage d'enfants obèses :\pause
	\begin{equation*}
	\num{12.8} - \num{9.2} = \num{3.6} \quad soit \; \num{3.6} \%.
	\end{equation*}\pause
	
	\item \begin{enumerate}[a]
		\item Dans l'agglomération parisienne, il y a \num{5} \% d'enfants obèses et \num{16.6} \% en surpoids; la proportion d'enfants obèses parmi ceux en surpoids est donc égale à $\dfrac{5}{\num{16.6}}=\num{0.301} \%,$ soit environ un peu plus de 3 enfants souffrant d'obésité pour 10 en surpoids. L'affirmation est donc juste.\pause
		
		\item Les effectifs pour les différents types d'agglomération ne sont pas connus. On ne peut donc rien affirmer concernant le nombre d'enfants en surpoids.
	\end{enumerate}
	
	
\end{enumerate}
\end{frame}

\section{Pourcentage d'évolution, coefficient multiplicateur}

\subsection{Variation relative (taux d'évolution)}

\begin{frame}
\

\begin{Large}
	\textcolor{Red}{\underline{II.Pourcentage d'évolution, coefficient multiplicateur (TP 2 page 10)}}
\end{Large}\pause
\

\vspace*{1cm}

\textcolor{Green}{\underline{1) Variation relative (taux d'évolution)}}
\end{frame}

\begin{frame}
\begin{enumerate}
	\item 
	\begin{enumerate} [a]
		\item Variation absolue du nombre de médecins généralistes en France entre 1990 et 2009 :\pause
		
		\begin{equation*}
		\num{101667} - \num{93380} = \num{8287}
		\end{equation*}
		$\rightarrow$  Soit une hausse de \num{8287} médecins.\pause
		
		
		\item Variation relative (ou taux d'évolution) du nombre de généralistes entre \num{1990} et \num{2009} :\pause
		
		\begin{table}[h!]
			\centering
			
			
			\begin{tabular}{|ccc|}
				\hline
				en \num{1990} &    & en \num{2009} \\
				\num{93380} médecins& $\rightarrow$ & \num{101667} médecins \\
				\hline
			\end{tabular}
		\end{table}
		
		\begin{equation*}
		\dfrac{(\num{101667} - \num{93380})}{\num{93380}} \times 100 = \num{8.874}...
		\end{equation*}
		$\rightarrow$ Soit une hausse d'environ \num{8.87} \%.\pause
		
		\item Entre 1990 et 2009 le nombre de médecins généralistes en France à augmenté de \num{8.87} \%.
	\end{enumerate}
\end{enumerate}
\end{frame}

\begin{frame}
			\begin{block}{Remarque}
			\begin{eqnarray*}
				\num{1.0887} & = & 1 + \num{0.0887} \\
				& = & 1 + \dfrac{\num{8.87}}{100}
			\end{eqnarray*}
%			
			Ainsi pour augmenter une grandeur de \num{8.87} \% il suffit de multiplier cette grandeur par $1 + \dfrac{8.87}{100}$ soit \num{1.0887}. Ce nombre s'appelle le \mykw{coefficient multiplicateur} associé à une augmentation de \num{8.87} \%.
		\end{block}
\end{frame}


\begin{frame}{}

\begin{enumerate}%[label=\arabic*.]
	\setcounter{enumi}{1}
	\item \begin{enumerate}[a]
		\item Variation absolue du nombre de médecins généralistes :\pause
		\begin{equation*}
		\num{99670} - \num{101667} = \num{-1997}
		\end{equation*}
		Soit une baisse de \num{1997} médecins.\pause
		
		\vspace*{0.25cm}
		\item Taux d'évolution correspondant :\pause
		\begin{equation*}
		\dfrac{(\num{99670} - \num{101667})}{\num{101667}}\times 100 \approx -\num{-1.96}
		\end{equation*}
		
		Soit une baisse d'environ \num{-1.96} \%. \pause
		\vspace*{0.25cm}
		\item Entre \num{2009} et \num{2015}, le nombre de médecins généralistes en France devrait baisser d'environ \num{1.96} \%.
		
		
		
		
	\end{enumerate}
\end{enumerate}
\end{frame}


\begin{frame}
\begin{block}{Remarque}
	\begin{center}
		\begin{tabular}{|ccc|}
		\hline
		en \num{2009} & - \num{1.96} \%  & en \num{2015} \\
		& {\LARGE $\rightarrow$} &			\\
		\num{101667} médecins& $\times \num{0.9804}$ & \num{99670} médecins \\
		\hline
	\end{tabular}
	\end{center}
	
	%\vspace*{0.5cm} 
	On a : $\dfrac{99670}{101667} \approx \num{0.9804}$.
	Et $1 - \dfrac{\num{1.96}}{\num{100}} = \num{0.9804}$
	
	\vspace*{0.5cm} 
	Pour diminuer une grandeur de \num{1.96} \%, il suffit de multiplier cette grandeur par $1 - \frac{\num{1.96}}{\num{100}}$, soit \num{0.9804}. \num{0.9804} est le \mykw{coefficient multiplicateur} associé à une baisse de \num{1.96} \%.
\end{block}
\end{frame}

\begin{frame}

	\begin{alertblock}{\`A retenir : Taux d'évolution et coefficient multiplicateur}
		Le taux d'évolution $t$ (ou variation relative) d'une quantité passant de la valeur $y_1$ à une valeur $y_2$ est égal à :
		\begin{equation*}
		t = \dfrac{y_2 - y_1}{y_1} \left(\dfrac{V_{arrivée} - V_{départ}}{V_{départ}}\right)
		\end{equation*}
		
		\underline{Remarque} : Un taux d'évolution positif traduit une hausse, un taux d'évolution négatif traduit une baisse.\\
		
		
		
		
	\end{alertblock}

\end{frame}

\begin{frame}

\begin{alertblock}{\`A retenir : Taux d'évolution et coefficient multiplicateur (suite)}
	\underline{Coefficients multiplicateurs :} 
	\begin{itemize}

		\item \mykw{Augmenter} une grandeur de $t \%$ revient à multiplier cette grandeur par $\left(1 + \dfrac{t}{100}\right)$.
		
		\item \underline{Exemple :} $+ 5 \% = \times \num{1.05}$ ; $+ 20 \% = \times \num{1.20}$ \\
		
		\item \mykw{Diminuer} une grandeur de $t \%$ revient à multiplier cette grandeur par $\qquad \left(1 - \dfrac{t}{100}\right)$.
		\item \underline{Exemple :} $- 12 \% = \times \num{0.88}$ ; $- 3 \% = \times \num{0.97}$ \\
		
		\item Dans le cas d'une \mykw{hausse}, le coefficient multiplicateur est \mykw{supérieur à 1}.
		
		\item Dans le cas d'une \mykw{baisse}, le coefficient multiplicateur est \mykw{inférieur à 1}.
	\end{itemize}
	
	
	
\end{alertblock}

\end{frame}

\begin{frame}{}

\begin{enumerate}%[label=\arabic*.]
	\setcounter{enumi}{2}
	\item Nombre de médecins des spécialités médicales en 2009 :\pause
	
	\begin{table}[h!]
		\centering
		\begin{tabular}{|ccc|}
			\hline
			en \num{1990} & + \num{23.63} \%  & en \num{2009} \\
			& {\LARGE $\rightarrow$} &			\\
			\num{48040} médecins& $ $ & ? médecins \\
			\hline
		\end{tabular}
	\end{table}\pause
	
	D'où : $\num{48040} \times \num{1.2363} = \num{59391.8}...$, soit environ \num{52392} médecins.\pause
	
	\item Nombre de médecins des spécialités chirurgicales en 2015 :\pause
	
	\begin{table}[h!]
		\centering
		\begin{tabular}{|ccc|}
			\hline
			en \num{2009} & - \num{8.22} \%  & en \num{2015} \\
			& {\LARGE $\rightarrow$} &			\\
			\num{25163} médecins& $ $ & ? médecins \\
			\hline
		\end{tabular}
	\end{table}\pause
	
	D'où : $\num{25163} \times \num{0.9178} = \num{23094.60}...$, soit environ \num{23095} médecins.
\end{enumerate}
\end{frame}


\begin{frame}{}

\begin{enumerate}%[label=\arabic*.]
	\setcounter{enumi}{4}
	\item  Nombre de médecins des spécialités chirurgicales en 1990 : \pause
	
	\begin{table}[h!]
		\centering
		\begin{tabular}{|ccc|}
			\hline
			en \num{1990} & + \num{17.21} \%  & en \num{2009} \\
			& {\LARGE $\rightarrow$} &			\\
			? médecins& $ $ & 25163 médecins \\
			\hline
			& {\LARGE $\leftarrow$} & \\
			& $ $ & \\
			\hline
		\end{tabular}
	\end{table}\pause
	
	D'où : $\num{25163} \div \num{1.1721} = \num{21468.30475}...$, soit environ \num{21468} médecins.
\end{enumerate}
\end{frame}

\subsection{\'Evolutions successives, évolution réciproque}


\begin{frame}
\

\textcolor{Green}{\underline{2) \'Evolutions successives, évolution réciproque)}}\\
 
 \vspace*{1cm}
 
A. \'Evolutions successives
\end{frame}

\begin{frame}{}

	\begin{enumerate}[1)]
		\item \
		
			\setlength{\tabcolsep}{4pt}
			\begin{table}[h!]
				\centering
				\begin{tabular}{|ccc|c|}
					\hline
					$P_1$ & + \num{25} \%  & $P_2$ &\\
					& {\LARGE $\rightarrow$} &	&	$P_2 = \num{16} \times \num{1.25} = 20$, soit 20 \$ .	\\
					\num{16} \$ & $\times \num{1.25}$ & ? \$ & \\
					\hline
				\end{tabular}
				
			\end{table}
		
		\pause
		
		\item \ 
		
		\begin{table}[h!]
			\centering
			\begin{tabular}{|ccc|c|}
				\hline
				$P_2$ & + \num{30} \%  & $P_3$ &\\
				& {\LARGE$\rightarrow$} &	&	$P_2 = \num{20} \times \num{1.30} = 26$, soit 26 \$ .	\\
				\num{20} \$ & $\times \num{1.30}$ & ? \$ & \\
				\hline
			\end{tabular}
			
		\end{table} \pause
		
				
		\item \ 
		\begin{table}[h!]
			\centering
			\begin{tabular}{|ccc|c|}
				\hline
				$P_1$ & + ... \%  & $P_3$ & \\
				& {\LARGE$\rightarrow$} &	&	 $\dfrac{\num{26} - \num{16}}{\num{16}} = \num{0.625}$	\\
				\num{16} \$ & $\times ...$ & 26 \$ & Soit une hausse globale de \num{62.5} \% \\
				\hline
			\end{tabular}
			
		\end{table}
		
		
	\end{enumerate}

\end{frame}


\begin{frame}{}

		\underline{Calcul du coefficient multiplicateur :}
		\begin{equation*}
		k = \dfrac{26}{16} = \num{1.625}
		\end{equation*}\pause
				
		On peut aussi calculer indépendamment des prix : $\num{1.25} \times \num{1.30} = \num{1.625}$, soit une hausse globale de \num{62.5} \%. \pause
		
		\begin{block}{Remarque}
			Le pourcentage de hausse globale \num{62.5} \% n'est pas égal à la somme des deux pourcentages de hausse successives \num{25} \% et \num{30} \%, car ces deux pourcentages ne s'appliquent pas sur le même prix, donc ne s'additionnent pas.
		\end{block}\pause
		
		\begin{alertblock}{\`A retenir : \'Evolutions successives}
			Deux évolutions (hausse ou baisse) successives de coefficients multiplicateurs $c$ et $c'$  correspondent  une évolution globale (hausse ou baisse) de $c \times c'$ (on multiplie).
		\end{alertblock}
	


\end{frame}


\begin{frame}
\

B. \'Evolution réciproque \pause
%\end{frame}


%\begin{frame}
	\begin{enumerate}[1.]
		\item 
		\begin{enumerate}[a.]
			\item \ 
			\begin{table}[h!]
				\centering{\ }
				\begin{tabular}{|@{\ \ }c@{\ \ }c@{\ \ }c@{\ \ }c@{\ \ }c@{\ \ }|}
					\hline
					$P_1$ & +\num{25} \%  & $P_2$ & -\num{25} \%  & $P'_3$ \\
					& {\LARGE$\rightarrow$} &	&	 {\LARGE$\rightarrow$} &	\\
					\num{16} \$ & $\times \num{1.25} $ & 20 \$ &  $\times \num{0.75}$ & \num{15} \$ \\
					\hline
				\end{tabular}
				
			\end{table}\pause
			
			\item On constate que la baisse de \num{25} \% n'annule pas la hausse de \num{25} \%.\pause
		\end{enumerate}
		
		\begin{block}{Remarque}
			\begin{eqnarray*}
				P'_3 &=& \num{16} \times \num{1.25} \times \num{0.75} \\
				P'_3 &=& \num{16} \times \num{0.9375} \\
				On\;a\; & &\num{0.9375} \neq 1
			\end{eqnarray*}
		\end{block}
		

	\end{enumerate}
\end{frame}


\begin{frame}{}

\begin{enumerate}[1.]
	\setcounter{enumi}{1}
	\item 
		\begin{enumerate}[a.]
			\item \begin{eqnarray*}
				t &=& \frac{16 - 20}{20}\\
				t &=& \num{-0.2}
			\end{eqnarray*} 
			
			Une baisse de 20 \% annule l'effet d'une hausse de 50 \%.\pause 
		\end{enumerate}
		
	
		
		
\end{enumerate}

\begin{block}{Remarque}
	\begin{equation*}
		\num{1.25} \times \num{0.8} = 1
	\end{equation*}\pause
\end{block}



\end{frame}




\begin{frame}{}

\begin{enumerate}[a.]
	\setcounter{enumi}{1}
	\item \
	
	\begin{table}[h!]
		\centering{\ }
		\begin{tabular}{|@{\ \ }c@{\ \ }c@{\ \ }c@{\ \ }c@{\ \ }c@{\ \ }|}
			\hline
			$P'_1$ & +\num{25} \%  & $P'_2$ & -\num{20} \%  & $P'_3$ \\
			& {\LARGE$\rightarrow$} &	&	 {\LARGE$\rightarrow$} &	\\
			\num{40} \$ & $\times \num{1.25} $ & 50 \$ &  $\times \num{0.8}$ & \num{40} \$ \\
			\hline
		\end{tabular}
		
	\end{table}

Oui une baisse de 20 \% compense une augmentation de 25 \% pour un prix de départ de 40 \$.
	
	
\end{enumerate}


\begin{alertblock}{\`A retenir : \'Evolution réciproque}
	Deux évolutions (hausse et baisse) successives sont réciproques si et seulement si leur \mykw{coefficients multiplicateurs $c$ et $c'$ sont inverses} : $c \times c' = 1$
\end{alertblock}

\end{frame}

\begin{frame}
\begin{enumerate}[1.]
	\setcounter{enumi}{3}
	\item 
	
	On recherche le coefficient multiplicateur $c$ qui annule l'augmentation de 50 \% :
	
	\begin{eqnarray*}
		\num{1.5} \times c &=& 1 \\
		c &=& \frac{1}{\num{1.5}} \\
		c &\approx& \num{0.6667}
	\end{eqnarray*}  
	
	Une baisse de \num{66.67} \% compense une hausse de 50 \%. 
	
\end{enumerate}
\end{frame}

\end{document}


\begin{frame}{}

\begin{enumerate}%[label=\arabic*.]
	\setcounter{enumi}{1}
	\item \begin{enumerate}[a]
		\item \pause
		
		\item \pause
		
		\item \pause
		
		
		
		
	\end{enumerate}
\end{enumerate}
\end{frame}