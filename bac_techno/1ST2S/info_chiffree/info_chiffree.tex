\documentclass[12pt,a4paper]{article}

%\usepackage[left=1.5cm,right=1.5cm,top=1cm,bottom=2cm]{geometry}
\usepackage[in, plain]{fullpage}
\usepackage{array}
\usepackage{../../../pas-math}
\usepackage{../../../moncours}


%\usepackage{pas-cours}
%-------------------------------------------------------------------------------
%          -Packages nécessaires pour écrire en Français et en UTF8-
%-------------------------------------------------------------------------------
\usepackage[utf8]{inputenc}
\usepackage[frenchb]{babel}
\usepackage[T1]{fontenc}
\usepackage{lmodern}
%-------------------------------------------------------------------------------

%-------------------------------------------------------------------------------
%                          -Outils de mise en forme-
%-------------------------------------------------------------------------------
\usepackage{hyperref}
\hypersetup{pdfstartview=XYZ}
\usepackage{enumerate}
\usepackage{graphicx}
\usepackage{multicol}

\usepackage{anysize} %%pour pouvoir mettre les marges qu'on veut
%\marginsize{2.5cm}{2.5cm}{2.5cm}{2.5cm}

\usepackage{indentfirst} %%pour que les premier paragraphes soient aussi indentés
%-------------------------------------------------------------------------------


%-------------------------------------------------------------------------------
%                  -Nécessaires pour écrire des mathématiques-
%-------------------------------------------------------------------------------
\usepackage{amsfonts}
\usepackage{amssymb}
\usepackage{amsmath}
\usepackage{amsthm}
\usepackage{tikz}
%-------------------------------------------------------------------------------

%-------------------------------------------------------------------------------
%                     -Mise en forme d'exercices-
%-------------------------------------------------------------------------------
\newtheoremstyle{exostyle}
{\topsep}% espace avant
{\topsep}% espace apres
{}% Police utilisee par le style de thm
{}% Indentation (vide = aucune, \parindent = indentation paragraphe)
{\bfseries}% Police du titre de thm
{.}% Signe de ponctuation apres le titre du thm
{ }% Espace apres le titre du thm (\newline = linebreak)
{\thmname{#1}\thmnumber{ #2}\thmnote{. \normalfont{\textit{#3}}}}% composants du titre du thm : \thmname = nom du thm, \thmnumber = numéro du thm, \thmnote = sous-titre du thm

\theoremstyle{exostyle}
\newtheorem{exercice}{Exercice}

\newenvironment{questions}{
\begin{enumerate}[\hspace{12pt}\bfseries\itshape a.]}{\end{enumerate}
} %mettre un 1 à la place du a si on veut des numéros au lieu de lettres pour les questions 
%-------------------------------------------------------------------------------



%-------------------------------------------------------------------------------
%                    - Racourcis d'écriture -
%-------------------------------------------------------------------------------

% Angles orientés (couples de vecteurs)
\newcommand{\aopp}[2]{(\vec{#1}, \vec{#2})} %Les deuc vecteurs sont positifs
\newcommand{\aopn}[2]{(\vec{#1}, -\vec{#2})} %Le second vecteur est négatif
\newcommand{\aonp}[2]{(-\vec{#1}, \vec{#2})} %Le premier vecteur est négatif
\newcommand{\aonn}[2]{(-\vec{#1}, -\vec{#2})} %Les deux vecteurs sont négatifs

%Ensembles mathématiques
\newcommand{\naturels}{\mathbb{N}} %Nombres naturels
\newcommand{\relatifs}{\mathbb{Z}} %Nombres relatifs
\newcommand{\rationnels}{\mathbb{Q}} %Nombres rationnels
\newcommand{\reels}{\mathbb{R}} %Nombres réels
\newcommand{\complexes}{\mathbb{C}} %Nombres complexes
%-------------------------------------------------------------------------------




%\makeatletter
%\renewcommand*{\@seccntformat}[1]{\csname the#1\endcsname\hspace{0.1cm}}
%\makeatother


%\author{Olivier FINOT}
\date{}
\title{Information chiffrée }

%\newcommand{\disp}{false}

\lhead{CH1 : Info chiffrée}
\rhead{O. FINOT}
%
%\rfoot{Page \thepage}
\begin{document}
%\maketitle

\chap[num=1, color=red]{Information chiffrée}{Olivier FINOT, \today }

\begin{myobj}
	\begin{itemize}
		\item Reconnaître un segment, une demie-droite, une droite et savoir les tracer;
		\item Tracer avec l’équerre la droite perpendiculaire à une droite donnée passant par un point donné;
		\item Tracer avec la règle et l’équerre la droite parallèle à une droite donnée passant par un point donné;
		\item Déterminer la distance entre deux points, entre un point et une droite;
		\item Savoir coder et lire une figure.
	\end{itemize}
\end{myobj}

\begin{mycomp}
	\begin{itemize}
		\item \kw{Modéliser} 
		\item \kw{Représenter} 
		\item \kw{Raisonner} 
		\item \kw{Communiquer}
		
	\end{itemize}
\end{mycomp}

\section{Effectifs et proportions (Activité : TP 1 page 8)}


\subsection{Expression d'une proportion à l'aide d'un pourcentage}
\begin{enumerate}[label=\arabic*.]
	\item 
	\begin{enumerate} [label=\alph*) ]
		\item Proportion des "pratiquants de roller" parmi les personnes interrogées:
		\begin{itemize}
			\item Sous forme de fraction : $\dfrac{1192}{13685}$
			\item Sous forme d'un nombre décimal arrondi à $10^{-4}$ : $\approx 0,0871$ ($10^{-4} = 0,0001=\frac{1}{10000}=\frac{1}{10^4}$)
			\item Sous la forme d'un pourcentage arrondi à $10^{-2} \approx 8,71 \% $
		\end{itemize}
		
		\vspace*{1cm}
		
		\hspace*{-5cm} \begin{mybilan2}{Proportion}
			La \kw{proportion ou fréquence} d'une partie $A$ d'une population $E$, est le rapport $p$ des effectifs de $A$ et de $E$ :
			
			\begin{eqnarray*}
			p = \dfrac{n_A}{n_E} \; \left(\dfrac{Effectif de A}{Effectif de E}\right)
			\end{eqnarray*}
		\end{mybilan2}
				
		\item Pourcentage de femmes parmi ces "pratiquants du roller"
		
		\begin{eqnarray*}
			 \dfrac{657}{1192} \times 100 = 55,117, soit\; environ\; 55,12 \%
		\end{eqnarray*}
		%	\dfrac{657}{1192}  100 =  55,117, soit environ 55,12 
						

	\end{enumerate}
	
	\item \begin{enumerate}[label=\alph*) ]
		\item Nombre des 16-25 ans interrogés qui pratiquent le roller :
		
		\begin{eqnarray*}
			\dfrac{521 \times 19}{100} = 521 \times 0,19 = 98,99
		\end{eqnarray*}
		
		Soit environ 99 "16-25 ans"
		
		
		\item Soit $N$ le nombre des "12-24" ans interrogées. On a :
		
		\begin{eqnarray*}
			N \times \dfrac{43,15}{100} &= 356 \\
			N &= \dfrac{356 \times 100}{43,15} = 825,02
		\end{eqnarray*}
		Soit environ 825 "12-24 ans".
	\end{enumerate}
	
	\item Pourcentage de "porteurs de casque" parmi les "pratiquants de roller" :
	
	\begin{eqnarray*}
		657 \times 0,088 + 535 \times 0,144 = 134,856 = 135\; porteurs\; de\; casque
	\end{eqnarray*}
	
	\begin{eqnarray*}
		\dfrac{135}{1192}=0,11325 \approx 11,33 \%
	\end{eqnarray*}
\end{enumerate}



\begin{myexos}
	2, 3, 4 page 21
\end{myexos}

\subsection{Comparaison de deux pourcentages, pourcentages de pourcentages}
\end{document}