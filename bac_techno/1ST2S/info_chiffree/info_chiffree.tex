\documentclass[12pt,a4paper]{article}

%\usepackage[left=1.5cm,right=1.5cm,top=1cm,bottom=2cm]{geometry}
\usepackage[in, plain]{fullpage}
\usepackage{array}
\usepackage{../../../pas-math}
\usepackage{../../../moncours}


%\usepackage{pas-cours}
%-------------------------------------------------------------------------------
%          -Packages nécessaires pour écrire en Français et en UTF8-
%-------------------------------------------------------------------------------
\usepackage[utf8]{inputenc}
\usepackage[frenchb]{babel}
\usepackage[T1]{fontenc}
\usepackage{lmodern}
\usepackage{textcomp}



%-------------------------------------------------------------------------------

%-------------------------------------------------------------------------------
%                          -Outils de mise en forme-
%-------------------------------------------------------------------------------
\usepackage{hyperref}
\hypersetup{pdfstartview=XYZ}
%\usepackage{enumerate}
\usepackage{graphicx}
\usepackage{multicol}
\usepackage{tabularx}
\usepackage{multirow}


\usepackage{anysize} %%pour pouvoir mettre les marges qu'on veut
%\marginsize{2.5cm}{2.5cm}{2.5cm}{2.5cm}

\usepackage{indentfirst} %%pour que les premier paragraphes soient aussi indentés
\usepackage{verbatim}
\usepackage{enumitem}
\usepackage[usenames,dvipsnames,svgnames,table]{xcolor}

\usepackage{variations}

%-------------------------------------------------------------------------------


%-------------------------------------------------------------------------------
%                  -Nécessaires pour écrire des mathématiques-
%-------------------------------------------------------------------------------
\usepackage{amsfonts}
\usepackage{amssymb}
\usepackage{amsmath}
\usepackage{amsthm}
\usepackage{tikz}
\usepackage{xlop}
%-------------------------------------------------------------------------------



%-------------------------------------------------------------------------------


%-------------------------------------------------------------------------------
%                    - Mise en forme avancée
%-------------------------------------------------------------------------------

\usepackage{ifthen}
\usepackage{ifmtarg}


\newcommand{\ifTrue}[2]{\ifthenelse{\equal{#1}{true}}{#2}{$\qquad \qquad$}}

%-------------------------------------------------------------------------------

%-------------------------------------------------------------------------------
%                     -Mise en forme d'exercices-
%-------------------------------------------------------------------------------
%\newtheoremstyle{exostyle}
%{\topsep}% espace avant
%{\topsep}% espace apres
%{}% Police utilisee par le style de thm
%{}% Indentation (vide = aucune, \parindent = indentation paragraphe)
%{\bfseries}% Police du titre de thm
%{.}% Signe de ponctuation apres le titre du thm
%{ }% Espace apres le titre du thm (\newline = linebreak)
%{\thmname{#1}\thmnumber{ #2}\thmnote{. \normalfont{\textit{#3}}}}% composants du titre du thm : \thmname = nom du thm, \thmnumber = numéro du thm, \thmnote = sous-titre du thm

%\theoremstyle{exostyle}
%\newtheorem{exercice}{Exercice}
%
%\newenvironment{questions}{
%\begin{enumerate}[\hspace{12pt}\bfseries\itshape a.]}{\end{enumerate}
%} %mettre un 1 à la place du a si on veut des numéros au lieu de lettres pour les questions 
%-------------------------------------------------------------------------------

%-------------------------------------------------------------------------------
%                    - Mise en forme de tableaux -
%-------------------------------------------------------------------------------

\renewcommand{\arraystretch}{1.7}

\setlength{\tabcolsep}{1.2cm}

%-------------------------------------------------------------------------------



%-------------------------------------------------------------------------------
%                    - Racourcis d'écriture -
%-------------------------------------------------------------------------------

% Angles orientés (couples de vecteurs)
\newcommand{\aopp}[2]{(\vec{#1}, \vec{#2})} %Les deuc vecteurs sont positifs
\newcommand{\aopn}[2]{(\vec{#1}, -\vec{#2})} %Le second vecteur est négatif
\newcommand{\aonp}[2]{(-\vec{#1}, \vec{#2})} %Le premier vecteur est négatif
\newcommand{\aonn}[2]{(-\vec{#1}, -\vec{#2})} %Les deux vecteurs sont négatifs

%Ensembles mathématiques
\newcommand{\naturels}{\mathbb{N}} %Nombres naturels
\newcommand{\relatifs}{\mathbb{Z}} %Nombres relatifs
\newcommand{\rationnels}{\mathbb{Q}} %Nombres rationnels
\newcommand{\reels}{\mathbb{R}} %Nombres réels
\newcommand{\complexes}{\mathbb{C}} %Nombres complexes


%Intégration des parenthèses aux cosinus
\newcommand{\cosP}[1]{\cos\left(#1\right)}
\newcommand{\sinP}[1]{\sin\left(#1\right)}


%Probas stats
\newcommand{\stat}{statistique}
\newcommand{\stats}{statistiques}
%-------------------------------------------------------------------------------

%-------------------------------------------------------------------------------
%                    - Mise en page -
%-------------------------------------------------------------------------------

\newcommand{\twoCol}[1]{\begin{multicols}{2}#1\end{multicols}}


\setenumerate[1]{font=\bfseries,label=\textit{\alph*})}
\setenumerate[2]{font=\bfseries,label=\arabic*)}


%-------------------------------------------------------------------------------
%                    - Elements cours -
%-------------------------------------------------------------------------------





%\makeatletter
%\renewcommand*{\@seccntformat}[1]{\csname the#1\endcsname\hspace{0.1cm}}
%\makeatother


%\author{Olivier FINOT}
\date{}
\title{Information chiffrée }

%\newcommand{\disp}{false}

\lhead{CH1 : Info chiffrée}
\rhead{O. FINOT}
%
%\rfoot{Page \thepage}
\begin{document}
%\maketitle

\chap[num=1, color=red]{Information chiffrée}{Olivier FINOT, \today }

\begin{myobj}
	\begin{itemize}
		
		\item Construire le symétrique d’un point ou d'une figure par rapport à une droite à la main où à l’aide d’un logiciel;
		\item Construire le symétrique d’un point ou d'une figure par rapport à un point, à la main où à l’aide d’un logiciel;
		\item Utiliser les propriétés de la symétrie axiale ou centrale;
		\item Identifier des symétries dans des figures.		
	\end{itemize}
\end{myobj}

\begin{mycomp}
	\begin{itemize}
		\item \kw{Chercher (Ch2)} :  s’engager    dans    une    démarche    scientifique, observer, questionner, manipuler, expérimenter (sur une feuille de papier, avec des objets, à l’aide de logiciels), émettre des hypothèses, chercher des exemples ou des contre-exemples, simplifier ou particulariser une situation, émettre une conjecture ;
		\item \kw{Raisonner (Ra3)} :  démontrer : utiliser un raisonnement logique et des règles établies (propriétés, théorèmes, formules) pour parvenir à une conclusion ;
		\item \kw{Communiquer (Co2)} :  expliquer à l’oral ou à l’écrit (sa démarche, son raisonnement, un calcul, un protocole   de   construction   géométrique, un algorithme), comprendre les explications d’un autre et argumenter dans l’échange ; 
		
	\end{itemize}
\end{mycomp}




\section{Effectifs et proportions (Activité : TP 1 page 8)}


\subsection{Expression d'une proportion à l'aide d'un pourcentage}
\begin{enumerate}[label=\arabic*.]
	\item 
	\begin{enumerate} [label=\alph*) ]
		\item Proportion des "pratiquants de roller" parmi les personnes interrogées:
		\begin{itemize}
			\item Sous forme de fraction : $\dfrac{1192}{13685}$
			\item Sous forme d'un nombre décimal arrondi à $10^{-4}$ : $\approx 0,0871$ ($10^{-4} = 0,0001=\frac{1}{10000}=\frac{1}{10^4}$)
			\item Sous la forme d'un pourcentage arrondi à $10^{-2} \approx 8,71 \% $
		\end{itemize}
		
		\vspace*{1cm}
		
		\hspace*{-5cm} \begin{mybilan2}{Proportion}
			La \kw{proportion ou fréquence} d'une partie $A$ d'une population $E$, est le rapport $p$ des effectifs de $A$ et de $E$ :
			
			\begin{eqnarray*}
			p = \dfrac{n_A}{n_E} \; \left(\dfrac{Effectif de A}{Effectif de E}\right)
			\end{eqnarray*}
		\end{mybilan2}
				
		\item Pourcentage de femmes parmi ces "pratiquants du roller"
		
		\begin{eqnarray*}
			 \dfrac{657}{1192} \times 100 = 55,117, soit\; environ\; 55,12 \%
		\end{eqnarray*}
		%	\dfrac{657}{1192}  100 =  55,117, soit environ 55,12 
						

	\end{enumerate}
	
	\item \begin{enumerate}[label=\alph*) ]
		\item Nombre des 16-25 ans interrogés qui pratiquent le roller :
		
		\begin{eqnarray*}
			\dfrac{521 \times 19}{100} = 521 \times 0,19 = 98,99
		\end{eqnarray*}
		
		Soit environ 99 "16-25 ans"
		
		
		\item Soit $N$ le nombre des "12-24" ans interrogées. On a :
		
		\begin{eqnarray*}
			N \times \dfrac{43,15}{100} &= 356 \\
			N &= \dfrac{356 \times 100}{43,15} = 825,02
		\end{eqnarray*}
		Soit environ 825 "12-24 ans".
	\end{enumerate}
	
	\item Pourcentage de "porteurs de casque" parmi les "pratiquants de roller" :
	
	\begin{eqnarray*}
		657 \times 0,088 + 535 \times 0,144 = 134,856 = 135\; porteurs\; de\; casque
	\end{eqnarray*}
	
	\begin{eqnarray*}
		\dfrac{135}{1192}=0,11325 \approx 11,33 \%
	\end{eqnarray*}
\end{enumerate}



\begin{myexos}
	2, 3, 4, 5 page 21-22
\end{myexos}

\subsection{Comparaison de deux pourcentages, pourcentages de pourcentages}

\begin{enumerate}[label=\arabic*. ]
	\item \begin{enumerate}[label=\alph*)]
		\item \begin{equation*}
			\dfrac{73}{149} \approx \num{0.4899},\; soit\; \num{48.99}\; \%.
		\end{equation*}
		Il y a \num{48.99} \% d'hommes parmi les victimes d'accidents de roller de "35 ans et plus".
		
		\item \begin{equation*}
			\dfrac{343}{2075} \approx \num{0.1653},\; soit\; \num{16.53}\; \%.
		\end{equation*}
		\num{16.53} \% des victimes d'accidents de roller ont "9 ans et moins".
		
		\item \begin{equation*}
			\dfrac{312}{745} \approx \num{0.4188},\; soit\; \num{41.88}\; \%.
		\end{equation*}
		Il y a \num{41.88} \% de "10 à 14 ans" parmi l'ensemble des femmes victimes d'un accident de roller.
		
		\item \begin{equation*}
			\dfrac{1330}{2075} \approx \num{0.6410},\; soit\; \num{64.10}\; \%.
		\end{equation*}
		\num{64.10} \% des accidents de roller concernent des hommes.
	\end{enumerate}
	
	\medskip 
	
	\item \begin{enumerate}[label=\alph*)]
		\item \begin{equation*}
			\dfrac{174}{1330} \approx \num{0.1308},\; soit \; \num{13.08}\; \%.
		\end{equation*}
		\num{13.08} \% des hommes victimes d'accidents de roller ont "de 20 à 34 ans".
		
		\item \begin{equation*}
			\dfrac{127}{745} \approx \num{0.1705},\; soit \; \num{17.05}\; \%.
		\end{equation*}
		\num{17.05} \% des femmes victimes d'accidents de la route ont "de 20 à 34 ans".
		
		\item Dans les effectifs, il y a plus d'hommes que de femmes de 20 à 34 ans, mais en pourcentage il y a plus de femmes. Il y a moins de femmes pratiquantes du roller que d'hommes mais en proportion elles ont plus d'accidents.
	\end{enumerate}
	
	\medskip 
	
	\item \begin{enumerate}[label=\alph*)]
		\item Proportion $p_1$ de femmes parmi les accidentés de "15 à 19 ans" :
		\begin{equation*}
			p_1=\dfrac{47}{276}\approx \num{0.1703}, \; soit \; \num{17.03}\; \%.
		\end{equation*}
	
		\item Proportion $p_2$ des "15 à 19 ans" parmi les accidentés :
		\begin{equation*}
			p_2=\dfrac{276}{2075} \approx \num{0.1330}, \; soit \; \num{13.30}\; \%.
		\end{equation*}
		
		\item Proportion $p_3$ des femmes de "15 à 19 ans" parmi les accidentés :
		\begin{equation*}
			p_3 = \dfrac{47}{2075} \approx \num{0.0227}, \; soit \; \num{2.27} \; \%.
		\end{equation*}
		
		\begin{myrem}
			Remarque : $\dfrac{47}{276}  \times \dfrac{276}{2075} = \dfrac{47}{2075}$, donc $p_1 \times p_2 = p_3$.
			
			On peut faire : \num{17.03} \% de \num{13.30} \%
			\begin{equation*}
			\dfrac{\num{17.03}}{100} \times \dfrac{\num{13.30}}{100} = \num{0.022649}, \; soit \; environ \; \num{2.26}\; \%.
			\end{equation*}	
		\end{myrem}
		
	\end{enumerate}
\end{enumerate}

\begin{myexos}
	9, 10, 11 p 23-24
\end{myexos}
\end{document}