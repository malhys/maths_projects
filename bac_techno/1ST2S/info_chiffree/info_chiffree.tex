\documentclass[12pt,a4paper]{article}

%\usepackage[left=1.5cm,right=1.5cm,top=1cm,bottom=2cm]{geometry}
\usepackage[in, plain]{fullpage}
\usepackage{array}
\usepackage{../../../pas-math}
\usepackage{../../../moncours}


%\usepackage{pas-cours}
%-------------------------------------------------------------------------------
%          -Packages nécessaires pour écrire en Français et en UTF8-
%-------------------------------------------------------------------------------
\usepackage[utf8]{inputenc}
\usepackage[frenchb]{babel}
\usepackage[T1]{fontenc}
\usepackage{lmodern}
\usepackage{textcomp}



%-------------------------------------------------------------------------------

%-------------------------------------------------------------------------------
%                          -Outils de mise en forme-
%-------------------------------------------------------------------------------
\usepackage{hyperref}
\hypersetup{pdfstartview=XYZ}
%\usepackage{enumerate}
\usepackage{graphicx}
\usepackage{multicol}
\usepackage{tabularx}
\usepackage{multirow}


\usepackage{anysize} %%pour pouvoir mettre les marges qu'on veut
%\marginsize{2.5cm}{2.5cm}{2.5cm}{2.5cm}

\usepackage{indentfirst} %%pour que les premier paragraphes soient aussi indentés
\usepackage{verbatim}
\usepackage{enumitem}
\usepackage[usenames,dvipsnames,svgnames,table]{xcolor}

\usepackage{variations}

%-------------------------------------------------------------------------------


%-------------------------------------------------------------------------------
%                  -Nécessaires pour écrire des mathématiques-
%-------------------------------------------------------------------------------
\usepackage{amsfonts}
\usepackage{amssymb}
\usepackage{amsmath}
\usepackage{amsthm}
\usepackage{tikz}
\usepackage{xlop}
%-------------------------------------------------------------------------------



%-------------------------------------------------------------------------------


%-------------------------------------------------------------------------------
%                    - Mise en forme avancée
%-------------------------------------------------------------------------------

\usepackage{ifthen}
\usepackage{ifmtarg}


\newcommand{\ifTrue}[2]{\ifthenelse{\equal{#1}{true}}{#2}{$\qquad \qquad$}}

%-------------------------------------------------------------------------------

%-------------------------------------------------------------------------------
%                     -Mise en forme d'exercices-
%-------------------------------------------------------------------------------
%\newtheoremstyle{exostyle}
%{\topsep}% espace avant
%{\topsep}% espace apres
%{}% Police utilisee par le style de thm
%{}% Indentation (vide = aucune, \parindent = indentation paragraphe)
%{\bfseries}% Police du titre de thm
%{.}% Signe de ponctuation apres le titre du thm
%{ }% Espace apres le titre du thm (\newline = linebreak)
%{\thmname{#1}\thmnumber{ #2}\thmnote{. \normalfont{\textit{#3}}}}% composants du titre du thm : \thmname = nom du thm, \thmnumber = numéro du thm, \thmnote = sous-titre du thm

%\theoremstyle{exostyle}
%\newtheorem{exercice}{Exercice}
%
%\newenvironment{questions}{
%\begin{enumerate}[\hspace{12pt}\bfseries\itshape a.]}{\end{enumerate}
%} %mettre un 1 à la place du a si on veut des numéros au lieu de lettres pour les questions 
%-------------------------------------------------------------------------------

%-------------------------------------------------------------------------------
%                    - Mise en forme de tableaux -
%-------------------------------------------------------------------------------

\renewcommand{\arraystretch}{1.7}

\setlength{\tabcolsep}{1.2cm}

%-------------------------------------------------------------------------------



%-------------------------------------------------------------------------------
%                    - Racourcis d'écriture -
%-------------------------------------------------------------------------------

% Angles orientés (couples de vecteurs)
\newcommand{\aopp}[2]{(\vec{#1}, \vec{#2})} %Les deuc vecteurs sont positifs
\newcommand{\aopn}[2]{(\vec{#1}, -\vec{#2})} %Le second vecteur est négatif
\newcommand{\aonp}[2]{(-\vec{#1}, \vec{#2})} %Le premier vecteur est négatif
\newcommand{\aonn}[2]{(-\vec{#1}, -\vec{#2})} %Les deux vecteurs sont négatifs

%Ensembles mathématiques
\newcommand{\naturels}{\mathbb{N}} %Nombres naturels
\newcommand{\relatifs}{\mathbb{Z}} %Nombres relatifs
\newcommand{\rationnels}{\mathbb{Q}} %Nombres rationnels
\newcommand{\reels}{\mathbb{R}} %Nombres réels
\newcommand{\complexes}{\mathbb{C}} %Nombres complexes


%Intégration des parenthèses aux cosinus
\newcommand{\cosP}[1]{\cos\left(#1\right)}
\newcommand{\sinP}[1]{\sin\left(#1\right)}


%Probas stats
\newcommand{\stat}{statistique}
\newcommand{\stats}{statistiques}
%-------------------------------------------------------------------------------

%-------------------------------------------------------------------------------
%                    - Mise en page -
%-------------------------------------------------------------------------------

\newcommand{\twoCol}[1]{\begin{multicols}{2}#1\end{multicols}}


\setenumerate[1]{font=\bfseries,label=\textit{\alph*})}
\setenumerate[2]{font=\bfseries,label=\arabic*)}


%-------------------------------------------------------------------------------
%                    - Elements cours -
%-------------------------------------------------------------------------------





%\makeatletter
%\renewcommand*{\@seccntformat}[1]{\csname the#1\endcsname\hspace{0.1cm}}
%\makeatother


%\author{Olivier FINOT}
\date{}
\title{Information chiffrée }

%\newcommand{\disp}{false}

\lhead{CH1 : Info chiffrée}
\rhead{O. FINOT}
%
%\rfoot{Page \thepage}
\begin{document}
%\maketitle

\chap[num=1, color=red]{Information chiffrée}{Olivier FINOT, \today }

\begin{myobj}
	\begin{itemize}
		
		\item Construire le symétrique d’un point ou d'une figure par rapport à une droite à la main où à l’aide d’un logiciel;
		\item Construire le symétrique d’un point ou d'une figure par rapport à un point, à la main où à l’aide d’un logiciel;
		\item Utiliser les propriétés de la symétrie axiale ou centrale;
		\item Identifier des symétries dans des figures.		
	\end{itemize}
\end{myobj}

\begin{mycomp}
	\begin{itemize}
		\item \kw{Chercher (Ch2)} :  s’engager    dans    une    démarche    scientifique, observer, questionner, manipuler, expérimenter (sur une feuille de papier, avec des objets, à l’aide de logiciels), émettre des hypothèses, chercher des exemples ou des contre-exemples, simplifier ou particulariser une situation, émettre une conjecture ;
		\item \kw{Raisonner (Ra3)} :  démontrer : utiliser un raisonnement logique et des règles établies (propriétés, théorèmes, formules) pour parvenir à une conclusion ;
		\item \kw{Communiquer (Co2)} :  expliquer à l’oral ou à l’écrit (sa démarche, son raisonnement, un calcul, un protocole   de   construction   géométrique, un algorithme), comprendre les explications d’un autre et argumenter dans l’échange ; 
		
	\end{itemize}
\end{mycomp}




\section{Effectifs et proportions (Activité : TP 1 page 8)}


\subsection{Expression d'une proportion à l'aide d'un pourcentage}
\begin{enumerate}[label=\arabic*.]
	\item 
	\begin{enumerate} [label=\alph*) ]
		\item Proportion des "pratiquants de roller" parmi les personnes interrogées:
		\begin{itemize}
			\item Sous forme de fraction : $\dfrac{1192}{13685}$
			\item Sous forme d'un nombre décimal arrondi à $10^{-4}$ : $\approx 0,0871$ ($10^{-4} = 0,0001=\frac{1}{10000}=\frac{1}{10^4}$)
			\item Sous la forme d'un pourcentage arrondi à $10^{-2} \approx 8,71 \% $
		\end{itemize}
		
		\vspace*{1cm}
		
		\hspace*{-5cm} \begin{mybilan2}{Proportion}
			La \kw{proportion ou fréquence} d'une partie $A$ d'une population $E$, est le rapport $p$ des effectifs de $A$ et de $E$ :
			
			\begin{eqnarray*}
			p = \dfrac{n_A}{n_E} \; \left(\dfrac{Effectif de A}{Effectif de E}\right)
			\end{eqnarray*}
		\end{mybilan2}
				
		\item Pourcentage de femmes parmi ces "pratiquants du roller"
		
		\begin{eqnarray*}
			 \dfrac{657}{1192} \times 100 = 55,117, soit\; environ\; 55,12 \%
		\end{eqnarray*}
		%	\dfrac{657}{1192}  100 =  55,117, soit environ 55,12 
						

	\end{enumerate}
	
	\item \begin{enumerate}[label=\alph*) ]
		\item Nombre des 16-25 ans interrogés qui pratiquent le roller :
		
		\begin{eqnarray*}
			\dfrac{521 \times 19}{100} = 521 \times 0,19 = 98,99
		\end{eqnarray*}
		
		Soit environ 99 "16-25 ans"
		
		
		\item Soit $N$ le nombre des "12-24" ans interrogées. On a :
		
		\begin{eqnarray*}
			N \times \dfrac{43,15}{100} &= 356 \\
			N &= \dfrac{356 \times 100}{43,15} = 825,02
		\end{eqnarray*}
		Soit environ 825 "12-24 ans".
	\end{enumerate}
	
	\item Pourcentage de "porteurs de casque" parmi les "pratiquants de roller" :
	
	\begin{eqnarray*}
		657 \times 0,088 + 535 \times 0,144 = 134,856 = 135\; porteurs\; de\; casque
	\end{eqnarray*}
	
	\begin{eqnarray*}
		\dfrac{135}{1192}=0,11325 \approx 11,33 \%
	\end{eqnarray*}
\end{enumerate}



\begin{myexos}
	2, 3, 4, 5 page 21-22
\end{myexos}

\subsection{Comparaison de deux pourcentages, pourcentages de pourcentages}

\begin{enumerate}[label=\arabic*. ]
	\item \begin{enumerate}[label=\alph*)]
		\item \begin{equation*}
			\dfrac{73}{149} \approx \num{0.4899},\; soit\; \num{48.99}\; \%.
		\end{equation*}
		Il y a \num{48.99} \% d'hommes parmi les victimes d'accidents de roller de "35 ans et plus".
		
		\item \begin{equation*}
			\dfrac{343}{2075} \approx \num{0.1653},\; soit\; \num{16.53}\; \%.
		\end{equation*}
		\num{16.53} \% des victimes d'accidents de roller ont "9 ans et moins".
		
		\item \begin{equation*}
			\dfrac{312}{745} \approx \num{0.4188},\; soit\; \num{41.88}\; \%.
		\end{equation*}
		Il y a \num{41.88} \% de "10 à 14 ans" parmi l'ensemble des femmes victimes d'un accident de roller.
		
		\item \begin{equation*}
			\dfrac{1330}{2075} \approx \num{0.6410},\; soit\; \num{64.10}\; \%.
		\end{equation*}
		\num{64.10} \% des accidents de roller concernent des hommes.
	\end{enumerate}
	
	\medskip 
	
	\item \begin{enumerate}[label=\alph*)]
		\item \begin{equation*}
			\dfrac{174}{1330} \approx \num{0.1308},\; soit \; \num{13.08}\; \%.
		\end{equation*}
		\num{13.08} \% des hommes victimes d'accidents de roller ont "de 20 à 34 ans".
		
		\item \begin{equation*}
			\dfrac{127}{745} \approx \num{0.1705},\; soit \; \num{17.05}\; \%.
		\end{equation*}
		\num{17.05} \% des femmes victimes d'accidents de la route ont "de 20 à 34 ans".
		
		\item Dans les effectifs, il y a plus d'hommes que de femmes de 20 à 34 ans, mais en pourcentage il y a plus de femmes. Il y a moins de femmes pratiquantes du roller que d'hommes mais en proportion elles ont plus d'accidents.
	\end{enumerate}
	
	\medskip 
	
	\item \begin{enumerate}[label=\alph*)]
		\item Proportion $p_1$ de femmes parmi les accidentés de "15 à 19 ans" :
		\begin{equation*}
			p_1=\dfrac{47}{276}\approx \num{0.1703}, \; soit \; \num{17.03}\; \%.
		\end{equation*}
	
		\item Proportion $p_2$ des "15 à 19 ans" parmi les accidentés :
		\begin{equation*}
			p_2=\dfrac{276}{2075} \approx \num{0.1330}, \; soit \; \num{13.30}\; \%.
		\end{equation*}
		
		\item Proportion $p_3$ des femmes de "15 à 19 ans" parmi les accidentés :
		\begin{equation*}
			p_3 = \dfrac{47}{2075} \approx \num{0.0227}, \; soit \; \num{2.27} \; \%.
		\end{equation*}
		
		\begin{myrem}
			Remarque : $\dfrac{47}{276}  \times \dfrac{276}{2075} = \dfrac{47}{2075}$, donc $p_1 \times p_2 = p_3$.
			
			On peut faire : \num{17.03} \% de \num{13.30} \%
			\begin{equation*}
			\dfrac{\num{17.03}}{100} \times \dfrac{\num{13.30}}{100} = \num{0.022649}, \; soit \; environ \; \num{2.26}\; \%.
			\end{equation*}	
		\end{myrem}
		
	\end{enumerate}
\end{enumerate}

\begin{myexos}
	9, 10, 11 p 23-24
\end{myexos}

\newpage

\subsection{Additionner et comparer des pourcentages}

\begin{enumerate}[label=\arabic*. ]
	\item  Pourcentage d'enfants en surpoids dans les zones rurales :
	\begin{equation*}
		\num{100} - \num{87.2} = \num{12.8} \quad soit \; \num{12.8} \%.
	\end{equation*}
	
	\medskip 
	
	\item  Pourcentage d'enfants obèses :
	\begin{equation*}
		\num{12.8} - \num{9.2} = \num{3.6} \quad soit \; \num{3.6} \%.
	\end{equation*}
	
	\item \begin{enumerate}[label=\alph*)]
		\item Dans l'agglomération parisienne, il y a \num{5} \% d'enfants obèses et \num{16} \% en surpoids; la proportion d'enfants obèses parmi ceux en surpoids est donc égale à $\dfrac{5}{16}=\num{0.301} \%,$ soit environ un peu plus de 3 enfants souffrant d'obésité pour 10 en surpoids. L'affirmation est donc juste.
		
		\item Les effectifs pour les différents types d'agglomération ne sont pas connus. On ne peut donc rien affirmer concernant le nombre d'enfants en surpoids.
	\end{enumerate}
	
	
\end{enumerate}

\section{Pourcentage d'évolution, coefficient multiplicateur}

TP2 page 10

\subsection{Variation relative (taux d'évolution)}

\begin{enumerate}[label=\arabic*. ]
	
	\item \begin{enumerate}[label=\alph*)]
		\item Variation absolue du nombre de médecins généralistes en France entre 1990 et 2009 :
		
		\begin{equation*}
			\num{107667} - \num{93380} = \num{8287}
		\end{equation*}
		$\rightarrow$  Soit une hausse de \num{8287} médecins.
		
		\item Variation relative (ou taux d'évolution) du nombre de généralistes entre \num{1990} et \num{2009} :
		
		\begin{table}[h!]
			\centering
			
			
			\begin{tabular}{|ccc|}
				\hline
				en \num{1990} & + \num{8.87} \%  & en \num{2009} \\
				\num{93380} médecins& $\rightarrow$ & \num{101667} médecins \\
				\hline
			\end{tabular}
		\end{table}
		
		\begin{equation*}
			\dfrac{(\num{101667} - \num{93380})}{\num{93380}} \times 100 = \num{8.874}...
		\end{equation*}
		$\rightarrow$ Soit une hausse d'environ \num{8.87} \%.
		
		\item Entre 1990 et 2009 le nombre de médecins généralistes en France à augmenté de \num{8.87} \%.
		
	\end{enumerate}
	
	 
	
	\begin{myrem}
		\begin{eqnarray*}
			\num{1.0887} & = & 1 + \num{0.0887} \\
						& = & 1 + \frac{\num{8.87}}{100}
		\end{eqnarray*}
		
		Ainsi pour augmenter une grandeur de \num{8.87} \% il suffit de multiplier cette grandeur par $1 + \dfrac{8.87}{100}$ soit \num{1.0887}. Ce nombre s'appelle le \kw{coefficient multiplicateur} associé à une augmentation de \num{8.87} \%.
	\end{myrem}
	
	\item \begin{enumerate}[label=\alph*)]
		\item Variation absolue du nombre de médecins généralistes :
		\begin{equation*}
			\num{99670} - \num{101667} = \num{-1997}
		\end{equation*}
		Soit une baisse de \num{1997} médecins.
		
		\item Taux d'évolution correspondant
			\begin{equation*}
				\dfrac{(\num{99670} - \num{101667})}{\num{101667}}\times 100 \approx -\num{-1.96}
			\end{equation*}
			
			Soit une baisse d'environ \num{-1.96} \%. 
			
		\item Entre \num{2009} et \num{2015}, le nombre de médecins généralistes en France devrait baisser d'environ \num{1.96} \%.
	\end{enumerate}
	
	\begin{myrem}
	%\end{myrem}
	
	%	\begin{table}[h!]
	%		\centering
			
			
			\begin{tabular}{|ccc|}
				\hline
				en \num{2009} & - \num{1.96} \%  & en \num{2015} \\
									& {\LARGE $\rightarrow$} &			\\
				\num{101667} médecins& $\times \num{0.9804}$ & \num{99670} médecins \\
				\hline
			\end{tabular}

		\vspace*{1cm} 
		On a : $\dfrac{99670}{101667} \approx \num{0.9804}$.
		Et $1 - \dfrac{\num{1.96}}{\num{100}} = \num{0.9804}$
	
	
		Pour diminuer une grandeur de \num{1.96} \%, il suffit de multiplier cette grandeur par $1 - \dfrac{\num{1.96}}{\num{100}}$, soit \num{0.9804}. \num{0.9804} est le \kw{coefficient multiplicateur} associé à une baisse de \num{1.96} \%.
	\end{myrem}
	
	\begin{mybilan2}{Taux d'évolution et coefficient multiplicateur}
		Le taux d'évolution $t$ (ou variation relative) d'une quantité passant de la valeur $y_1$ à une valeur $y_2$ est égal à :
		\begin{equation*}
			t = \dfrac{y_2 - y_1}{y_1} \left(\dfrac{V_{arrivée} - V_{départ}}{V_{départ}}\right)
		\end{equation*}
		
		\underline{Remarque} : Un taux d'évolution positif traduit une hausse, un taux d'évolution négatif traduit une baisse.\\
		
		\underline{Coefficients multiplicateurs :} 
		\begin{itemize}
			\item \kw{Augmenter} une grandeur de $t \%$ revient à multiplier cette grandeur par $(1 + \dfrac{t}{100})$.
			
			\item \underline{Exemple :} $+ 5 \% = \times \num{1.05}$ ; $+ 20 \% = \times \num{1.20}$ \\
			
			\item \kw{Diminuer} une grandeur de $t \%$ revient à multiplier cette grandeur par $\qquad (1 - \dfrac{t}{100})$.
			\item \underline{Exemple :} $- 12 \% = \times \num{0.88}$ ; $- 3 \% = \times \num{0.97}$ \\
			
			\item Dans le cas d'une \kw{hausse}, le coefficient multiplicateur est \kw{supérieur à 1}.
			
			\item Dans le cas d'une \kw{baisse}, le coefficient multiplicateur est \kw{inférieur à 1}.
		\end{itemize}
		
		
		
	\end{mybilan2}
	
	\item Nombre de médecins des spécialités médicales en 2009 
	
	\begin{table}[h!]
		\centering
		\begin{tabular}{|ccc|}
			\hline
			en \num{1990} & + \num{23.63} \%  & en \num{2009} \\
			& {\LARGE $\rightarrow$} &			\\
			\num{48040} médecins& $\times \num{1.2363}$ & ? médecins \\
			\hline
		\end{tabular}
	\end{table}
	
	D'où : $\num{48040} \times \num{1.2363} = \num{59391.8}...$, soit environ \num{52392} médecins.
	
	\item Nombre de médecins des spécialités chirurgicales en 2015 
	
	\begin{table}[h!]
		\centering
		\begin{tabular}{|ccc|}
			\hline
			en \num{2009} & - \num{8.22} \%  & en \num{2015} \\
			& {\LARGE $\rightarrow$} &			\\
			\num{25163} médecins& $\times \num{0.9178}$ & ? médecins \\
			\hline
		\end{tabular}
	\end{table}
	
	D'où : $\num{25163} \times \num{0.9178} = \num{23094.60}...$, soit environ \num{23095} médecins.
	
	\vspace*{2cm}
	\item Nombre de médecins des spécialités chirurgicales en 1990 
	
	\begin{table}[h!]
		\centering
		\begin{tabular}{|ccc|}
			\hline
			en \num{1990} & + \num{17.21} \%  & en \num{2009} \\
			& {\LARGE $\rightarrow$} &			\\
			? médecins& $\times \num{1.1721}$ & 25163 médecins \\
			\hline
			& {\LARGE $\leftarrow$} & \\
			& $\div \num{1.1721}$ & \\
			\hline
		\end{tabular}
	\end{table}
	
	D'où : $\num{25163} \div \num{1.1721} = \num{21468.30475}...$, soit environ \num{21468} médecins.
\end{enumerate}

\begin{myexos}
	Application exerices : 12, 13, 14 et 15 page 25
\end{myexos}

\subsection{\'Evolutions successives, évolution réciproque}

\begin{enumerate}[label=\Alph*.]
	\item \'Evolutions successives
	
		\begin{enumerate}[label=\arabic*)]
			\item \ 
			
				\begin{table}[h!]
					\centering
					\begin{tabular}{|ccc|c|}
						\hline
						$P_1$ & + \num{25} \%  & $P_2$ &\\
						& {\LARGE $\rightarrow$} &	&	$P_2 = \num{16} \times \num{1.25} = 20$, soit 20 \$ .	\\
						\num{16} \$ & $\times \num{1.25}$ & ? \$ \\
						\hline
					\end{tabular}
					
				\end{table}
				
			\item \ 
			
				\begin{table}[h!]
					\centering
					\begin{tabular}{|ccc|c|}
						\hline
						$P_2$ & + \num{30} \%  & $P_3$ &\\
						& {\LARGE$\rightarrow$} &	&	$P_2 = \num{20} \times \num{1.30} = 26$, soit 26 \$ .	\\
						\num{20} \$ & $\times \num{1.30}$ & ? \$ & \\
						\hline
					\end{tabular}
					
				\end{table}
			\item \ 
			
			\begin{table}[h!]
				\centering
				\begin{tabular}{|ccc|c|}
					\hline
					$P_1$ & + ... \%  & $P_3$ & \\
							& {\LARGE$\rightarrow$} &	&	 $\dfrac{\num{26} - \num{16}}{\num{16}} = \num{0.625}$	\\
					\num{16} \$ & $\times ...$ & 26 \$ & Soit une hausse globale de \num{62.5} \% \\
					\hline
				\end{tabular}
											
			\end{table}
			
			\underline{Calcul du coefficient multiplicateur :}
			\begin{equation*}
				 k = \dfrac{26}{16} = \num{1.625}
			\end{equation*}
			
			On peut aussi calculer indépendamment des prix : $\num{1.25} \times \num{1.30} = \num{1.625}$, soit une hausse globale de \num{62.5} \%.\\
			
			\begin{myrem}
				Le pourcentage de hausse globale \num{62.5} \% n'est pas égal à la somme des deux pourcentages de hausse successives \num{25} \% et \num{30} \%, car ces deux pourcentages ne s'appliquent pas sur le même prix, donc ne s'additionnent pas.
			\end{myrem}
			
			\begin{mybilan2}{\'Evolutions successives}
				Deux évolutions (hausse ou baisse) successives de coefficients multiplicateurs $c$ et $c'$  correspondent  une évolution globale (hausse ou baisse) de $c \times c'$ (on multiplie).
			\end{mybilan2}
		\end{enumerate}
		
\end{enumerate}
\end{document}