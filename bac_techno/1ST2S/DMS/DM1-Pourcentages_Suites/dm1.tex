\documentclass[a4paper,11pt]{exam}
\printanswers % pour imprimer les réponses (corrigé)
%\noprintanswers % Pour ne pas imprimer les réponses (énoncé)
\addpoints % Pour compter les points
% \noaddpoints % pour ne pas compter les points
%\qformat{\textbf{\thequestion ) } }
\qformat{\textbf{\thequestion )}} % Pour définir le style des questions (facultatif)
\usepackage{color} % définit une nouvelle couleur
\shadedsolutions % définit le style des réponses
% \framedsolutions % définit le style des réponses
\definecolor{SolutionColor}{rgb}{0.8,0.9,1} % bleu ciel
\renewcommand{\solutiontitle}{\noindent\textbf{Solution:}\par\noindent} % Définit le titre des solutions

\usepackage{fullpage}



\makeatletter

\def\maketitle{{\centering%
	\par{\huge\textbf{\@title}}%
	\par{\@date}%
	\par}}

\makeatother

\lhead{NOM Pr\'enom :}
\rhead{\textbf{Les r\'eponses doivent \^etre justifi\'ees}}
\cfoot{\thepage / \pageref{LastPage}}


%\usepackage{../../pas-math}
%\usepackage{../../moncours}


%\usepackage{pas-cours}
%-------------------------------------------------------------------------------
%          -Packages nécessaires pour écrire en Français et en UTF8-
%-------------------------------------------------------------------------------
\usepackage[utf8]{inputenc}
\usepackage[frenchb]{babel}
\usepackage[T1]{fontenc}
\usepackage{lmodern}
\usepackage{textcomp}



%-------------------------------------------------------------------------------

%-------------------------------------------------------------------------------
%                          -Outils de mise en forme-
%-------------------------------------------------------------------------------
\usepackage{hyperref}
\hypersetup{pdfstartview=XYZ}
%\usepackage{enumerate}
\usepackage{graphicx}
\usepackage{multicol}
\usepackage{tabularx}
\usepackage{multirow}


\usepackage{anysize} %%pour pouvoir mettre les marges qu'on veut
%\marginsize{2.5cm}{2.5cm}{2.5cm}{2.5cm}

\usepackage{indentfirst} %%pour que les premier paragraphes soient aussi indentés
\usepackage{verbatim}
\usepackage{enumitem}
\usepackage[usenames,dvipsnames,svgnames,table]{xcolor}

\usepackage{variations}

%-------------------------------------------------------------------------------


%-------------------------------------------------------------------------------
%                  -Nécessaires pour écrire des mathématiques-
%-------------------------------------------------------------------------------
\usepackage{amsfonts}
\usepackage{amssymb}
\usepackage{amsmath}
\usepackage{amsthm}
\usepackage{tikz}
\usepackage{xlop}
%-------------------------------------------------------------------------------



%-------------------------------------------------------------------------------


%-------------------------------------------------------------------------------
%                    - Mise en forme avancée
%-------------------------------------------------------------------------------

\usepackage{ifthen}
\usepackage{ifmtarg}


\newcommand{\ifTrue}[2]{\ifthenelse{\equal{#1}{true}}{#2}{$\qquad \qquad$}}

%-------------------------------------------------------------------------------

%-------------------------------------------------------------------------------
%                     -Mise en forme d'exercices-
%-------------------------------------------------------------------------------
%\newtheoremstyle{exostyle}
%{\topsep}% espace avant
%{\topsep}% espace apres
%{}% Police utilisee par le style de thm
%{}% Indentation (vide = aucune, \parindent = indentation paragraphe)
%{\bfseries}% Police du titre de thm
%{.}% Signe de ponctuation apres le titre du thm
%{ }% Espace apres le titre du thm (\newline = linebreak)
%{\thmname{#1}\thmnumber{ #2}\thmnote{. \normalfont{\textit{#3}}}}% composants du titre du thm : \thmname = nom du thm, \thmnumber = numéro du thm, \thmnote = sous-titre du thm

%\theoremstyle{exostyle}
%\newtheorem{exercice}{Exercice}
%
%\newenvironment{questions}{
%\begin{enumerate}[\hspace{12pt}\bfseries\itshape a.]}{\end{enumerate}
%} %mettre un 1 à la place du a si on veut des numéros au lieu de lettres pour les questions 
%-------------------------------------------------------------------------------

%-------------------------------------------------------------------------------
%                    - Mise en forme de tableaux -
%-------------------------------------------------------------------------------

\renewcommand{\arraystretch}{1.7}

\setlength{\tabcolsep}{1.2cm}

%-------------------------------------------------------------------------------



%-------------------------------------------------------------------------------
%                    - Racourcis d'écriture -
%-------------------------------------------------------------------------------

% Angles orientés (couples de vecteurs)
\newcommand{\aopp}[2]{(\vec{#1}, \vec{#2})} %Les deuc vecteurs sont positifs
\newcommand{\aopn}[2]{(\vec{#1}, -\vec{#2})} %Le second vecteur est négatif
\newcommand{\aonp}[2]{(-\vec{#1}, \vec{#2})} %Le premier vecteur est négatif
\newcommand{\aonn}[2]{(-\vec{#1}, -\vec{#2})} %Les deux vecteurs sont négatifs

%Ensembles mathématiques
\newcommand{\naturels}{\mathbb{N}} %Nombres naturels
\newcommand{\relatifs}{\mathbb{Z}} %Nombres relatifs
\newcommand{\rationnels}{\mathbb{Q}} %Nombres rationnels
\newcommand{\reels}{\mathbb{R}} %Nombres réels
\newcommand{\complexes}{\mathbb{C}} %Nombres complexes


%Intégration des parenthèses aux cosinus
\newcommand{\cosP}[1]{\cos\left(#1\right)}
\newcommand{\sinP}[1]{\sin\left(#1\right)}


%Probas stats
\newcommand{\stat}{statistique}
\newcommand{\stats}{statistiques}
%-------------------------------------------------------------------------------

%-------------------------------------------------------------------------------
%                    - Mise en page -
%-------------------------------------------------------------------------------

\newcommand{\twoCol}[1]{\begin{multicols}{2}#1\end{multicols}}


\setenumerate[1]{font=\bfseries,label=\textit{\alph*})}
\setenumerate[2]{font=\bfseries,label=\arabic*)}


%-------------------------------------------------------------------------------
%                    - Elements cours -
%-------------------------------------------------------------------------------





%\usepackage{fullpage}
\author{\ }
\date{A rendre pour le 12 Novembre 2018}
\title{$1^{ère}$ $ST_2S$ : DM num\'ero 1}


\begin{document}
%	\usepackage{fancyhdr}
%	
%	\pagestyle{fancy}
%	\fancyhf{}
	%\rhead{Share\LaTeX}

	\maketitle


\section{A propos des acides gras oméga-3}
On a récemment découvert que les acides gras oméga-3, présents dans des poissons comme la truite ou le saumon, ont un effet protecteur contre les maladies cardio-vasculaires.
\emph{Les pourcentages demandés seront arrondis à \num{e-2}. \%} 

\begin{questions}
	\question Une portion de 180 g de saumon d'élevage fournit environ \num{1.5}g d'oméga-3. Calculer le pourcentage d'oméga-3 dans le saumon d'élevage.
	\begin{solution}
		$\dfrac{\num{1.5}}{\num{180}} \approx \num{0.0083}$, soit \num{0.83} \% d'oméga-3 dans le saumon d'élevage.
	\end{solution}
	
	\question Le pourcentage d'omega-3 dans le saumon sauvage est de \num{0.78} \%. En déduire la quantité d'oméga-3 contenue dans une portion de 180g de saumon sauvage (arrondir à \num{0.1} g).
	\begin{solution}
		$180 \times \frac{\num{0.78}}{100} = \num{1.404}$, soit environ \num{1.4} g d'oméga-3 pour 180 g de saumon sauvage.
	\end{solution}
	
	\question Consigner les résultats précédents dans le tableau suivant et finir de le compléter. 
	
	{\footnotesize \begin{tabular}{@{\ }c@{\ }|@{\ }c@{\ }|@{\ }c@{\ }|@{\ }c@{\ }|@{\ }c@{\ }|}
		\cline{2-5}
		& \multicolumn{2}{c|}{Elevage}               & \multicolumn{2}{c|}{Sauvage}               \\ \cline{2-5} 
		& Pourcentage d'oméga-3 & Quantité d'oméga-3 & Pourcentage d'oméga-3 & Quantité d'oméga-3 \\ \hline
		\multicolumn{1}{|@{\ }c@{\ }|}{Saumon (180 g)} &                       & \num{1.5} g      & \num{0.78}\%        &                    \\ \hline
		\multicolumn{1}{|@{\ }c@{\ }|}{Truite (180 g)} &                       & \num{1.3} g      & \num{0.22}\%        &                    \\ \hline
	\end{tabular}}
	
	\begin{solution}
		{\footnotesize \begin{tabular}{@{\ }c@{\ }|@{\ }c@{\ }|@{\ }c@{\ }|@{\ }c@{\ }|@{\ }c@{\ }|}
				\cline{2-5}
				& \multicolumn{2}{c|}{Elevage}               & \multicolumn{2}{c|}{Sauvage}               \\ \cline{2-5} 
				& Pourcentage d'oméga-3 & Quantité d'oméga-3 & Pourcentage d'oméga-3 & Quantité d'oméga-3 \\ \hline
				\multicolumn{1}{|@{\ }c@{\ }|}{Saumon (180 g)} &  \num{0.83} \%   & \num{1.5} g      & \num{0.78}\%        &    \num{1.4} g                \\ \hline
				\multicolumn{1}{|@{\ }c@{\ }|}{Truite (180 g)} &  \num{0.72} \%  & \num{1.3} g      & \num{0.22}\%        &  \num{0.4} g\\ \hline
			\end{tabular}}
	\end{solution}
	
	\question La consommation d'une portion de 180 g de truite d'élevage couvre environ 37 \% des besoins hebdomadaires d'un être humain. Montrer que ces besoins, arrondis à \num{0.1} g, sont de \num{3.5} g.
	
	\begin{solution}
		Soit $x$ le besoin hebdomadaire en oméga-3 d'un être humain. On a $\dfrac{37 \times x}{100}=\num{1.3}$.
		
		Donc $x=\dfrac{\num{1.3}}{\num{0.37}} \approx \num{3.51}$ g.
	\end{solution} 
	
	\question Retrouver la réponse précédente sachant que ces besoins hebdomadaires sont exactement couverts si on consomme 450 g de saumon sauvage.
	\begin{solution}
		$\dfrac{450 \times \num{0.78}}{100} = \num{3.51}$ soit \num{3.5} g.
	\end{solution}
	
	\question Calculer la quantité de truite sauvage qu'il faudrait consommer pour couvrir la totalité de ces besoins hebdomadaires (arrondir à 10 g).
	\begin{solution}
		$\dfrac{\num{3.5}}{\num{0.4}}\times 180 = 1575$, soit 1570 g de truite sauvage pour couvrir les besoins hebdomadaires en oméga-3. 
	\end{solution}
\end{questions}


%\newpage

\section{Efficacité d'un médicament}

120 personnes atteintes d'une maladie ont accepté de tester l'efficacité d'un nouveau médicament. 

Pendant un mois 80 personnes parmi elles ont pris le médicament, les autres ont prit le placebo.

A l'issue de l'expérimentation :
\begin{itemize}
	\item parmi les personnes ayant pris le médicament, 75\% ont vu leur santé s'améliorer;
	\item parmi les personnes ayant prit le placebo, seulement 5 personnes ont vu leur santé s'améliorer.
\end{itemize} 
\begin{center}
	
	{\small \begin{tabular}{|@{\ }c@{\ }|@{\ }c@{\ }|@{\ }c@{\ }|@{\ }c@{\ }|}
			\hline
			& Ont vu leur santé s'améliorer & N'ont pas vu leur santé s'améliorer &  Total  \\
			\hline
			Ont pris le médicament &  &  &    \\
			\hline
			Ont prit le placebo &  &  &   \\
			\hline
			Total &  &  & 120   \\
			\hline
	\end{tabular}}
\end{center}


\begin{questions}
	\question Compléter le tableau.
	
	\begin{solution}
		{\small \begin{tabular}{|@{\ }c@{\ }|@{\ }c@{\ }|@{\ }c@{\ }|@{\ }c@{\ }|}
				\hline
				& Ont vu leur santé s'améliorer & N'ont pas vu leur santé s'améliorer &  Total  \\
				\hline
				Ont pris le médicament &  60 & 20 & 80   \\
				\hline
				Ont prit le placebo & 5 & 35 &  40 \\
				\hline
				Total &  65 & 55 & 120   \\
				\hline
		\end{tabular}}
		
	\end{solution}
	\question \begin{parts}
		\part Calculer le pourcentage de personnes qui ont vu leu santé s'améliorer (arrondir à \num{0.01}\% près).
		\begin{solution}
			$ \dfrac{65}{120} \approx \num{0.5417} $ Soit \num{54.17} \%.
		\end{solution}
		\part Parmi les personnes qui n'ont pas prit le médicament, calculer le pourcentage de celles qui ont vu leur santé s'améliorer.
		\begin{solution}
			$ \dfrac{60}{80} = \num{0.75} $ Soit \num{75} \%.
		\end{solution}
		\part Parmi les personnes qui ont vu leur santé s'améliorer, calculer le pourcentage de celles qui ont prit le médicament (arrondir à \num{0.1}\% près).
		\begin{solution}
			$ \dfrac{60}{65} \approx \num{0.923} $ Soit \num{92.3} \%.
		\end{solution}
	\end{parts}
\end{questions}


\section{Cultures marines}

Une ferme aquacole de Vendée décide de cultiver des micro-algues sur de l'eau de forage.

Elle fait appel à une entreprise A pour creuser un puits. Le coût prévu pour ce travail comprend :
\begin{itemize}
	\item un forfait de mise en place du matériel de 800 €;
	\item 200 € par mètre creusé.	
\end{itemize}

Le montant forfaitaire est noté $u_0$ (forage à 0 mètres), $u_1$ est le coût du forage à 1 mètre, $u_2$ le coût du forage à 2 mètres..., $u_n$ le coût du forage à $n$ mètres.
\begin{questions}
	\question Calculer $u_1$, $u_2$ et $u_3$.
	\begin{solution}
		\begin{itemize}
			\item $u_1$ correspond à un forage à 1 mètre de profondeur donc $u_1 = 800 + 200 = 1000$.
			\item Pour $u_2$ on ajoute 1 mètre et donc 200 €, d'où : $u_2 = u_1 + 200 = 1200$.
			
			\item Même chose pour $u_3$ ; $u_3 = u_2 + 200 = 1400$.
		\end{itemize}
	\end{solution}
	%\question Donner la nature de la suite $(u_n)$, préciser sa raison.
	\question
		\begin{parts}
			\part Comment calculer le montant du forage en fonction de sa profondeur $n$.
			\begin{solution}
				Pour calculer le montant du forage on prend le montant forfaitaire et on lui ajoute $n$ fois 200 €.
			\end{solution}
			\part Exprimer $u_n$ en fonction de $n$.
			\begin{solution}
				On a donc :
					\begin{eqnarray*}
						u_n &=& u_0 + n \times 200 \\
						u_n &=& 800 + n \times 200
					\end{eqnarray*} 
			\end{solution}
			\part Calculer $u_{13}$.
			\begin{solution}
				
				\begin{eqnarray*}
					u_{13} &=& u_0 + 13 \times 200 \\
					u_{13} &=& 800 + 13 \times 200 \\
					u_{13} &=& 800 + 2600 \\
					u_{13} &=& 3400
				\end{eqnarray*}
			\end{solution}
		\end{parts}
	
	\question La profondeur prévue pour le forage est 12 mètres. Une autre entreprise, l'entreprise B propose de forer à 12 mètres pour un coût global de \num{3500} €. Parmi les deux entreprises, laquelle est la plus avantageuse ?
		\begin{solution}
			Je calcule le coût d'un forage à 12 mètres avec l'entreprise A :
			
			\begin{eqnarray*}
				u_{12} &=& u_0 + 12 \times 200 \\
				u_{12} &=& 800 + 12 \times 200 \\
				u_{12} &=& 800 + 2400 \\
				u_{12} &=& 3200 
			\end{eqnarray*}
		
		3500 > 3200, donc pour un forage à 12 mètres de profondeur, l'entreprise A est la plus avantageuse.
		\end{solution}
	
\end{questions}
	\label{LastPage}
	

\end{document}