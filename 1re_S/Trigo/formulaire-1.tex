\documentclass[12pt,a4paper]{article}
%\usepackage[utf8]{inputenc}
%\usepackage[frenchb]{babel}
%\usepackage[T1]{fontenc}
%\usepackage{graphicx}

\usepackage[utf8]{inputenc}
\usepackage[T1]{fontenc}
\usepackage{lmodern}
\usepackage{textcomp}

%
\usepackage[left=1.5cm,right=1.5cm,top=1cm,bottom=2cm]{geometry}
\usepackage{fullpage}
%\usepackage{multicol}
%
%\usepackage{amsmath}
%\usepackage{amssymb}
%\usepackage{multirow}


%-------------------------------------------------------------------------------
%          -Packages nécessaires pour écrire en Français et en UTF8-
%-------------------------------------------------------------------------------
\usepackage[utf8]{inputenc}
\usepackage[frenchb]{babel}
\usepackage[T1]{fontenc}
\usepackage{lmodern}
%-------------------------------------------------------------------------------

%-------------------------------------------------------------------------------
%                          -Outils de mise en forme-
%-------------------------------------------------------------------------------
\usepackage{hyperref}
\hypersetup{pdfstartview=XYZ}
\usepackage{enumerate}
\usepackage{graphicx}
\usepackage{multicol}

\usepackage{anysize} %%pour pouvoir mettre les marges qu'on veut
%\marginsize{2.5cm}{2.5cm}{2.5cm}{2.5cm}

\usepackage{indentfirst} %%pour que les premier paragraphes soient aussi indentés
%-------------------------------------------------------------------------------


%-------------------------------------------------------------------------------
%                  -Nécessaires pour écrire des mathématiques-
%-------------------------------------------------------------------------------
\usepackage{amsfonts}
\usepackage{amssymb}
\usepackage{amsmath}
\usepackage{amsthm}
\usepackage{tikz}
%-------------------------------------------------------------------------------

%-------------------------------------------------------------------------------
%                     -Mise en forme d'exercices-
%-------------------------------------------------------------------------------
\newtheoremstyle{exostyle}
{\topsep}% espace avant
{\topsep}% espace apres
{}% Police utilisee par le style de thm
{}% Indentation (vide = aucune, \parindent = indentation paragraphe)
{\bfseries}% Police du titre de thm
{.}% Signe de ponctuation apres le titre du thm
{ }% Espace apres le titre du thm (\newline = linebreak)
{\thmname{#1}\thmnumber{ #2}\thmnote{. \normalfont{\textit{#3}}}}% composants du titre du thm : \thmname = nom du thm, \thmnumber = numéro du thm, \thmnote = sous-titre du thm

\theoremstyle{exostyle}
\newtheorem{exercice}{Exercice}

\newenvironment{questions}{
\begin{enumerate}[\hspace{12pt}\bfseries\itshape a.]}{\end{enumerate}
} %mettre un 1 à la place du a si on veut des numéros au lieu de lettres pour les questions 
%-------------------------------------------------------------------------------



%-------------------------------------------------------------------------------
%                    - Racourcis d'écriture -
%-------------------------------------------------------------------------------

% Angles orientés (couples de vecteurs)
\newcommand{\aopp}[2]{(\vec{#1}, \vec{#2})} %Les deuc vecteurs sont positifs
\newcommand{\aopn}[2]{(\vec{#1}, -\vec{#2})} %Le second vecteur est négatif
\newcommand{\aonp}[2]{(-\vec{#1}, \vec{#2})} %Le premier vecteur est négatif
\newcommand{\aonn}[2]{(-\vec{#1}, -\vec{#2})} %Les deux vecteurs sont négatifs

%Ensembles mathématiques
\newcommand{\naturels}{\mathbb{N}} %Nombres naturels
\newcommand{\relatifs}{\mathbb{Z}} %Nombres relatifs
\newcommand{\rationnels}{\mathbb{Q}} %Nombres rationnels
\newcommand{\reels}{\mathbb{R}} %Nombres réels
\newcommand{\complexes}{\mathbb{C}} %Nombres complexes
%-------------------------------------------------------------------------------




%%%<
\usepackage{verbatim}
\usepackage[active,tightpage]{preview}
%\PreviewEnvironment{tikzpicture}
%\setlength\PreviewBorder{5pt}


\author{Olivier}
\date{}
\title{Angles Orientés \& Trigonométrie }
\begin{document}
\maketitle
%\thispagestyle{empty}



\section{Définitions}

\begin{multicols}{2}
	\begin{mydef}
	$a$ et $b$ sont deux nombres ($b$ $\neq$ 0).\pause Le \kw{quotient} de $a$ par $b$ se note $a \div b$ ou $\dfrac{a}{b}$, en écriture fractionnaire.\pause
\end{mydef}

\begin{myex}
	%\begin{itemize}
		%\item 
		Le quotient de 5 par 4 est $\dfrac{5}{4}$, c'est le nombre qui multiplié par 4 donne 5. \pause
		\begin{equation*}
			\dfrac{5}{4} \times 4 = 5
		\end{equation*}

		%\item Le quotient de 2 par 3 est $\dfrac{2}{3}$, c'est le nombre qui multiplié par 3 donne 2. $\dfrac{2}{3} \times 3 = 2 $.
	%\end{itemize}
\end{myex}

\begin{mydef}
	Si $a$ et $b$ sont entiers, alors $\dfrac{a}{b}$ est une \kw{fraction}.\pause $a$ est le\pause \kw{numérateur} et $b$ est le\pause \kw{dénominateur}.	
	
\end{mydef}

\begin{center}
	\includegraphics*[scale=0.5]{def}
\end{center}

\begin{myex}
	$\dfrac{\num{4.2}}{\num{2}}$, $\dfrac{\num{5}}{\num{2.4}}$, $\dfrac{\num{1.3}}{\num{3.7}}$ et $\dfrac{\num{2}}{\num{3}}$ sont toutes des écritures fractionnaires, mais seule $\dfrac{\num{2}}{\num{3}}$ est une fraction.
\end{myex}
\end{multicols}


%\subsection*{Définitions}

%\begin{multicols}{2}
\section{Cercle Trigonométrique}


\begin{center}
\input{./circle2}
\end{center}

\vspace*{2cm}

\section{Formules utiles}

\subsection*{Angles orientés}

\begin{multicols}{2}
\begin{multicols}{2}
%\item Formule de Chasles : 

Opposé d'un angle :
\begin{equation}
-\aopp{u}{v} = \aopp{v}{u}
\label{eq:oppose}
\end{equation}

Formule de Chasles :
\begin{equation}
\aopp{u}{v} + \aopp{v}{w} = \aopp{u}{w} [2\pi]
\label{eq:chasles}
\end{equation}

Remplacement d'un des vecteurs :
\begin{equation}
\aopn{u}{v} = \aonp{u}{v} = \aopp{u}{v} + \pi [2\pi]
\end{equation}

Remplacement des deux vecteurs :
\begin{eqnarray}
\aopp{u}{v} = \aonn{u}{v}
\label{eq:opposes2}
\end{eqnarray}

\end{multicols}
\end{multicols}


\subsection*{Trigonométrie}

\begin{multicols}{2}
	

%Il est intéressant d'apprendre ces quelques formules parce qu'on peut en déduire toutes les autres.
\begin{eqnarray}
-1 \leq \cos{(x)} \leq 1 & \forall x \in \reels \\  
-1 \leq \sin{(x)} \leq 1 & \forall x \in \reels
\end{eqnarray}


\begin{equation}
\cos^2{(x)}+\sin^2{(x)} = 1
\label{cos2plussin2}
\end{equation}

%Le cosinus est une fonction paire :
\begin{eqnarray}
\cos(-x) & = & \cos{(x)}\\
\sin(-x) & = & -\sin{(x)}
\end{eqnarray}

\begin{eqnarray}
\cos(\pi - x) & = & -\cos(x)\\
\sin(\pi - x) & = &\sin(x)
\end{eqnarray}

\begin{eqnarray}
\cos(\pi + x) & = & -\cos(x)\\
\sin(\pi + x) & = & -\sin(x)
\end{eqnarray}

\begin{eqnarray}
\cos(\frac{\pi}{2} - x) & = & \sin(x)\\
\sin(\frac{\pi}{2} - x) & = &\cos(x)
\end{eqnarray}


\begin{eqnarray}{ }
\forall x \in \reels, \forall k \in \relatifs & \cos (x + 2k \pi)  =  \cos{x} \\ 
\forall x \in \reels, \forall k \in \relatifs & \sin (x + 2k \pi)  =  \sin{x}  
\end{eqnarray}


\begin{eqnarray}
(\cos(x) = \cos(a)) & \Leftrightarrow & (x = \pm a \,[2\pi]) \\
(\cos(x) = \cos(b)) & \Leftrightarrow & (x = b \, [2\pi] \; ou\\
& & x = \pi - b \, [2\pi])
\end{eqnarray}

%Dérivée \& primitive :
%\begin{eqnarray}
%\cos'(A) &=& -\sin(A) \\
%\sin'(A) &=& \cos(A) \\
%\int \cos(A)dA &=& \sin(A)+C \\
%\int \sin(A)dA &=& -\cos(A)+C
%\end{eqnarray}

%\begin{equation}
%\cos(A+B) = \cos{A}\cos{B} - \sin{A}\sin{B}
%\label{cosaplusb}
%\end{equation}
%
%Moyen mnémotechnique : on dit que \og le cosinus est sectaire et menteur \fg{} parce que les sinus sont à part et qu'il y a un signe moins.
%
%\begin{equation}
%\sin(A+B) = \sin{A}\cos{B} + \cos{A}\sin{B}
%\label{sinaplusb}
%\end{equation}
%
%Pour le sinus, c'est l'inverse. Les cosinus et les sinus sont mélangés et il y a un signe plus.

\end{multicols}


%\section{Formules d'addition et de différence des arcs}
%L'équation (\ref{cosaplusb}) en remplaçant $B$ par $-B$ et en appliquant (\ref{cosinuspair}) et (\ref{sinusimpair}) :

\begin{equation}
\cos(A-B) = \cos{A}\cos{B} + \sin{A}\sin{B}
\label{cosamoinsb}
\end{equation}

L'équation (\ref{sinaplusb}) en remplaçant $B$ par $-B$ et en appliquant (\ref{cosinuspair}) et (\ref{sinusimpair}) :

\begin{equation}
\sin(A-B) = \sin{A}\cos{B} - \cos{A}\sin{B}
\label{sinamoinsb}
\end{equation}

L'équation (\ref{cosaplusb}) en posant $B=A$, puis en utilisant (\ref{cos2plussin2}) :
\begin{equation}
\cos(2A) = \cos^2{A} - \sin^2{A} = \cos^2{A} - (1-\cos^2{A}) = 2\cos^2A-1
\label{cos2a}
\end{equation}

L'équation (\ref{sinaplusb}) en posant $B=A$ :
\begin{equation}
\sin(2A) = 2\sin{A}\cos{A}
\end{equation}


%\section{Formules de Simpson}

%Ces formules sont utiles (entre autres) pour le calcul intégral. En effet elles permettent par étapes successives d'isoler les sinus et les cosinus et de se débarrasser des puissances, de façon à ce qu'on puisse facilement trouver la primitive.

La demi-somme des équations (\ref{cosaplusb}) et (\ref{cosamoinsb}) donne :
\begin{equation}
\cos{A}\cos{B} = \frac{1}{2}[\cos(A+B)+\cos(A-B)]
\label{cosacosb}
\end{equation}

La demi-différence des équations (\ref{cosamoinsb}) et (\ref{cosaplusb}) donne :
\begin{equation}
\sin{A}\sin{B} = \frac{1}{2}[\cos(A-B)-\cos(A+B)]
\label{sinasinb}
\end{equation}

La demi-somme des équations (\ref{sinaplusb}) et (\ref{sinamoinsb}) donne :
\begin{equation}
\sin{A}\cos{B} = \frac{1}{2}[\sin(A+B) + \sin(A-B)]
\label{sinacosb}
\end{equation}

L'équation (\ref{cosacosb}) en posant $B=A$ sachant que $\cos 0 = 1$ :
\begin{equation}
\cos^2{A} = \frac{1}{2}[1+\cos(2A)]
\end{equation}

L'équation (\ref{sinasinb}) en posant $B=A$ sachant que $\cos 0 = 1$ :
\begin{equation}
\sin^2{A} = \frac{1}{2}[1-\cos(2A)]
\end{equation}

%\section{Exemple d'utilisation}
%Par exemple, pour trouver une primitive de $\sin^3{A}$, on écrit $\sin^3{A} = \sin{A}\sin^2{A} = \sin{A}(\frac{1}{2}[1-\cos(2A)]) = \frac{1}{2}\sin{A} - \frac{1}{2}\sin{A}\cos(2A) = \frac{1}{2}\sin{A} - \frac{1}{4}(\sin(3A) - \sin{A})$. Donc la primitive est $-\frac{1}{2}\cos{A} + \frac{1}{12}\cos(3A) - \frac{1}{4}\cos{A}$.


%\end{multicols}
\end{document}