Ces formules sont utiles (entre autres) pour le calcul intégral. En effet elles permettent par étapes successives d'isoler les sinus et les cosinus et de se débarrasser des puissances, de façon à ce qu'on puisse facilement trouver la primitive.

La demi-somme des équations (\ref{cosaplusb}) et (\ref{cosamoinsb}) donne :
\begin{equation}
\cos{A}\cos{B} = \frac{1}{2}[\cos(A+B)+\cos(A-B)]
\label{cosacosb}
\end{equation}

La demi-différence des équations (\ref{cosamoinsb}) et (\ref{cosaplusb}) donne :
\begin{equation}
\sin{A}\sin{B} = \frac{1}{2}[\cos(A-B)-\cos(A+B)]
\label{sinasinb}
\end{equation}

La demi-somme des équations (\ref{sinaplusb}) et (\ref{sinamoinsb}) donne :
\begin{equation}
\sin{A}\cos{B} = \frac{1}{2}[\sin(A+B) + \sin(A-B)]
\label{sinacosb}
\end{equation}

L'équation (\ref{cosacosb}) en posant $B=A$ sachant que $\cos 0 = 1$ :
\begin{equation}
\cos^2{A} = \frac{1}{2}[1+\cos(2A)]
\end{equation}

L'équation (\ref{sinasinb}) en posant $B=A$ sachant que $\cos 0 = 1$ :
\begin{equation}
\sin^2{A} = \frac{1}{2}[1-\cos(2A)]
\end{equation}