%Il est intéressant d'apprendre ces quelques formules parce qu'on peut en déduire toutes les autres.

\begin{multicols}{2}

\begin{eqnarray}
-1 \leq \cos{(x)} \leq 1 & \forall x \in \reels \\  
-1 \leq \sin{(x)} \leq 1 & \forall x \in \reels 
\end{eqnarray}


%Le cosinus est une fonction paire :
\begin{eqnarray}
\cos(-x) & = & \cos{(x)}\\
\sin(-x) & = & -\sin{(x)}
\end{eqnarray}

\begin{eqnarray}
\cos(\pi - x) & = & -\cos(x)\\
\sin(\pi - x) & = &\sin(x)
\end{eqnarray}

\begin{eqnarray}
\cos(\pi + x) & = & -\cos(x)\\
\sin(\pi + x) & = & -\sin(x)
\end{eqnarray}

\begin{eqnarray}
\cos(\frac{\pi}{2} - x) & = & \sin(x)\\
\sin(\frac{\pi}{2} - x) & = &\cos(x)
\end{eqnarray}

\begin{equation}
\cos^2{(x)}+\sin^2{(x)} = 1\\
%\label{cos2plussin2}
\end{equation}

\end{multicols}


\begin{eqnarray}{ }
\forall x \in \reels, \forall k \in \relatifs & \cos (x + 2k \pi)  =  \cos(x) \\ 
\forall x \in \reels, \forall k \in \relatifs & \sin (x + 2k \pi)  =  \sin(x)  
\end{eqnarray}

\begin{multicols}{2}

\begin{equation}
\cos(a+b) = \cos{a}\cos{b} - \sin(a)\sin{b}
\label{cosaplusb}
\end{equation}
%
%Moyen mnémotechnique : on dit que \og le cosinus est sectaire et menteur \fg{} parce que les sinus sont à part et qu'il y a un signe moins.
%
\begin{equation}
\sin(a+b) = \sin{a}\cos{b} + \cos{a}\sin{b}
\label{sinaplusb}
\end{equation}

\end{multicols}




