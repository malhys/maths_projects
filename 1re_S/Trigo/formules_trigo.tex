%Il est intéressant d'apprendre ces quelques formules parce qu'on peut en déduire toutes les autres.
\begin{eqnarray}
-1 \leq \cos{(x)} \leq 1 & \forall x \in \reels \\  
-1 \leq \sin{(x)} \leq 1 & \forall x \in \reels
\end{eqnarray}


\begin{equation}
\cos^2{(x)}+\sin^2{(x)} = 1
\label{cos2plussin2}
\end{equation}

%Le cosinus est une fonction paire :
\begin{eqnarray}
\cos(-x) & = & \cos{(x)}\\
\sin(-x) & = & -\sin{(x)}
\end{eqnarray}

\begin{eqnarray}
\cos(\pi - x) & = & -\cos(x)\\
\sin(\pi - x) & = &\sin(x)
\end{eqnarray}

\begin{eqnarray}
\cos(\pi + x) & = & -\cos(x)\\
\sin(\pi + x) & = & -\sin(x)
\end{eqnarray}

\begin{eqnarray}
\cos(\frac{\pi}{2} - x) & = & \sin(x)\\
\sin(\frac{\pi}{2} - x) & = &\cos(x)
\end{eqnarray}


\begin{eqnarray}{ }
\forall x \in \reels, \forall k \in \relatifs & \cos (x + 2k \pi)  =  \cos{x} \\ 
\forall x \in \reels, \forall k \in \relatifs & \sin (x + 2k \pi)  =  \sin{x}  
\end{eqnarray}


\begin{eqnarray}
(\cos(x) = \cos(a)) & \Leftrightarrow & (x = \pm a \,[2\pi]) \\
(\cos(x) = \cos(b)) & \Leftrightarrow & (x = b \, [2\pi] \; ou\\
& & x = \pi - b \, [2\pi])
\end{eqnarray}

%Dérivée \& primitive :
%\begin{eqnarray}
%\cos'(A) &=& -\sin(A) \\
%\sin'(A) &=& \cos(A) \\
%\int \cos(A)dA &=& \sin(A)+C \\
%\int \sin(A)dA &=& -\cos(A)+C
%\end{eqnarray}

%\begin{equation}
%\cos(A+B) = \cos{A}\cos{B} - \sin{A}\sin{B}
%\label{cosaplusb}
%\end{equation}
%
%Moyen mnémotechnique : on dit que \og le cosinus est sectaire et menteur \fg{} parce que les sinus sont à part et qu'il y a un signe moins.
%
%\begin{equation}
%\sin(A+B) = \sin{A}\cos{B} + \cos{A}\sin{B}
%\label{sinaplusb}
%\end{equation}
%
%Pour le sinus, c'est l'inverse. Les cosinus et les sinus sont mélangés et il y a un signe plus.
