\begin{multicols}{2}

\begin{enumerate}
\item Un angle orienté est défini par un couple de vecteurs non nuls $(\vec{u}, \vec{v})$

\item Angles particuliers :
\begin{itemize}
	\item l'angle \textbf{nul} : $\hat{0} = \aopp{u}{v} , \forall \vec{u} \neq \vec{0} $  (0\degres, 0 $rad$)
	\item l'angle \textbf{plat} :  $\mathcal{P} = \aopn{u}{u} (180\degres, \pi \:rad $)
	\item l'angle \textbf{droit direct} : $\aopp{i}{j} (90\degres, \frac{\pi}{2} \: rad$)
	\item l'angle \textbf{droit indirect} : $\aopn{i}{j} (90\degres, -\frac{\pi}{2} \: rad$)
\end{itemize}

\item La \textbf{mesure principale} d'un angle $\aopp{u}{v}$ est \textbf{l'unique réel} de l'intervalle $]-\pi, \pi]$.

\item Deux angles orientés sont \textbf{égaux} lorsqu'ils ont \textbf{la même mesure principale, modulo $2\pi$}.

\item \textbf{L'opposé} de $\aopp{u}{v}$ est l'angle $\aopp{v}{u}$. %, on note $$.

\item En remplaçant \textbf{les deux vecteurs par leurs opposés} on ne \textbf{change pas l'angle}. %, \aopp{u}{v} $=$ \aonn{u}{v}.
.
\end{enumerate}
\end{multicols}