L'équation (\ref{cosaplusb}) en remplaçant $B$ par $-B$ et en appliquant (\ref{cosinuspair}) et (\ref{sinusimpair}) :

\begin{equation}
\cos(A-B) = \cos{A}\cos{B} + \sin{A}\sin{B}
\label{cosamoinsb}
\end{equation}

L'équation (\ref{sinaplusb}) en remplaçant $B$ par $-B$ et en appliquant (\ref{cosinuspair}) et (\ref{sinusimpair}) :

\begin{equation}
\sin(A-B) = \sin{A}\cos{B} - \cos{A}\sin{B}
\label{sinamoinsb}
\end{equation}

L'équation (\ref{cosaplusb}) en posant $B=A$, puis en utilisant (\ref{cos2plussin2}) :
\begin{equation}
\cos(2A) = \cos^2{A} - \sin^2{A} = \cos^2{A} - (1-\cos^2{A}) = 2\cos^2A-1
\label{cos2a}
\end{equation}

L'équation (\ref{sinaplusb}) en posant $B=A$ :
\begin{equation}
\sin(2A) = 2\sin{A}\cos{A}
\end{equation}
