\documentclass[12pt,a4paper]{article}

%\usepackage[left=1.5cm,right=1.5cm,top=1cm,bottom=2cm]{geometry}
\usepackage[in, plain]{fullpage}
\usepackage{array}
\usepackage{../../pas-math}
\usepackage{../../moncours}


%\usepackage{pas-cours}
%-------------------------------------------------------------------------------
%          -Packages nécessaires pour écrire en Français et en UTF8-
%-------------------------------------------------------------------------------
\usepackage[utf8]{inputenc}
\usepackage[frenchb]{babel}
\usepackage[T1]{fontenc}
\usepackage{lmodern}
\usepackage{textcomp}



%-------------------------------------------------------------------------------

%-------------------------------------------------------------------------------
%                          -Outils de mise en forme-
%-------------------------------------------------------------------------------
\usepackage{hyperref}
\hypersetup{pdfstartview=XYZ}
%\usepackage{enumerate}
\usepackage{graphicx}
\usepackage{multicol}
\usepackage{tabularx}
\usepackage{multirow}


\usepackage{anysize} %%pour pouvoir mettre les marges qu'on veut
%\marginsize{2.5cm}{2.5cm}{2.5cm}{2.5cm}

\usepackage{indentfirst} %%pour que les premier paragraphes soient aussi indentés
\usepackage{verbatim}
\usepackage{enumitem}
\usepackage[usenames,dvipsnames,svgnames,table]{xcolor}

\usepackage{variations}

%-------------------------------------------------------------------------------


%-------------------------------------------------------------------------------
%                  -Nécessaires pour écrire des mathématiques-
%-------------------------------------------------------------------------------
\usepackage{amsfonts}
\usepackage{amssymb}
\usepackage{amsmath}
\usepackage{amsthm}
\usepackage{tikz}
\usepackage{xlop}
%-------------------------------------------------------------------------------



%-------------------------------------------------------------------------------


%-------------------------------------------------------------------------------
%                    - Mise en forme avancée
%-------------------------------------------------------------------------------

\usepackage{ifthen}
\usepackage{ifmtarg}


\newcommand{\ifTrue}[2]{\ifthenelse{\equal{#1}{true}}{#2}{$\qquad \qquad$}}

%-------------------------------------------------------------------------------

%-------------------------------------------------------------------------------
%                     -Mise en forme d'exercices-
%-------------------------------------------------------------------------------
%\newtheoremstyle{exostyle}
%{\topsep}% espace avant
%{\topsep}% espace apres
%{}% Police utilisee par le style de thm
%{}% Indentation (vide = aucune, \parindent = indentation paragraphe)
%{\bfseries}% Police du titre de thm
%{.}% Signe de ponctuation apres le titre du thm
%{ }% Espace apres le titre du thm (\newline = linebreak)
%{\thmname{#1}\thmnumber{ #2}\thmnote{. \normalfont{\textit{#3}}}}% composants du titre du thm : \thmname = nom du thm, \thmnumber = numéro du thm, \thmnote = sous-titre du thm

%\theoremstyle{exostyle}
%\newtheorem{exercice}{Exercice}
%
%\newenvironment{questions}{
%\begin{enumerate}[\hspace{12pt}\bfseries\itshape a.]}{\end{enumerate}
%} %mettre un 1 à la place du a si on veut des numéros au lieu de lettres pour les questions 
%-------------------------------------------------------------------------------

%-------------------------------------------------------------------------------
%                    - Mise en forme de tableaux -
%-------------------------------------------------------------------------------

\renewcommand{\arraystretch}{1.7}

\setlength{\tabcolsep}{1.2cm}

%-------------------------------------------------------------------------------



%-------------------------------------------------------------------------------
%                    - Racourcis d'écriture -
%-------------------------------------------------------------------------------

% Angles orientés (couples de vecteurs)
\newcommand{\aopp}[2]{(\vec{#1}, \vec{#2})} %Les deuc vecteurs sont positifs
\newcommand{\aopn}[2]{(\vec{#1}, -\vec{#2})} %Le second vecteur est négatif
\newcommand{\aonp}[2]{(-\vec{#1}, \vec{#2})} %Le premier vecteur est négatif
\newcommand{\aonn}[2]{(-\vec{#1}, -\vec{#2})} %Les deux vecteurs sont négatifs

%Ensembles mathématiques
\newcommand{\naturels}{\mathbb{N}} %Nombres naturels
\newcommand{\relatifs}{\mathbb{Z}} %Nombres relatifs
\newcommand{\rationnels}{\mathbb{Q}} %Nombres rationnels
\newcommand{\reels}{\mathbb{R}} %Nombres réels
\newcommand{\complexes}{\mathbb{C}} %Nombres complexes


%Intégration des parenthèses aux cosinus
\newcommand{\cosP}[1]{\cos\left(#1\right)}
\newcommand{\sinP}[1]{\sin\left(#1\right)}


%Probas stats
\newcommand{\stat}{statistique}
\newcommand{\stats}{statistiques}
%-------------------------------------------------------------------------------

%-------------------------------------------------------------------------------
%                    - Mise en page -
%-------------------------------------------------------------------------------

\newcommand{\twoCol}[1]{\begin{multicols}{2}#1\end{multicols}}


\setenumerate[1]{font=\bfseries,label=\textit{\alph*})}
\setenumerate[2]{font=\bfseries,label=\arabic*)}


%-------------------------------------------------------------------------------
%                    - Elements cours -
%-------------------------------------------------------------------------------





%\makeatletter
%\renewcommand*{\@seccntformat}[1]{\csname the#1\endcsname\hspace{0.1cm}}
%\makeatother


%\author{Olivier FINOT}
\date{}


%\newcommand{\disp}{false}

\lhead{Structures de données}
\rhead{O. FINOT}
%
%\rfoot{Page \thepage}
\begin{document}
%\maketitle

\chap[num=1, color=red]{Structures de données linéaires}{Olivier FINOT, \today }

\section{Structures de données linéaires}

\begin{itemize}
	\item Besoin de stocker/organiser plus d'information / d'éléments dans une variable
\end{itemize}

\subsection{Tableau}
	\begin{itemize}
		\item Éléments stockés dans des espaces mémoire contigus;
		\item $\oplus$  Accès indicé aux éléments contenus ($t[i]$);
		\item $\ominus$ Taille fixe.
	\end{itemize}
	

\subsection{Liste chainée}

	\begin{itemize}
		\item \'Eléments non contigus, chaque "case" renvoie vers la suivante;
		\item  $\oplus$ Pas besoin de définir la taille à la création, ajout d'éléments aisé;
		\item $\ominus$ Accès séquentiel, pas indicé.
	\end{itemize}

\section{Recherche dans une structure non triée}

\subsection{Présence d'une valeur dans la structure}

\begin{itemize}
	\item Recherche séquentielle, parcours linéaire de la structure;
	\item Fin du parcours si la valeur recherchée est trouvée;
	\item Si la fin est atteinte alors la valeur n'est pas présente.
	\item complexité : $O(n)$
\end{itemize}

\begin{algorithm}[h!]
	\Entree{$v$ : la valeur recherchée \\ $tab$ : le tableau dans lequel on cherche}
	\Sortie{$vrai$ si $v$ est présent dans $tab$, $faux$ sinon}
	\Deb{n $\leftarrow $ longueur de $tab$ \\
	\Pour{i de 0 à n-1}{
		\Si{$tab[i]$ == $n$}{\Retour{$vrai$}}
	}
	\Retour{$faux$}}
\end{algorithm}

\newpage

\subsection{Recherche d'un minimum/maximum local}

\begin{itemize}
	\item Choix d'une valeur référence (premier élément);
	\item Chaque élément est comparé à la valeur référence;
	\item Si il est plus petit (plus grand) on met à jour la valeur référence.
	\item complexité : $O(n)$
\end{itemize}

	\begin{algorithm}[h!]
		\Entree{$tab$ : le tableau dans lequel on cherche}
		\Sortie{La valeur minimale présente dans le tableau}
		\Deb{n $\leftarrow $ longueur de $tab$ \\
			min $\leftarrow $ $tab[0]$ \\
			\Pour{i de 1 à n-1}{
				\Si{$tab[i]$ < $min$}{min $\leftarrow $ $tab[i]$}
			}
			\Retour{min}}
	\end{algorithm}

\newpage

\section{Recherche dans une structure triée}


\subsection{Présence d'une valeur dans la structure}

	\begin{itemize}
		\item Recherche dichotomique;
		\item Besoin d'un accès indicé (impossible avec une liste chaînée).
		\item complexité : $O(log n)$
	\end{itemize}

	\begin{function}[h!]{search($tab$, $v$, $debut$, $fin$)\\}
		\Entree{$v$ : la valeur recherchée \\ $tab$ : le tableau dans lequel on cherche \\$debut$ : indice de début de la recherche (par défaut 0) \\ $fin$ : indice de fin (par défaut longueur de $tab$)}
		
		\Sortie{$vrai$ si $v$ est présent dans $tab$, $faux$ sinon}
		\Deb{
			
			milieu $\leftarrow$ $(debut + fin) \div 2$
			
			\eSi{fin - debut $\leq$ 1}{\Retour{$faux$}}{
				
				
				\eSi{$tab[milieu] = v$}{\Retour{$vrai$}}{
					\eSi{$t[milieu] > v$}{\Retour{search(tab, v, debut, milieu)}} {\Retour{search(tab, v, milieu, fin)}}
				}
			}		
		}
		 
			
	\end{function}
\subsection{Recherche d'un minimum/maximum local}

\begin{itemize}
	\item Prendre le premier / dernier élément de la structure
	\item complexité : $O(1)$
\end{itemize}


\end{document}