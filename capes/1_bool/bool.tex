\documentclass[12pt,a4paper]{article}

%\usepackage[left=1.5cm,right=1.5cm,top=1cm,bottom=2cm]{geometry}
\usepackage[in, plain]{fullpage}
\usepackage{array}
\usepackage{../../pas-math}
\usepackage{../../moncours}


%\usepackage{pas-cours}
%-------------------------------------------------------------------------------
%          -Packages nécessaires pour écrire en Français et en UTF8-
%-------------------------------------------------------------------------------
\usepackage[utf8]{inputenc}
\usepackage[frenchb]{babel}
\usepackage[T1]{fontenc}
\usepackage{lmodern}
%-------------------------------------------------------------------------------

%-------------------------------------------------------------------------------
%                          -Outils de mise en forme-
%-------------------------------------------------------------------------------
\usepackage{hyperref}
\hypersetup{pdfstartview=XYZ}
\usepackage{enumerate}
\usepackage{graphicx}
\usepackage{multicol}

\usepackage{anysize} %%pour pouvoir mettre les marges qu'on veut
%\marginsize{2.5cm}{2.5cm}{2.5cm}{2.5cm}

\usepackage{indentfirst} %%pour que les premier paragraphes soient aussi indentés
%-------------------------------------------------------------------------------


%-------------------------------------------------------------------------------
%                  -Nécessaires pour écrire des mathématiques-
%-------------------------------------------------------------------------------
\usepackage{amsfonts}
\usepackage{amssymb}
\usepackage{amsmath}
\usepackage{amsthm}
\usepackage{tikz}
%-------------------------------------------------------------------------------

%-------------------------------------------------------------------------------
%                     -Mise en forme d'exercices-
%-------------------------------------------------------------------------------
\newtheoremstyle{exostyle}
{\topsep}% espace avant
{\topsep}% espace apres
{}% Police utilisee par le style de thm
{}% Indentation (vide = aucune, \parindent = indentation paragraphe)
{\bfseries}% Police du titre de thm
{.}% Signe de ponctuation apres le titre du thm
{ }% Espace apres le titre du thm (\newline = linebreak)
{\thmname{#1}\thmnumber{ #2}\thmnote{. \normalfont{\textit{#3}}}}% composants du titre du thm : \thmname = nom du thm, \thmnumber = numéro du thm, \thmnote = sous-titre du thm

\theoremstyle{exostyle}
\newtheorem{exercice}{Exercice}

\newenvironment{questions}{
\begin{enumerate}[\hspace{12pt}\bfseries\itshape a.]}{\end{enumerate}
} %mettre un 1 à la place du a si on veut des numéros au lieu de lettres pour les questions 
%-------------------------------------------------------------------------------



%-------------------------------------------------------------------------------
%                    - Racourcis d'écriture -
%-------------------------------------------------------------------------------

% Angles orientés (couples de vecteurs)
\newcommand{\aopp}[2]{(\vec{#1}, \vec{#2})} %Les deuc vecteurs sont positifs
\newcommand{\aopn}[2]{(\vec{#1}, -\vec{#2})} %Le second vecteur est négatif
\newcommand{\aonp}[2]{(-\vec{#1}, \vec{#2})} %Le premier vecteur est négatif
\newcommand{\aonn}[2]{(-\vec{#1}, -\vec{#2})} %Les deux vecteurs sont négatifs

%Ensembles mathématiques
\newcommand{\naturels}{\mathbb{N}} %Nombres naturels
\newcommand{\relatifs}{\mathbb{Z}} %Nombres relatifs
\newcommand{\rationnels}{\mathbb{Q}} %Nombres rationnels
\newcommand{\reels}{\mathbb{R}} %Nombres réels
\newcommand{\complexes}{\mathbb{C}} %Nombres complexes
%-------------------------------------------------------------------------------




%\makeatletter
%\renewcommand*{\@seccntformat}[1]{\csname the#1\endcsname\hspace{0.1cm}}
%\makeatother


%\author{Olivier FINOT}
\date{}


%\newcommand{\disp}{false}

\lhead{CH1 : Logique booléenne}
\rhead{O. FINOT}
%
%\rfoot{Page \thepage}
\begin{document}
%\maketitle

\chap[num=1, color=red]{\large Logique Booléenne et instructions conditionnelles}{Olivier FINOT, \today }

\section{Contexte et exemple}

	\subsection{Contexte}
	
	\begin{itemize}
		\item Programmes rarement linéaires;
		\item Besoin de faire des choix;
		\item Selon la situation, plusieurs choix sont possibles;
		\item[$\Rightarrow$] Besoin d'indiquer au programme quel choix il doit faire.
	\end{itemize}
	
	
	\subsection{Exemple : Loto}
	
	\begin{itemize}
		\item L'animateur tire un numéro;
		\item Le joueur pose un pion s'il a le numéro;
		\item Si la grille est pleine il a gagné;
\end{itemize}

\section{Instructions conditionnelles}

	\subsection{Expression booléenne}
		\begin{itemize}
			\item Expression qui peut être vraie ou fausse (rien d'autre)
			\item Comparaison 
			\item Structure : \texttt{valeur opérateur valeur}
			\item Opérateurs de comparaison
		\end{itemize}
	
	\subsection{Si...Alors}
	
		\begin{algorithm}[h!]
			\If{condition}{instructions si condition vraie}
		\end{algorithm}
	
		%\begin{itemize}
		%	\item Les instruction ne sont exécutées que si la condition est vraie. 
		%\end{itemize}
	
	\subsection{Si...Alors...Sinon}
	
		\begin{algorithm}[h!]
			\eIf{condition}{instructions si vrai}{instrauctions si faux}
		\end{algorithm}
	\ 
	
	\subsection{Conditions imbriquées}
	
	\ 
		\begin{algorithm}[h!]
			\eIf{condition1}{\eIf{condition2}{instructions si condition1 et condition 2 vraies}{instructions si condition1 vraie et condition2 fausse}}{instrauctions si condition1 faux}
		\end{algorithm}

\section{Conditions complexes}

	\subsection{Principe et présentation}
	
		\begin{itemize}
			\item Dans cas réalistes la condition peut être plus complexe qu'une simple comparaison;
			\item Utilisation d'opérateurs spécifiques;
			\item Notion de table de vérité.
		\end{itemize}
		
	\subsection{Opérateur \textbf{ET}}
	
	\begin{tabular}{@{ }c@{ }@{ }c | c@{}@{ }c@{ }@{ }c@{ }@{ }c@{ }@{}c@{ }}
		A & B & ( & A & $\&$ & B & )\\
		\hline 
		1 & 1 &  & 1 & \textcolor{red}{1} & 1 & \\
		1 & 0 &  & 1 & \textcolor{red}{0} & 0 & \\
		0 & 1 &  & 0 & \textcolor{red}{0} & 1 & \\
		0 & 0 &  & 0 & \textcolor{red}{0} & 0 & \\
	\end{tabular}

	\subsection{Opérateur \textbf{OU}}
	
	\begin{tabular}{@{ }c@{ }@{ }c | c@{}@{ }c@{ }@{ }c@{ }@{ }c@{ }@{}c@{ }}
		A & B & ( & A & $\lor$ & B & )\\
		\hline 
		1 & 1 &  & 1 & \textcolor{red}{1} & 1 & \\
		1 & 0 &  & 1 & \textcolor{red}{1} & 0 & \\
		0 & 1 &  & 0 & \textcolor{red}{1} & 1 & \\
		0 & 0 &  & 0 & \textcolor{red}{0} & 0 & \\
	\end{tabular}

	\subsection{Opérateur \textbf{NON}}
	
	\begin{tabular}{@{ }c | c@{ }@{ }c}
		A & $\sim$ & A\\
		\hline 
		1 & \textcolor{red}{0} & 1\\
		0 & \textcolor{red}{1} & 0\\
	\end{tabular}

	\subsection{Combinaison d'opérateurs}
	
	\begin{itemize}
		\item Possibilité de combiner les opérateurs logiques
	\end{itemize}

	\begin{tabular}{@{ }c@{ }@{ }c | c@{ }@{}c@{}@{ }c@{ }@{ }c@{ }@{ }c@{ }@{}c@{ }}
		A & B & $\sim$ & ( & A & $\&$ & B & )\\
		\hline 
		1 & 1 & \textcolor{red}{0} &  & 1 & 1 & 1 & \\
		1 & 0 & \textcolor{red}{1} &  & 1 & 0 & 0 & \\
		0 & 1 & \textcolor{red}{1} &  & 0 & 0 & 1 & \\
		0 & 0 & \textcolor{red}{1} &  & 0 & 0 & 0 & \\
	\end{tabular}

	\ \\

	\begin{tabular}{@{ }c@{ }@{ }c@{ }@{ }c | c@{}@{}c@{}@{ }c@{ }@{ }c@{ }@{ }c@{ }@{}c@{}@{ }c@{ }@{ }c@{ }@{}c@{ }}
		A & B & C & ( & ( & A & $\&$ & B & ) & $\lor$ & C & )\\
		\hline 
		1 & 1 & 1 &  &  & 1 & 1 & 1 &  & \textcolor{red}{1} & 1 & \\
		1 & 1 & 0 &  &  & 1 & 1 & 1 &  & \textcolor{red}{1} & 0 & \\
		1 & 0 & 1 &  &  & 1 & 0 & 0 &  & \textcolor{red}{1} & 1 & \\
		1 & 0 & 0 &  &  & 1 & 0 & 0 &  & \textcolor{red}{0} & 0 & \\
		0 & 1 & 1 &  &  & 0 & 0 & 1 &  & \textcolor{red}{1} & 1 & \\
		0 & 1 & 0 &  &  & 0 & 0 & 1 &  & \textcolor{red}{0} & 0 & \\
		0 & 0 & 1 &  &  & 0 & 0 & 0 &  & \textcolor{red}{1} & 1 & \\
		0 & 0 & 0 &  &  & 0 & 0 & 0 &  & \textcolor{red}{0} & 0 & \\
	\end{tabular}
\end{document}