\documentclass[xcolor={dvipsnames}]{beamer}
%\usepackage[utf8]{inputenc}
\usetheme{CambridgeUS}

\input{../../utils_maths_beamer}


\usepackage{../../pas-math}
\usepackage{../../moncours_beamer}


\graphicspath{{./img/}}

\title{Introduction à la cryptologie}
\author{O. FINOT}\institute{Lycée S$^t$ Vincent de Paul}


\AtBeginSection[]
{
	\begin{frame}
		\frametitle{Sommaire}
		\tableofcontents[currentsection, hideallsubsections]
	\end{frame} 
}


\AtBeginSubsection[]
{
	\begin{frame}
		\frametitle{Sommaire}
		\tableofcontents[currentsection, currentsubsection]
	\end{frame} 
}

\begin{document}



\begin{frame}
  \titlepage 
\end{frame}

\section{Introduction}

\begin{frame}
\frametitle{Introduction}	
	
	\begin{block}{Historique}
		\begin{itemize}
			\item Utilisé depuis toujours
			\item Cacher, dissimuler des informations essentielles / confidentielles\pause
			\item[$\Rightarrow$] Cryptologie
		\end{itemize}	\pause
	\end{block}
	
	\begin{block}{Aujourd'hui : Sur internet}
		\begin{itemize}
			\item Informations confidentielles
			\item Impôts
			\item Paiements en ligne\pause
			\item[$\Rightarrow$] Données ne doivent pas circuler "en clair"
		\end{itemize}
	\end{block}
	
\end{frame}

\begin{frame}
	\frametitle{Sommaire}
	\tableofcontents[hideallsubsections]
\end{frame} 

\section{Vocabulaire}

\begin{frame}
\frametitle{Définitions I}


	\begin{alertblock}{Cryptologie}
		\begin{itemize}
			\item Science des messages secrets
			\item Cryptographie vs. Cryptanalyse
		\end{itemize}
	\end{alertblock}\pause
	
	\begin{block}{Cryptographie}
		"Art" de transformer un message pour le rendre illisible
	\end{block}\pause
	
	\begin{block}{Cryptananlyse}
		"Art" de rendre un message transformé lisible
	\end{block}
\end{frame}

\begin{frame}
\frametitle{Définitions II}	

	\begin{block}{Chiffrer / Crypter}
		Transformer un message
	\end{block}\pause
	
	\begin{block}{Décrypter}
		Rendre un message lisible
	\end{block}
	

\end{frame}
\section{Chiffrements par substitution monoalphabétique}

\subsection{Principe}

\begin{frame}
\frametitle{Principe des chiffrements par substitution}

\begin{itemize}
	\item Chaque lettre remplacée par une autre
	\item Toujours la même lettre d'arrivée pour une lettre donnée
\end{itemize}

\end{frame}


\subsection{Exemples}

\begin{frame}
\frametitle{Chiffre de César}
\framesubtitle{Présentation}

\begin{block}{Historique}
	\begin{itemize}
		\item Utilisé par César 
		\item Transmission des ordres à ses généraux 
	\end{itemize}
	\end{block}
	
	\begin{block}{Principe}
		\begin{itemize}
			\item Choix d'une distance (26 possibilités)
			\item Remplacement d'une lettre par celle qui se trouve à la distance choisie   
		\end{itemize}
		
		\begin{center}
			\includegraphics[scale=0.2]{cesar}			
		\end{center}

	\end{block}
	
\end{frame}




\begin{frame}
\frametitle{Chiffre de césar}
\framesubtitle{Exemple avec une distance de 3}


	\begin{center}
		\includegraphics[scale=0.2]{cesar}			
	\end{center}

\begin{itemize}
	\item ALEA JACTA EST\pause
	\item[$\Rightarrow$] DOHD MDFWD HVW
\end{itemize}


\end{frame}


\begin{frame}
	\frametitle{Substitution "aléatoire"}
	
	\begin{block}{Principe}
		\begin{itemize}
			\item Pas de distance fixe
			\item Choix d'une lettre de remplacement pour chaque lettre d'origine
		\end{itemize}
	\end{block}
	
	\begin{exampleblock}{Exemple de substitution}
		\begin{itemize}
			\item ABCDEFGHIJKLMNOPQRSTUVWXYZ
			\item AZERTYUIOPQSDFGHJKLMWXCVBN\pause
			\item []
			\item[$\Rightarrow$] SUBSTITUTION devient LWZLMOMWMOGF
		\end{itemize}
		
		
	\end{exampleblock}
	
	
\end{frame}


\subsection{Bilan}

\begin{frame}
\frametitle{Décryptage}

\begin{block}{Analyse fréquentielle}
	\begin{itemize}
		\item Repérer les lettres qui apparaissent le plus
		\item En français : E
	\end{itemize}
\end{block}

\begin{center}
	\includegraphics[scale=0.6]{freq}\footnote{Source wikipédia}
\end{center}	

\end{frame}

\begin{frame}
\frametitle{Avantages / Inconvénients}

\begin{exampleblock}{Avantages}
	\begin{itemize}
		\item Faciles à comprendre
		\item Faciles à utiliser
	\end{itemize}
\end{exampleblock}

\begin{alertblock}{Incovénients}
	\begin{itemize}
		\item Faciles à casser
	\end{itemize}
\end{alertblock}
\end{frame}
\section{Chiffrements par clé}

\subsection{Chiffrements symétriques}

\begin{frame}
\frametitle{Principe du chiffrement symétrique}

\begin{itemize}
	\item Une clé pour chiffrer un message
	\item La même pour décrypter
\end{itemize}

\begin{center}
	\includegraphics[scale=0.5]{sym}
\end{center}
\end{frame}

\begin{frame}
\frametitle{Chiffre de Vigénère}

\begin{block}{Principe}
	\begin{itemize}
		\item Choix d'une clé
		\item Correspondance entre le texte en clair et la clé
	\end{itemize}
	
	\begin{center}
		\includegraphics[scale=0.08]{vigenere}
	\end{center}
\end{block}
\end{frame}

\begin{frame}
	\frametitle{Bilan}
	
	\begin{exampleblock}{Avantage}
		\begin{itemize}
			\item Très sûr (si clé assez longue)			
		\end{itemize}
	\end{exampleblock}
	
	\begin{alertblock}{Inconvénient}
		\begin{itemize}
			\item \'Echange de la clé
		\end{itemize}
	\end{alertblock}
	
\end{frame}

\subsection{Chiffrements Asymétriques}

\begin{frame}
	\frametitle{Principe du chiffrement asymétrique}
	
	\begin{block}{Principe}
	
		\begin{itemize}
			\item 2 clés
			\item 1 clé publique distribuée à tout le monde
			\item 1 clé privée gardée pour soi
		\end{itemize}
	\end{block}
	
	\begin{center}
		\includegraphics[scale=0.5]{asym}
	\end{center}	
	
\end{frame}

\section{Conclusion}


%\begin{columns}[c]
%	\begin{column}{6cm}
%		\begin{itemize}
%			\item $f(x)=2x$
%			\item $g(x)=-x+2$
%		\end{itemize}
%	\end{column}
%	\begin{column}{6cm}
%		\begin{itemize}
%			\item $h(x)=3x-4$
%			\item $i(x)=5$
%		\end{itemize}
%	\end{column}				
%\end{columns}



\end{document}