\documentclass{beamer}
%\usepackage[utf8]{inputenc}
%\usetheme{Warsaw}
%\usetheme{boadilla}
\usetheme{CambridgeUS}
\usecolortheme{beaver}

\usepackage[utf8]{inputenc} 
\usepackage[T1]{fontenc}
\usepackage[upright]{fourier} 
%\usepackage[usenames,dvipsnames]{xcolor}

\usepackage{color}
%\usepackage[dvipsnames]{xcolor}

\usepackage{tkz-kiviat,numprint} 
\usetikzlibrary{arrows}
\usepackage{pgfplots}
\usepackage{pgfplotstable}
%\usepackage{tikz}
\usepackage[francais]{babel}
%\usepgfplotslibrary{external}
% 
%\tikzexternalize 

\pgfplotsset{width=7cm,	compat=1.12}



\AtBeginSection[]
{
	\begin{frame}
		\frametitle{Plan}
		\tableofcontents[currentsection, currentsubsection]
	\end{frame} 
}


\definecolor{mygreen2}{rgb}{0.0, 0.26, 0.15}
\definecolor{mygreen}{rgb}{0.0, 0.42, 0.24}
\definecolor{pastelgray}{rgb}{0.81, 0.81, 0.77}

\title{Conseils pour donner une présentation}
\author{Olivier FINOT}\institute[]{Groupe Scolaire S$^t$ Vincent, S$^t$ Bernard}

\begin{document}
	
	
	
\begin{frame}
	\titlepage
\end{frame}

\section{Introduction}
\begin{frame}
	\frametitle{Pourquoi cette présentation ?}  
	\framesubtitle{}
	
	blabla
	
\end{frame}

\begin{frame}
	\frametitle{Plan}
	
	\tableofcontents
\end{frame}

\section{Objectifs}
\begin{frame}
	\frametitle{Objectifs d'une présentation}
	
	Une présentation ;
	
	\begin{itemize}
		\item Fait passer un message ;
		\item Dresse un bilan d'un travail effectué;
		\item Donne une image de vous ;
		\item Permet de vous mettre en valeur.
		
	\end{itemize}
	
\end{frame}

\section{Discours}
\begin{frame}
	\frametitle{Plan}
	blabla
	
\end{frame}


\section{Diaporama}
\begin{frame}

	blabla

\end{frame}


\end{document}


\begin{frame}
	\frametitle{}  
	\framesubtitle{}	
	
\end{frame}




