\documentclass[a4paper, 11pt]{article}

\author{}


\title{Charte de classe et fonctionnement du cours de mathématiques}

\usepackage{fullpage}
\date{}
\begin{document}
\maketitle	
\thispagestyle{empty}

\vspace*{-1cm}
	
\section{Engagements de l'élève}

\begin{itemize}
	
	\item \`A chaque cours je dois apporter :
		\begin{itemize}
			\item Carnet de liaison
			\item Cahiers de leçon et d'exercices
			\item Manuel (au moins un pour deux)
			\item Copies simples et doubles
			%\item Cahier de brouillon
			\item Le matériel de géométrie si demandé
			
		\end{itemize}
	
	\item Pour que tout le monde puisse travailler dans de bonnes conditions du calme est nécessaire. C'est pourquoi je serai attentif et respecterai le travail de mes camarades et du professeur.
	
	\item Durant un travail sur des exercices ou en groupe, je peux échanger avec mes voisins, mais toujours dans le respect des autres.
	
	\item En classe où à la maison je fais le travail demandé. Si ce n'est pas le cas, j'aurai des exercices supplémentaires à faire.
	
	\item Je tiens mes cahiers correctement et reprend la correction des exercices.
	
	\item En cas d'absence, je récupère ce qui a été fait et le travail à faire.
	
	\item J'apprends régulièrement mon cours. 
\end{itemize}
	

\section{Fonctionnement du cours et évaluations}

\begin{itemize}
	\item Les devoirs surveillés durent la totalité de l'heure, ils peuvent porter sur plusieurs chapitres et seront annoncés à l'avance.
	
	\item Des interrogations rapides auront lieu régulièrement pour vérifier que le cours a été travaillé. Elles ne seront pas annoncées.
	
	\item En cas d'absence le jour d'une évaluation, elle sera rattrapée dès le retour de l'élève.
	
	\item Les sujets de devoirs à la maison seront donnés plusieurs jours à l'avance, vous pouvez travailler dessus à plusieurs mais la rédaction doit être personnelle. Les retards et copies identiques seront sanctionnés.
	
	\item Les cahiers seront relevés au moins une fois par trimestre.
	
	\item Les activités en classe ou en salle informatique pourront donner lieu à une évaluation.
	
	
	
\end{itemize}

\vspace*{0.5cm}
\begin{tabular}{ccr}
	%\hline
	
	Signature de l'élève : & \hspace*{5cm} & Signature d'un parent : \\
	&  &                      
\end{tabular}
	
\end{document}