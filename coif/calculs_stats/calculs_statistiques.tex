\documentclass[12pt,a4paper]{article}

%\usepackage[left=1.5cm,right=1.5cm,top=1cm,bottom=2cm]{geometry}
\usepackage[in, plain]{fullpage}
\usepackage{array}
\usepackage{../../pas-math}
\usepackage{../../moncours}


%\usepackage{pas-cours}
%-------------------------------------------------------------------------------
%          -Packages nécessaires pour écrire en Français et en UTF8-
%-------------------------------------------------------------------------------
\usepackage[utf8]{inputenc}
\usepackage[frenchb]{babel}
\usepackage[T1]{fontenc}
\usepackage{lmodern}
\usepackage{textcomp}



%-------------------------------------------------------------------------------

%-------------------------------------------------------------------------------
%                          -Outils de mise en forme-
%-------------------------------------------------------------------------------
\usepackage{hyperref}
\hypersetup{pdfstartview=XYZ}
%\usepackage{enumerate}
\usepackage{graphicx}
\usepackage{multicol}
\usepackage{tabularx}
\usepackage{multirow}


\usepackage{anysize} %%pour pouvoir mettre les marges qu'on veut
%\marginsize{2.5cm}{2.5cm}{2.5cm}{2.5cm}

\usepackage{indentfirst} %%pour que les premier paragraphes soient aussi indentés
\usepackage{verbatim}
\usepackage{enumitem}
\usepackage[usenames,dvipsnames,svgnames,table]{xcolor}

\usepackage{variations}

%-------------------------------------------------------------------------------


%-------------------------------------------------------------------------------
%                  -Nécessaires pour écrire des mathématiques-
%-------------------------------------------------------------------------------
\usepackage{amsfonts}
\usepackage{amssymb}
\usepackage{amsmath}
\usepackage{amsthm}
\usepackage{tikz}
\usepackage{xlop}
%-------------------------------------------------------------------------------



%-------------------------------------------------------------------------------


%-------------------------------------------------------------------------------
%                    - Mise en forme avancée
%-------------------------------------------------------------------------------

\usepackage{ifthen}
\usepackage{ifmtarg}


\newcommand{\ifTrue}[2]{\ifthenelse{\equal{#1}{true}}{#2}{$\qquad \qquad$}}

%-------------------------------------------------------------------------------

%-------------------------------------------------------------------------------
%                     -Mise en forme d'exercices-
%-------------------------------------------------------------------------------
%\newtheoremstyle{exostyle}
%{\topsep}% espace avant
%{\topsep}% espace apres
%{}% Police utilisee par le style de thm
%{}% Indentation (vide = aucune, \parindent = indentation paragraphe)
%{\bfseries}% Police du titre de thm
%{.}% Signe de ponctuation apres le titre du thm
%{ }% Espace apres le titre du thm (\newline = linebreak)
%{\thmname{#1}\thmnumber{ #2}\thmnote{. \normalfont{\textit{#3}}}}% composants du titre du thm : \thmname = nom du thm, \thmnumber = numéro du thm, \thmnote = sous-titre du thm

%\theoremstyle{exostyle}
%\newtheorem{exercice}{Exercice}
%
%\newenvironment{questions}{
%\begin{enumerate}[\hspace{12pt}\bfseries\itshape a.]}{\end{enumerate}
%} %mettre un 1 à la place du a si on veut des numéros au lieu de lettres pour les questions 
%-------------------------------------------------------------------------------

%-------------------------------------------------------------------------------
%                    - Mise en forme de tableaux -
%-------------------------------------------------------------------------------

\renewcommand{\arraystretch}{1.7}

\setlength{\tabcolsep}{1.2cm}

%-------------------------------------------------------------------------------



%-------------------------------------------------------------------------------
%                    - Racourcis d'écriture -
%-------------------------------------------------------------------------------

% Angles orientés (couples de vecteurs)
\newcommand{\aopp}[2]{(\vec{#1}, \vec{#2})} %Les deuc vecteurs sont positifs
\newcommand{\aopn}[2]{(\vec{#1}, -\vec{#2})} %Le second vecteur est négatif
\newcommand{\aonp}[2]{(-\vec{#1}, \vec{#2})} %Le premier vecteur est négatif
\newcommand{\aonn}[2]{(-\vec{#1}, -\vec{#2})} %Les deux vecteurs sont négatifs

%Ensembles mathématiques
\newcommand{\naturels}{\mathbb{N}} %Nombres naturels
\newcommand{\relatifs}{\mathbb{Z}} %Nombres relatifs
\newcommand{\rationnels}{\mathbb{Q}} %Nombres rationnels
\newcommand{\reels}{\mathbb{R}} %Nombres réels
\newcommand{\complexes}{\mathbb{C}} %Nombres complexes


%Intégration des parenthèses aux cosinus
\newcommand{\cosP}[1]{\cos\left(#1\right)}
\newcommand{\sinP}[1]{\sin\left(#1\right)}


%Probas stats
\newcommand{\stat}{statistique}
\newcommand{\stats}{statistiques}
%-------------------------------------------------------------------------------

%-------------------------------------------------------------------------------
%                    - Mise en page -
%-------------------------------------------------------------------------------

\newcommand{\twoCol}[1]{\begin{multicols}{2}#1\end{multicols}}


\setenumerate[1]{font=\bfseries,label=\textit{\alph*})}
\setenumerate[2]{font=\bfseries,label=\arabic*)}


%-------------------------------------------------------------------------------
%                    - Elements cours -
%-------------------------------------------------------------------------------





%\makeatletter
%\renewcommand*{\@seccntformat}[1]{\csname the#1\endcsname\hspace{0.1cm}}
%\makeatother


%\author{Olivier FINOT}
\date{}
\title{Calculs statistiques }

%\newcommand{\disp}{false}

\lhead{CH1 : Calculs Stats}
\rhead{O. FINOT}
%
%\rfoot{Page \thepage}
\begin{document}
%\maketitle

\chap[num=1, color=red]{Calculs statistiques}{Olivier FINOT, \today }

\begin{myobj}
	Être capable : 
\begin{enumerate}
	\item d'identifier, dans une situation simple, le caractère étudié et sa nature :qualificatif ou quantitatif (révision);
	\item de lire les données d'une série statistique présentées dans un tableau ou graphiquement (révision);
	\item de représenter par un diagramme en bâtons ou en secteurs circulaires une série donnant les valeurs d'une caractère qualitatif (révision);
	\item de lire les données d'un tableau à double entrée donnant des effectifs (révision);
	\item de calculer et interpréter les sommes par lignes ou par colonnes d'un tableau d'effectifs (révision);
	\item de déterminer le maximum, le minimum d'une série numérique;
	\item de calculer des fréquences;
	\item de calculer la moyenne d'une série statistique à partir de la somme des données et du nombre d'éléments dans la série.
	
\end{enumerate}
\end{myobj}
\section{Révisions de CAP1}





\subsection{Tableaux Statistiques}


\begin{myact}{1 : Utiliser des tableaux statistiques (page 5)}
	
	\begin{enumerate}[label=\alph*) ]
		\item 100 personnes ont été interrogées.
		On appelle \kw{population} un ensemble de personnes, d'objets, ... sur lequel porte une enquête statistique ; \kw{l'effectif total} de la population est noté \kw{N} (ici N = 100 personnes). Chaque personne est \kw{un individu}.
		
		\item 35 personnes pratiquent l'activité sport. 35 est \kw{l'effectif de la valeur} sport.
		
		\item Le sujet étudié est l'activité préférée, c'est \kw{le caractère de la population}. 
		
		\item Ce caractère est \kw{qualitatif} car il n'est pas mesurable. Un caractère \kw{quantitatif} est mesurable, exemples : la taille, la masse, l'âge.
	\end{enumerate}
\end{myact}


\begin{myact}{2 : Lire un tableau statistique (page 5)}
	\begin{enumerate}
		\item Le caractère étudié est le \kw{nombre d'heures par jour à regarder la télévision.}
		\item C'est un caractère mesurable, donc il est \kw{quantitatif}.
		\item 5 personnes regardent la télévision entre 2 et 3 heures par jour. L'intervalle $[2;3[$ ($2$ inclus et $3$ exclus) est \kw{une classe}. 
	\end{enumerate}
\end{myact}

\begin{myapp}
	Comment compléter et exploiter un tableau statistique : page 6 
\end{myapp}

\begin{myexos}
	\begin{itemize}
		\item 1, 2 et 3 page 11
		\item problème 1 page 14
	\end{itemize}
	
\end{myexos}

\subsection{Graphiques statistiques}
%\begin{mybox}
%
%		
%
%	 Le \kw{diagramme en secteurs (ou circulaire)} est une représentation adaptée une série à \kw{caractère qualitatif}.
%	
%	 Chaque valeur est représentée par un secteur circulaire dont l'aire et l'angle sont proportionnels à l'effectif $n_i$ (ou à la fréquence $f_i$).
%	
%\end{mybox}	



\end{document}