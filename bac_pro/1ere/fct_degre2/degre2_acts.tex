\newcommand{\actOne}[1]{
	Parmi les fonctions suivantes, désigner par une croix celles qui représentent  une fonction du second degré. Pour celles-cis donner les valeurs de $a$, $b$ et $c$.
	
\begin{small}
	

\begin{center}
	\begin{tabular}{|m{3.1cm}|m{0.2cm}|m{0.2cm}|m{0.2cm}|m{0.2cm}|}
		\hline
		\textbf{Fonction} & \textbf{R} & \textbf{$a$} & \textbf{$b$} & \textbf{$c$} \\ 
		\hline
		\hline
		$f(x)=2x+5 $ & & & & \\
		\hline
		$f(x)=2x^2+3x $ & \ifTrue{#1}{$\times$} & \ifTrue{#1}{$2$} & \ifTrue{#1}{$3$} & \ifTrue{#1}{$0$} \\
		\hline
		$f(x)=x^2-x-1$ & \ifTrue{#1}{$\times$}& \ifTrue{#1}{$1$} & \ifTrue{#1}{$-1$} & \ifTrue{#1}{$-1$} \\
		\hline
		$f(x)=2x^3-3x+1$& & & & \\
		\hline
		$f(x)=3x^2+3$& \ifTrue{#1}{$\times$}& \ifTrue{#1}{$3$} & \ifTrue{#1}{$0$} & \ifTrue{#1}{$3$} \\
		\hline
		$f(x)=-x^2+x-8$& \ifTrue{#1}{$\times$}& \ifTrue{#1}{$-1$} & \ifTrue{#1}{$1$} & \ifTrue{#1}{$-8$} \\
		\hline
		
	\end{tabular}

\end{center}

\end{small}	

}

\newcommand{\actTwo}{
A l'aide de votre calculatrice étudiez les fonctions suivantes :

\begin{multicols}{2}
	\begin{itemize}
		\item $f(x) =  2x^2+3x$ \\
		\item $g(x) =  x^2-x-1$ \\
		\item $h(x) =  3x^2+3$ \\
		\item $i(x) =  -x^2+x-8$ \\
	\end{itemize}
\end{multicols}

Pour chaque fonction :
\begin{enumerate}
	\item Selon la situation, déterminez son minimum ou son maximum.
	\item Dressez son tableau de variations.
	\item Quel lien pouvez vous faire entre les coefficients ($a$, $b$ et $c$) et les variations observées ?
\end{enumerate}

%	A l'aide de votre calculatrice étudiez les fonctions suivantes :
%	
%	\begin{multicols}{2}
%		\begin{align}
%			$f(x)$ & $=$ & $2x^2+3x$ \\
%			$g(x)$ & $=$ & $x^2-x-1$ \\
%			$h(x)$ & $=$ & $3x^2+3$ \\
%			$i(x)$ & $=$ & $-x^2+x-8$ \\
%		\end{align}
%	\end{multicols}
}