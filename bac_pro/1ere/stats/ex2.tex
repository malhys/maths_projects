\begin{myex}


Le 1$^{er}$ novembre 2012, on a relevé le prix du gazole sur 10 points de vente du département du Territoire de Belfort. Les 10 prix rangés dans l'ordre croissant sont :

\begin{center}
	\begin{tabular}{|@{\ }l@{\ } | @{\ }c@{\ } | @{\ }c@{\ } | @{\ }c@{\ } |@{\ }c@{\ } |@{\ }c@{\ } |@{\ }c@{\ }|@{\ }c@{\ }|@{\ }c@{\ }|@{\ }c@{\ }|@{\ }c@{\ }|}
		\hline
		Rang & 1 & 2 & 3 & 4 & \kw{5} & \kw{6} & 7 & 8 & 9& 10 \\ \hline  
		Prix & 1,368 & 1,369 & 1,374 & 1,375 & \kw{1,377} & \kw{1,379} & 1,385 & 1,408 & 1,450 & 1,460 \\ \hline			
	\end{tabular}
\end{center}

Ici $N = 10$ est pair.

La \kw{médiane $Me$} de la série est donc la moyenne entre les $5^e$ et $6^e$ valeurs :

\begin{center}
	$Me = \dfrac{n_5 + n_6}{2} = \dfrac{1,377 + 1,379}{2} = 1,378$
\end{center}
La moitié des prix pratiqués est donc inférieure ou égale à 1,378 €.
\end{myex}