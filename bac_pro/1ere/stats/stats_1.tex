\documentclass[12pt,a4paper]{article}

%\usepackage[left=1.5cm,right=1.5cm,top=1cm,bottom=2cm]{geometry}
\usepackage[in, plain]{fullpage}
\usepackage{array}
\usepackage{../../../pas-math}
\usepackage{../../../moncours}


%\usepackage{pas-cours}
%-------------------------------------------------------------------------------
%          -Packages nécessaires pour écrire en Français et en UTF8-
%-------------------------------------------------------------------------------
\usepackage[utf8]{inputenc}
\usepackage[frenchb]{babel}
\usepackage[T1]{fontenc}
\usepackage{lmodern}
%-------------------------------------------------------------------------------

%-------------------------------------------------------------------------------
%                          -Outils de mise en forme-
%-------------------------------------------------------------------------------
\usepackage{hyperref}
\hypersetup{pdfstartview=XYZ}
\usepackage{enumerate}
\usepackage{graphicx}
\usepackage{multicol}

\usepackage{anysize} %%pour pouvoir mettre les marges qu'on veut
%\marginsize{2.5cm}{2.5cm}{2.5cm}{2.5cm}

\usepackage{indentfirst} %%pour que les premier paragraphes soient aussi indentés
%-------------------------------------------------------------------------------


%-------------------------------------------------------------------------------
%                  -Nécessaires pour écrire des mathématiques-
%-------------------------------------------------------------------------------
\usepackage{amsfonts}
\usepackage{amssymb}
\usepackage{amsmath}
\usepackage{amsthm}
\usepackage{tikz}
%-------------------------------------------------------------------------------

%-------------------------------------------------------------------------------
%                     -Mise en forme d'exercices-
%-------------------------------------------------------------------------------
\newtheoremstyle{exostyle}
{\topsep}% espace avant
{\topsep}% espace apres
{}% Police utilisee par le style de thm
{}% Indentation (vide = aucune, \parindent = indentation paragraphe)
{\bfseries}% Police du titre de thm
{.}% Signe de ponctuation apres le titre du thm
{ }% Espace apres le titre du thm (\newline = linebreak)
{\thmname{#1}\thmnumber{ #2}\thmnote{. \normalfont{\textit{#3}}}}% composants du titre du thm : \thmname = nom du thm, \thmnumber = numéro du thm, \thmnote = sous-titre du thm

\theoremstyle{exostyle}
\newtheorem{exercice}{Exercice}

\newenvironment{questions}{
\begin{enumerate}[\hspace{12pt}\bfseries\itshape a.]}{\end{enumerate}
} %mettre un 1 à la place du a si on veut des numéros au lieu de lettres pour les questions 
%-------------------------------------------------------------------------------



%-------------------------------------------------------------------------------
%                    - Racourcis d'écriture -
%-------------------------------------------------------------------------------

% Angles orientés (couples de vecteurs)
\newcommand{\aopp}[2]{(\vec{#1}, \vec{#2})} %Les deuc vecteurs sont positifs
\newcommand{\aopn}[2]{(\vec{#1}, -\vec{#2})} %Le second vecteur est négatif
\newcommand{\aonp}[2]{(-\vec{#1}, \vec{#2})} %Le premier vecteur est négatif
\newcommand{\aonn}[2]{(-\vec{#1}, -\vec{#2})} %Les deux vecteurs sont négatifs

%Ensembles mathématiques
\newcommand{\naturels}{\mathbb{N}} %Nombres naturels
\newcommand{\relatifs}{\mathbb{Z}} %Nombres relatifs
\newcommand{\rationnels}{\mathbb{Q}} %Nombres rationnels
\newcommand{\reels}{\mathbb{R}} %Nombres réels
\newcommand{\complexes}{\mathbb{C}} %Nombres complexes
%-------------------------------------------------------------------------------




%\makeatletter
%\renewcommand*{\@seccntformat}[1]{\csname the#1\endcsname\hspace{0.1cm}}
%\makeatother


%\author{Olivier FINOT}
\date{}
\title{}

%\newcommand{\disp}{false}

\lhead{CH1 : Stats et représentations graphiques}
\rhead{O. FINOT}
%
%\rfoot{Page \thepage}
\begin{document}
%\maketitle
\chap[num=1, color=red]{Indicateurs statistiques}{Olivier FINOT, \today }

\section{Indicateurs de tendance centrale}


\subsection{Mode}
\begin{mydef}
	Le \kw{mode} (ou la valeur modale) d'une série statistique est la valeur qui a \kw{l'effectif le plus important}.
\end{mydef}


\begin{myillus}

		Courbe représentative de la fonction $f(x) = x^2$ et tableau de variations associé:
	\begin{multicols}{2}

	


	\begin{center}
		\includegraphics[scale=0.6]{./img/carre}
	\end{center}
	
	

	\vspace*{1cm}
	\begin{center}
%		\begin{tikzpicture}
%		\tkzTabInit{$x$/1,$f(x)$/2}{$- \infty$,$0$,$+ \infty$}
%		%\tkzTabLine{,-,z,+}
%		\tkzTabVar{+/$+ \infty$,-/$0$,+/$+ \infty$}
%		\end{tikzpicture}	

		\begin{variations}
			x & \mI & & 0 & & \pI \\
		\filet
			\m{x^2} & \h\pi & \d & 0 & \c & \h\pI \\				
		\end{variations}
	\end{center}
	\end{multicols}
\end{myillus}


\subsection{Médiane}

\begin{mydef}
	La \kw{médiane $Me$} d'une série statistique est le nombre qui \kw{partage la série en deux} séries ayant \kw{le même effectif}.
	
	La moitié (ou 50 \%)  des valeurs de la série sont inférieures ou égales à la médiane et l'autre moitié (50 \%) lui sont supérieures ou égales.
\end{mydef}

\begin{mymeth}
	Pour calculer la valeur $Me$ de la médiane d'une série statistiques :
	\begin{itemize}
		\item ranger les valeurs par ordre croissant (du plus petit grand);
		\item \begin{itemize}
			\item si l'effectif total ($N$) est impair, $Me$ est la $ \left( \dfrac{N+1}{2}\right)  ^e$ valeur de la série.
			\item si $N$ est pair, $Me$ est la moyenne entre les $\left(\dfrac{N}{2}\right)^e$ et  $\left(\dfrac{N}{2} + 1\right)^e$ valeurs.
			
		\end{itemize}
	\end{itemize}
\end{mymeth}

\begin{myex}
	La droite d'ajustement obtenue grâce au tableur passe par le point moyen $G$ dont nous avons calculé les coordonnées.
	\begin{center}
		\includegraphics[scale =0.7]{./img/ex2}		
	\end{center}

\end{myex}

\subsection{Moyenne}

\begin{mydef}
	On note $x_1, x_2, ..., x_p$ les valeurs du caractère étudié et $n_1, n_2, ...,n_p$ les effectifs correspondants.
	
	La \kw{moyenne $\bar{x}$} de la série statistique est $\bar{x} = \dfrac{n_1x_1 + n_2x_2 + ... + n_px_p}{N} = \dfrac{\Sigma n_ix_i}{N} $
	
\end{mydef}

\begin{myex}
	On lance un dé à 6 faces truqué. Une étude statistique donne le tableau suivant :
		\begin{center}
			
			\begin{tabular}{|@{\ }l@{\ }|@{\ }c@{\ }|@{\ }c@{\ }|@{\ }c@{\ }|@{\ }c@{\ }|@{\ }c@{\ }|@{\ }c@{\ }|}
				\hline
				Issue $x_i$ & 1 & 2 & 3 & 4 & 5 & 6 \\\hline
				Probabilité $p_i$ & $0,125$ & $0,125$ & $0,125$ & $0,125$ & $0,2$ & $0,3$ \\ \hline
			\end{tabular}
		
		\end{center}
	
	On s'intéresse à l'événement A : <<le nombre obtenu est pair>>.	On a : 
	
	\begin{align*}
		p(A) &= p_2 + p_4 + p_6 \\
			 &= 0,125 + 0,125 + 0,3 \\
			 &= 0,55
	\end{align*}
	
	La probabilité d'obtenir un nombre pair est de 0,55.
\end{myex}



\begin{mywarning}
	Ne pas confondre moyenne et médiane qui sont deux indicateurs statistiques très différents.\\
	
	
	En 2013 le salaire moyen français était de 7.800 € bruts par mois, alors que le salaire médian était de 1.675 € bruts par mois\footnote{Source : http://www.lefigaro.fr/social/2014/03/13/09010-20140313ARTFIG00167-derriere-le-salaire-moyen-de-fortes-disparites.php}.
\end{mywarning}

\section{Indicateurs de dispersion}

\subsection{\'Etendue}
	
	\begin{mydef}
		L'\kw{étendue $e$} d'une série statistique est la différence entre la plus grande et la plus petite valeur de la série.
	\end{mydef}	

	\begin{myex}
		Les 10 prix rangés par ordre croissant sont :
		\begin{center}
			\begin{tabular}{|@{\ }l@{\ } | @{\ }c@{\ } | @{\ }c@{\ } | @{\ }c@{\ } |@{\ }c@{\ } |@{\ }c@{\ } |@{\ }c@{\ }|@{\ }c@{\ }|@{\ }c@{\ }|@{\ }c@{\ }|@{\ }c@{\ }|}
				\hline
				Rang & 1 & 2 & 3 & 4 & \kw{5} & \kw{6} & 7 & 8 & 9& 10 \\ \hline  
				Prix & 1,368 & 1,369 & 1,374 & 1,375 & \kw{1,377} & \kw{1,379} & 1,385 & 1,408 & 1,450 & 1,460 \\ \hline			
			\end{tabular}
		\end{center}
		
		L'étendue de la série est $e = 1,460 - 1,368 = 0,092$ €.
\end{myex}

\subsection{Quartiles}

\begin{mydef}
	\begin{itemize}
		\item Le \kw{premier quartile $Q_1$}, est la plus petite valeur à laquelle un quart (ou 25 \%) des valeurs sont inférieures ou égales.
		\item Le \kw{troisième quartile $Q_3$}, est la plus petite valeur à laquelle trois quarts (ou 75 \%) des valeurs sont inférieures ou égales.
		\item L'\kw{écart interquartile $Q_3-Q_1$} est la différence entre les 3$^e$ et 1$^er$ quartiles : $Q_3 - Q_1$. Il regroupe au moins 50 \% des effectifs de la série avec un nombre égal de valeurs réparties de part et d'autre de la médiane $Me$.
	\end{itemize}
	 
\end{mydef}	

\begin{mymeth}
	Pour calculer les valeurs $Q_1$ et $Q_3$ des quartiles :
	\begin{itemize}
		\item ranger les valeurs de la série par ordre croissant;
		\item calculer $r_1 = 0,25 \times N$ et $r_3 = 0,75 \times N$;
		\item \begin{itemize}
				\item Si $N$ est un multiple de 4, $Q_1$ est la $r_1^e$ valeur de la série et $Q_3$ est la $r_3^e$ valeur de la série.
				\item Si $N$ n'est pas un multiple de 4, \begin{itemize}
					\item le plus petit entier supérieur à $r_1$ (ou $ \lceil r_1\rceil$) donne le rang de $Q_1$.
					\item le plus petit entier supérieur à $r_3$ (ou $\lceil r_3\rceil$) donne le rang de $Q_3$.
					
				\end{itemize}
			\end{itemize}
	\end{itemize}
\end{mymeth}

\begin{myex}
	
	
	\begin{itemize}		

		\item Dans l'exemple ci-dessus, on a $N = 10$, donc $N$ n'est pas un multiple de 4.
		
		$r_1 = 0,25 \times N = 0,25 \times 10 = 2,5$ et $ r_3 = 0,75 \times N = 0,75 \times 10 = 7,5 $
		

		\item Calcul du \kw{premier quartile $Q_1$} :
		\begin{itemize}
			\item le plus petit entier supérieur à $r_1 = 2,5$ est 3 ;
			\item $Q_1$ correspond à la 3$^e$ valeur de la série : $Q_1 = 1,374$
		\end{itemize}
		
		\item Calcul du \kw{troisième quartile $Q_3$} :
		\begin{itemize}
			\item le plus petit entier supérieur à $r_3 = 7,5$ est 8 ;
			\item $Q_3$ correspond à la 8$^e$ valeur de la série : $Q_3 = 1,408$
		\end{itemize}
		
		25 \% des prix pratiqués sont inférieurs ou égaux à 1,374 € et 75 \% des prix pratiqués sont inférieurs à 1,408 €.
		
		
		\item L'\kw{écart interquartile $Q_3 - Q_1$} vaut $1,408 - 1,374 = 0,034$ €.
	\end{itemize}
\end{myex}


\subsection{\'Ecart type}

\begin{mydef}
	L'\kw{écart type $\sigma$} (sigma), fourni par la calculatrice ou le tableur, mesure la dispersion de la série autour de la moyenne $\bar{x}$. 
	
	Plus l'écart type $\sigma$ est grand, plus les valeurs sont <<\kw{dispersées}>> autour de la moyenne. 
	
	Inversement, plus l'écart type $\sigma$ est grand, plus les valeurs sont <<\kw{resserrées}>> autour de la moyenne.
\end{mydef}	

\begin{myex}
	Ces deux graphiques représentent deux séries de même effectif et de de même moyenne $\bar{x} = 11$.
	
	\includegraphics[scale=0.9, angle=-1.1, origin=c]{img/ex_ecart_type}
	
	$\sigma _A < \sigma _B$ : les valeurs de la série $\mathbf{B}$ sont plus dispersées que celles de la série $\mathbf{A}$ autour de $\bar{x}$. 
\end{myex}
			

\section{Diagrammes en boîte à moustaches}

\begin{mydef}
	Le \kw{diagramme en boîte à moustaches} est un dessin à l'échelle, où la <<\kw{boîte}>> est un rectangle limité par $Q_1$ et $Q_3$, et regroupe donc 50 \% des valeurs.
	
	La médiane $Me$ est repérée par un segment dans le rectangle.
	
	Le minimum $x_{min}$ et le maximum $x_{max}$ correspondent aux extrémités des <<\kw{moustaches}>>.
\end{mydef}

\begin{myex}
	Boite à moustache correspondant à l'exemple :
		\begin{center}
		 \includegraphics[scale=0.67]{img/moustache}
	\end{center}
\end{myex}
\end{document}