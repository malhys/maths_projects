\begin{myex}
	
	
	\begin{itemize}		

		\item Dans l'exemple ci-dessus, on a $N = 10$, donc $N$ n'est pas un multiple de 4.
		
		$r_1 = 0,25 \times N = 0,25 \times 10 = 2,5$ et $ r_3 = 0,75 \times N = 0,75 \times 10 = 7,5 $
		

		\item Calcul du \kw{premier quartile $Q_1$} :
		\begin{itemize}
			\item le plus petit entier supérieur à $r_1 = 2,5$ est 3 ;
			\item $Q_1$ correspond à la 3$^e$ valeur de la série : $Q_1 = 1,374$
		\end{itemize}
		
		\item Calcul du \kw{troisième quartile $Q_3$} :
		\begin{itemize}
			\item le plus petit entier supérieur à $r_3 = 7,5$ est 8 ;
			\item $Q_3$ correspond à la 8$^e$ valeur de la série : $Q_3 = 1,408$
		\end{itemize}
		
		25 \% des prix pratiqués sont inférieurs ou égaux à 1,374 € et 75 \% des prix pratiqués sont inférieurs à 1,408 €.
		
		
		\item L'\kw{écart interquartile $Q_3 - Q_1$} vaut $1,408 - 1,374 = 0,034$ €.
	\end{itemize}
\end{myex}