\documentclass[12pt,a4paper]{article}

%\usepackage[left=1.5cm,right=1.5cm,top=1cm,bottom=2cm]{geometry}
\usepackage[in, plain]{fullpage}
\usepackage{array}
\usepackage{../../../pas-math}
\usepackage{../../../moncours}


%\usepackage{pas-cours}
%-------------------------------------------------------------------------------
%          -Packages nécessaires pour écrire en Français et en UTF8-
%-------------------------------------------------------------------------------
\usepackage[utf8]{inputenc}
\usepackage[frenchb]{babel}
\usepackage[T1]{fontenc}
\usepackage{lmodern}
%-------------------------------------------------------------------------------

%-------------------------------------------------------------------------------
%                          -Outils de mise en forme-
%-------------------------------------------------------------------------------
\usepackage{hyperref}
\hypersetup{pdfstartview=XYZ}
\usepackage{enumerate}
\usepackage{graphicx}
\usepackage{multicol}

\usepackage{anysize} %%pour pouvoir mettre les marges qu'on veut
%\marginsize{2.5cm}{2.5cm}{2.5cm}{2.5cm}

\usepackage{indentfirst} %%pour que les premier paragraphes soient aussi indentés
%-------------------------------------------------------------------------------


%-------------------------------------------------------------------------------
%                  -Nécessaires pour écrire des mathématiques-
%-------------------------------------------------------------------------------
\usepackage{amsfonts}
\usepackage{amssymb}
\usepackage{amsmath}
\usepackage{amsthm}
\usepackage{tikz}
%-------------------------------------------------------------------------------

%-------------------------------------------------------------------------------
%                     -Mise en forme d'exercices-
%-------------------------------------------------------------------------------
\newtheoremstyle{exostyle}
{\topsep}% espace avant
{\topsep}% espace apres
{}% Police utilisee par le style de thm
{}% Indentation (vide = aucune, \parindent = indentation paragraphe)
{\bfseries}% Police du titre de thm
{.}% Signe de ponctuation apres le titre du thm
{ }% Espace apres le titre du thm (\newline = linebreak)
{\thmname{#1}\thmnumber{ #2}\thmnote{. \normalfont{\textit{#3}}}}% composants du titre du thm : \thmname = nom du thm, \thmnumber = numéro du thm, \thmnote = sous-titre du thm

\theoremstyle{exostyle}
\newtheorem{exercice}{Exercice}

\newenvironment{questions}{
\begin{enumerate}[\hspace{12pt}\bfseries\itshape a.]}{\end{enumerate}
} %mettre un 1 à la place du a si on veut des numéros au lieu de lettres pour les questions 
%-------------------------------------------------------------------------------



%-------------------------------------------------------------------------------
%                    - Racourcis d'écriture -
%-------------------------------------------------------------------------------

% Angles orientés (couples de vecteurs)
\newcommand{\aopp}[2]{(\vec{#1}, \vec{#2})} %Les deuc vecteurs sont positifs
\newcommand{\aopn}[2]{(\vec{#1}, -\vec{#2})} %Le second vecteur est négatif
\newcommand{\aonp}[2]{(-\vec{#1}, \vec{#2})} %Le premier vecteur est négatif
\newcommand{\aonn}[2]{(-\vec{#1}, -\vec{#2})} %Les deux vecteurs sont négatifs

%Ensembles mathématiques
\newcommand{\naturels}{\mathbb{N}} %Nombres naturels
\newcommand{\relatifs}{\mathbb{Z}} %Nombres relatifs
\newcommand{\rationnels}{\mathbb{Q}} %Nombres rationnels
\newcommand{\reels}{\mathbb{R}} %Nombres réels
\newcommand{\complexes}{\mathbb{C}} %Nombres complexes
%-------------------------------------------------------------------------------




%\makeatletter
%\renewcommand*{\@seccntformat}[1]{\csname the#1\endcsname\hspace{0.1cm}}
%\makeatother


%\author{Olivier FINOT}
\date{}
\title{}

%\newcommand{\disp}{false}

\lhead{CH5 : \'Equations du second degré}
\rhead{O. FINOT}
%
%\rfoot{Page \thepage}
\begin{document}
%\maketitle
\chap[num=5, color=red]{\'Equations du second degré}{Olivier FINOT, \today }

\section{Résolution d'une équation du second degré}

\begin{mydef}
	Une équation du second degré est une équation du type $ax^2+bx+c=0$, où $a$, $b$ et $c$ sont des nombres quelconques avec $a \neq 0$.
	
	Ce type d'équation possède \kw{zéro, une ou deux solutions}.
\end{mydef}

\begin{mymeth}
	\begin{enumerate}
		\item Pour résoudre une équation du second degré, il faut d'abord calculer \kw{le discriminant $\Delta$} (delta) de l'équation.
		
		On a : 
		\begin{align*}
			\Delta = b^2 - 4ac
		\end{align*}
		
		\item Le \kw{nombre de solutions} de l'équation dépend du \kw{signe de $\Delta$} :
		
		\begin{itemize}
		
			\item Si \kw{$\Delta > 0$}, alors il existe \kw{deux solutions} distinctes ($x_1$ et $x_2$). On a :
			
			\begin{multicols}{2}
				\begin{align*}
					x_1=\dfrac{-b -\sqrt{\Delta}}{2a}
				\end{align*}
				
				\begin{align*}
					x_2=\dfrac{-b +\sqrt{\Delta}}{2a}
				\end{align*}
			\end{multicols}
			
			\item Si \kw{$\Delta = 0$}, alors il existe \kw{une unique solution} ($x_1$). On a :
			
			\begin{align*}
				x_1=\dfrac{-b}{2a}
			\end{align*}
			
			\item Si \kw{$\Delta < 0$}, alors il n'existe \kw{aucune solution}. 
			
		\end{itemize} 
	\end{enumerate}
	
	
	
\end{mymeth}

\begin{mymeth}

\begin{tikzpicture}[node distance=2cm]

\node (start) [startstop] {D\'ebut};
\node (pr1) [process, below of=start] {Caluler le discriminant  $\Delta = b^2-4ac$};
\node (dec1) [decision, below of=pr1, yshift=-0.5cm] {$\Delta = 0 ?$};
\node (dec2) [decision, below of=dec1, yshift=-1cm] {$\Delta > 0 ?$};
\node (pr3) [process, below of=dec2, yshift=-0.5cm] {2 solutions $\dfrac{-b +/- \sqrt{\Delta}}{2a}$};
\node (pr2) [process, left of=pr3, xshift=-2cm] {1 solution $\dfrac{-b}{2a}$ };
\node (pr4) [process, right of=pr3, xshift= 2cm] {Aucune solution };
\node (stop) [startstop, below of=pr3] {Fin};


\draw [arrow] (start) --  (pr1);
\draw [arrow] (pr1) --  (dec1);
\draw [arrow] (dec1) -| node[anchor=east] {oui} (pr2);
\draw [arrow] (dec1) -- node[anchor=east] {non} (dec2);
\draw [arrow] (dec2) -- node[anchor=west] {oui} (pr3);
\draw [arrow] (dec2) -| node[anchor=west] {non} (pr4);
\draw [arrow] (pr2) |-  (stop);
\draw [arrow] (pr3) --  (stop);
\draw [arrow] (pr4) |-  (stop);
\end{tikzpicture}

\end{mymeth}

\section{Signe d'un polynôme du second degré}

\begin{mydef}
	Un polynôme du second degré (ou trinôme) est une expression de la forme $ax^2+bx+c$ ($a \neq 0$).
\end{mydef}
\end{document}