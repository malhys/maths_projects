\documentclass[12pt,a4paper]{article}

%\usepackage[left=1.5cm,right=1.5cm,top=1cm,bottom=2cm]{geometry}
\usepackage[in, plain]{fullpage}
\usepackage{array}
\usepackage{../../../pas-math}
\usepackage{../../../moncours}



%\usepackage{pas-cours}
%-------------------------------------------------------------------------------
%          -Packages nécessaires pour écrire en Français et en UTF8-
%-------------------------------------------------------------------------------
\usepackage[utf8]{inputenc}
\usepackage[frenchb]{babel}
\usepackage[T1]{fontenc}
\usepackage{lmodern}
%-------------------------------------------------------------------------------

%-------------------------------------------------------------------------------
%                          -Outils de mise en forme-
%-------------------------------------------------------------------------------
\usepackage{hyperref}
\hypersetup{pdfstartview=XYZ}
\usepackage{enumerate}
\usepackage{graphicx}
\usepackage{multicol}

\usepackage{anysize} %%pour pouvoir mettre les marges qu'on veut
%\marginsize{2.5cm}{2.5cm}{2.5cm}{2.5cm}

\usepackage{indentfirst} %%pour que les premier paragraphes soient aussi indentés
%-------------------------------------------------------------------------------


%-------------------------------------------------------------------------------
%                  -Nécessaires pour écrire des mathématiques-
%-------------------------------------------------------------------------------
\usepackage{amsfonts}
\usepackage{amssymb}
\usepackage{amsmath}
\usepackage{amsthm}
\usepackage{tikz}
%-------------------------------------------------------------------------------

%-------------------------------------------------------------------------------
%                     -Mise en forme d'exercices-
%-------------------------------------------------------------------------------
\newtheoremstyle{exostyle}
{\topsep}% espace avant
{\topsep}% espace apres
{}% Police utilisee par le style de thm
{}% Indentation (vide = aucune, \parindent = indentation paragraphe)
{\bfseries}% Police du titre de thm
{.}% Signe de ponctuation apres le titre du thm
{ }% Espace apres le titre du thm (\newline = linebreak)
{\thmname{#1}\thmnumber{ #2}\thmnote{. \normalfont{\textit{#3}}}}% composants du titre du thm : \thmname = nom du thm, \thmnumber = numéro du thm, \thmnote = sous-titre du thm

\theoremstyle{exostyle}
\newtheorem{exercice}{Exercice}

\newenvironment{questions}{
\begin{enumerate}[\hspace{12pt}\bfseries\itshape a.]}{\end{enumerate}
} %mettre un 1 à la place du a si on veut des numéros au lieu de lettres pour les questions 
%-------------------------------------------------------------------------------



%-------------------------------------------------------------------------------
%                    - Racourcis d'écriture -
%-------------------------------------------------------------------------------

% Angles orientés (couples de vecteurs)
\newcommand{\aopp}[2]{(\vec{#1}, \vec{#2})} %Les deuc vecteurs sont positifs
\newcommand{\aopn}[2]{(\vec{#1}, -\vec{#2})} %Le second vecteur est négatif
\newcommand{\aonp}[2]{(-\vec{#1}, \vec{#2})} %Le premier vecteur est négatif
\newcommand{\aonn}[2]{(-\vec{#1}, -\vec{#2})} %Les deux vecteurs sont négatifs

%Ensembles mathématiques
\newcommand{\naturels}{\mathbb{N}} %Nombres naturels
\newcommand{\relatifs}{\mathbb{Z}} %Nombres relatifs
\newcommand{\rationnels}{\mathbb{Q}} %Nombres rationnels
\newcommand{\reels}{\mathbb{R}} %Nombres réels
\newcommand{\complexes}{\mathbb{C}} %Nombres complexes
%-------------------------------------------------------------------------------



%\makeatletter
%\renewcommand*{\@seccntformat}[1]{\csname the#1\endcsname\hspace{0.1cm}}
%\makeatother


%\author{Olivier FINOT}
\date{}
\title{}

\newcommand{\disp}{true}

\lhead{CH3 : Suites numériques}
\rhead{O. FINOT}
%
%\rfoot{Page \thepage}
\begin{document}
%\maketitle
\chap[num=3, color=red]{Suites Numériques}{Olivier FINOT, \today }

\section{Suite numérique}

\begin{mydef}
	\begin{itemize}
		\item Une \kw{suite numérique} est constituée de \kw{plusieurs nombres rangés dans un certain ordre}.
		\item Ces nombres sont les \kw{termes} de la suite.
		\item $u_1$ est le premier terme de la suite, $u_2$ le deuxième, $u_n$ est le n-ième. Le terme suivant est noté $u_{n+1}$.
	\end{itemize}
	
\end{mydef}

\begin{myex}
	On considère le prix d'un litre de gazole relevé dans une même station au premier janvier entre 1999 et 2008.
	\begin{align*}
	0,62 \:;\: 0,95 \:;\: 0,82 \:; \:0,78 \:;\: 0,81 \:;\: 0,80 \:;\: 0,92 \:;\: 1,05 \:;\: 1,01 \:; \:1,20
	\end{align*}
	
	Le premier terme est $0,62$ ; le deuxième terme est $0,95$ ; le troisième est $0,82$ , ...
	On a
	$u_1=0,62$, $u_2=0,95$, $u_3=0,82$ , ...
\end{myex}

\section{Suite arithmétique}

\subsection{Définition}
\begin{mydef}
	Une \kw{suite arithmétique} est une suite de nombres, où chaque terme, à partir du deuxième est obtenu en ajoutant au précédent un même nombre, la \kw{raison} de la suite (notée $r$).	
	On note :
	\kw{{\LARGE \begin{align*}
		u_{n+1} = u_n + r 
	\end{align*}}}
\end{mydef}

\begin{myprop}
	La différence entre \kw{deux termes consécutifs} quelconques d'une suite arithmétique est constante (c'est sa raison).
\end{myprop}

\subsection{Identification d'une suite arithmétique}
\begin{mymeth}
	Peur prouver qu'une suite numérique est une suite arithmétique, il faut vérifier que la différence entre deux termes consécutifs est constante.
	
	Pour \kw{chaque couple de termes consécutifs}, on calcule \kw{leur différence} ($u_2 - u_1$, $u_3 - u_2$, $u_4-u_3$, ... ). Si le \kw{résultat est toujours identique}, la suite est arithmétique.

\end{mymeth}

\begin{myexs}
	\begin{enumerate}
		\item On considère la suite : $10,6$ ; $14,4$ ; $18,2$ ; $22$ ; $25,8$ ; $29,6$.
		
		On a $14,4-10,6=3,8$ ; $18,2 -14,4=3,8$ ; $22 -18,8=3,8$ ; $25,8-22=3,8$.
		La suite est donc \kw{arithmétique}, et sa \kw{raison est $3,8$}.
		
		\item On considère la suite : $12$ ; $9$ ; $6$ ; $3$ ; $0$ ; $-2$ ; $-5$.
		
		On a : $9-2=-3$  ; $6-9=-3$ ; $3-6=-3$ ; $0-3=-3$ ; \kw{$-2-0=-2$} ; $-5-(-2)=-3$.
		Le résultat n'est pas toujours le même donc la suite n'est pas arithmétique.
	\end{enumerate}
\end{myexs}

\section{Suite géométrique}

\subsection{Définition}
\begin{mydef}
	Une \kw{suite géométrique} est une suite de nombres, où chaque terme, à partir du deuxième est obtenu en multipliant le précédent par un même nombre, la \kw{raison} de la suite (notée $q$).	
	On note
	\kw{{\LARGE \begin{align*}
			u_{n+1} = u_n \times q 
			\end{align*}}}
\end{mydef}

\begin{myprop}
	Le rapport entre \kw{deux termes consécutifs} quelconques d'une suite géométrique est constante (c'est sa raison).
\end{myprop}

\subsection{Identification d'une suite géométrique}
\begin{mymeth}
	Peur prouver qu'une suite numérique est une suite géométrique, il faut vérifier que le rapport entre deux termes consécutifs quelconques est constant.
	
	Pour \kw{chaque couple de termes consécutifs}, on calcule \kw{leur rapport} ($\dfrac{u_2}{u_1}$, $\dfrac{u_3}{u_2}$, $\dfrac{u_4}{u_3}$, ...). Si le \kw{résultat est toujours identique}, la suite est géométrique.
	
\end{mymeth}

\begin{myexs}
	\begin{enumerate}
		\item On considère la suite : $200$ ; $160$ ; $128$ ; $102,4$ ; $81,92$.
		
		On a $\dfrac{160}{200}=0,8$ ; $\dfrac{128}{160}=0,8$ ; $\dfrac{102,4}{128}=0,8$ ; $\dfrac{81,92}{102,4}=0,8$
		La suite est donc \kw{géométrique}, et sa \kw{raison est $0,8$}.
		
		\item On considère la suite : $8$ ; $16$ ; $32$ ; $64$ ; $128$ ; $255$.
		
		On a $\dfrac{16}{8}=2$ ; $\dfrac{32}{16}=2$ ; $\dfrac{64}{32}=2$ ; \kw{$\dfrac{128}{64}\approx 1,99$}.
		Le résultat n'est pas toujours le même donc la suite n'est pas géométrique.
	\end{enumerate}
\end{myexs}

\end{document}
