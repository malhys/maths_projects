\documentclass[12pt,a4paper]{article}

%\usepackage[left=1.5cm,right=1.5cm,top=1cm,bottom=2cm]{geometry}
\usepackage{fullpage}

%-------------------------------------------------------------------------------
%          -Packages nécessaires pour écrire en Français et en UTF8-
%-------------------------------------------------------------------------------
\usepackage[utf8]{inputenc}
\usepackage[frenchb]{babel}
\usepackage[T1]{fontenc}
\usepackage{lmodern}
%-------------------------------------------------------------------------------

%-------------------------------------------------------------------------------
%                          -Outils de mise en forme-
%-------------------------------------------------------------------------------
\usepackage{hyperref}
\hypersetup{pdfstartview=XYZ}
\usepackage{enumerate}
\usepackage{graphicx}
\usepackage{multicol}

\usepackage{anysize} %%pour pouvoir mettre les marges qu'on veut
%\marginsize{2.5cm}{2.5cm}{2.5cm}{2.5cm}

\usepackage{indentfirst} %%pour que les premier paragraphes soient aussi indentés
%-------------------------------------------------------------------------------


%-------------------------------------------------------------------------------
%                  -Nécessaires pour écrire des mathématiques-
%-------------------------------------------------------------------------------
\usepackage{amsfonts}
\usepackage{amssymb}
\usepackage{amsmath}
\usepackage{amsthm}
\usepackage{tikz}
%-------------------------------------------------------------------------------

%-------------------------------------------------------------------------------
%                     -Mise en forme d'exercices-
%-------------------------------------------------------------------------------
\newtheoremstyle{exostyle}
{\topsep}% espace avant
{\topsep}% espace apres
{}% Police utilisee par le style de thm
{}% Indentation (vide = aucune, \parindent = indentation paragraphe)
{\bfseries}% Police du titre de thm
{.}% Signe de ponctuation apres le titre du thm
{ }% Espace apres le titre du thm (\newline = linebreak)
{\thmname{#1}\thmnumber{ #2}\thmnote{. \normalfont{\textit{#3}}}}% composants du titre du thm : \thmname = nom du thm, \thmnumber = numéro du thm, \thmnote = sous-titre du thm

\theoremstyle{exostyle}
\newtheorem{exercice}{Exercice}

\newenvironment{questions}{
\begin{enumerate}[\hspace{12pt}\bfseries\itshape a.]}{\end{enumerate}
} %mettre un 1 à la place du a si on veut des numéros au lieu de lettres pour les questions 
%-------------------------------------------------------------------------------



%-------------------------------------------------------------------------------
%                    - Racourcis d'écriture -
%-------------------------------------------------------------------------------

% Angles orientés (couples de vecteurs)
\newcommand{\aopp}[2]{(\vec{#1}, \vec{#2})} %Les deuc vecteurs sont positifs
\newcommand{\aopn}[2]{(\vec{#1}, -\vec{#2})} %Le second vecteur est négatif
\newcommand{\aonp}[2]{(-\vec{#1}, \vec{#2})} %Le premier vecteur est négatif
\newcommand{\aonn}[2]{(-\vec{#1}, -\vec{#2})} %Les deux vecteurs sont négatifs

%Ensembles mathématiques
\newcommand{\naturels}{\mathbb{N}} %Nombres naturels
\newcommand{\relatifs}{\mathbb{Z}} %Nombres relatifs
\newcommand{\rationnels}{\mathbb{Q}} %Nombres rationnels
\newcommand{\reels}{\mathbb{R}} %Nombres réels
\newcommand{\complexes}{\mathbb{C}} %Nombres complexes
%-------------------------------------------------------------------------------





\author{}
\date{}
\title{Programme de mathématiques seconde professionnelle}

%\newcommand{\disp}{false}

\begin{document}
\maketitle

\section{Statistiques et probabilités}

\subsection{Statistiques et représentations graphiques }\label{ch:2:stats_base}

\begin{multicols}{2}
	

\subsubsection*{Compétences}

\begin{enumerate}
	\item Organiser des données statistiques à l'aide d'une représentation adaptée
	\item Utiliser une calculatrice ou un tableur
	\item Extraire des informations d'une représentation d'une série statistique
\end{enumerate}

\subsubsection*{Connaissances}

\begin{enumerate}
	\item Vocabulaire de base (Population, individu, Recensement, Sondage, échantillon) 
	\item Qualifier le caractère d'une propriété (qualitatif, quantitatif, discret, continu, classes, amplitude)
	\item Savoir calculer une fréquence
	\item Utilisations des différents types de diagrammes	
\end{enumerate}

\end{multicols}
\subsection{Indicateurs statistiques}\label{ch:2:indicateurs}

%\subsubsection*{Pré-requis}
%	Statistiques de base (Chapitre \ref{ch:2:stats_base} )

\begin{multicols}{2}	
\subsubsection*{Compétences}
	\begin{enumerate}
		\item Comparer les indicateurs de tendance centrale d'une même série statistique obtenus à l'aide d'une calculatrice ou d'un tableur
		\item Interpréter les résultats
		\item Comparer deux séries statistiques à l'aide d'indicateurs de tendance centrale et de dispersion
	\end{enumerate}

\subsubsection*{Connaissances}
	\begin{enumerate}
		\item Moyenne (arithmétique et pondérée)
		\item Médiane
		\item Indicateurs de dispersion (étendue, quartiles)
	\end{enumerate}
\end{multicols}

\subsection{Fluctuations d'une fréquence, probabilités}

\begin{multicols}{2}
\subsubsection*{Compétences}
\begin{enumerate}
	\item Expérimenter la prise d'échantillons aléatoires de taille $n$ fixée, extraits d'une population où la fréquence $p$ relative à un caractère est connu
	\item Déterminer l'étendue des fréquences de la série d'échantillons obtenue
\end{enumerate}

\subsubsection*{Connaissances}
\begin{enumerate}
	\item Tirage aléatoire  et avec remise de $n$ éléments dans une population où la fréquence $p$ relative à un caractère est connue.
	\item Fluctuation d'une fréquence relative à un caractère, sur des échantillons de taille $n$ fixée
\end{enumerate}
\end{multicols}


\subsection{Probabilités}
\begin{multicols}{2}
	\subsubsection*{Compétences}
	\begin{enumerate}
		\item Évaluer la probabilité d'un évènement à partir des fréquences
		\item Évaluer la probabilité d'un évènement dans le cadre d'une situation aléatoire simple
		\item Faire preuve d'esprit critique face à une situation aléatoire simple
		\end{enumerate}
		
		\subsubsection*{Connaissances}
		\begin{enumerate}
			\item Stabilisation relative des fréquences vers la probabilité de l'évènement quand $n$ augmente
			\end{enumerate}
\end{multicols}

\section{Algèbre et analyse}

\subsection{Information chiffrée, proportionnalité}
\begin{multicols}{2}
	\subsubsection*{Compétences}
	\begin{enumerate}
		\item Reconnaître que deux suites de nombres sont proportionnelles
		\item Résoudre un problème dans une situation de proportionnalité clairement identifiée
		\item Utiliser des pourcentages dans des situations issues de la vie courante et professionnelle
		\item Utiliser les TIC pour traiter des problèmes de proportionnalité
	\end{enumerate}
	
	\subsubsection*{Connaissances}
	\begin{enumerate}
		\item Suites de nombres proportionnelles
		\item Pourcentages, taux d'évolution
		\item \'Echelles
		\item Indices simples
		\item Proportions
		\item Représentation graphique d'une situation de proportionnalité
	\end{enumerate}
\end{multicols}

\subsection{\'Equations et inéquations}
\begin{multicols}{2}
	\subsubsection*{Compétences}
	\begin{enumerate}
		\item Rechercher et organiser l'information
		\item Traduire un problème à l'aide d'une équation ou d'une inéquation
		\item La résoudre
		\item Critiquer le résultat, rendre compte
		\item Choisir une méthode de résolution adaptée au problème (algébrique, graphique, informatique)
	\end{enumerate}
	
	\subsubsection*{Connaissances}
	Méthodes de résolution
	\begin{enumerate}
		\item d'une équation du premier degré à une inconnue
		\item d'une inéquation du premier degré à une inconnue
		\item d'un système de deux équations du premier degré à deux inconnues
	\end{enumerate}
\end{multicols}

\subsection{Notion de fonctions}
\begin{multicols}{2}
	\subsubsection*{Compétences}
	\begin{enumerate}
		\item Utiliser une calculatrice ou un tableur pour obtenir, sur un intervalle
		\begin{itemize}
			\item l'image d'un nombre réel par une fonction donnée (valeur exacte ou arrondie)
			\item un tableau de valeurs d'une fonction donnée (valeur exacte ou non)
			\item la représentation graphique d'une fonction donnée
		\end{itemize}
		\item Exploiter une représentation graphique d'une fonction sur un intervalle donné pour obtenir
		\begin{itemize}
			\item l'image d'un nombre réel par une fonction donnée
			\item un tableau de valeurs d'une fonction donnée
		\end{itemize}
		\item Décrire les variations d'une fonction avec un vocabulaire adapté ou un tableau de variation
	\end{enumerate}
	
	\subsubsection*{Connaissances}
	\begin{enumerate}
		\item Notion d'intervalle
		\item Vocabulaire de base
		\begin{itemize}
			\item image
			\item antécédent 
			\item croissance
			\item décroissance
			\item minimum
			\item maximum
		\end{itemize}
	
	\end{enumerate}
\end{multicols}

\subsection{Fonctions affines}
\begin{multicols}{2}
	\subsubsection*{Compétences}
	\begin{enumerate}
		\item Représenter une fonction affine
		\item Déterminer le sens de variation d'une fonction affine
		\item Déterminer l'expression algébrique d'une fonction affine à partir de la donnée de deux nombres et de leurs images
		\item Déterminer par la calcul si un point M du plan appartient ou non à une droite d'équation donnée
	\end{enumerate}
	
	\subsubsection*{Connaissances}
	
	\begin{enumerate}
		
		\item Fonction affine
		\begin{itemize}
			\item ordonnée à l'origine, coefficient directeur
			\item sens de variation
			\item représentation graphique
			\item cas particulier, fonction linéaire (lien avec la proportionnalité)
			
		\end{itemize}
		\item \'Equation de droite de la forme $y=ax+b$
	\end{enumerate}
\end{multicols}

\subsection{fonctions de référence}
\begin{multicols}{2}
	\subsubsection*{Compétences}
	\begin{enumerate}
		\item 
		\item 
	\end{enumerate}
	
	\subsubsection*{Connaissances}
	\begin{enumerate}
		\item 
		\item 
	\end{enumerate}
\end{multicols}
\end{document}

\subsection{}
\begin{multicols}{2}
	\subsubsection*{Compétences}
	\begin{enumerate}
		\item 
		\item 
	\end{enumerate}
	
	\subsubsection*{Connaissances}
	\begin{enumerate}
		\item 
		\item 
	\end{enumerate}
\end{multicols}


