\documentclass[12pt,a4paper]{article}

%\usepackage[left=1.5cm,right=1.5cm,top=1cm,bottom=2cm]{geometry}
\usepackage[in, plain]{fullpage}
\usepackage{array}
\usepackage{../../../pas-math}
\usepackage{../../../moncours}


%\usepackage{pas-cours}
%-------------------------------------------------------------------------------
%          -Packages nécessaires pour écrire en Français et en UTF8-
%-------------------------------------------------------------------------------
\usepackage[utf8]{inputenc}
\usepackage[frenchb]{babel}
\usepackage[T1]{fontenc}
\usepackage{lmodern}
%-------------------------------------------------------------------------------

%-------------------------------------------------------------------------------
%                          -Outils de mise en forme-
%-------------------------------------------------------------------------------
\usepackage{hyperref}
\hypersetup{pdfstartview=XYZ}
\usepackage{enumerate}
\usepackage{graphicx}
\usepackage{multicol}

\usepackage{anysize} %%pour pouvoir mettre les marges qu'on veut
%\marginsize{2.5cm}{2.5cm}{2.5cm}{2.5cm}

\usepackage{indentfirst} %%pour que les premier paragraphes soient aussi indentés
%-------------------------------------------------------------------------------


%-------------------------------------------------------------------------------
%                  -Nécessaires pour écrire des mathématiques-
%-------------------------------------------------------------------------------
\usepackage{amsfonts}
\usepackage{amssymb}
\usepackage{amsmath}
\usepackage{amsthm}
\usepackage{tikz}
%-------------------------------------------------------------------------------

%-------------------------------------------------------------------------------
%                     -Mise en forme d'exercices-
%-------------------------------------------------------------------------------
\newtheoremstyle{exostyle}
{\topsep}% espace avant
{\topsep}% espace apres
{}% Police utilisee par le style de thm
{}% Indentation (vide = aucune, \parindent = indentation paragraphe)
{\bfseries}% Police du titre de thm
{.}% Signe de ponctuation apres le titre du thm
{ }% Espace apres le titre du thm (\newline = linebreak)
{\thmname{#1}\thmnumber{ #2}\thmnote{. \normalfont{\textit{#3}}}}% composants du titre du thm : \thmname = nom du thm, \thmnumber = numéro du thm, \thmnote = sous-titre du thm

\theoremstyle{exostyle}
\newtheorem{exercice}{Exercice}

\newenvironment{questions}{
\begin{enumerate}[\hspace{12pt}\bfseries\itshape a.]}{\end{enumerate}
} %mettre un 1 à la place du a si on veut des numéros au lieu de lettres pour les questions 
%-------------------------------------------------------------------------------



%-------------------------------------------------------------------------------
%                    - Racourcis d'écriture -
%-------------------------------------------------------------------------------

% Angles orientés (couples de vecteurs)
\newcommand{\aopp}[2]{(\vec{#1}, \vec{#2})} %Les deuc vecteurs sont positifs
\newcommand{\aopn}[2]{(\vec{#1}, -\vec{#2})} %Le second vecteur est négatif
\newcommand{\aonp}[2]{(-\vec{#1}, \vec{#2})} %Le premier vecteur est négatif
\newcommand{\aonn}[2]{(-\vec{#1}, -\vec{#2})} %Les deux vecteurs sont négatifs

%Ensembles mathématiques
\newcommand{\naturels}{\mathbb{N}} %Nombres naturels
\newcommand{\relatifs}{\mathbb{Z}} %Nombres relatifs
\newcommand{\rationnels}{\mathbb{Q}} %Nombres rationnels
\newcommand{\reels}{\mathbb{R}} %Nombres réels
\newcommand{\complexes}{\mathbb{C}} %Nombres complexes
%-------------------------------------------------------------------------------




%\makeatletter
%\renewcommand*{\@seccntformat}[1]{\csname the#1\endcsname\hspace{0.1cm}}
%\makeatother


%\author{Olivier FINOT}
\date{}
\title{}

%\newcommand{\disp}{false}

%
%\rfoot{Page \thepage}
\begin{document}
	%\maketitle
	\chap[num=6, color=red]{Fonctions affines}{Olivier FINOT, \today }


\section{Fonctions affines, fonctions linéaires}


\begin{mydefs}
	
$a$ et $b$ sont des nombres quelconques ; la fonction qui à tout nombre $x$, associe le nombre \kw{$ax+b$}, est une \kw{fonction affine}.\\
	
	
Cas particuliers :
\begin{itemize}
	\item Si \kw{$b = 0$}, la fonction est \kw{linéaire}.
	\item Si \kw{$a=0$}, la fonction est \kw{constante}.
\end{itemize}
\end{mydefs}

\begin{myexs}
	On considère les fonctions $f,g,h$ et $i$ :
	\begin{multicols}{2}
		\begin{itemize}
			\item $f(x)=2x$
			\item $g(x)=-x+2$
			\item $h(x)=3x-4$
			\item $i(x)=5$
		\end{itemize}
	\end{multicols}
	
	\begin{itemize}
		\item $f$ est une fonction \kw{linéaire (On a $a=2$ et $b=0$)}.
		\item $g$ est une fonction \kw{affine (On a $a=-1$ et $b=2$)}.
		\item $h$ est une fonction \kw{affine (On a $a=3$ et $b=-4$)}.
		\item $i$ est une fonction \kw{constante (On a $a=0$ et $b=5$)}.
	\end{itemize}
\end{myexs}
	
\section{Représentation graphique et variations}

\subsection{Représentation graphique d'une fonction affine}	

\begin{myprops}
	\begin{itemize}
		\item La \kw{représentation graphique} d'une fonction affine $f(x)=ax+b$ et \kw{une droite}. On dit que $y=ax+b$ est l'équation de la droite.
		\kw{a} est le \kw{coefficient directeur} (ou la pente) de la droite.
		\kw{b} est l'\kw{ordonnée à l'origine}.
		\item La droite passe par le point de coordonnées $(0;b)$, si la fonction est linéaire elle passe par l'origine du repère.
	\end{itemize}
	
\end{myprops}

\begin{myex}
	\begin{multicols}{2}
		\vspace*{1.5cm}
		On considère la fonction affine $f(x)=2x+4$. Elle ne passe pas par l'origine du repère, elle n'est pas linéaire. Elle passe par le point de coordonnées $(0;4)$.
		
		\includegraphics[scale=0.5]{img/ex1}
	\end{multicols}
\end{myex}

\subsection{Calcul de l'expression algébrique}
\subsection{Sens de variation}

\section{Résolution graphique}
\end{document}	