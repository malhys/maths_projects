\documentclass[xcolor={dvipsnames}]{beamer}
%\usepackage[utf8]{inputenc}
\usetheme{CambridgeUS}

\input{../../../../utils_maths_beamer}


\usepackage{../../../../pas-math}
\usepackage{../../../../moncours_beamer}


\graphicspath{{../img/}}

\title{Fonctions affines}
\author{}\institute{}


\AtBeginSection[]
{
	\begin{frame}
		\frametitle{Sommaire}
		\tableofcontents[currentsection, hideallsubsections]
	\end{frame} 
}


\AtBeginSubsection[]
{
	\begin{frame}
		\frametitle{Sommaire}
		\tableofcontents[currentsection, currentsubsection]
	\end{frame} 
}

\begin{document}



\begin{frame}
  \titlepage 
\end{frame}

\section{Fonctions affines, fonctions linéaires}

\begin{frame}
\begin{mydefs}
	
	$a$ et $b$ sont des nombres quelconques ; la fonction qui à tout nombre $x$, associe le nombre \kw{$ax+b$}, est une \kw{fonction affine}.\\
	
	
	Cas particuliers :
	\begin{itemize}
		\item Si \kw{$b = 0$}, la fonction est \kw{linéaire}.
		\item Si \kw{$a=0$}, la fonction est \kw{constante}.
	\end{itemize}
\end{mydefs}
\end{frame}

\begin{frame}
	\begin{myexs}
		On considère les fonctions $f,g,h$ et $i$ :
		\begin{columns}[c]
			\begin{column}{6cm}
				\begin{itemize}
					\item $f(x)=2x$
					\item $g(x)=-x+2$
				\end{itemize}
			\end{column}
			\begin{column}{6cm}
				\begin{itemize}
					\item $h(x)=3x-4$
					\item $i(x)=5$
				\end{itemize}
			\end{column}				
		\end{columns}
		
		\vspace*{1cm}
		
		\begin{itemize}
			\item[$\rightarrow$] $f$ est une fonction \kw{linéaire (On a $a=2$ et $b=0$)}.
			\item[$\rightarrow$] $g$ est une fonction \kw{affine (On a $a=-1$ et $b=2$)}.
			\item[$\rightarrow$] $h$ est une fonction \kw{affine (On a $a=3$ et $b=-4$)}.
			\item[$\rightarrow$] $i$ est une fonction \kw{constante (On a $a=0$ et $b=5$)}.
		\end{itemize}
	\end{myexs}
\end{frame}

\section{Représentation graphique et variations}

\subsection{Représentation graphique d'une fonction affine}	

\begin{frame}
	\begin{myprops}
		\begin{itemize}
			\item La \kw{représentation graphique} d'une fonction affine $f(x)=ax+b$ et \kw{une droite}. 
			%On dit que $y=ax+b$ est l'équation de la droite.
			\kw{a} est le \kw{coefficient directeur} (ou la pente) de la droite.
			\kw{b} est l'\kw{ordonnée à l'origine}.
			\item La droite passe par le point de coordonnées $(0;b)$, si la fonction est linéaire elle passe par l'origine du repère.
		\end{itemize}
	
	\end{myprops}
\end{frame}

\begin{frame}
\begin{myex}
		\begin{columns}[c]
			\begin{column}{6cm}
				On considère la fonction affine $f(x)=2x+2$. Elle ne passe pas par l'origine du repère, elle n'est pas linéaire. Elle passe par le point de coordonnées $(0;2)$.
			\end{column}
			
			\begin{column}{6cm}
				\includegraphics[scale=0.5]{ex1}
			\end{column}			
			
		\end{columns}
	\end{myex}
\end{frame}

\subsection{Sens de variation}

\begin{frame}

\begin{myprop}
	Le sens de variation d'une fonction affine dépend du signe de $a$ :
	\begin{itemize}
		\item Si \kw{$a > 0$}, la droite \kw{"monte"}, la fonction est \kw{croissante};
		\item Si \kw{$a < 0$}, la droite \kw{"descend"} la fonction est \kw{décroissante};
		\item Si \kw{$a = 0$}, la droite est \kw{horizontale}, la fonction est \kw{constante}.
	\end{itemize}
\end{myprop}

\end{frame}


\begin{frame}
	%\frametitle{Exemples}

	\begin{myex}
		$f, g$ et $h$ sont des fonctions affines telles que :
	\end{myex}
	
	\begin{columns}[c]
		
		
		\begin{column}{3.7cm}
			
			\begin{exampleblock}{$f(x)=2x+1$}
				
				
				\begin{center}
					\includegraphics[scale=0.5]{ex2_1}
				\end{center}		
				
				$a=2$; $a>0$, la droite "monte", la fonction est croissante. 	
			\end{exampleblock}
			
		\end{column}
		
		\begin{column}{3.7cm}
			\begin{exampleblock}{$g(x)=-x+1$}
			
				\begin{center}
					\includegraphics[scale=0.5]{ex2_2}
				\end{center}		
			
				$a=-1$; $a<0$, la droite "descend", la fonction est décroissante. 
			\end{exampleblock}
		\end{column}
		
		\begin{column}{3.7cm}
			\begin{exampleblock}{$h(x)=3$}
			
				\begin{center}
					\includegraphics[scale=0.45]{ex2_3}
				\end{center}		
				$a=0$, la droite est horizontale, la fonction est constante. 
			\end{exampleblock}
		\end{column}
	
	\end{columns}



\end{frame}

\subsection{Calcul du coefficient directeur}


\begin{frame}
	\begin{mymeth}
		Pour calculer le coefficient directeur d'une fonction affine $f$, on a besoin de deux nombres distincts $x_1$ et $x_2$ et de leurs images par $f$, $f(x_1)$ et $f(x_2)$. On a alors :

		\kw{\begin{align*}
			a = \dfrac{f(x_2) - f(x_1)}{x_2 - x_1}
		\end{align*}}
	\end{mymeth}
\end{frame}	

\begin{frame}
	\begin{myex}
		La fonction passe par les points de coordonnées $(2;4)$ et $(4;8)$, on a :
		
		\begin{align*}
		a &= \frac{8-4}{4-2} \\
		a &= \frac{4}{2} \\
		a &= 2
		\end{align*}
	\end{myex}
	
\end{frame}

\end{document}