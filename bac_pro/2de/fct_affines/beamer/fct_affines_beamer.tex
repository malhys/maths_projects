\documentclass[xcolor={dvipsnames}]{beamer}
%\usepackage[utf8]{inputenc}
\usetheme{CambridgeUS}

\input{../../../../utils_maths_beamer}


\usepackage{../../../../pas-math}
\usepackage{../../../../moncours_beamer}

\title{Fonctions affines}
\author{}\institute{}


\AtBeginSection[]
{
	\begin{frame}
		\frametitle{Sommaire}
		\tableofcontents[currentsection, hideothersubsections]
	\end{frame} 
}

\begin{document}



\begin{frame}
  \titlepage 
\end{frame}

\section{Fonctions affines, fonctions linéaires}

\begin{frame}
\begin{mydefs}
	
	$a$ et $b$ sont des nombres quelconques ; la fonction qui à tout nombre $x$, associe le nombre \kw{$ax+b$}, est une \kw{fonction affine}.\\
	
	
	Cas particuliers :
	\begin{itemize}
		\item Si \kw{$b = 0$}, la fonction est \kw{linéaire}.
		\item Si \kw{$a=0$}, la fonction est \kw{constante}.
	\end{itemize}
\end{mydefs}
\end{frame}


\begin{frame}


\begin{myobj}
	Toto
\end{myobj}
\end{frame}
\end{document}