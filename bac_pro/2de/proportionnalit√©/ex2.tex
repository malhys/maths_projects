\begin{myex}
	Un catalogue de vente de fleurs propose 25 bulbes de gla\"{\i}euls pour 4,50 €.
	
	Combien coûterait l'achat de 350 bulbes de gla\"{\i}euls pour fleurir le parvis d'un hôtel de ville ?\\
	
	\mysp
	
	On peut établir le \kw{tableau de proportionnalité} suivant où $x$ représente la valeur cherchée.
	
	\begin{center}
		\begin{tabular}{|@{\ }l@{\ }|@{\ }c@{\ }|@{\ }c@{\ }|}
			\hline
			Nombre de bulbes & 25 & 350  \\ \hline
			Prix à payer (€) & 4,5 & $x$  \\ \hline
		\end{tabular}
	\end{center}
	
	En utilisant le \kw{produit en croix}, on obtient : \\
	
	$\dfrac{25}{4,5} = \dfrac{350}{x}$ on a alors : $x = \dfrac{4,5 \times 350}{25} = 63$\\
	
	On peut donc conclure que fleurir le parvis de l'hôtel de ville coûtera 63 €.
\end{myex}

