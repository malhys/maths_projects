\documentclass[12pt,a4paper]{article}

%\usepackage[left=1.5cm,right=1.5cm,top=1cm,bottom=2cm]{geometry}
\usepackage[in, plain]{fullpage}
\usepackage{array}
\usepackage{../../../pas-math}
\usepackage{../../../moncours}


%\usepackage{pas-cours}
%-------------------------------------------------------------------------------
%          -Packages nécessaires pour écrire en Français et en UTF8-
%-------------------------------------------------------------------------------
\usepackage[utf8]{inputenc}
\usepackage[frenchb]{babel}
\usepackage[T1]{fontenc}
\usepackage{lmodern}
\usepackage{textcomp}



%-------------------------------------------------------------------------------

%-------------------------------------------------------------------------------
%                          -Outils de mise en forme-
%-------------------------------------------------------------------------------
\usepackage{hyperref}
\hypersetup{pdfstartview=XYZ}
%\usepackage{enumerate}
\usepackage{graphicx}
\usepackage{multicol}
\usepackage{tabularx}
\usepackage{multirow}


\usepackage{anysize} %%pour pouvoir mettre les marges qu'on veut
%\marginsize{2.5cm}{2.5cm}{2.5cm}{2.5cm}

\usepackage{indentfirst} %%pour que les premier paragraphes soient aussi indentés
\usepackage{verbatim}
\usepackage{enumitem}
\usepackage[usenames,dvipsnames,svgnames,table]{xcolor}

\usepackage{variations}

%-------------------------------------------------------------------------------


%-------------------------------------------------------------------------------
%                  -Nécessaires pour écrire des mathématiques-
%-------------------------------------------------------------------------------
\usepackage{amsfonts}
\usepackage{amssymb}
\usepackage{amsmath}
\usepackage{amsthm}
\usepackage{tikz}
\usepackage{xlop}
%-------------------------------------------------------------------------------



%-------------------------------------------------------------------------------


%-------------------------------------------------------------------------------
%                    - Mise en forme avancée
%-------------------------------------------------------------------------------

\usepackage{ifthen}
\usepackage{ifmtarg}


\newcommand{\ifTrue}[2]{\ifthenelse{\equal{#1}{true}}{#2}{$\qquad \qquad$}}

%-------------------------------------------------------------------------------

%-------------------------------------------------------------------------------
%                     -Mise en forme d'exercices-
%-------------------------------------------------------------------------------
%\newtheoremstyle{exostyle}
%{\topsep}% espace avant
%{\topsep}% espace apres
%{}% Police utilisee par le style de thm
%{}% Indentation (vide = aucune, \parindent = indentation paragraphe)
%{\bfseries}% Police du titre de thm
%{.}% Signe de ponctuation apres le titre du thm
%{ }% Espace apres le titre du thm (\newline = linebreak)
%{\thmname{#1}\thmnumber{ #2}\thmnote{. \normalfont{\textit{#3}}}}% composants du titre du thm : \thmname = nom du thm, \thmnumber = numéro du thm, \thmnote = sous-titre du thm

%\theoremstyle{exostyle}
%\newtheorem{exercice}{Exercice}
%
%\newenvironment{questions}{
%\begin{enumerate}[\hspace{12pt}\bfseries\itshape a.]}{\end{enumerate}
%} %mettre un 1 à la place du a si on veut des numéros au lieu de lettres pour les questions 
%-------------------------------------------------------------------------------

%-------------------------------------------------------------------------------
%                    - Mise en forme de tableaux -
%-------------------------------------------------------------------------------

\renewcommand{\arraystretch}{1.7}

\setlength{\tabcolsep}{1.2cm}

%-------------------------------------------------------------------------------



%-------------------------------------------------------------------------------
%                    - Racourcis d'écriture -
%-------------------------------------------------------------------------------

% Angles orientés (couples de vecteurs)
\newcommand{\aopp}[2]{(\vec{#1}, \vec{#2})} %Les deuc vecteurs sont positifs
\newcommand{\aopn}[2]{(\vec{#1}, -\vec{#2})} %Le second vecteur est négatif
\newcommand{\aonp}[2]{(-\vec{#1}, \vec{#2})} %Le premier vecteur est négatif
\newcommand{\aonn}[2]{(-\vec{#1}, -\vec{#2})} %Les deux vecteurs sont négatifs

%Ensembles mathématiques
\newcommand{\naturels}{\mathbb{N}} %Nombres naturels
\newcommand{\relatifs}{\mathbb{Z}} %Nombres relatifs
\newcommand{\rationnels}{\mathbb{Q}} %Nombres rationnels
\newcommand{\reels}{\mathbb{R}} %Nombres réels
\newcommand{\complexes}{\mathbb{C}} %Nombres complexes


%Intégration des parenthèses aux cosinus
\newcommand{\cosP}[1]{\cos\left(#1\right)}
\newcommand{\sinP}[1]{\sin\left(#1\right)}


%Probas stats
\newcommand{\stat}{statistique}
\newcommand{\stats}{statistiques}
%-------------------------------------------------------------------------------

%-------------------------------------------------------------------------------
%                    - Mise en page -
%-------------------------------------------------------------------------------

\newcommand{\twoCol}[1]{\begin{multicols}{2}#1\end{multicols}}


\setenumerate[1]{font=\bfseries,label=\textit{\alph*})}
\setenumerate[2]{font=\bfseries,label=\arabic*)}


%-------------------------------------------------------------------------------
%                    - Elements cours -
%-------------------------------------------------------------------------------





%\makeatletter
%\renewcommand*{\@seccntformat}[1]{\csname the#1\endcsname\hspace{0.1cm}}
%\makeatother


%\author{Olivier FINOT}
\date{}
\title{\'Equations et Inéquations}

%\newcommand{\disp}{false}

%\lhead{CH1 : Stats et représentations graphiques}
%\rhead{O. FINOT}
%
%\rfoot{Page \thepage}
\begin{document}
%\maketitle

\chap[num=3, color=red]{Problèmes du premier degré}{Olivier FINOT, \today }


\begin{myobj}
	Être capable de :
	\begin{itemize}
		\item traduire un problème en équation ou en inéquation.
		\item résoudre une équation.
		\item résoudre une inéquation.
		\item évaluer ou critiquer un résultat obtenu
		\item replacer un résultat dans le contexte du problème.
	\end{itemize}
\end{myobj}

\section*{Activités préparatoires}

\begin{myact}{1}
	Dans un magasin Jean achète 5 CD et paye 45 euros en caisse. 
	
	Combien coûte un CD ?
\end{myact}

\begin{myrep}
	
	%\begin{multicols}{2}
		\begin{itemize}
			\item On pose $x$ le prix d'un CD.
			\item $NbCDs \times PrixCD = Total$
			\item[$\Rightarrow$] $5 \times x = 45$ 
		\end{itemize}
		
		\begin{align*}
			5x &= 45 \\
			x &= 45 \div 5\\
			x &= 9
		\end{align*}
	%\end{multicols}
	
	Un CD coûte 9 euros.
	
\end{myrep}

\begin{myact}{2}
	Sophie paie 42 € pour 3 mangas et des romans. Un manga vaut 6 € et un roman 12 €. 
	
	Combien a-t-elle acheté de romans ?
\end{myact}

\begin{myrep}
	%\begin{multicols}{2}
		\begin{itemize}
			\item On pose $x$ le nombre de romans achetés.
			\item $NbMangas \times PrixManga + NbRomans \times PrixRoman = Total $
			\item[$\Rightarrow$]$3 \times 6 + x \times 12 = 42$
		\end{itemize}
		
		\begin{align*}
			18 + 12x &= 42\\
			12x &= 42 - 18 \\
			12x &= 24 \\
			x &= 24 \div 12\\
			x &= 2
		\end{align*}
	%\end{multicols}
	
	Elle a acheté 2 romans.
\end{myrep}

\begin{myact}{3}
	Jeanne a 25 ans de moins que sa mère, et elles ont 43 ans à eux deux. 
	
	Quel âge ont-elles ?
\end{myact}

\begin{myrep}
	\begin{itemize}
		\item On cherche l'âge de Jeanne et celui de sa mère.
		\item On pose $x$ l'âge de Jeanne.
		\item On a :
		\begin{itemize}
			
		\item $AgeMère = AgeJeanne + 25$.
		\item[$\Rightarrow$] $AgeMère = x + 25 $
		\item $AgeMère + AgeJeanne = 43$
		\item[$\Rightarrow$] $(x + 25) + x = 43$ 
		\end{itemize} 
	\end{itemize}
	
	\begin{align*}
		2x + 25 &= 43\\
		2x &= 43 - 25\\
		2x &= 18 \\
		x &= 18 \div 2 \\
		x &= 9
	\end{align*}
	
	Jeanne a 9 ans et sa mère 34 (9 + 25).
\end{myrep}

\begin{myact}{4}
	Thomas possède une collection de 248 vidéos. Il les classe en trois catégories : séries télé, films et documentaire.
	
	Il a 3 fois plus de films que de séries télé et 40 documentaires.
	
	Combien de vidéos de chaque catégorie possède-t-il ?
	
\end{myact}

\begin{myrep}
\begin{itemize}
	\item On cherche le nombre de séries et le nombre de films.
	\item On pose $x$ le nombre de séries.
	\item On a :
		\begin{itemize}
			\item $NbFilms = 3 \times NbSéries$
			\item[$\Rightarrow$] $NbFilms = 3 \times x$
			\item $NbFilms + NbSéries + NbDocs = Total$
			\item[$\Rightarrow$] $(3 \times x) + x + 40 = 248$
		\end{itemize} 
\end{itemize}

\begin{align*}
	4x + 40 &= 248\\
	4x &= 248 - 40 \\
	4x &= 208 \\
	x &= 208 \div 4 \\
	x &= 52 \\
\end{align*}

Il a 52 séries, 156 films ($52 \times 3$) et 40 documentaires. 
\end{myrep}
	
%
%\begin{mydefs}
%\begin{itemize}
%	\item Une \kw{équation du premier degré} est une \kw{égalité qui contient une inconnue}, une lettre dont on ne connaît pas la valeur.
%	\item \kw{Résoudre une équation} c'est trouver la valeur de l'inconnue (si elle existe) qui rend l'égalité vraie. Cette valeur est la \kw{solution} de l'équation.
%\end{itemize}
%
%
%\begin{myex}
%		$12x - 7 = 8x + 5$ est un exemple d'équation.
%		$x$ est l'inconnue de cette équation.
%		
%		En remplaçant $x$ par $3$, on a:
%		$12 \times 3 -7 = 36 - 7 = 29$
%		$8 \times 3 + 5 = 24 + 5 = 29$
%		
%		Donc $3$ est une \kw{solution} de l'équation.
%\end{myex}
%	
%\end{mydefs}

\section{Mettre un problème en équation}



\section{Inéquations}


\end{document}