\documentclass[12pt,a4paper]{article}

%\usepackage[left=1.5cm,right=1.5cm,top=1cm,bottom=2cm]{geometry}
\usepackage[in, plain]{fullpage}
\usepackage{array}
\usepackage{../../../pas-math}
\usepackage{../../../moncours}


%\usepackage{pas-cours}
%-------------------------------------------------------------------------------
%          -Packages nécessaires pour écrire en Français et en UTF8-
%-------------------------------------------------------------------------------
\usepackage[utf8]{inputenc}
\usepackage[frenchb]{babel}
\usepackage[T1]{fontenc}
\usepackage{lmodern}
%-------------------------------------------------------------------------------

%-------------------------------------------------------------------------------
%                          -Outils de mise en forme-
%-------------------------------------------------------------------------------
\usepackage{hyperref}
\hypersetup{pdfstartview=XYZ}
\usepackage{enumerate}
\usepackage{graphicx}
\usepackage{multicol}

\usepackage{anysize} %%pour pouvoir mettre les marges qu'on veut
%\marginsize{2.5cm}{2.5cm}{2.5cm}{2.5cm}

\usepackage{indentfirst} %%pour que les premier paragraphes soient aussi indentés
%-------------------------------------------------------------------------------


%-------------------------------------------------------------------------------
%                  -Nécessaires pour écrire des mathématiques-
%-------------------------------------------------------------------------------
\usepackage{amsfonts}
\usepackage{amssymb}
\usepackage{amsmath}
\usepackage{amsthm}
\usepackage{tikz}
%-------------------------------------------------------------------------------

%-------------------------------------------------------------------------------
%                     -Mise en forme d'exercices-
%-------------------------------------------------------------------------------
\newtheoremstyle{exostyle}
{\topsep}% espace avant
{\topsep}% espace apres
{}% Police utilisee par le style de thm
{}% Indentation (vide = aucune, \parindent = indentation paragraphe)
{\bfseries}% Police du titre de thm
{.}% Signe de ponctuation apres le titre du thm
{ }% Espace apres le titre du thm (\newline = linebreak)
{\thmname{#1}\thmnumber{ #2}\thmnote{. \normalfont{\textit{#3}}}}% composants du titre du thm : \thmname = nom du thm, \thmnumber = numéro du thm, \thmnote = sous-titre du thm

\theoremstyle{exostyle}
\newtheorem{exercice}{Exercice}

\newenvironment{questions}{
\begin{enumerate}[\hspace{12pt}\bfseries\itshape a.]}{\end{enumerate}
} %mettre un 1 à la place du a si on veut des numéros au lieu de lettres pour les questions 
%-------------------------------------------------------------------------------



%-------------------------------------------------------------------------------
%                    - Racourcis d'écriture -
%-------------------------------------------------------------------------------

% Angles orientés (couples de vecteurs)
\newcommand{\aopp}[2]{(\vec{#1}, \vec{#2})} %Les deuc vecteurs sont positifs
\newcommand{\aopn}[2]{(\vec{#1}, -\vec{#2})} %Le second vecteur est négatif
\newcommand{\aonp}[2]{(-\vec{#1}, \vec{#2})} %Le premier vecteur est négatif
\newcommand{\aonn}[2]{(-\vec{#1}, -\vec{#2})} %Les deux vecteurs sont négatifs

%Ensembles mathématiques
\newcommand{\naturels}{\mathbb{N}} %Nombres naturels
\newcommand{\relatifs}{\mathbb{Z}} %Nombres relatifs
\newcommand{\rationnels}{\mathbb{Q}} %Nombres rationnels
\newcommand{\reels}{\mathbb{R}} %Nombres réels
\newcommand{\complexes}{\mathbb{C}} %Nombres complexes
%-------------------------------------------------------------------------------




%\makeatletter
%\renewcommand*{\@seccntformat}[1]{\csname the#1\endcsname\hspace{0.1cm}}
%\makeatother


%\author{Olivier FINOT}
\date{}
\title{Proportionnalité}

%\newcommand{\disp}{false}

%\lhead{CH1 : Stats et représentations graphiques}
%\rhead{O. FINOT}
%
%\rfoot{Page \thepage}
\begin{document}
\maketitle

\section{Situation de proportionnalité}

\begin{mydefs}
	\begin{itemize}
		\item 	Deux grandeurs sont \kw{en situation de proportionnalité} lorsque les suites de nombres qui correspondent à leurs mesures sont \kw{proportionnelles}.
		
		\item Dans un tableau, si les valeurs d'une lignes s'obtiennent en \kw{multipliant} ou en \kw{divisant} celles de l'autre ligne par un \kw{même nombre} (noté \kw{$k$}); alors les suites de nombres présentées dans ce tableau sont proportionnelles. $k$ est le \kw{coefficient de proportrionnalité} .
		
		\item Lorsque les grandeurs proportionnelles sont présentées sous forme de \kw{graphique}, les points correspondant à ces deux grandeurs sont alignés sur une droite qui passe par l'origine du repère. 
		
	\end{itemize}

	
\end{mydefs}

\begin{myraps}
	\begin{itemize}
		\item Dans un repère \kw{orthogonal} le plan est défini par deux axes perpendiculaires.
		\item L'axe horizontal est l'\kw{axe des abscisses}.
		\item L'axe vertical est l'\kw{axe des ordonnées}.
		\item Les \kw{coordonnées} d'un point du plan sont constituées d'un couple de nombres \kw{$(x;y)$} où $x$ est une valeur sur l'axe des abscisses et $y$ sur l'axe des ordonnées.
		\item Leur point d'intersection est l'\kw{origine} du repère.
	\end{itemize}
\end{myraps}

\begin{myillus}

		Courbe représentative de la fonction $f(x) = x^2$ et tableau de variations associé:
	\begin{multicols}{2}

	


	\begin{center}
		\includegraphics[scale=0.6]{./img/carre}
	\end{center}
	
	

	\vspace*{1cm}
	\begin{center}
%		\begin{tikzpicture}
%		\tkzTabInit{$x$/1,$f(x)$/2}{$- \infty$,$0$,$+ \infty$}
%		%\tkzTabLine{,-,z,+}
%		\tkzTabVar{+/$+ \infty$,-/$0$,+/$+ \infty$}
%		\end{tikzpicture}	

		\begin{variations}
			x & \mI & & 0 & & \pI \\
		\filet
			\m{x^2} & \h\pi & \d & 0 & \c & \h\pI \\				
		\end{variations}
	\end{center}
	\end{multicols}
\end{myillus}

\section{Recherche d'un quatrième proportionnelle}

\begin{mymeth}
	L'égalité $\dfrac{a}{b}=\dfrac{c}{d}$ est une proportion.\\
	
	La règle du \kw{produit en croix} permet de calculer un des quatre nombres ($a$, $b$, $c$ ou $d$) si les trois autres sont connus :
	
	\begin{center}
		Si $\dfrac{a}{b} = \dfrac{c}{d}$ alors $a \times d = b \times c$
	\end{center}
\end{mymeth}

\begin{myex}
	La droite d'ajustement obtenue grâce au tableur passe par le point moyen $G$ dont nous avons calculé les coordonnées.
	\begin{center}
		\includegraphics[scale =0.7]{./img/ex2}		
	\end{center}

\end{myex}

\end{document}