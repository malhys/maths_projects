\begin{myexs}
	\begin{enumerate}
		\item 	Soit ($u_n$) la suite géométrique de terme initial $u_0 = 2,4$ et de raison $q = 0,6$.
		
		Le terme de rang $n$ est $u_n = 2,4 \times 0,6^n$.
		
		On a ainsi : 
		\begin{itemize}
			\item $u_4 = 2,4 \times 0,6^4 = 0,31104$;
			\item $u_{100} = 2,4 \times 0,6^{100} = 0$.
		\end{itemize}
		
		\item Soit ($u_n$) la suite arithmétique de terme initial $u_1 = 0,7$ et de raison $q = 2,2$.
		
		Le terme de rang $n$ est $u_n = 0,7 \times 2,2^{n-1}$.
		
		On a ainsi : 
		\begin{itemize}
			\item $u_5 = 0,7 \times 2,2^{4}= 16,39792$;
			\item $u_11 = 0,7 \times 2,2^{10} = 1859$.
		\end{itemize}
	\end{enumerate}
	
	
	
\end{myexs}