\begin{myexs}
	\begin{enumerate}
		\item 	Soit ($u_n$) la suite arithmétique de terme initial $u_0 = 1,5$ et de raison $r = -7$.
		
		Le terme de rang $n$ est $u_n = 1,5 + n \times (-7)$ c'est à dire $u_n=1,5 - 7n$.
		
		On a ainsi : 
		\begin{itemize}
			\item $u_4 = 1,5 - 7 \times 4 = -26,5$
			\item $u_{100} = 1,5 - 7 \times 100 = -698,5$
		\end{itemize}
		
		\item Soit ($u_n$) la suite arithmétique de terme initial $u_1 = 14$ et de raison $r = 1,3$.
		
		Le terme de rang $n$ est $u_n = 14 + (n-1) \times 1,3$; c'est à dire $u_n = 12,7 + 1,3n$.

		On a ainsi : 
		\begin{itemize}
			\item $u_4 = 12,7 + 1,3 \times 4 = 17,9$;
			\item $u_{100} = 12,7 + 1,3 \times 100 = 142,7$.
		\end{itemize}
	\end{enumerate}

	
	
\end{myexs}