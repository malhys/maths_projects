\documentclass[xcolor={dvipsnames}]{beamer}
%\usepackage[utf8]{inputenc}
\usetheme{CambridgeUS}

\input{../../../../utils_maths_beamer}


\usepackage{../../../../pas-math}
\usepackage{../../../../moncours_beamer}


\graphicspath{{../img/}}

\title{Exponentielles et logarithme décimal}
\author{}\institute{}


\AtBeginSection[]
{
	\begin{frame}
		\frametitle{Sommaire}
		\tableofcontents[currentsection, hideallsubsections]
	\end{frame} 
}


\AtBeginSubsection[]
{
	\begin{frame}
		\frametitle{Sommaire}
		\tableofcontents[currentsection, currentsubsection]
	\end{frame} 
}

\begin{document}



\begin{frame}
  \titlepage 
\end{frame}

\section{Fonction exponentielle de base $q$}

\subsection{Définition}



\begin{frame}
\begin{mydef}
	$q$ est un nombre strictement positif ($q > 0$).
	La fonction qui à tout nombre $x$ associe $q^x$, est appelée \kw{fonction exponentielle} de base $q$.
\end{mydef}\pause

\begin{myex}
	\begin{itemize}
		\item La fonction $f$, définie par $f(x)=2^x$, est la \kw{fonction exponentielle de base $2$}. 
		\item La fonction $g$, définie par $g(x)=0,5^x$, est la \kw{fonction exponentielle de base $0,5$}.
	\end{itemize}
	
\end{myex}

\end{frame}

\subsection{Valeurs particulières et variations}
\begin{frame}
\begin{myprops}
	\begin{enumerate}
		\item Valeurs particulières :
		
		\begin{center}
			\begin{align*}
			q^0 = 1 \qquad\qquad\qquad\qquad q^1 = q 			
			\end{align*}
		\end{center}
		
		
		\item Variations :
		\begin{itemize}
			\item Si \kw{$q > 0$}, alors la fonction est \kw{croissante}.
			\item Si \kw{$q < 0$}, alors la fonction est \kw{décroissante}.
		\end{itemize}
	\end{enumerate}
\end{myprops}
\end{frame}
%\begin{columns}[c]
%	\begin{column}{6cm}
%		\begin{itemize}
%			\item $f(x)=2x$
%			\item $g(x)=-x+2$
%		\end{itemize}
%	\end{column}
%	\begin{column}{6cm}
%		\begin{itemize}
%			\item $h(x)=3x-4$
%			\item $i(x)=5$
%		\end{itemize}
%	\end{column}				
%\end{columns}

\begin{frame}
\begin{myex}
	\begin{columns}{c}
		\begin{column}{6cm}
					
			$f(x)= 2^x$, $2 > 1$\\
			la fonction $f$ est croissante
		
			\begin{center}
				\includegraphics[scale=0.25]{var1}
			\end{center}
		
		\end{column}
		
		\begin{column}{6cm}
			$g(x)= 0,5^x$, $0,5 > 1$\\
			la fonction $g$ est décroissante
			
			\begin{center}
				\includegraphics[scale=0.2]{var2}
			\end{center}
			
		\end{column}
		
	\end{columns}
\end{myex}

\end{frame}

\subsection{Règles de calcul}

\begin{frame}
	\begin{myprops}
		Les règles de calculs sont les mêmes que pour les puissances entières.\\
		$a$ et $b$ sont deux nombres quelconques et $q$ un nombre strictement positif.
		
		\begin{align*}
		q^a = q^b &\Leftrightarrow a = b\\
		q^x \times q^y &= q^{a+b}\\
		\frac{q^a}{q^b} &= q^{a-b}\\
		(q^a)^b &= q^{a \times b}
		\end{align*}
		\end{myprops}
			
\end{frame}

\begin{frame}
	\begin{myex}
		
		\begin{align*}
		2^{-4} \times 2^{1,5} &= 2^{-2,5}\\\\
		\frac{0,1^3}{0,1^{1,8}} &= 0,1^{1,2}\\\\
		(3^{0,4})^{-2} &= 3^{-0,8}
		\end{align*}
	\end{myex}
\end{frame}


\section{Fonction logarithme décimal}

\subsection{Définition}

\begin{frame}
	\begin{mydef}
		$a$ est un nombre strictement positif ($a>0$), le nombre $b$ tel que \kw{$10^b=a$}, est le \kw{logarithme décimal}, noté $\log a$.
	\end{mydef}
\end{frame}


\subsection{Valeurs particulières et variations}

\begin{frame}
	\begin{myprops}
		\begin{enumerate}
			\item Valeurs particulières :
			\begin{align*}
			\log 1 &= 0 \\
			\log 10 &= 1\\
			\log 100 &= 2
			\end{align*}
			
			\item Signe et variations :
			\begin{itemize}
				\item La fonction $\log x$ est \kw{croissante} pour $x > 0$.
				\item Si $0 \leq x < 1$, alors $\log x$ est négatif.
				\item Si $x \geq 1$, alors $\log x$ est positif.
			\end{itemize}			
			
			\begin{center}
				\includegraphics[scale = 0.22]{var_log}
			\end{center}	
		\end{enumerate}
		
		
	\end{myprops}
\end{frame}


\subsection{Règles de calcul}
\begin{frame}
	\begin{myprops}
		$a$ et $b$ sont deux nombres strictement positifs :
		
		\begin{align*}
		\log a = b &\Leftrightarrow a = 10^b \\
		10^b = a &\Leftrightarrow b = \log a \\
		\log a = \log b &\Leftrightarrow a = b \\
		\log a < \log b &\Leftrightarrow a < b \\
		\log (a \times b) &= \log a + \log b \\
		\log  \left( \frac{a}{b} \right) &= \log a - \log b \\
		\log(a^x) &= x \times \log a 
		\end{align*}
	\end{myprops}	
\end{frame}

\end{document}