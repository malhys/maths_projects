\documentclass[12pt,a4paper]{article}

%\usepackage[left=1.5cm,right=1.5cm,top=1cm,bottom=2cm]{geometry}
\usepackage[in, plain]{fullpage}
\usepackage{array}
\usepackage{../../../pas-math}
\usepackage{../../../moncours}


%\usepackage{pas-cours}
%-------------------------------------------------------------------------------
%          -Packages nécessaires pour écrire en Français et en UTF8-
%-------------------------------------------------------------------------------
\usepackage[utf8]{inputenc}
\usepackage[frenchb]{babel}
\usepackage[T1]{fontenc}
\usepackage{lmodern}
%-------------------------------------------------------------------------------

%-------------------------------------------------------------------------------
%                          -Outils de mise en forme-
%-------------------------------------------------------------------------------
\usepackage{hyperref}
\hypersetup{pdfstartview=XYZ}
\usepackage{enumerate}
\usepackage{graphicx}
\usepackage{multicol}

\usepackage{anysize} %%pour pouvoir mettre les marges qu'on veut
%\marginsize{2.5cm}{2.5cm}{2.5cm}{2.5cm}

\usepackage{indentfirst} %%pour que les premier paragraphes soient aussi indentés
%-------------------------------------------------------------------------------


%-------------------------------------------------------------------------------
%                  -Nécessaires pour écrire des mathématiques-
%-------------------------------------------------------------------------------
\usepackage{amsfonts}
\usepackage{amssymb}
\usepackage{amsmath}
\usepackage{amsthm}
\usepackage{tikz}
%-------------------------------------------------------------------------------

%-------------------------------------------------------------------------------
%                     -Mise en forme d'exercices-
%-------------------------------------------------------------------------------
\newtheoremstyle{exostyle}
{\topsep}% espace avant
{\topsep}% espace apres
{}% Police utilisee par le style de thm
{}% Indentation (vide = aucune, \parindent = indentation paragraphe)
{\bfseries}% Police du titre de thm
{.}% Signe de ponctuation apres le titre du thm
{ }% Espace apres le titre du thm (\newline = linebreak)
{\thmname{#1}\thmnumber{ #2}\thmnote{. \normalfont{\textit{#3}}}}% composants du titre du thm : \thmname = nom du thm, \thmnumber = numéro du thm, \thmnote = sous-titre du thm

\theoremstyle{exostyle}
\newtheorem{exercice}{Exercice}

\newenvironment{questions}{
\begin{enumerate}[\hspace{12pt}\bfseries\itshape a.]}{\end{enumerate}
} %mettre un 1 à la place du a si on veut des numéros au lieu de lettres pour les questions 
%-------------------------------------------------------------------------------



%-------------------------------------------------------------------------------
%                    - Racourcis d'écriture -
%-------------------------------------------------------------------------------

% Angles orientés (couples de vecteurs)
\newcommand{\aopp}[2]{(\vec{#1}, \vec{#2})} %Les deuc vecteurs sont positifs
\newcommand{\aopn}[2]{(\vec{#1}, -\vec{#2})} %Le second vecteur est négatif
\newcommand{\aonp}[2]{(-\vec{#1}, \vec{#2})} %Le premier vecteur est négatif
\newcommand{\aonn}[2]{(-\vec{#1}, -\vec{#2})} %Les deux vecteurs sont négatifs

%Ensembles mathématiques
\newcommand{\naturels}{\mathbb{N}} %Nombres naturels
\newcommand{\relatifs}{\mathbb{Z}} %Nombres relatifs
\newcommand{\rationnels}{\mathbb{Q}} %Nombres rationnels
\newcommand{\reels}{\mathbb{R}} %Nombres réels
\newcommand{\complexes}{\mathbb{C}} %Nombres complexes
%-------------------------------------------------------------------------------



\date{}
\title{}


\begin{document}
%\maketitle
\chap[num=2, color=red]{Statistiques à deux variables}{Olivier FINOT, \today }

\section{Série statistique double}

\begin{mydefs}
	\begin{itemize}
		\item Lorsqu'on étudie \kw{deux caractères statistiques} sur une même population, on obtient une \kw{série statistique double}.
		
		\item La représentation d'une série statistique double forme un \kw{nuage de points}.
		
		\item Le \kw{point moyen} $G$ d'un  nuage de points a pour coordonnées \kw{$(\bar{x}; \bar{y})$}.
	\end{itemize}
	
\end{mydefs}

\begin{myillus}

		Courbe représentative de la fonction $f(x) = x^2$ et tableau de variations associé:
	\begin{multicols}{2}

	


	\begin{center}
		\includegraphics[scale=0.6]{./img/carre}
	\end{center}
	
	

	\vspace*{1cm}
	\begin{center}
%		\begin{tikzpicture}
%		\tkzTabInit{$x$/1,$f(x)$/2}{$- \infty$,$0$,$+ \infty$}
%		%\tkzTabLine{,-,z,+}
%		\tkzTabVar{+/$+ \infty$,-/$0$,+/$+ \infty$}
%		\end{tikzpicture}	

		\begin{variations}
			x & \mI & & 0 & & \pI \\
		\filet
			\m{x^2} & \h\pi & \d & 0 & \c & \h\pI \\				
		\end{variations}
	\end{center}
	\end{multicols}
\end{myillus}

\newpage

\section{Ajustement affine et prévisions}


\subsection{Ajustement affine}
\begin{mydefs}
	\begin{itemize}
		\item Si le nuage de points a une forme <<allongée>>, on peut calculer un \kw{ajustement affine} du nuage. 
		\item On obtient ainsi une \kw{droite d'ajustement} (ou droite de régression) qui passe par le point moyen $G$ et au plus près des autres points du nuage.		
	\end{itemize}
\end{mydefs}

%\begin{mymeth}
%	La droite d'ajustement est obtenue à l'aide du tableur ou de la calculatrice.
%\end{mymeth}

\begin{myex}
	La droite d'ajustement obtenue grâce au tableur passe par le point moyen $G$ dont nous avons calculé les coordonnées.
	\begin{center}
		\includegraphics[scale =0.7]{./img/ex2}		
	\end{center}

\end{myex}

\subsection{Prévisions}

\begin{myprop}
	\begin{itemize}
		\item La droite d'ajustement donne la <<tendance>> de l'évolution de la grandeur $y$ en fonction de celle de $x$.
		\item En supposant que la tendance se poursuive, il est possible d'estimer une valeur future par lecture graphique ou à partir de l'équation de la droite. 
	\end{itemize}
	
	 
	%L'équation de type $y = ax +b $ de la droite d'ajustement donne  Cette équation permet de réaliser des estimations en supposant que la tendance observée se poursuive.
\end{myprop}

\begin{myex}
	On lance un dé à 6 faces truqué. Une étude statistique donne le tableau suivant :
		\begin{center}
			
			\begin{tabular}{|@{\ }l@{\ }|@{\ }c@{\ }|@{\ }c@{\ }|@{\ }c@{\ }|@{\ }c@{\ }|@{\ }c@{\ }|@{\ }c@{\ }|}
				\hline
				Issue $x_i$ & 1 & 2 & 3 & 4 & 5 & 6 \\\hline
				Probabilité $p_i$ & $0,125$ & $0,125$ & $0,125$ & $0,125$ & $0,2$ & $0,3$ \\ \hline
			\end{tabular}
		
		\end{center}
	
	On s'intéresse à l'événement A : <<le nombre obtenu est pair>>.	On a : 
	
	\begin{align*}
		p(A) &= p_2 + p_4 + p_6 \\
			 &= 0,125 + 0,125 + 0,3 \\
			 &= 0,55
	\end{align*}
	
	La probabilité d'obtenir un nombre pair est de 0,55.
\end{myex}



\end{document}