%-------------------------------------------------------------------------------
%          -Packages nécessaires pour écrire en Français et en UTF8-
%-------------------------------------------------------------------------------
\usepackage[utf8]{inputenc}
\usepackage[frenchb]{babel}
\usepackage{numprint}
\usepackage[T1]{fontenc}
\usepackage{lmodern}
\usepackage{textcomp}
\usepackage[french, boxed]{algorithm2e}


%-------------------------------------------------------------------------------

%-------------------------------------------------------------------------------
%                          -Outils de mise en forme-
%-------------------------------------------------------------------------------
\usepackage{hyperref}
\hypersetup{pdfstartview=XYZ}
%\usepackage{enumerate}
\usepackage{graphicx}
\usepackage{multicol}
\usepackage{tabularx}
\usepackage{multirow}
\usepackage{color}


\usepackage{anysize} %%pour pouvoir mettre les marges qu'on veut
%\marginsize{2.5cm}{2.5cm}{2.5cm}{2.5cm}

\usepackage{indentfirst} %%pour que les premier paragraphes soient aussi indentés
\usepackage{verbatim}
\usepackage{enumitem}
\usepackage[usenames,dvipsnames,svgnames,table]{xcolor}

\usepackage{variations}

%-------------------------------------------------------------------------------


%-------------------------------------------------------------------------------
%                  -Nécessaires pour écrire des mathématiques-
%-------------------------------------------------------------------------------
\usepackage{amsfonts}
\usepackage{amssymb}
\usepackage{amsmath}
\usepackage{amsthm}
\usepackage{tikz}
\usepackage{xlop}
\usepackage[output-decimal-marker={,}]{siunitx}
%-------------------------------------------------------------------------------



%-------------------------------------------------------------------------------


%-------------------------------------------------------------------------------
%                    - Mise en forme avancée
%-------------------------------------------------------------------------------

\usepackage{ifthen}
\usepackage{ifmtarg}


\newcommand{\ifTrue}[2]{\ifthenelse{\equal{#1}{true}}{#2}{$\qquad \qquad$}}

%\newcommand{\kword}[1]{\textcolor{red}{\underline{#1}}}
%-------------------------------------------------------------------------------

%-------------------------------------------------------------------------------
%                     -Mise en forme d'exercices-
%-------------------------------------------------------------------------------
%\newtheoremstyle{exostyle}
%{\topsep}% espace avant
%{\topsep}% espace apres
%{}% Police utilisee par le style de thm
%{}% Indentation (vide = aucune, \parindent = indentation paragraphe)
%{\bfseries}% Police du titre de thm
%{.}% Signe de ponctuation apres le titre du thm
%{ }% Espace apres le titre du thm (\newline = linebreak)
%{\thmname{#1}\thmnumber{ #2}\thmnote{. \normalfont{\textit{#3}}}}% composants du titre du thm : \thmname = nom du thm, \thmnumber = numéro du thm, \thmnote = sous-titre du thm

%\theoremstyle{exostyle}
%\newtheorem{exercice}{Exercice}
%
%\newenvironment{questions}{
%\begin{enumerate}[\hspace{12pt}\bfseries\itshape a.]}{\end{enumerate}
%} %mettre un 1 à la place du a si on veut des numéros au lieu de lettres pour les questions 
%-------------------------------------------------------------------------------

%-------------------------------------------------------------------------------
%                    - Mise en forme de tableaux -
%-------------------------------------------------------------------------------

\renewcommand{\arraystretch}{1.7}

\setlength{\tabcolsep}{1.2cm}

%-------------------------------------------------------------------------------



%-------------------------------------------------------------------------------
%                    - Racourcis d'écriture -
%-------------------------------------------------------------------------------
%Droites
\newcommand{\dte}[1]{$(#1)$}
\newcommand{\fig}[1]{figure $#1$}
\newcommand{\sym}{symétrique}
\newcommand{\syms}{symétriques}
\newcommand{\asym}{axe de symétrie}
\newcommand{\asyms}{axes de symétrie}
\newcommand{\seg}[1]{$[#1]$}
\newcommand{\monAngle}[1]{$\widehat{#1}$}
\newcommand{\bissec}{bissectrice}
\newcommand{\mediat}{médiatrice}
\newcommand{\ddte}[1]{$[#1)$}


% Angles orientés (couples de vecteurs)
\newcommand{\aopp}[2]{(\vec{#1}, \vec{#2})} %Les deuc vecteurs sont positifs
\newcommand{\aopn}[2]{(\vec{#1}, -\vec{#2})} %Le second vecteur est négatif
\newcommand{\aonp}[2]{(-\vec{#1}, \vec{#2})} %Le premier vecteur est négatif
\newcommand{\aonn}[2]{(-\vec{#1}, -\vec{#2})} %Les deux vecteurs sont négatifs

%Ensembles mathématiques
\newcommand{\naturels}{\mathbb{N}} %Nombres naturels
\newcommand{\relatifs}{\mathbb{Z}} %Nombres relatifs
\newcommand{\rationnels}{\mathbb{Q}} %Nombres rationnels
\newcommand{\reels}{\mathbb{R}} %Nombres réels
\newcommand{\complexes}{\mathbb{C}} %Nombres complexes


%Intégration des parenthèses aux cosinus
\newcommand{\cosP}[1]{\cos\left(#1\right)}
\newcommand{\sinP}[1]{\sin\left(#1\right)}


%Probas stats
\newcommand{\stat}{statistique}
\newcommand{\stats}{statistiques}


\newcommand{\homo}{homothétie}
\newcommand{\homos}{homothéties}


\newcommand{\mycoord}[3]{(\textcolor{red}{\num{#1}} ; \textcolor{Green}{\num{#2}} ; \textcolor{blue}{\num{#3}})}
%-------------------------------------------------------------------------------

%-------------------------------------------------------------------------------
%                    - Mise en page -
%-------------------------------------------------------------------------------

\newcommand{\twoCol}[1]{\begin{multicols}{2}#1\end{multicols}}


\setenumerate[1]{font=\bfseries,label=\textit{\alph*})}
\setenumerate[2]{font=\bfseries,label=\arabic*)}


%-------------------------------------------------------------------------------
%                    - Elements cours -
%-------------------------------------------------------------------------------

%-------------------------------------------------------------------------------
%                    - raccourcis d'écriture -
%-------------------------------------------------------------------------------
