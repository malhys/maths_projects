\documentclass[xcolor={dvipsnames}, handout]{beamer}
%\usepackage[utf8]{inputenc}
%\usetheme{Madrid}
\usetheme{Malmoe}
\usecolortheme{beaver}
%\usecolortheme{rose}

%-------------------------------------------------------------------------------
%          -Packages nécessaires pour écrire en Français et en UTF8-
%-------------------------------------------------------------------------------
\usepackage[utf8]{inputenc}
\usepackage[frenchb]{babel}
\usepackage[T1]{fontenc}
\usepackage{lmodern}
\usepackage{textcomp}

%-------------------------------------------------------------------------------

%-------------------------------------------------------------------------------
%                          -Outils de mise en forme-
%-------------------------------------------------------------------------------
\usepackage{hyperref}
\hypersetup{pdfstartview=XYZ}
\usepackage{enumerate}
\usepackage{graphicx}
%\usepackage{multicol}
%\usepackage{tabularx}

%\usepackage{anysize} %%pour pouvoir mettre les marges qu'on veut
%\marginsize{2.5cm}{2.5cm}{2.5cm}{2.5cm}

\usepackage{indentfirst} %%pour que les premier paragraphes soient aussi indentés
\usepackage{verbatim}
%\usepackage[table]{xcolor}  
%\usepackage{multirow}
\usepackage{ulem}
%-------------------------------------------------------------------------------


%-------------------------------------------------------------------------------
%                  -Nécessaires pour écrire des mathématiques-
%-------------------------------------------------------------------------------
\usepackage{amsfonts}
\usepackage{amssymb}
\usepackage{amsmath}
\usepackage{amsthm}
\usepackage{tikz}
\usepackage{xlop}
\usepackage[output-decimal-marker={,}]{siunitx}
%-------------------------------------------------------------------------------


%-------------------------------------------------------------------------------
%                    - Mise en forme 
%-------------------------------------------------------------------------------

\newcommand{\bu}[1]{\underline{\textbf{#1}}}


\usepackage{ifthen}


\newcommand{\ifTrue}[2]{\ifthenelse{\equal{#1}{true}}{#2}{$\qquad \qquad$}}

\newcommand{\kword}[1]{\textcolor{red}{\underline{#1}}}


%-------------------------------------------------------------------------------



%-------------------------------------------------------------------------------
%                    - Racourcis d'écriture -
%-------------------------------------------------------------------------------

% Angles orientés (couples de vecteurs)
\newcommand{\aopp}[2]{(\vec{#1}, \vec{#2})} %Les deuc vecteurs sont positifs
\newcommand{\aopn}[2]{(\vec{#1}, -\vec{#2})} %Le second vecteur est négatif
\newcommand{\aonp}[2]{(-\vec{#1}, \vec{#2})} %Le premier vecteur est négatif
\newcommand{\aonn}[2]{(-\vec{#1}, -\vec{#2})} %Les deux vecteurs sont négatifs

%Ensembles mathématiques
\newcommand{\naturels}{\mathbb{N}} %Nombres naturels
\newcommand{\relatifs}{\mathbb{Z}} %Nombres relatifs
\newcommand{\rationnels}{\mathbb{Q}} %Nombres rationnels
\newcommand{\reels}{\mathbb{R}} %Nombres réels
\newcommand{\complexes}{\mathbb{C}} %Nombres complexes


%Intégration des parenthèses aux cosinus
\newcommand{\cosP}[1]{\cos\left(#1\right)}
\newcommand{\sinP}[1]{\sin\left(#1\right)}

%Fractions
\newcommand{\myfrac}[2]{{\LARGE $\frac{#1}{#2}$}}

%Vocabulaire courrant
\newcommand{\cad}{c'est-à-dire}

%Droites
\newcommand{\dte}[1]{droite $(#1)$}
\newcommand{\fig}[1]{figure $#1$}
\newcommand{\sym}{symétrique}
\newcommand{\syms}{symétriques}
\newcommand{\asym}{axe de symétrie}
\newcommand{\asyms}{axes de symétrie}
\newcommand{\seg}[1]{$[#1]$}
\newcommand{\monAngle}[1]{$\widehat{#1}$}
\newcommand{\bissec}{bissectrice}
\newcommand{\mediat}{médiatrice}
\newcommand{\ddte}[1]{$[#1)$}

%Figures
\newcommand{\para}{parallélogramme}
\newcommand{\paras}{parallélogrammes}
\newcommand{\myquad}{quadrilatère}
\newcommand{\myquads}{quadrilatères}
\newcommand{\co}{côtés opposés}
\newcommand{\diag}{diagonale}
\newcommand{\diags}{diagonales}
\newcommand{\supp}{supplémentaires}
\newcommand{\car}{carré}
\newcommand{\cars}{carrés}
\newcommand{\rect}{rectangle}
\newcommand{\rects}{rectangles}
\newcommand{\los}{losange}
\newcommand{\loss}{losanges}


%----------------------------------------------------


\usepackage{../../../../pas-math}
\usepackage{../../../../moncours_beamer}

\usepackage{amssymb,amsmath}


\newcommand{\myitem}{\item[\textbullet]}

\graphicspath{{../img/}}

\title{Séquence 5 : Nombres relatifs}
%\author{O. FINOT}\institute{Collège S$^t$ Bernard}

%
\AtBeginSection[]
{
	\begin{frame}
		\frametitle{}
		\tableofcontents[currentsection, hideallsubsections]
	\end{frame} 

}
%
%
%\AtBeginSubsection[]
%{
%	\begin{frame}
%		\frametitle{Sommaire}
%		\tableofcontents[currentsection, currentsubsection]
%	\end{frame} 
%}

\begin{document}



%\begin{frame}
%  \titlepage 
%\end{frame}


	

\begin{frame}
	\frametitle{Exercice 5 page 82}
	
	\begin{block}{Nombres positifs}
		+8; 0; \num{+3.5}; \num{+0.9}; +$\dfrac{4}{7}$; +125
	\end{block}

	\begin{block}{Nombres négatifs}
		-7 ; -5; 0; -$\dfrac{1}{3}$; -12
	\end{block}
	
	
\end{frame}



\begin{frame}
	\frametitle{Exercice 6 page 82}
	
	\begin{enumerate}
		\item -42 et 42 ont la même distance à zéro mais pas le même signe. \pause
		\item +\num{12.04} et $+\dfrac{7}{8}$ ont le même signe mais pas la même distance à zéro.\pause
		\item -\num{52.345} et +512 n'ont ni le même signe, ni la même distance à zéro.
	\end{enumerate}	
	
\end{frame}

\begin{frame}
	\frametitle{Exercice 8 page 83}
	\ \pause
	\begin{enumerate}
		\item Cela signifie que je dois 563 euros à la banque, je suis à découvert.\pause
		\item L'altitude de l'arrivée était plus haute que celle de départ de 542 mètres.\pause
		\item \`A la mi-temps, nous avions 2 buts de retard sur les adversaires, mais à la fin du match nous en avions 3 de plus qu'eux.\pause
		\item En vacances, la température était de 35°C, chez moi elles étaient en dessous de zéro.
	\end{enumerate}
\end{frame}

\begin{frame}
	%\frametitle{}
	
	\begin{alertblock}{Exercice 9 page 83}
		
	\ \pause
	
	\begin{itemize}
		\item L'américain appuiera sur le bouton 3;\pause
		\item L'espagnol appuiera sur le bouton -2;\pause
		\item Le français appuiera sur le bouton 0.\pause
	\end{itemize}
	\end{alertblock}
%\end{frame}
%
%\begin{frame}
%	\frametitle{Exercice 10 page 83}
%	\ \pause

	\begin{alertblock}{Exercice 10 page 83}
	 \ \pause	
	\begin{enumerate}
		\item Les nombres de la dernière ligne expriment la différence entre les points marqués et les points encaissés.\pause
		\item Une équipe aura un nombre négatif dans cette ligne si elle encaisse plus de points qu'elle en marque.\pause
		\item Si une équipe marque autant de points qu'elle en encaisse, il y aura un 0 dans cette ligne.%\pause
	\end{enumerate}
	\end{alertblock}
\end{frame}

\begin{frame}
	\frametitle{Exercice 11 page 83}
	\ \pause
	\begin{enumerate}
		\item L'écart à la moyenne a été aux alentours de -2\degree C en 1962; 1970 et 1984.\pause
		\item L'écart à la moyenne a été aux alentours de +2\degree C en 2011.\pause
		\item Ce graphique semble montrer que la température à tendance à augmenter par rapport à la moyenne ces dernières années.\pause
		
	\end{enumerate}
\end{frame}

\end{document}