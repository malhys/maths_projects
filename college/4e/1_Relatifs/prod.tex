\iftoggle{eleve}{%
	\begin{myprops}
		\begin{itemize}
			\item Le produit de \hrulefill
			
			\vspace*{0.2cm}
			\ \hrulefill
			
			
			\item Le produit de \hrulefill 
			
			\vspace*{0.2cm}
			\ \hrulefill
		\end{itemize}
	\end{myprops}
	
	\begin{mymeth}
		Pour calculer un produit \hrulefill 
		
		\vspace*{0.2cm}
		\ \hrulefill
	\end{mymeth}
}{%
	\begin{myprops}
		\begin{itemize}
			\item Le produit de deux nombres de \kw{même signe} est \kw{positif}.
			\item Le produit de deux nombres de \kw{signes différents} est \kw{négatif}.
		\end{itemize}
	\end{myprops}
	
	\begin{mymeth}
		Pour calculer un produit on détermine le signe puis on multiplie les distances à zéro.
	\end{mymeth}
	
}

\begin{myexs}
	
	\vspace*{-1cm}
	
	\begin{multicols}{2}
		
		
		
		\iftoggle{eleve}{%
			
			\begin{eqnarray*}
				A &=& 5 \times 2 \\
				A &=& \hrulefill 				
			\end{eqnarray*}
			Les deux facteurs sont \hrulefill
			
			le résultat est \hrulefill
			
			\begin{eqnarray*}
				B &=& (-\num{12.2}) \times (-3)  \\
				B &=& \hrulefill 			
			\end{eqnarray*}  
		
			Les deux facteurs sont \hrulefill
			
			le résultat est \hrulefill
		
		\end{multicols}
	\begin{multicols}{2}
		
			\begin{eqnarray*}
				C &=& -\num{12.2} \times (5)  \\
				C &=& \hrulefill 			
			\end{eqnarray*}  
		
			\begin{eqnarray*}
				D &=& \num{5.4} \times (-\num{1.5})  \\
				D &=& \hrulefill 			
			\end{eqnarray*} 
		
		\end{multicols}
		Les deux facteurs sont \hrulefill
		}{%
			\begin{eqnarray*}
				A &=& 5 \times 2 \\
				A &=& 10 				
			\end{eqnarray*} 
		%
			Les deux facteurs sont positifs, 
			
			le résultat est positif.
			\ \\
			\begin{eqnarray*}
				B &=& (-\num{12.2}) \times (-3)  \\
				B &=&  \num{36.6}				
			\end{eqnarray*} 
		%
		 Les deux facteurs sont négatifs, 
		 
		 le résultat est positif.
		 
		\end{multicols}
		\begin{multicols}{2}
		 \begin{eqnarray*}
		 	C &=& -\num{12.2} \times (5)  \\
		 	C &=& - 61 			
		 \end{eqnarray*}  
		 
		 \begin{eqnarray*}
		 	D &=& \num{5.4} \times (-\num{1.5})  \\
		 	D &=& - \num{8.1} 			
		 \end{eqnarray*} 
	
		 \end{multicols}
			 Les deux facteurs sont de signe différent, le résultat est négatif.
		}
		
	
\end{myexs}

\begin{mymeth}
	\iftoggle{eleve}{%
		Pour déterminer \hrulefill
		
		\vspace*{0.2cm}
		\ \hrulefill
		
		\begin{itemize}
			\item s'il est pair, \hrulefill
			\item s'il est impair, \hrulefill
		\end{itemize}
	}{%
		Pour déterminer le signe d'un produit de plusieurs facteurs on compte le nombre de facteurs négatifs :
		
		\begin{itemize}
			\item s'il est pair, le résultat est positif;
			\item s'il est impair, il est négatif;
		\end{itemize}
	}
\end{mymeth}

\begin{myexs}
	\vspace*{-1cm}
	
	\iftoggle{eleve}{%
		\begin{eqnarray*}
			A &=& -3  \times 2 \times (-1) \times 6\\
			A &=& \hrulefill
		\end{eqnarray*} 
		Il y a \hrulefill
		
		
		\begin{eqnarray*}
			B &=& -5 \times 2  \times (-4) \times (-1) \times 8\\
			B &=& \hrulefill				
		\end{eqnarray*} 
		Il y a \hrulefill
	}{%
		\begin{eqnarray*}
			A &=& -3  \times 2 \times (-1) \times 6\\
			A &=& 36 				
		\end{eqnarray*} 
		Il y a deux facteurs négatifs, le produit est positif.
		
		
		\begin{eqnarray*}
			B &=& -5 \times 2  \times (-4) \times (-1) \times 8\\
			B &=& -320 				
		\end{eqnarray*} 
		Il y a trois facteurs négatifs, le produit est négatif.
	}
\end{myexs}
