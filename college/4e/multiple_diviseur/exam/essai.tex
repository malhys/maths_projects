\documentclass[a4paper,11pt]{exam}
\printanswers % pour imprimer les réponses (corrigé)
%\noprintanswers % Pour ne pas imprimer les réponses (énoncé)
%\addpoints % Pour compter les points
 \noaddpoints % pour ne pas compter les points
\qformat{\textbf{\thequestion ) }\hfill} % Pour définir le style des questions (facultatif)
\usepackage{color} % définit une nouvelle couleur
\shadedsolutions % définit le style des réponses
% \framedsolutions % définit le style des réponses
\definecolor{SolutionColor}{rgb}{0.8,0.9,1} % bleu ciel
\renewcommand{\solutiontitle}{\noindent\textbf{Solution:}\par\noindent} % Définit le titre des solutions

\makeatletter

\def\maketitle{%
	\par\centering\Huge\textbf{\@title}%
	%\par{\@author}%
	\par}

\makeatother


%\usepackage{../../pas-math}
%\usepackage{../../moncours}


%\usepackage{pas-cours}
%-------------------------------------------------------------------------------
%          -Packages nécessaires pour écrire en Français et en UTF8-
%-------------------------------------------------------------------------------
\usepackage[utf8]{inputenc}
\usepackage[frenchb]{babel}
\usepackage[T1]{fontenc}
\usepackage{lmodern}
\usepackage{textcomp}



%-------------------------------------------------------------------------------

%-------------------------------------------------------------------------------
%                          -Outils de mise en forme-
%-------------------------------------------------------------------------------
\usepackage{hyperref}
\hypersetup{pdfstartview=XYZ}
%\usepackage{enumerate}
\usepackage{graphicx}
\usepackage{multicol}
\usepackage{tabularx}
\usepackage{multirow}


\usepackage{anysize} %%pour pouvoir mettre les marges qu'on veut
%\marginsize{2.5cm}{2.5cm}{2.5cm}{2.5cm}

\usepackage{indentfirst} %%pour que les premier paragraphes soient aussi indentés
\usepackage{verbatim}
\usepackage{enumitem}
\usepackage[usenames,dvipsnames,svgnames,table]{xcolor}

\usepackage{variations}

%-------------------------------------------------------------------------------


%-------------------------------------------------------------------------------
%                  -Nécessaires pour écrire des mathématiques-
%-------------------------------------------------------------------------------
\usepackage{amsfonts}
\usepackage{amssymb}
\usepackage{amsmath}
\usepackage{amsthm}
\usepackage{tikz}
\usepackage{xlop}
%-------------------------------------------------------------------------------



%-------------------------------------------------------------------------------


%-------------------------------------------------------------------------------
%                    - Mise en forme avancée
%-------------------------------------------------------------------------------

\usepackage{ifthen}
\usepackage{ifmtarg}


\newcommand{\ifTrue}[2]{\ifthenelse{\equal{#1}{true}}{#2}{$\qquad \qquad$}}

%-------------------------------------------------------------------------------

%-------------------------------------------------------------------------------
%                     -Mise en forme d'exercices-
%-------------------------------------------------------------------------------
%\newtheoremstyle{exostyle}
%{\topsep}% espace avant
%{\topsep}% espace apres
%{}% Police utilisee par le style de thm
%{}% Indentation (vide = aucune, \parindent = indentation paragraphe)
%{\bfseries}% Police du titre de thm
%{.}% Signe de ponctuation apres le titre du thm
%{ }% Espace apres le titre du thm (\newline = linebreak)
%{\thmname{#1}\thmnumber{ #2}\thmnote{. \normalfont{\textit{#3}}}}% composants du titre du thm : \thmname = nom du thm, \thmnumber = numéro du thm, \thmnote = sous-titre du thm

%\theoremstyle{exostyle}
%\newtheorem{exercice}{Exercice}
%
%\newenvironment{questions}{
%\begin{enumerate}[\hspace{12pt}\bfseries\itshape a.]}{\end{enumerate}
%} %mettre un 1 à la place du a si on veut des numéros au lieu de lettres pour les questions 
%-------------------------------------------------------------------------------

%-------------------------------------------------------------------------------
%                    - Mise en forme de tableaux -
%-------------------------------------------------------------------------------

\renewcommand{\arraystretch}{1.7}

\setlength{\tabcolsep}{1.2cm}

%-------------------------------------------------------------------------------



%-------------------------------------------------------------------------------
%                    - Racourcis d'écriture -
%-------------------------------------------------------------------------------

% Angles orientés (couples de vecteurs)
\newcommand{\aopp}[2]{(\vec{#1}, \vec{#2})} %Les deuc vecteurs sont positifs
\newcommand{\aopn}[2]{(\vec{#1}, -\vec{#2})} %Le second vecteur est négatif
\newcommand{\aonp}[2]{(-\vec{#1}, \vec{#2})} %Le premier vecteur est négatif
\newcommand{\aonn}[2]{(-\vec{#1}, -\vec{#2})} %Les deux vecteurs sont négatifs

%Ensembles mathématiques
\newcommand{\naturels}{\mathbb{N}} %Nombres naturels
\newcommand{\relatifs}{\mathbb{Z}} %Nombres relatifs
\newcommand{\rationnels}{\mathbb{Q}} %Nombres rationnels
\newcommand{\reels}{\mathbb{R}} %Nombres réels
\newcommand{\complexes}{\mathbb{C}} %Nombres complexes


%Intégration des parenthèses aux cosinus
\newcommand{\cosP}[1]{\cos\left(#1\right)}
\newcommand{\sinP}[1]{\sin\left(#1\right)}


%Probas stats
\newcommand{\stat}{statistique}
\newcommand{\stats}{statistiques}
%-------------------------------------------------------------------------------

%-------------------------------------------------------------------------------
%                    - Mise en page -
%-------------------------------------------------------------------------------

\newcommand{\twoCol}[1]{\begin{multicols}{2}#1\end{multicols}}


\setenumerate[1]{font=\bfseries,label=\textit{\alph*})}
\setenumerate[2]{font=\bfseries,label=\arabic*)}


%-------------------------------------------------------------------------------
%                    - Elements cours -
%-------------------------------------------------------------------------------



\makeatletter

\def\maketitle{{\centering%
	\par{\huge\textbf{\@title}}%
	\par{\@date}%
	\par}}

\makeatother

\lhead{NOM Pr\'enom :}
\rhead{\textbf{Les r\'eponses doivent \^etre justifi\'ees}}
\cfoot{\thepage / \pageref{LastPage}}
%\author{}
\date{28 Septembre 2016}
\title{DS num\'ero 1}




\begin{document}
	
	\maketitle
\begin{small}
\begin{center}
	\begin{tabular}{|@{\ }l@{}|@{\ }c@{\ }|}
		\hline
		Effectuer une division euclidienne &  \\
		\hline
		Déterminer si un entier est ou n'est pas multiple ou diviseur d'un autre entier &  $\qquad$\\
		\hline
		Démontrer : utiliser un raisonnement logique et des règles établies pour parvenir à une conclusion. &  \\
		\hline
		Présenter la démarche suivie, les résultats obtenus, communiquer à l’aide d’un langage adapté. &  \\
		\hline
	\end{tabular}
\end{center}
\end{small}	
	

\section{Heure de fin d'un film}

Le film Vice-Versa dure \num{5700} secondes. 


\begin{questions}
	\question Si la séance débute à 20 heures, à quelle heure se terminera-t-elle ?
	
	\begin{solution}
		\begin{eqnarray*}
			5700 \div 60 &=& 95  \\
			95 \div 60 &=& 1 \quad (reste\; 35)\\
		\end{eqnarray*}
		
		Le film dure 95 minutes, soit 1 heure et 35 minutes. Il terminera donc à 21 h 30.
	\end{solution}
\end{questions}
	
\section{Répartition d'une somme d'argent}
	Un groupe de moins de 40 personnes doit se répartir équitablement une somme de 229 €. Il reste alors 19 euros. Une autre fois, ce même groupe doit se répartir équitablement 474  €, cette fois ci il reste 12 €.
	\begin{questions} % Début de l'examen
		
		\question Combien y a t-il de personnes dans le groupe ?
		\begin{solution}
			\begin{eqnarray*}
				\num{229} & - & \num{19} = \num{210} \\
				\num{474} & - & \num{12} = \num{462} \\
			\end{eqnarray*}
			
			Le reste le plus grand est 19, il y a donc plus de 19 personnes.
			\num{210} et \num{462} sont des multiples de 21 ($21 \times 10$ et $21 \times 22$)? Il y a donc 21 personnes dans le groupe. 
			
		\end{solution}
		
		\question Ce groupe de 21 personnes décide de se répartir ce qu'il reste équitablement. Combien chaque personne reçoit-elle en plus ? Quelle somme auront-ils reçu au total ?
		
		\begin{solution}
			\begin{eqnarray*}
				\num{19} + \num{12} &= \num{31} \\
				\num{31} \div \num{21} &= 1 & (reste \; 10) \\
			\end{eqnarray*}
			Chaque personne reçoit une pièce de plus.
			
			\begin{equation*}
				\num{10} + \num{22} + \num{1} = \num{33}
			\end{equation*}
			Au total, ils reçoivent chacun \num{33} pièces.
		\end{solution}
	\end{questions}
	
\section{Trouver un nombre}
	
	Je suis un nombre entier de 4 chiffres, multiple de 9 et de 10.
	Mon chiffre des dizaines est le même que mon chiffre des centaines.
	Mon chiffre des milliers divise tous les nombres.
	
	
	\begin{questions}
		\question Qui suis-je ?
		\begin{solution}
			Je suis \num{1440}.
		\end{solution}
	\end{questions}
	
%	\newpage
	
\section{Rangées de pièces}
	Zoé possède 72 pièces de 1 €. Elle souhaite les disposer en rangées parallèles contenant toutes le même nombre de pièces et qu'il n'en reste aucune non rangée.
	
	\begin{questions}
		\question Sur chaque rangée, Zoé peut-elle disposer 5 pièces ? 4 pièces ? Si oui, combien y a-t-il de rangées ?
		\begin{solution}
			72 est un multiple de 4 mais pas de 5. Donc elle peut disposer 4 pièces sur ses rangées ($72 \div 4 = 18$, donc 18 rangées), mais pas 5.
		\end{solution}
		
		\question Déterminer toutes les dispositions possibles de ces 72 pièces.
		\begin{solution}
			Les possibilités :
			\begin{multicols}{2}
				\begin{itemize}
					\item 2 rangées de 36 pièces;
					\item 3 rangées de 24 pièces;
					\item 4 rangées de 18 pièces;
					\item 6 rangées de 12 pièces;
					\item 8 rangées de 9 pièces;
					\item 9 rangées de 8 pièces;
					\item 12 rangées de 6 pièces;
					\item 18 rangées de 4 pièces;
					\item 24 rangées de 3 pièces;
					\item 36 rangées de 2 pièces.
				\end{itemize}
			\end{multicols}
			
		\end{solution}
	\end{questions}
	
	\section{Compléter un nombre}
	
	\begin{questions}
		\question Par quels chiffres peut-on remplacer le symbole $\bullet$ pour que le nombre 56$\bullet$ soit divisible par 3 ?
		\begin{solution}
			$5+6 = 11$, pour obtenir un multiple de 3, on peut remplacer $\bullet$ par 1; 4 ou 7.
		\end{solution}
	\end{questions} 
	
\section{Trouver un nombre 2}

On écrit un nombre avec les seuls chiffres 0, 1, 6 et 8 utilisés une seule fois chacun.
\begin{questions}
	\question \'Ecrire le plus grand multiple de 5, de 2.
	\begin{solution}
		\num{8610} est le plus grand multiple de 5 et 2 qu'il est possible d'écrire avec ces chiffres.
	\end{solution}
	
	\question \'Ecrire le plus grand multiple de 4.
	\begin{solution}
		Le plus grand multiple de 4 qu'il est possible d'écrire est \num{8160}.
	\end{solution}
	
	\question Est-il possible d'écrire un multiple de 3 ? Si oui, donner le plus grand possible.
	\begin{solution}
		$ 0 + 1 + 6 + 8 = 15$. 15 est un multiple de 3, donc on peut écrire un multiple de 3. Le plus grand est \num{8610}.
	\end{solution}
\end{questions}
\end{document}