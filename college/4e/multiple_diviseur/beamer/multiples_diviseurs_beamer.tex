\documentclass[xcolor={dvipsnames}]{beamer}
%\usepackage[utf8]{inputenc}
\usetheme{CambridgeUS}
\usecolortheme{rose}

%-------------------------------------------------------------------------------
%          -Packages nécessaires pour écrire en Français et en UTF8-
%-------------------------------------------------------------------------------
\usepackage[utf8]{inputenc}
\usepackage[frenchb]{babel}
\usepackage[T1]{fontenc}
\usepackage{lmodern}
\usepackage{textcomp}

%-------------------------------------------------------------------------------

%-------------------------------------------------------------------------------
%                          -Outils de mise en forme-
%-------------------------------------------------------------------------------
\usepackage{hyperref}
\hypersetup{pdfstartview=XYZ}
\usepackage{enumerate}
\usepackage{graphicx}
%\usepackage{multicol}
%\usepackage{tabularx}

%\usepackage{anysize} %%pour pouvoir mettre les marges qu'on veut
%\marginsize{2.5cm}{2.5cm}{2.5cm}{2.5cm}

\usepackage{indentfirst} %%pour que les premier paragraphes soient aussi indentés
\usepackage{verbatim}
%\usepackage[table]{xcolor}  
%\usepackage{multirow}
\usepackage{ulem}
%-------------------------------------------------------------------------------


%-------------------------------------------------------------------------------
%                  -Nécessaires pour écrire des mathématiques-
%-------------------------------------------------------------------------------
\usepackage{amsfonts}
\usepackage{amssymb}
\usepackage{amsmath}
\usepackage{amsthm}
\usepackage{tikz}
\usepackage{xlop}
\usepackage[output-decimal-marker={,}]{siunitx}
%-------------------------------------------------------------------------------


%-------------------------------------------------------------------------------
%                    - Mise en forme 
%-------------------------------------------------------------------------------

\newcommand{\bu}[1]{\underline{\textbf{#1}}}


\usepackage{ifthen}


\newcommand{\ifTrue}[2]{\ifthenelse{\equal{#1}{true}}{#2}{$\qquad \qquad$}}

\newcommand{\kword}[1]{\textcolor{red}{\underline{#1}}}


%-------------------------------------------------------------------------------



%-------------------------------------------------------------------------------
%                    - Racourcis d'écriture -
%-------------------------------------------------------------------------------

% Angles orientés (couples de vecteurs)
\newcommand{\aopp}[2]{(\vec{#1}, \vec{#2})} %Les deuc vecteurs sont positifs
\newcommand{\aopn}[2]{(\vec{#1}, -\vec{#2})} %Le second vecteur est négatif
\newcommand{\aonp}[2]{(-\vec{#1}, \vec{#2})} %Le premier vecteur est négatif
\newcommand{\aonn}[2]{(-\vec{#1}, -\vec{#2})} %Les deux vecteurs sont négatifs

%Ensembles mathématiques
\newcommand{\naturels}{\mathbb{N}} %Nombres naturels
\newcommand{\relatifs}{\mathbb{Z}} %Nombres relatifs
\newcommand{\rationnels}{\mathbb{Q}} %Nombres rationnels
\newcommand{\reels}{\mathbb{R}} %Nombres réels
\newcommand{\complexes}{\mathbb{C}} %Nombres complexes


%Intégration des parenthèses aux cosinus
\newcommand{\cosP}[1]{\cos\left(#1\right)}
\newcommand{\sinP}[1]{\sin\left(#1\right)}

%Fractions
\newcommand{\myfrac}[2]{{\LARGE $\frac{#1}{#2}$}}

%Vocabulaire courrant
\newcommand{\cad}{c'est-à-dire}

%Droites
\newcommand{\dte}[1]{droite $(#1)$}
\newcommand{\fig}[1]{figure $#1$}
\newcommand{\sym}{symétrique}
\newcommand{\syms}{symétriques}
\newcommand{\asym}{axe de symétrie}
\newcommand{\asyms}{axes de symétrie}
\newcommand{\seg}[1]{$[#1]$}
\newcommand{\monAngle}[1]{$\widehat{#1}$}
\newcommand{\bissec}{bissectrice}
\newcommand{\mediat}{médiatrice}
\newcommand{\ddte}[1]{$[#1)$}

%Figures
\newcommand{\para}{parallélogramme}
\newcommand{\paras}{parallélogrammes}
\newcommand{\myquad}{quadrilatère}
\newcommand{\myquads}{quadrilatères}
\newcommand{\co}{côtés opposés}
\newcommand{\diag}{diagonale}
\newcommand{\diags}{diagonales}
\newcommand{\supp}{supplémentaires}
\newcommand{\car}{carré}
\newcommand{\cars}{carrés}
\newcommand{\rect}{rectangle}
\newcommand{\rects}{rectangles}
\newcommand{\los}{losange}
\newcommand{\loss}{losanges}


%----------------------------------------------------


\usepackage{../../../../pas-math}
\usepackage{../../../../moncours_beamer}


\graphicspath{{../img/}}

\title{Effectuer des calculs numériques}
\author{}\institute{}


\AtBeginSection[]
{
	\begin{frame}
		\frametitle{Sommaire}
		\tableofcontents[currentsection, hideallsubsections]
	\end{frame} 
}


%\AtBeginSubsection[]
%{
%	\begin{frame}
%		\frametitle{Sommaire}
%		\tableofcontents[currentsection, currentsubsection]
%	\end{frame} 
%}

\begin{document}



\begin{frame}
  \titlepage 
\end{frame}

\section{Division euclidienne}

\section{Multiples et diviseurs}



\begin{frame}
	
	\begin{alertblock}{A retenir :}

	Effectuer la \kword{division euclidienne} (ou division entière) d’un nombre entier $a$ par un nombre entier $b$, c’est trouver le \kword{quotient entier} et le \kword{reste} de la division de a par b.
	
	
	Le nombre a est appelé le \kword{dividende} et le nombre b est appelé le \kword{diviseur}.
		
		
	\end{alertblock}
	
	\begin{exampleblock}<2>{Exemples}	
		\begin{itemize}
			\item $48$ est un multiple de $12$, car $48 = 4 \times 12;$
			\item $0$ est multiple de tout nombre entier.
		\end{itemize}
	\end{exampleblock}
	
\end{frame}

\section{Critères de divisibilité}

\begin{frame}
	
	\begin{alertblock}{A retenir :}
		Un nombre entier est divisible :
		\begin{itemize}
			\item<alert@2> \textbf{par \num{2}} \onslide*<2->{si son chiffre des unités est \num{0}, \num{2}, \num{4}, \num{6} ou \num{8};}
			\item<alert@3> \textbf{par \num{3}} \onslide*<3->{si la somme de ses chiffres est divisible par \num{3};}
			\item<alert@4> \textbf{par \num{4}} \onslide*<4->{si le nombre formé par ses deux derniers chiffres est divisible par \num{4};}
			\item<alert@5> \textbf{par \num{5}} \onslide*<5->{si son chiffre des unités est \num{0} ou \num{5};}
			\item<alert@6> \textbf{par \num{9}} \onslide*<6->{si la somme de ses chiffres est divisible par \num{9};}
			\item<alert@7> \textbf{par \num{10}} \onslide*<7->{si son chiffre des unités est \num{0}.}
		\end{itemize}
	\end{alertblock}
	
	\begin{exampleblock}{Exemple : $\num{2160}$}<8->
		\begin{itemize}
			\item<9-> le chiffre des unités est \num{0}, donc \num{2160} est divisible par \num{2}, \num{5} et \num{10};
			\item<10-> \num{60} est divisible par \num{4}, donc \num{2160} est divisible par \num{4};
			\item<11-> $\num{2}+\num{1}+\num{7}+\num{0}=\num{9}$, \num{9} est divisible par \num{3} et \num{9}, donc \num{2170} est divisible par \num{3} et \num{9}.
		\end{itemize}
	\end{exampleblock}
\end{frame}
\end{document}