\documentclass[xcolor={dvipsnames}]{beamer}
%\usepackage[utf8]{inputenc}
\usetheme{CambridgeUS}
\usecolortheme{rose}

\input{../../../../utils_maths_beamer}


\usepackage{../../../../pas-math}
\usepackage{../../../../moncours_beamer}


\graphicspath{{../img/}}

\title{Effectuer des calculs numériques}
\author{}\institute{}


\AtBeginSection[]
{
	\begin{frame}
		\frametitle{Sommaire}
		\tableofcontents[currentsection, hideallsubsections]
	\end{frame} 
}


%\AtBeginSubsection[]
%{
%	\begin{frame}
%		\frametitle{Sommaire}
%		\tableofcontents[currentsection, currentsubsection]
%	\end{frame} 
%}

\begin{document}



\begin{frame}
  \titlepage 
\end{frame}

\section{Division euclidienne}

\section{Multiples et diviseurs}



\begin{frame}
	
	\begin{alertblock}{A retenir :}

	Effectuer la \kword{division euclidienne} (ou division entière) d’un nombre entier $a$ par un nombre entier $b$, c’est trouver le \kword{quotient entier} et le \kword{reste} de la division de a par b.
	
	
	Le nombre a est appelé le \kword{dividende} et le nombre b est appelé le \kword{diviseur}.
		
		
	\end{alertblock}
	
	\begin{exampleblock}<2>{Exemples}	
		\begin{itemize}
			\item $48$ est un multiple de $12$, car $48 = 4 \times 12;$
			\item $0$ est multiple de tout nombre entier.
		\end{itemize}
	\end{exampleblock}
	
\end{frame}

\section{Critères de divisibilité}

\begin{frame}
	
	\begin{alertblock}{A retenir :}
		Un nombre entier est divisible :
		\begin{itemize}
			\item<alert@2> \textbf{par \num{2}} \onslide*<2->{si son chiffre des unités est \num{0}, \num{2}, \num{4}, \num{6} ou \num{8};}
			\item<alert@3> \textbf{par \num{3}} \onslide*<3->{si la somme de ses chiffres est divisible par \num{3};}
			\item<alert@4> \textbf{par \num{4}} \onslide*<4->{si le nombre formé par ses deux derniers chiffres est divisible par \num{4};}
			\item<alert@5> \textbf{par \num{5}} \onslide*<5->{si son chiffre des unités est \num{0} ou \num{5};}
			\item<alert@6> \textbf{par \num{9}} \onslide*<6->{si la somme de ses chiffres est divisible par \num{9};}
			\item<alert@7> \textbf{par \num{10}} \onslide*<7->{si son chiffre des unités est \num{0}.}
		\end{itemize}
	\end{alertblock}
	
	\begin{exampleblock}{Exemple : $\num{2160}$}<8->
		\begin{itemize}
			\item<9-> le chiffre des unités est \num{0}, donc \num{2160} est divisible par \num{2}, \num{5} et \num{10};
			\item<10-> \num{60} est divisible par \num{4}, donc \num{2160} est divisible par \num{4};
			\item<11-> $\num{2}+\num{1}+\num{7}+\num{0}=\num{9}$, \num{9} est divisible par \num{3} et \num{9}, donc \num{2170} est divisible par \num{3} et \num{9}.
		\end{itemize}
	\end{exampleblock}
\end{frame}
\end{document}