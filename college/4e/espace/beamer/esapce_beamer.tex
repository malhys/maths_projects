\documentclass[xcolor={dvipsnames}]{beamer}
%\usepackage[utf8]{inputenc}
\usetheme{CambridgeUS}
\usecolortheme{rose}

\input{../../../../utils_maths_beamer}


\usepackage{../../../../pas-math}
\usepackage{../../../../moncours_beamer}


\graphicspath{{../img/}}

\title{Utiliser le théorème de Thalès}
\author{}\institute{}


\AtBeginSection[]
{
	\begin{frame}
		\frametitle{Sommaire}
		\tableofcontents[currentsection, hideallsubsections]
	\end{frame} 
}


%\AtBeginSubsection[]
%{
%	\begin{frame}
%		\frametitle{Sommaire}
%		\tableofcontents[currentsection, currentsubsection]
%	\end{frame} 
%}

\begin{document}



\begin{frame}
  \titlepage 
\end{frame}

\section{Repérage dans un parallélépipède rectangle}





\section{Repérage sur Terre}

\begin{frame}
	\begin{alertblock}{Définition}
		On considère que la Terre est une sphère. L'origine du repère est le centre de la Terre, les axes sont :
		\begin{itemize}
			\item Un cercle : l'\kword{équateur};
			\item Un demi-cercle : le \kword{méridien de Greenwich}.
		\end{itemize}
		
		
	\end{alertblock}
\end{frame}

\begin{frame}
	\begin{alertblock}{Définition}
		La Terre est quadrillée par des cercles \kword{parallèles} à l'équateur, et des demi-cercles allant d'un pôle à l'autre, appelés \kword{méridiens} :
		\begin{columns}
			\begin{column}{7cm}
				\begin{itemize}
					\item l'abscisse d'un point correspond à l'angle, orienté Ouest ou Est, entre le méridien de Greenwich et et le méridien du point. C'est sa \kword{longitude}. 
					\item L'ordonnée d'un point correspond à l'angle, orienté Nord ou Sud, entre l'équateur et le parallèle du point. C'est sa \kword{latitude}.
				\end{itemize}
			\end{column}
			
			\begin{column}{5cm}
				\begin{center}
					\includegraphics[scale=0.45]{../img/terre}
				\end{center}
			\end{column}
		\end{columns}
		
		
		
	\end{alertblock}
\end{frame}

\begin{frame}
	\begin{exampleblock}{Exemples}
		
		\begin{center}
			\includegraphics[scale=0.6]{../img/terre1}
		\end{center}
				
		La longitude de New York est \pause $74$° Ouest

	\end{exampleblock}
\end{frame}			


\begin{frame}
	\begin{exampleblock}{Exemples}
		
		\begin{center}
			\includegraphics[scale=0.6]{../img/terre2}
		\end{center}
		
		La latitude de Madrid est \pause $40$° Nord.
		
	\end{exampleblock}
\end{frame}			


\end{document}