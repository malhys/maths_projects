\documentclass[xcolor={dvipsnames}]{beamer}
%\usepackage[utf8]{inputenc}
\usetheme{CambridgeUS}
\usecolortheme{rose}

\input{../../../../utils_maths_beamer}


\usepackage{../../../../pas-math}
\usepackage{../../../../moncours_beamer}


\graphicspath{{../img/}}

\title{Utiliser le théorème de Thalès}
\author{}\institute{}


\AtBeginSection[]
{
	\begin{frame}
		\frametitle{Sommaire}
		\tableofcontents[currentsection, hideallsubsections]
	\end{frame} 
}


%\AtBeginSubsection[]
%{
%	\begin{frame}
%		\frametitle{Sommaire}
%		\tableofcontents[currentsection, currentsubsection]
%	\end{frame} 
%}

\begin{document}



\begin{frame}
  \titlepage 
\end{frame}

\section{Homothéties}

\begin{frame}
	\frametitle{Homothéties}
	\framesubtitle{}
	
	\begin{alertblock}{Définition}
		
		\begin{columns}
			\begin{column}{8cm}
				
				Le point $M'$ est l'image du point $M$ par l'\kw{\homo\ de centre $O$ et de rapport $k$} ($k$ est un nombre différent de $0$) lorsque :
				\begin{itemize}
					\item si $k$ est positif : $M' \in [OM) $ ou si $k$ est négatif : $O \in [MM']$
					\item $OM' = k \times OM$ si $k$ est positif, $OM'=-k \times OM$ si $k$ est négatif.
				\end{itemize}	
				
			\end{column}
			\begin{column}{4cm}
				\begin{center}
					\includegraphics[scale=0.65]{../img/homo_v2}
				\end{center}
			\end{column}
		\end{columns}
		
			
			
	\end{alertblock}
\end{frame}

\begin{frame}
	\frametitle{Homothéties}
	\framesubtitle{}
	
	\begin{block}{Remarque}
		\begin{itemize}
			\item Si $k>1$ ou $k<-1$, \pause la figure est un agrandissement de la figure initiale.\pause
			\item Si $-1<k<0$ ou $0<k<1$,\pause la figure est une réduction de a figure initiale. 
		\end{itemize}
	\end{block}
	
	\begin{block}{Propriétés}
		Par une \homo\ de rapport $k$, l'image :
		\begin{itemize}
			\item d'une droite est une droite qui lui est parallèle;
			\item d'un segment $[MN]$ est un segment $[M'N']$ de longueur $k \times MN$ (si $k>0$) ou $-k \times MN$ (si $k<0$)
		\end{itemize}
	\end{block}
\end{frame}

\section{Théorème de Thalès}

\begin{frame}
	%\frametitle{Théorème de Thalès}

	\begin{alertblock}{Propriété}
		Si deux droites \dte{BM} et \dte{CN} sécantes en $A$ sont coupées par deux droites parallèles \dte{BC} et \dte{MN}, \textbf{alors :}
		
		\begin{equation*}
		\dfrac{A\textcolor{Blue}{M}}{A\textcolor{Red}{B}}=\dfrac{A\textcolor{Blue}{N}}{A\textcolor{Red}{C}}=\dfrac{\textcolor{Blue}{MN}}{\textcolor{Red}{BC}}
		\end{equation*}
	\end{alertblock}
	
	\begin{block}{Configurations de Thalès}
		\begin{columns}
			\begin{column}{3cm}
				\includegraphics[scale=0.3]{../img/thales3}
			\end{column}
			\begin{column}{3cm}
				\includegraphics[scale=0.27]{../img/thales2}
			\end{column}
			
			\begin{column}{3cm}
				\includegraphics[scale=0.3]{../img/thales1}
			\end{column}
		\end{columns}
		
		Le triangle AMN est l'image du triangle ABC par une \homo\ de centre A.
	\end{block}
\end{frame}


\section{Réciproque du théorème de Thalès}

\end{document}