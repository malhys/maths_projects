\documentclass[12pt,a4paper]{article}

%\usepackage[left=1.5cm,right=1.5cm,top=1cm,bottom=2cm]{geometry}
\usepackage[in, plain]{fullpage}
\usepackage{array}
\usepackage{../../../pas-math}
\usepackage{../../../moncours}


%\usepackage{pas-cours}
%-------------------------------------------------------------------------------
%          -Packages nécessaires pour écrire en Français et en UTF8-
%-------------------------------------------------------------------------------
\usepackage[utf8]{inputenc}
\usepackage[frenchb]{babel}
\usepackage[T1]{fontenc}
\usepackage{lmodern}
\usepackage{textcomp}



%-------------------------------------------------------------------------------

%-------------------------------------------------------------------------------
%                          -Outils de mise en forme-
%-------------------------------------------------------------------------------
\usepackage{hyperref}
\hypersetup{pdfstartview=XYZ}
%\usepackage{enumerate}
\usepackage{graphicx}
\usepackage{multicol}
\usepackage{tabularx}
\usepackage{multirow}


\usepackage{anysize} %%pour pouvoir mettre les marges qu'on veut
%\marginsize{2.5cm}{2.5cm}{2.5cm}{2.5cm}

\usepackage{indentfirst} %%pour que les premier paragraphes soient aussi indentés
\usepackage{verbatim}
\usepackage{enumitem}
\usepackage[usenames,dvipsnames,svgnames,table]{xcolor}

\usepackage{variations}

%-------------------------------------------------------------------------------


%-------------------------------------------------------------------------------
%                  -Nécessaires pour écrire des mathématiques-
%-------------------------------------------------------------------------------
\usepackage{amsfonts}
\usepackage{amssymb}
\usepackage{amsmath}
\usepackage{amsthm}
\usepackage{tikz}
\usepackage{xlop}
%-------------------------------------------------------------------------------



%-------------------------------------------------------------------------------


%-------------------------------------------------------------------------------
%                    - Mise en forme avancée
%-------------------------------------------------------------------------------

\usepackage{ifthen}
\usepackage{ifmtarg}


\newcommand{\ifTrue}[2]{\ifthenelse{\equal{#1}{true}}{#2}{$\qquad \qquad$}}

%-------------------------------------------------------------------------------

%-------------------------------------------------------------------------------
%                     -Mise en forme d'exercices-
%-------------------------------------------------------------------------------
%\newtheoremstyle{exostyle}
%{\topsep}% espace avant
%{\topsep}% espace apres
%{}% Police utilisee par le style de thm
%{}% Indentation (vide = aucune, \parindent = indentation paragraphe)
%{\bfseries}% Police du titre de thm
%{.}% Signe de ponctuation apres le titre du thm
%{ }% Espace apres le titre du thm (\newline = linebreak)
%{\thmname{#1}\thmnumber{ #2}\thmnote{. \normalfont{\textit{#3}}}}% composants du titre du thm : \thmname = nom du thm, \thmnumber = numéro du thm, \thmnote = sous-titre du thm

%\theoremstyle{exostyle}
%\newtheorem{exercice}{Exercice}
%
%\newenvironment{questions}{
%\begin{enumerate}[\hspace{12pt}\bfseries\itshape a.]}{\end{enumerate}
%} %mettre un 1 à la place du a si on veut des numéros au lieu de lettres pour les questions 
%-------------------------------------------------------------------------------

%-------------------------------------------------------------------------------
%                    - Mise en forme de tableaux -
%-------------------------------------------------------------------------------

\renewcommand{\arraystretch}{1.7}

\setlength{\tabcolsep}{1.2cm}

%-------------------------------------------------------------------------------



%-------------------------------------------------------------------------------
%                    - Racourcis d'écriture -
%-------------------------------------------------------------------------------

% Angles orientés (couples de vecteurs)
\newcommand{\aopp}[2]{(\vec{#1}, \vec{#2})} %Les deuc vecteurs sont positifs
\newcommand{\aopn}[2]{(\vec{#1}, -\vec{#2})} %Le second vecteur est négatif
\newcommand{\aonp}[2]{(-\vec{#1}, \vec{#2})} %Le premier vecteur est négatif
\newcommand{\aonn}[2]{(-\vec{#1}, -\vec{#2})} %Les deux vecteurs sont négatifs

%Ensembles mathématiques
\newcommand{\naturels}{\mathbb{N}} %Nombres naturels
\newcommand{\relatifs}{\mathbb{Z}} %Nombres relatifs
\newcommand{\rationnels}{\mathbb{Q}} %Nombres rationnels
\newcommand{\reels}{\mathbb{R}} %Nombres réels
\newcommand{\complexes}{\mathbb{C}} %Nombres complexes


%Intégration des parenthèses aux cosinus
\newcommand{\cosP}[1]{\cos\left(#1\right)}
\newcommand{\sinP}[1]{\sin\left(#1\right)}


%Probas stats
\newcommand{\stat}{statistique}
\newcommand{\stats}{statistiques}
%-------------------------------------------------------------------------------

%-------------------------------------------------------------------------------
%                    - Mise en page -
%-------------------------------------------------------------------------------

\newcommand{\twoCol}[1]{\begin{multicols}{2}#1\end{multicols}}


\setenumerate[1]{font=\bfseries,label=\textit{\alph*})}
\setenumerate[2]{font=\bfseries,label=\arabic*)}


%-------------------------------------------------------------------------------
%                    - Elements cours -
%-------------------------------------------------------------------------------





%\makeatletter
%\renewcommand*{\@seccntformat}[1]{\csname the#1\endcsname\hspace{0.1cm}}
%\makeatother


%\author{Olivier FINOT}
\date{}
\title{}

%\newcommand{\disp}{false}

\lhead{CH1 : Calculs numériques}
\rhead{O. FINOT}
%
%\rfoot{Page \thepage}
\begin{document}
%\maketitle

\chap[num=1, color=red]{Effectuer des calculs numériques}{Olivier FINOT, \today }

\begin{myobj}
	\begin{itemize}
		
		\item Construire le symétrique d’un point ou d'une figure par rapport à une droite à la main où à l’aide d’un logiciel;
		\item Construire le symétrique d’un point ou d'une figure par rapport à un point, à la main où à l’aide d’un logiciel;
		\item Utiliser les propriétés de la symétrie axiale ou centrale;
		\item Identifier des symétries dans des figures.		
	\end{itemize}
\end{myobj}

\begin{mycomp}
	\begin{itemize}
		\item \kw{Chercher (Ch2)} :  s’engager    dans    une    démarche    scientifique, observer, questionner, manipuler, expérimenter (sur une feuille de papier, avec des objets, à l’aide de logiciels), émettre des hypothèses, chercher des exemples ou des contre-exemples, simplifier ou particulariser une situation, émettre une conjecture ;
		\item \kw{Raisonner (Ra3)} :  démontrer : utiliser un raisonnement logique et des règles établies (propriétés, théorèmes, formules) pour parvenir à une conclusion ;
		\item \kw{Communiquer (Co2)} :  expliquer à l’oral ou à l’écrit (sa démarche, son raisonnement, un calcul, un protocole   de   construction   géométrique, un algorithme), comprendre les explications d’un autre et argumenter dans l’échange ; 
		
	\end{itemize}
\end{mycomp}




\section{Règles de calcul sur les puissances}

\begin{myact}{1 : 1 p 11}
	 
	\begin{enumerate}[label=\alph*. ]
		\item Dans une minute, il y a 60 secondes, dans une heure il y a 60 minutes et dans une journée il y a 24 heures.
			\begin{equation*}
				60 \times 60 \times 24 \times 365,25 = 3,156 \times 10^7
			\end{equation*}
			
			Dans une année, il y a $3,156 \times 10^7$ secondes.
		
		\item Dans le vide la lumière se déplace à environ $3 \times 10^5$ kilomètres par secondes. 
			\begin{eqnarray*}
				3,156 \times 10^7 \times 3 \times 10^5 &=& 3,156 \times 3 \times 10^7 \times 10^5 \\
														&=& 9,468 \times 10^{12}
			\end{eqnarray*}
			
			Une année lumière correspond à $ \num{9.468 e12}$ kilomètres.
			
		\item Kepler-69c se trouve à $\num{2 700}$ années lumières de la Terre.
		
		\begin{equation*}
			\num{2.7 e3} \times \num{9.468 e12} = \num{2.55636 e16}
		\end{equation*}
		
		$\num{2.55630} > \num{1.324}$, donc Kepler-552b est la plus proche de la Terre.
	\end{enumerate}
		
		
\end{myact}

\begin{myex}
	\begin{itemize}
		\item $5^2 \times 5^4 = 5 \times 5 \times 5 \times 5 \times 5 \times 5 = 5^6$
		\item $\dfrac{\num{e5}}{\num{e3}} = \dfrac{\textcolor{red}{ 10 \times 10 \times 10} \times 10 \times 10}{\textcolor{red}{10 \times 10 \times 10}} = \num{e2}$
		\item $(3 \times 7)^2 = (3 \times 7) \times (3 \times 7) = 3 \times 3 \times 7 \times 7 = 3^2 \times 7^2$
		\item $(\num{e3})^2 = \num{e3} \times \num{e3} = \num{e6}$
	\end{itemize}
\end{myex}
	
	
\begin{mybilan}	
	$a$ et $b$ sont des nombres relatifs et $m$ et $n$ des nombres entiers relatifs.
	
	\begin{itemize}
		\item $a^n \times a^m = a^{n+m}$
		\item $\dfrac{a^n}{a^m}= a^{n-m}$
		\item $(a \times b)^n = a^n \times b^n$
		\item $(a^n)^m = a^{n+m}$
	\end{itemize}
	
\end{mybilan}

\begin{myexos}
	\begin{itemize}
		\item 11, 12, 13 p 14 (Projeté $\rightarrow$ Oral)
		\item 28, 29, 30, 33, 34, 35, 40 - 46 page 15
	\end{itemize}
	
\end{myexos}

\section{Notation scientifique}

\subsection*{Actvité 2 : 2 p 11}
	
\begin{enumerate}[label=\arabic*) ]
	\item Dimension des bactéries sous forme décimale :
		\begin{itemize}
			\item Cellule humaine : $\num{0.00001}$ m
			\item  Salmonelle : $\num{0.000003}$ m
			\item Fièvre jaune : $\num{0.00000002}$ m
			\item Tétanos : $\num{0.000004}$ m
			\item Staphylocoque : $\num{0.000001}$ m
			\item Globule rouge : $\num{0.0000075}$ m
			\item Grippe : $\num{0.00000012}$ m
		\end{itemize}
		
		$\num{0.00000002} < \num{0.00000012} < \num{0.000001} < \num{0.000003} < \num{0.000004} < \num{0.0000075} < \num{0.00001} $.
		
		On a donc, dans l'ordre croissant : Virus de la fièvre jaune, Virus de la grippe, Staphylocoque, Bactérie de la salmonelle, Bacille du tétanos, Globule rouge et Cellule humaine.
		
	\item 
		\begin{enumerate}[label=\alph*. ]
			\item $\num{0.003 e-3} = \num{3e-3} \times \num{e-3}= \num{3} \times 10^{-3-3} = \num{3e-6}$.
			
			Donc \num{3e-6} est la notation scientifique de la longueur de la bactérie de la salmonelle.	
			
			\item Dimension des bactéries en notation scientifique :
			\begin{itemize}
				\item Cellule humaine : $\num{1e-5}$ m
				\item Salmonelle : $\num{3e-6}$ m
				\item Fièvre jaune : $\num{2e-8}$ m
				\item Tétanos : $\num{4e-6}$ m
				\item Staphylocoque : $\num{1e-6}$ m
				\item Globule rouge : $\num{7.5e-6}$ m
				\item Grippe : $\num{1.2e-7}$ m
			\end{itemize}
		\end{enumerate}
		
		On a $\num{2e-8} < \num{1.2e-7} < \num{1e-6} < \num{3e-6} < \num{4e-6} < \num{7.5e-6} < \num{1e-5}$.

		On a donc, dans l'ordre croissant : Virus de la fièvre jaune, Virus de la grippe, Staphylocoque, Bactérie de la salmonelle, Bacille du tétanos, Globule rouge et Cellule humaine.	
	
\end{enumerate}


\end{document}