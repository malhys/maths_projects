\documentclass[a4paper,11pt]{exam}
\printanswers % pour imprimer les réponses (corrigé)
%\noprintanswers % Pour ne pas imprimer les réponses (énoncé)
%\addpoints % Pour compter les points
 \noaddpoints % pour ne pas compter les points
\qformat{\textbf{\thequestion ) }\hfill} % Pour définir le style des questions (facultatif)
\usepackage{color} % définit une nouvelle couleur
\shadedsolutions % définit le style des réponses
% \framedsolutions % définit le style des réponses
\definecolor{SolutionColor}{rgb}{0.8,0.9,1} % bleu ciel
\renewcommand{\solutiontitle}{\noindent\textbf{Solution:}\par\noindent} % Définit le titre des solutions

\makeatletter

\def\maketitle{%
	\par\centering\Huge\textbf{\@title}%
	%\par{\@author}%
	\par}

\makeatother


%\usepackage{../../pas-math}
%\usepackage{../../moncours}


%\usepackage{pas-cours}
%-------------------------------------------------------------------------------
%          -Packages nécessaires pour écrire en Français et en UTF8-
%-------------------------------------------------------------------------------
\usepackage[utf8]{inputenc}
\usepackage[frenchb]{babel}
\usepackage[T1]{fontenc}
\usepackage{lmodern}
\usepackage{textcomp}



%-------------------------------------------------------------------------------

%-------------------------------------------------------------------------------
%                          -Outils de mise en forme-
%-------------------------------------------------------------------------------
\usepackage{hyperref}
\hypersetup{pdfstartview=XYZ}
%\usepackage{enumerate}
\usepackage{graphicx}
\usepackage{multicol}
\usepackage{tabularx}
\usepackage{multirow}


\usepackage{anysize} %%pour pouvoir mettre les marges qu'on veut
%\marginsize{2.5cm}{2.5cm}{2.5cm}{2.5cm}

\usepackage{indentfirst} %%pour que les premier paragraphes soient aussi indentés
\usepackage{verbatim}
\usepackage{enumitem}
\usepackage[usenames,dvipsnames,svgnames,table]{xcolor}

\usepackage{variations}

%-------------------------------------------------------------------------------


%-------------------------------------------------------------------------------
%                  -Nécessaires pour écrire des mathématiques-
%-------------------------------------------------------------------------------
\usepackage{amsfonts}
\usepackage{amssymb}
\usepackage{amsmath}
\usepackage{amsthm}
\usepackage{tikz}
\usepackage{xlop}
%-------------------------------------------------------------------------------



%-------------------------------------------------------------------------------


%-------------------------------------------------------------------------------
%                    - Mise en forme avancée
%-------------------------------------------------------------------------------

\usepackage{ifthen}
\usepackage{ifmtarg}


\newcommand{\ifTrue}[2]{\ifthenelse{\equal{#1}{true}}{#2}{$\qquad \qquad$}}

%-------------------------------------------------------------------------------

%-------------------------------------------------------------------------------
%                     -Mise en forme d'exercices-
%-------------------------------------------------------------------------------
%\newtheoremstyle{exostyle}
%{\topsep}% espace avant
%{\topsep}% espace apres
%{}% Police utilisee par le style de thm
%{}% Indentation (vide = aucune, \parindent = indentation paragraphe)
%{\bfseries}% Police du titre de thm
%{.}% Signe de ponctuation apres le titre du thm
%{ }% Espace apres le titre du thm (\newline = linebreak)
%{\thmname{#1}\thmnumber{ #2}\thmnote{. \normalfont{\textit{#3}}}}% composants du titre du thm : \thmname = nom du thm, \thmnumber = numéro du thm, \thmnote = sous-titre du thm

%\theoremstyle{exostyle}
%\newtheorem{exercice}{Exercice}
%
%\newenvironment{questions}{
%\begin{enumerate}[\hspace{12pt}\bfseries\itshape a.]}{\end{enumerate}
%} %mettre un 1 à la place du a si on veut des numéros au lieu de lettres pour les questions 
%-------------------------------------------------------------------------------

%-------------------------------------------------------------------------------
%                    - Mise en forme de tableaux -
%-------------------------------------------------------------------------------

\renewcommand{\arraystretch}{1.7}

\setlength{\tabcolsep}{1.2cm}

%-------------------------------------------------------------------------------



%-------------------------------------------------------------------------------
%                    - Racourcis d'écriture -
%-------------------------------------------------------------------------------

% Angles orientés (couples de vecteurs)
\newcommand{\aopp}[2]{(\vec{#1}, \vec{#2})} %Les deuc vecteurs sont positifs
\newcommand{\aopn}[2]{(\vec{#1}, -\vec{#2})} %Le second vecteur est négatif
\newcommand{\aonp}[2]{(-\vec{#1}, \vec{#2})} %Le premier vecteur est négatif
\newcommand{\aonn}[2]{(-\vec{#1}, -\vec{#2})} %Les deux vecteurs sont négatifs

%Ensembles mathématiques
\newcommand{\naturels}{\mathbb{N}} %Nombres naturels
\newcommand{\relatifs}{\mathbb{Z}} %Nombres relatifs
\newcommand{\rationnels}{\mathbb{Q}} %Nombres rationnels
\newcommand{\reels}{\mathbb{R}} %Nombres réels
\newcommand{\complexes}{\mathbb{C}} %Nombres complexes


%Intégration des parenthèses aux cosinus
\newcommand{\cosP}[1]{\cos\left(#1\right)}
\newcommand{\sinP}[1]{\sin\left(#1\right)}


%Probas stats
\newcommand{\stat}{statistique}
\newcommand{\stats}{statistiques}
%-------------------------------------------------------------------------------

%-------------------------------------------------------------------------------
%                    - Mise en page -
%-------------------------------------------------------------------------------

\newcommand{\twoCol}[1]{\begin{multicols}{2}#1\end{multicols}}


\setenumerate[1]{font=\bfseries,label=\textit{\alph*})}
\setenumerate[2]{font=\bfseries,label=\arabic*)}


%-------------------------------------------------------------------------------
%                    - Elements cours -
%-------------------------------------------------------------------------------



\makeatletter

\def\maketitle{{\centering%
	\par{\huge\textbf{\@title}}%
	\par{\@date}%
	\par}}

\makeatother

\lhead{NOM Pr\'enom :}
\rhead{\textbf{Les r\'eponses doivent \^etre justifi\'ees}}
\cfoot{\thepage / \pageref{LastPage}}
%\author{}
\date{28 Septembre 2016}
\title{DS num\'ero 1}




\begin{document}
	
	\maketitle

\begin{small}
\begin{center}
	\begin{tabular}{|@{\ }l@{}|@{\ }c@{\ }|}
		\hline
		\'Ecrire scientifiquement un nombre &  \ \ \ \\
		\hline
		Effectuer des opérations avec les puissances &  \\
		\hline
		Démontrer : utiliser un raisonnement logique et des règles établies pour parvenir à une conclusion. &  \\
		\hline
		Présenter la démarche suivie, les résultats obtenus, communiquer à l’aide d’un langage adapté. &  \\
		\hline
	\end{tabular}
\end{center}
\end{small}	
	

\section{Calculs de puissances}

Donner le résultat sous la forme d'une seule puissance.

%\begin{multicols}{2}
	
	\begin{questions}
		\question $10^3 \times 10^4$
	
		\begin{solution}
			$10^{3+4} = 10^7$
		\end{solution}
	
		\question $(10^2)^4$
	
		\begin{solution}
			$10^{2 \times 4} = 10^8$
		\end{solution}
	
		\question  $\dfrac{10 \times \num{e5} \times \num{10}}{\num{e2}} $
		\begin{solution}
			$\dfrac{10^{1+5+1}}{\num{e2}}=\dfrac{\num{e7}}{\num{e2}}=\num{e5}$
		\end{solution}
	    
	    \vspace*{0.5cm}
	     
		\question $\dfrac{5^3 \times 5^4}{5^2 \times 5^3}$
		\begin{solution}
			$\dfrac{5^{3+4}}{5^{2+3}} = \dfrac{5^7}{5^5} = 5^2$
		\end{solution}
	
		\question $\dfrac{3^{4} \times 3^{-2}}{3 \times 3^{5}}$
		\begin{solution}
			$\dfrac{3^{4-2}}{3^{1+5}} = \dfrac{3^{2}}{3^{6}} = 3^{2-6} = 3^{-4}$
		\end{solution}
		
		\question $\dfrac{5^{5}}{25}$
		\begin{solution}
			$\dfrac{5^{5}}{5^{2}} = 5^{5-2} = 5^3$
		\end{solution}
	\end{questions}
%\end{multicols}
	
\section{Les jurons du capitaine Haddock}
	
	Dans les \emph{Aventures de Tintin}, le capitaine HADDOCK est célèbre pour ses jurons. Recopier chaque juron et donner son écriture à l'aide d'une puissance de 10.
	\begin{multicols}{2}

		\begin{questions} % Début de l'examen
			
			\question Mille tonnerres !
			\begin{solution}
				\num{e3}
			\end{solution}
			
			\question Mille millions de mille sabords!
			\begin{solution}
				\num{e12}
			\end{solution}
						
			\question Mille millions de tonnerres de Brest!
			\begin{solution}
				\num{e9}
			\end{solution}
			
			\question Mille milliards de mille sabords!
			\begin{solution}
				\num{e15}
			\end{solution}
			
			\question Mille millions de mille milliards de mille tonnerres de Brest !
			\begin{solution}
				\num{e24}
			\end{solution}
		\end{questions}
	\end{multicols}
	
	
\section{Volume globulaire moyen}
	
Le volume globulaire moyen (VGM) est le volume moyen d'un globule rouge d'une personne. Il se mesure lors q'une prise de sang. Chez un adolescent, le VGM est d'environ 90 femtolitres (1 femtolitre = 1fL = \num{e-15} L).

	\begin{questions}
		\question Combien de litres occupent les vingt-cinq mille milliards de globules rouges présents en moyenne dans le corps ?  
		
		\begin{solution}
			Vingt-cinq mille milliards = $25 \times \num{e3} \times \num{e9} = \num{25e12} = \num{2.5e13}$.
			
			90 femtolitres = $\num{90e-15}=\num{9e-14}$.
			
			$\num{2.5e13} \times \num{9e-14} = \num{22.5} \times 10^{13-14} = \num{22.5e-1} = \num{2.5}$.
			
			Donc, dans le corps d'un adolescent \num{2.5} litres sont occupés par des globules rouges. 
		\end{solution}
	\end{questions}

\newpage	
\section{Encadrement de nombres}

Donner un encadrement par deux puissances de 10 consécutives 

	\begin{questions}
		\question En nombre d'années de l'âge de la Terre : environ \num{4.5} milliards d'années.
		\begin{solution}
			$\num{e9}<\num{4.5e9}<\num{e10}$
		\end{solution}
		
		
		\question En mètre, du diamètre d'une bactérie qui peut atteindre \num{3} $\mu m$.
		\begin{solution}
			$\num{e-6}<\num{3e-6}<\num{e-5}$
			\end{solution}
		

		\question En hertz, de la fréquence d'un processeur tournant à \num{4.1} GHz.
		\begin{solution}
			$\num{e9}<\num{4.1e9}<\num{e10}$
		\end{solution}
	\end{questions}
	
\section{Battements du c\oe ur}	

Le c\oe ur humain effectue environ \num{5000} battements par heure.

	\begin{questions}
		\question \'Ecrire \num{5000} en notation scientifique.
		\begin{solution}
			$\num{5000} = \num{5e3}$
		\end{solution}
		
		\question Calculer le nombre de battements effectués en un jour.
		\begin{solution}
			$ 24 \times \num{5e3} = \num{120e3} = \num{1.2e5}$
		\end{solution}
		
		\question Calculer le nombre de battements effectués pendant une vie de 80 ans. On considère qu'une année comporte 365 jours. Donner la réponse en notation scientifique.
		\begin{solution}
			\begin{itemize}
				\item Nombre de battements pour une année : $365 \times \num{1.2e5} = \num{438e5} = \num{4.38e7}$;
				\item Nombre de battements pour 80 ans : $80 \times \num{4.38e7} = \num{350.4e7} = \num{3.504e9}.$
				
				Soit \num{3.504e9} battements de c\oe ur pendant 80 ans.
			\end{itemize}
		\end{solution}
	\end{questions}
	
\section{Coupe de la Terre}	
	La structure interne de la Terre a été découpées en plusieurs couches en fonction des différentes densités de matière calculées :
	\begin{itemize}
		\item La croûte terrestre, épaisse d'une centaine de km;
		\item Le manteau supérieur qui s'enfonce jusque 650 km;
		\item Le manteau inférieur qui s'étend sur près de \num{2000} km;
		\item Le noyau externe qui s'étend sur presque \num{2300} km;
		\item Le noyau interne.
		
		\begin{questions}
			\question Le rayon de la Terre étant de \num{6400} km environ, exprimer l'étendue de chaque couche en écriture scientifique (on donnera le résultat en km, puis un ordre de grandeur en cm).
			
			\begin{solution}
				\'Epaisseur de chaque couche en notation scientifique en km:
				\begin{itemize}
					\item Croûte terrestre : \num{e2}km;
					\item Manteau supérieur : $\num{6.5e2} - \num{e2} = \num{5.5e2}$ km;
					\item Manteau inférieur : \num{2e3} km;
					\item Noyau externe : \num{2.3e3} km;
					\item Noyau interne : $\num{6400} - (\num{100} + \num{550} + \num{2000} + \num{2300}) = \num{6400} - \num{4950} = \num{1450} = \num{1.45e3}$ km.\\
				\end{itemize}
				
				Ordre de grandeur en cm :
				\begin{itemize}
					\item Croûte terrestre : $\num{e2} \times \num{e2} = \num{e4}$ cm;
					\item Manteau supérieur : $\num{e3} \times \num{e2} =  \num{e5}$ cm;
					\item Manteau inférieur : $\num{e3} \times \num{e2} =  \num{e5}$ cm;
					\item Noyau externe : $\num{e3} \times \num{e2} =  \num{e5}$ cm;
					\item Noyau interne : $\num{e3} \times \num{e2} =  \num{e5}$ cm.
				\end{itemize}
			\end{solution}
		\end{questions}
	\end{itemize}
	\label{LastPage}
\end{document}