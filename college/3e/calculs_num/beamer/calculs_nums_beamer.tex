\documentclass[xcolor={dvipsnames}]{beamer}
%\usepackage[utf8]{inputenc}
\usetheme{CambridgeUS}
\usecolortheme{rose}

\input{../../../../utils_maths_beamer}


\usepackage{../../../../pas-math}
\usepackage{../../../../moncours_beamer}


\graphicspath{{../img/}}

\title{Effectuer des calculs numériques}
\author{}\institute{}


\AtBeginSection[]
{
	\begin{frame}
		\frametitle{Sommaire}
		\tableofcontents[currentsection, hideallsubsections]
	\end{frame} 
}


%\AtBeginSubsection[]
%{
%	\begin{frame}
%		\frametitle{Sommaire}
%		\tableofcontents[currentsection, currentsubsection]
%	\end{frame} 
%}

\begin{document}



\begin{frame}
  \titlepage 
\end{frame}

\section{Règles de calcul sur les puissances}

\section{Notation scientifique}

\begin{frame}
	\frametitle{Activité 2}
	\framesubtitle{1) Dimension des bactéries sous forme décimale}
	
	\begin{itemize}
		\item<alert@2> Cellule humaine : \onslide<2->{ $\num{0.00001}$ m }
		\item<alert@3>  Salmonelle : \onslide<3->{ $\num{0.000003}$ m }
		\item<alert@4> Fièvre jaune : \onslide<4->{ $\num{0.00000002}$ m }
		\item<alert@5> Tétanos : \onslide<5->{ $\num{0.000004}$ m }
		\item<alert@6> Staphylocoque : \onslide<6->{ $\num{0.000001}$ m }
		\item<alert@7> Globule rouge : \onslide<7->{ $\num{0.0000075}$ m }
		\item<alert@8> Grippe : \onslide<8->{ $\num{0.00000012}$ m }
	\end{itemize}
	
	\only<9>{On a : $\num{0.00000002} < \num{0.00000012} < \num{0.000001} < \num{0.000003} < \num{0.000004} < \num{0.0000075} < \num{0.00001} $.}
	
	\only<10>{Donc, dans l'ordre croissant : Virus de la fièvre jaune, Virus de la grippe, Staphylocoque, Bactérie de la salmonelle, Bacille du tétanos, Globule rouge et Cellule humaine.}
\end{frame}

\begin{frame}
	\frametitle{Activité 2}
	\framesubtitle{2) a.}

	\begin{equation*}
		\num{0.003 e-3} = \pause \num{3e-3} \times \num{e-3}= \pause \num{3} \times 10^{-3-3} = \pause \num{3e-6}.
	\end{equation*}	
		 
		
		Donc \num{3e-6} est la notation scientifique de la longueur de la bactérie de la salmonelle.	
	

\end{frame}

\begin{frame}
	\frametitle{Activité 2}
	\framesubtitle{2) b.}
	Dimension des bactéries en notation scientifique :
	\begin{itemize}
		\item<alert@2> Cellule humaine : \onslide<2->{$\num{1e-5}$ m}
		\item<alert@3> Salmonelle : \onslide<3->{$\num{3e-6}$ m}
		\item<alert@4> Fièvre jaune : \onslide<4->{$\num{2e-8}$ m}
		\item<alert@5> Tétanos : \onslide<5->{$\num{4e-6}$ m}
		\item<alert@6> Staphylocoque : \onslide<6->{$\num{1e-6}$ m}
		\item<alert@7> Globule rouge : \onslide<7->{$\num{7.5e-6}$ m}
		\item<alert@8> Grippe : \onslide<8->{$\num{1.2e-7}$ m}
	\end{itemize}
	
	
	\only<9>{On a : $\num{2e-8} < \num{1.2e-7} < \num{1e-6} < \num{3e-6} < \num{4e-6} < \num{7.5e-6} < \num{1e-5}$.}
	
	\only<10>{On a donc, dans l'ordre croissant : Virus de la fièvre jaune, Virus de la grippe, Staphylocoque, Bactérie de la salmonelle, Bacille du tétanos, Globule rouge et Cellule humaine.}
\end{frame}

\begin{frame}
	\frametitle{Activité 2}
	\framesubtitle{3) a.}
	Ordre de grandeur de la dimension des bactéries :
	\begin{itemize}
		\item<alert@2> Cellule humaine : \onslide<2->{$\num{e-5}$ m}
		\item<alert@3> Salmonelle : \onslide<3->{$\num{e-6}$ m}
		\item<alert@4> Fièvre jaune : \onslide<4->{$\num{e-8}$ m}
		\item<alert@5> Tétanos : \onslide<5->{$\num{e-6}$ m}
		\item<alert@6> Staphylocoque : \onslide<6->{$\num{e-6}$ m}
		\item<alert@7> Globule rouge : \onslide<7->{$\num{e-5}$ m}
		\item<alert@8> Grippe : \onslide<8->{$\num{e-7}$ m}
	\end{itemize}
	
	
	\only<10>{Plusieurs bactéries ont des dimensions dans le même ordre de grandeur, donc il n'est pas possible de les classer avec ce critère.}
\end{frame}


\begin{frame}
	\begin{alertblock}{A retenir}
		La \kword{notation scientifique} d'un nombre décimal différent de $0$ est la seule écriture de la forme $a \times 10^n$, où :
		
		\begin{itemize}
			\item $a$ est un nombre décimal avec \kword{un seul chiffre autre que $0$ avant la virgule};
			\item $n$ est un nombre entier relatif.
		\end{itemize}
	\end{alertblock}
	
	\begin{exampleblock}{Exemple : Notation scientifique de \num{1785800}}
		\begin{equation*}
			\num{1778500} = \pause \num{1.78500e5}\pause \;  soit \; \num{1778500} = \pause \num{1.785e5}.\pause
		\end{equation*}		
		
	\end{exampleblock}
	
	\begin{exampleblock}{Exemple : Notation scientifique de \num{0.00682}}
		\begin{equation*}
		\num{0.00682} = \pause \num{0006.82e-3}\pause \;  soit \; \num{0.00682} = \pause \num{6.82e-3}.
		\end{equation*}		
		
	\end{exampleblock}
\end{frame}
\end{document}