\documentclass{beamer}
%\usepackage[utf8]{inputenc}
%\usetheme{Warsaw}
\usetheme{CambridgeUS}
%\usecolortheme{seahorse}

\input{../../../utils_maths_beamer}

\title{Fractions}
\author{}\institute{}


\AtBeginSubsection[]
{
	\begin{frame}
		\frametitle{Sommaire}
		\tableofcontents[currentsection, currentsubsection]
	\end{frame} 
}

\begin{document}
	
	
	
\begin{frame}
	\titlepage
\end{frame}

\section{Quotient de deux nombres entiers}

\subsection{Définition}


\begin{frame}
\frametitle{}  
\framesubtitle{ }	
	
\begin{block}{Vocabulaire}
	bla
\end{block}	

\end{frame}


\subsection{Représentation sur une demi-droite graduée}
\begin{frame}
	\frametitle{}  
	\framesubtitle{ }	
	
	blabla
	
\end{frame}


\section{Utilisation des fractions}

\subsection{Multiplication d'une fraction par un nombre}

\begin{frame}
	\frametitle{}  
	\framesubtitle{}	
	
	\begin{block}{Propriété}
		\bu{Prendre une fraction} d'un nombre revient à multiplier cette fraction par ce nombre.	
	\end{block}
	
	\begin{block}{Méthode}
		On peut effectuer le calcul de plusieurs façons.
		$a, b$ et $k$ désignent trois nombres, avec $b \neq 0$ :
		\begin{itemize}
			\item[\ ] {\Large $\frac{a}{b}$} $\times k$
			\item[ou] {\Large $\frac{a \times k}{b}$}
			\item[ou] $a \times $ {\Large $\frac{k}{b}$}
		\end{itemize}
	\end{block}
			
\end{frame}


\begin{frame}
	\frametitle{}  
	\framesubtitle{\ }
	
	\begin{exampleblock}{Exemple}
	Une boisson de $350 cl$ contient {\LARGE $\frac{3}{5}$} de jus d'orange.
	Pour calculer le volume de jus d'orange contenu dans la boisson, on peut calculer {\LARGE $\frac{3}{5}$} de 350 cl de plusieurs manières :
	
	
		\begin{itemize}
			\item  \myfrac{3}{5} $ \times 350 = $\myfrac{3 \times 350}{5} $= 1050 : 5 = 210 cl $ 
			\item[ou]  \myfrac{3}{5} $ \times 350 = (350:5) \times 3 = 70 \times 3 = 210 cl $ 
			\item[ou] \myfrac{3}{5} $ \times 350 = (3:5) \times 350 = 0,6 \times 350 = 210 cl $ 
			\item[$\Rightarrow$] La boisson contient 210 cl de jus d'orange.
%			\item {\LARGE $ \frac{3}{5}$} $ \times 350 = $ {\LARGE $\frac{3 \times 350}{5}$} $ = 1050 : 5 = 210 cl $ 
%			\item[ou] {\LARGE $ \frac{3}{5}$} $ \times 350 = (350:5) \times 3 = 70 \times 3 = 210 cl $ 
%			\item[ou] {\LARGE $ \frac{3}{5}$} $ \times 350 = (3:5) \times 350 = 0,6 \times 350 = 210 cl $ 
%			\item[$\Rightarrow$] La boisson contient 210 cl de jus d'orange.
		\end{itemize}
	\end{exampleblock}
\end{frame}

\subsection{Quotients égaux}

\begin{frame}
	\frametitle{}  
	\framesubtitle{ \ }	
	
	\begin{block}{Propriété}
		\begin{itemize}
			
		\item Le \underline{quotient} de deux nombres \underline{ne change pas} lorsqu'on \underline{multiplie} (ou on divise) le numérateur \underline{et} le dénominateur par \underline{un même nombre} différent de zéro.\\
		
		\item[$\rightarrow$] $a$ et $b$ désignent deux nombres avec $b \neq 0$ et $k$ désigne un nombre avec $k \neq 0$.\\
		
		On écrit alors : {\LARGE $\frac{a}{b} = \frac{a \: \times  \: \textbf{k}}{b \: \times \: \textbf{k}}$} et {\LARGE $\frac{a}{b} = \frac{a \: : \: \textbf{k}}{b \: : \:\textbf{k}}$}
		\end{itemize}
	\end{block}
	
\end{frame}

\begin{frame}
	\frametitle{}  
	\framesubtitle{Exemples}	
	
	
	\begin{columns}[onlytextwidth]
		\begin{column}{0.465\textwidth}
			\begin{exampleblock}{Exemple 1}
				{\LARGE $\frac{1,5}{2,5} = \frac{1,5 \: \times  \: 2}{2,5 \: \times \: 2} = \frac{3}{5}$} 
				\vspace*{5mm}
				
				On a transformé ce quotient en fraction.\\
				
				On a remplacé les nombres décimaux du numérateur et du dénominateur par des entiers.\pause
			\end{exampleblock}
		\end{column}
		\begin{column}{0.465\textwidth}
			\begin{exampleblock}{Exemple 2}
				{\LARGE $\frac{24}{56} = \frac{24 \: :  \: 8}{56 \: : \: 8} = \frac{3}{7}$} 
				\vspace*{5mm}
				
				On a simplifié la fraction par 8.\\
				
				On a remplacé les entiers du numérateur et du dénominateur par des entiers plus simples.
			\end{exampleblock}
		\end{column}
	\end{columns}
	
	
	
\end{frame}

\subsection{Divisibilité}

\begin{frame}[allowframebreaks]
	\frametitle{} %Critères de divisibilité }  
	\framesubtitle{\ }	
	
	\begin{block}{Nombre divisible par 2}
		Un nombre est divisible par \bu{2} si et seulement si \underline{\textbf{le chiffre de ses unités est 0, 2, 4, 6 ou 8}.}\\
		Exemples : 2, 12, 26, 34, 50, 78 
	\end{block}
	
	\begin{block}{Nombre divisible par 3}
		Un nombre est divisible par \bu{2}2 si et seulement si \textbf{\underline{la somme de ses chiffres est divisible par 3}}.\\
		Exemples : 3, 6, 9, 18, 39, 78 
	\end{block}

	\begin{block}{Nombre divisible par 5}
		Un nombre est divisible par \bu{5} si et seulement si \underline{\textbf{le chiffre de ses unités est 0 ou 5}}.\\
		Exemples : 5, 10, 20, 35, 50, 75 
	\end{block}
	
	\begin{block}{Nombre divisible par 10}
		Un nombre est divisible par \bu{10} si et seulement si \underline{\textbf{le chiffre de ses unités est 0}}.\\
		Exemples : 10, 20, 40, 60, 90, 120 
	\end{block}		
		
		
\end{frame}

\subsection{Simplification de fractions}

\begin{frame}
	\frametitle{} %Simplification de fractions}  
	\framesubtitle{\ }	
	
	\begin{block}{Définition}
		Simplifier une fraction c'est signifie trouver une fraction égale avec un numérateur et un dénominateur plus petits.
	\end{block}	
	
	\begin{block}{Méthode}
		Pour simplifier une fraction, on cherche un \bu{diviseur commun} au numérateur et au dénominateur. Pour cela on utilise les tables de multiplications ou les critères de divisibilité.
	\end{block}
\end{frame}

\end{document}


\begin{frame}
	\frametitle{}  
	\framesubtitle{}	
	
\end{frame}