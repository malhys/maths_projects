\subsection{Proportionnalité}\label{ch_5_proba}

\subsubsection*{Compétences}
\begin{enumerate}
	\item Reconnaître si deux grandeurs sont proportionnelles
	\item Savoir calculer un coefficient de proportionnalité sous forme de quotient
	\item Connaître la règle de passage à l'unité
	\item Maîtriser les pourcentages
	\item Maîtriser la notion d'échelle
\end{enumerate}

\subsection{Statistiques}\label{ch_5_stats}

\subsubsection*{Compétences}
 

\begin{enumerate}
	\item Savoir lire et construire des tableaux
	\item Savoir calculer une fréquence
	\item Savoir répartir les données en classes
	\item Savoir lire et construire 
	\begin{itemize}
		\item des diagrammes circulaires
		\item des digrammes en tuyaux d'orgues
		\item des histogrammes
	\end{itemize}

\end{enumerate}

\subsection{Calcul littéral}\label{ch_5_lit}

\subsubsection*{Pré-requis}
 Opérations (ch \ref{ch_5_op})
 
 \subsubsection*{Compétences}
\begin{enumerate}
	\item Être capable de manipuler des expressions littérales
	\item Pouvoir utiliser le calcul littéral pour démonter quelque chose
	\item Savoir distribuer (distributivité simple) et factoriser
\end{enumerate}

\subsection{Calcul numérique}\label{ch_5_op}

\subsubsection*{Compétences}
\begin{enumerate}
	\item Diviser par un nombre décimal (si non vu en 6$^e$)
	\item Connaître les priorités des différents opérateurs
	\item Savoir enchaîner plusieurs opérations
%	\item Distributivité sur des exemples numériques et littéraux
	
\end{enumerate}

\subsection{Fractions}\label{ch_5_frac}

\subsubsection*{Pré-requis}

Calcul numérique (ch \ref{ch_5_op})

\subsubsection*{Compétences}

\begin{enumerate}

	\item Connaître les différentes significations d'une écriture du type  {\Large $\frac{a}{b}$} (fréquence, proportion)
	\item Savoir simplifier une fraction
	\item Savoir comparer des fractions
	\item Savoir multiplier des fractions
	\item Savoir additionner et soustraire des fractions
		\begin{itemize}
			\item Avec un même dénominateur
			\item Avec des dénominateurs multiples l'un de l'autre
		\end{itemize}

\end{enumerate}

\subsection{Nombres relatifs}\label{ch_5_rels}

\subsubsection*{Compétences}
\begin{enumerate}
	\item Connaître les nombres relatifs
	\item Savoir ce qui caractérise deux nombres opposés
	\item Savoir ordonner des nombres relatifs
	\item Pouvoir placer des nombres relatifs sur une droite graduée
	\item Savoir utiliser des nombres relatifs pour se repérer dans un plan
\end{enumerate}

\subsection{Addition et soustraction de nombres relatifs}\label{ch_5_add_rels}

\subsubsection*{Pré-requis}
 Nombres relatifs (ch \ref{ch_5_rels})
 
 \subsubsection*{Compétences}
\begin{enumerate}
	\item Savoir additionner et soustraire des nombres relatifs
	\item Pouvoir calculer la distance entre deux point d'une droite graduée 
	\item Savoir calculer une expression algébrique et la simplifier
\end{enumerate}

%\subsection{\'Equations}\label{ch_5_eq}
%\begin{enumerate}
%	\item Introduction
%\end{enumerate}
\subsection{Angles}\label{ch_5_angles}

\subsubsection*{Compétences}

\begin{enumerate}
	\item Savoir reconnaître des couples d'angles particuliers
	\begin{itemize}
		\item Adjacents
		\item Complémentaires
		\item Supplémentaires
		\item Opposés par le sommet
		\item Correspondants
		\item Alternes-internes
	\end{itemize}
	\item Être capable de raisonner avec ces angles
	\item Connaître les propriétés des mesures des angles d'un triangle
\end{enumerate}

\subsection{Parallélogramme}\label{ch_5_para}

\subsubsection*{Pré-requis}
Angles (ch \ref{ch_5_angles})

\subsubsection*{Compétences}
\begin{enumerate}
	\item Connaître la définition d'un parallélogramme
	\item Connaître les propriétés d'un parallélogramme
	\item Savoir reconnaître qu'un quadrilatère est un parallélogramme et le démonter
\end{enumerate}

\subsection{Parallélogrammes particuliers}\label{ch_5_para2}

\subsubsection*{Pré-requis}
Parallélogrammes (ch \ref{ch_5_para})
\begin{enumerate}
	\item Connaître les propriétés des parallélogrammes particuliers
	\item Pouvoir reconnaître et démonter qu'un quadrilatère est un 
		\begin{itemize}
			\item Rectangle
			\item Losange
			\item Carré
		\end{itemize}
\end{enumerate}

\subsection{Triangle}\label{ch_5_tri}

\subsubsection*{Pré-requis}

Angles (ch \ref{ch_5_angles})
\subsubsection*{Compétences}
	\begin{enumerate}
		\item Savoir construire un triangle
		\item Connaître l'inégalité triangulaire
		\item Connaître et savoir tracer les droites remarquables d'un triangle 
		\item Savoir ce qu'est le cercle circonscrit à un triangle
		\item Savoir le tracer
	\end{enumerate}
	
\subsection{Symétrie centrale}\label{ch_5_sym}	

\subsubsection*{Compétences}
\begin{enumerate}
	\item Être capable de tracer le symétrique
	\begin{itemize}
		\item d'un point
		\item d'un segment
		\item d'une droite
		\item d'une figure
		\item d'un cercle		
	\end{itemize}
	\item Pouvoir reconnaître qu'une figure possède un centre de symétrie
\end{enumerate}

\subsection{Prismes et cylindres de révolution}\label{ch_5_prismes}

\subsubsection*{Pré-requis}
	Triangles (ch \ref{ch_5_tri})
	
\subsubsection*{Compétences}	
\begin{enumerate}
	\item Savoir représenter un prisme droit et un cylindre de révolution
	\item Être capable de les construire
	\item Pouvoir calculer l'aire latérale, l'aire totale et le volume d'un prisme droit et d'un cylindre de révolution
\end{enumerate}