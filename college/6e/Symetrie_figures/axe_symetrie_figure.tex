\documentclass{beamer}
%\usepackage[utf8]{inputenc}
\usetheme{Warsaw}

\input{../../../utils_maths_beamer}

\title{Axes de symétrie d'une figure}
\author{Olivier FINOT}\institute{Collège Jules Ferry, Beaune}

\begin{document}



\begin{frame}
  \titlepage
\end{frame}


\section{Rappels}

\begin{frame}
\frametitle{Symétrique d'un point M par rapport à une droite (d)}  
\framesubtitle{Définition}

\begin{alertblock}<2->{Si M n'appartient pas à la droite (d)}
Dire que le point M' est le symétrique du point M par rapport à la droite (d) signifie que la droite (d) est la médiatrice du segment [MM'].
\end{alertblock}

\begin{alertblock}<3->{Si M appartient à la droite (d)}
Le symétrique M4 du point M par rapport à la droite (d) est lui-même.
C'est à dire, les points M et M' sont confondus.
\end{alertblock}

\end{frame}

\section{Axes(s) de symétrie d'une figure}

\subsection{Définition}

\begin{frame}
\frametitle{T}  
\framesubtitle{ST}
blabla
\end{frame}

\subsection{Axes de symétrie d'un segment}

\begin{frame}
\frametitle{T}  
\framesubtitle{ST}
blabla
\end{frame}

\subsection{Axe de symétrie d'un angle}

\begin{frame}
\frametitle{T}  
\framesubtitle{ST}
blabla
\end{frame}

\section{Axes de symétrie de triangles}


\subsection{Le Triangle Isocèle}

\begin{frame}
\frametitle{T}  
\framesubtitle{ST}
blabla
\end{frame}

\subsection{Le Triangle Équilatéral}

\begin{frame}
\frametitle{T}  
\framesubtitle{ST}
blabla
\end{frame}

\section{Axes de symétrie de quadrilatères usuels}

\subsection{Le Losange}

\begin{frame}
\frametitle{T}  
\framesubtitle{ST}
blabla
\end{frame}

\subsection{Le Rectangle}

\begin{frame}
\frametitle{T}  
\framesubtitle{ST}
blabla
\end{frame}

\subsection{Le Carré}

\begin{frame}
\frametitle{T}  
\framesubtitle{ST}
blabla
\end{frame}


\end{document}

\begin{frame}[label=lbl:]
\frametitle{}  
\framesubtitle{}

\begin{block}{title}
contenu...
\end{block}

\begin{alertblock}{block title}
contenu...
\end{alertblock}


\begin{exampleblock}{block title}
contenu...
\end{exampleblock}
\end{frame}
