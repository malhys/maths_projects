\subsection{Vocabulaire}

\begin{mydef}
	\iftoggle{eleve}{%
		Un \hrulefill
		
		\vspace*{0.2cm}
		\hrulefill
	}{
		Un \kw{produit} est le résultat de la \kw{multiplication} de deux \kw{facteurs}.
	}
		
\end{mydef}
	
\begin{myex}
	\begin{center}
		\iftoggle{eleve}{%
			\includegraphics*[scale=0.7]{img/produit2}
		}{%
	
			\includegraphics*[scale=0.7]{img/produit}
		}
	\end{center}
\end{myex}

\begin{myprop}
	\iftoggle{eleve}{%
		Dans une multiplication, \hrulefill
		
		\vspace*{0.2cm}
		\hrulefill
	}{%
	Dans une multiplication, \kw{l'ordre des facteurs n'a pas d'importance}.
	}
\end{myprop}

\begin{myex}
	\begin{itemize}
		\iftoggle{eleve}{%
			\item $4 \times 2 \times 5 = $ \hrulefill
			
			\vspace*{0.2cm}
			\item $\num{3.5} \times \num{2.5} \times 4 \times \num{2} = $ \hrulefill
		}{%
			\item $4 \times 2 \times 5 = 2 \times 5 \times 4 = 10 \times 4 = 40$
			\item $\num{3.5} \times \num{2.5} \times 4 \times \num{2} = \num{3.5}	 \times \num{2} \times 4 \times \num{2.5} = 7 \times 10 = 70$
		}
	\end{itemize}
\end{myex}

\subsection{Multiplier par 10, 100 ou 1000}

\begin{mymeth}
	
	Pour multiplier un nombre par 10, 100 ou 1000 :
	
	\iftoggle{eleve}{%
		\begin{enumerate}[label=\arabic*)]
			\item on repère \hrulefill 
			\item on la décale vers \hrulefill 
			
			\vspace*{0.2cm}
			
			\hrulefill
			\item on rajoute \hrulefill 
			
			\vspace*{0.2cm}
			
			\hrulefill
		\end{enumerate}
	}{%

		\begin{enumerate}[label=\arabic*)]
			\item on repère la virgule;
			\item on la décale vers la droite d'un rang ($\times 10$) , de deux rangs ($\times 100$) ou de trois ($\times 1000$);
			\item on rajoute des zéros si besoin entre le chiffre le plus à droite et la virgule.
		\end{enumerate}
	}
\end{mymeth}

\begin{myexs}
	\begin{multicols}{2}
		\begin{itemize}
			\iftoggle{eleve}{%
				\item $\num{25.26} \times 10 = $ 
				\item $\num{25.26} \times 100 = $
				%\item $\num{245.26} \times 1000 = \num{245260}$
				\item $\num{285} \times 10 = $ 
				\item $\num{285} \times 1000 = $ 
			}{%
		
				\item $\num{25.26} \times 10 = \num{252.6}$ 
				\item $\num{25.26} \times 100 = \num{2526.0} = \num{2526}$
				%\item $\num{245.26} \times 1000 = \num{245260}$
				\item $\num{285} \times 10 = \num{285.0} \times \num{10}= \num{2850}$ 
				\item $\num{285} \times 1000 = \num{285000}$ 
			}
		\end{itemize}	
	\end{multicols}
	
\end{myexs}

\subsection{Multiplier par \num{0.1}, \num{0.01} ou \num{0.001}}

\begin{mymeth}
	Pour multiplier un nombre par \num{0.1}, \num{0.01} ou \num{0.001} :
		\iftoggle{eleve}{%
		\begin{enumerate}[label=\arabic*)]
			\item on repère \hrulefill 
			\item on la décale vers \hrulefill 
			
			\vspace*{0.2cm}
			
			\hrulefill
			\item on rajoute \hrulefill 
			
			\vspace*{0.2cm}
			
			\hrulefill
		\end{enumerate}
	}{%
		\begin{enumerate}[label=\arabic*)]
			\item on repère la virgule;
			\item on la décale vers la gauche d'un rang ($\times \num{0.1}$) , de deux rangs ($\times \num{0.01}$) ou de trois ($\times \num{0.001}$);
			\item on rajoute des zéros si besoin entre le chiffre le plus à gauche et la virgule et avant la virgule.
		\end{enumerate}
	}
\end{mymeth}

\begin{myexs}
	\begin{multicols}{2}
		\begin{itemize}
			\iftoggle{eleve}{%
				\item $\num{25.26} \times \num{0.1} = $ 
				\item $\num{25.26} \times \num{0.01} = $
				%\item $\num{245.26} \times 1000 = \num{245260}$
				\item $\num{285} \times \num{0.01} = $ 
				\item $\num{28.5} \times \num{0.001} = $
			}{%
		
				\item $\num{25.26} \times \num{0.1} = \num{2.526}$ 
				\item $\num{25.26} \times \num{0.01} = \num{0.2526}$
				%\item $\num{245.26} \times 1000 = \num{245260}$
				\item $\num{285} \times \num{0.01} = \num{0.285} \times \num{0.01}= \num{0.0285}$ 
				\item $\num{28.5} \times \num{0.001} = \num{0.0285}$ 
			}
		\end{itemize}	
	\end{multicols}
	
\end{myexs}