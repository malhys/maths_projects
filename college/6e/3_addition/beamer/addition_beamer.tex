\documentclass[xcolor={dvipsnames}]{beamer}
%\usepackage[utf8]{inputenc}
\usetheme{Madrid}
%\usetheme{Malmoe}
\usecolortheme{beaver}
%\usecolortheme{rose}

\input{../../../../utils_maths_beamer}


\usepackage{../../../../pas-math}
\usepackage{../../../../moncours_beamer}

\usepackage{amssymb,amsmath}


\newcommand{\myitem}{\item[\textbullet]}

\graphicspath{{../img/}}

\title[3 : Addition, soustraction, multiplication]{Séquence 3 : Addition, soustraction, multiplication}
%\author{O. FINOT}\institute{Collège S$^t$ Bernard}

%
\AtBeginSection[]
{
	\begin{frame}
		\frametitle{}
		\tableofcontents[currentsection, hideallsubsections]
	\end{frame} 

}
%
%
%\AtBeginSubsection[]
%{
%	\begin{frame}
%		\frametitle{Sommaire}
%		\tableofcontents[currentsection, currentsubsection]
%	\end{frame} 
%}

\begin{document}



\begin{frame}
  \titlepage 
\end{frame}


	

\begin{frame}
	\begin{block}{Objectifs}
		\begin{itemize}
			\item Savoir additionner, soustraire et multiplier des nombres;
			\item Connaitre les propriétés de l’addition, la soustraction et la multiplication;
			\item Calculer astucieusement;
			\item Vérifier si un résultat semble correct avec un ordre de grandeur.
		\end{itemize}
	\end{block}
\end{frame}

\begin{frame}
	\begin{block}{Compétences travaillées}
		\begin{itemize}
			\item \kw{Calculer (Ca1)} : Calculer avec des nombres décimaux et des fractions simples de manière exacte ou approchée, en utilisant des stratégies ou des techniques appropriées (mentalement, en ligne, ou en posant les opérations); 
			\item \kw{Calculer (Ca2)} : Contrôler la vraisemblance de ses résultats ; 
			%\item \kw{Calculer (Ca3)} :	Utiliser une calculatrice pour trouver ou vérifier un résultat		
		\end{itemize}
	\end{block}
\end{frame}



\section{Additionner et soustraire}




\begin{frame}
	\begin{mydef}
		
		Le résultat d'une addition est une \kword{somme},\pause les nombres utilisés sont des \kword{termes}.\pause
	\end{mydef}
	
	\begin{myex}
		\begin{center}
			\includegraphics<3>[scale=0.9]{somme2}
			\includegraphics<4>[scale=0.9]{somme}
		\end{center}
	\end{myex}
	
\end{frame}

\begin{frame}
	\begin{mydef}
		Une \kword{différence} est le résultat de la soustraction de deux \kword{termes}.\pause
	\end{mydef}
	
	\begin{myex}
		\begin{center}
			\includegraphics<2>[scale=0.9]{difference2}
			\includegraphics<3>[scale=0.9]{difference}
		\end{center}
	\end{myex}
	
\end{frame}

\begin{frame}
		\begin{myprop}
		Dans une addition, \kword{l'ordre des termes n'a pas d'importance}.\pause
	\end{myprop}
	
	\begin{myex}
		\begin{itemize}
			\item $12 + 5 + 8 = $ \pause $ 12 + 8 + 5  = $ \pause $ 25$\pause
			\item $\num{3.5} + 5 + \num{6.5} + 2 = $ \pause $ \num{3.5} + \num{6.5} + 5 + 2 = $ \pause $ 17$
		\end{itemize}
	\end{myex}
\end{frame}

\begin{frame}
	\begin{mymeth}
		Pour avoir rapidement une idée du résultat attendu d'une addition ou d'une soustraction, on peut utiliser un \kword{ordre de grandeur}.\pause		
	\end{mymeth}
	
	\begin{myex}
		Je veux calculer la somme \num{48.7} + \num{97.584} : \pause
		\begin{enumerate}
			\item \num{48.7} est proche de \pause 50 \pause et \num{97.584} de \pause 100 \pause
			\item 50 + 100 =  \pause 150 \pause
			\item Donc cette somme est de l'ordre de 150 (ou voisine de 150).
		\end{enumerate}
	\end{myex}
\end{frame}
\end{document}