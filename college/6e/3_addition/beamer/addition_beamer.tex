\documentclass[xcolor={dvipsnames}]{beamer}
%\usepackage[utf8]{inputenc}
\usetheme{Madrid}
%\usetheme{Malmoe}
\usecolortheme{beaver}
%\usecolortheme{rose}

%-------------------------------------------------------------------------------
%          -Packages nécessaires pour écrire en Français et en UTF8-
%-------------------------------------------------------------------------------
\usepackage[utf8]{inputenc}
\usepackage[frenchb]{babel}
\usepackage[T1]{fontenc}
\usepackage{lmodern}
\usepackage{textcomp}

%-------------------------------------------------------------------------------

%-------------------------------------------------------------------------------
%                          -Outils de mise en forme-
%-------------------------------------------------------------------------------
\usepackage{hyperref}
\hypersetup{pdfstartview=XYZ}
\usepackage{enumerate}
\usepackage{graphicx}
%\usepackage{multicol}
%\usepackage{tabularx}

%\usepackage{anysize} %%pour pouvoir mettre les marges qu'on veut
%\marginsize{2.5cm}{2.5cm}{2.5cm}{2.5cm}

\usepackage{indentfirst} %%pour que les premier paragraphes soient aussi indentés
\usepackage{verbatim}
%\usepackage[table]{xcolor}  
%\usepackage{multirow}
\usepackage{ulem}
%-------------------------------------------------------------------------------


%-------------------------------------------------------------------------------
%                  -Nécessaires pour écrire des mathématiques-
%-------------------------------------------------------------------------------
\usepackage{amsfonts}
\usepackage{amssymb}
\usepackage{amsmath}
\usepackage{amsthm}
\usepackage{tikz}
\usepackage{xlop}
\usepackage[output-decimal-marker={,}]{siunitx}
%-------------------------------------------------------------------------------


%-------------------------------------------------------------------------------
%                    - Mise en forme 
%-------------------------------------------------------------------------------

\newcommand{\bu}[1]{\underline{\textbf{#1}}}


\usepackage{ifthen}


\newcommand{\ifTrue}[2]{\ifthenelse{\equal{#1}{true}}{#2}{$\qquad \qquad$}}

\newcommand{\kword}[1]{\textcolor{red}{\underline{#1}}}


%-------------------------------------------------------------------------------



%-------------------------------------------------------------------------------
%                    - Racourcis d'écriture -
%-------------------------------------------------------------------------------

% Angles orientés (couples de vecteurs)
\newcommand{\aopp}[2]{(\vec{#1}, \vec{#2})} %Les deuc vecteurs sont positifs
\newcommand{\aopn}[2]{(\vec{#1}, -\vec{#2})} %Le second vecteur est négatif
\newcommand{\aonp}[2]{(-\vec{#1}, \vec{#2})} %Le premier vecteur est négatif
\newcommand{\aonn}[2]{(-\vec{#1}, -\vec{#2})} %Les deux vecteurs sont négatifs

%Ensembles mathématiques
\newcommand{\naturels}{\mathbb{N}} %Nombres naturels
\newcommand{\relatifs}{\mathbb{Z}} %Nombres relatifs
\newcommand{\rationnels}{\mathbb{Q}} %Nombres rationnels
\newcommand{\reels}{\mathbb{R}} %Nombres réels
\newcommand{\complexes}{\mathbb{C}} %Nombres complexes


%Intégration des parenthèses aux cosinus
\newcommand{\cosP}[1]{\cos\left(#1\right)}
\newcommand{\sinP}[1]{\sin\left(#1\right)}

%Fractions
\newcommand{\myfrac}[2]{{\LARGE $\frac{#1}{#2}$}}

%Vocabulaire courrant
\newcommand{\cad}{c'est-à-dire}

%Droites
\newcommand{\dte}[1]{droite $(#1)$}
\newcommand{\fig}[1]{figure $#1$}
\newcommand{\sym}{symétrique}
\newcommand{\syms}{symétriques}
\newcommand{\asym}{axe de symétrie}
\newcommand{\asyms}{axes de symétrie}
\newcommand{\seg}[1]{$[#1]$}
\newcommand{\monAngle}[1]{$\widehat{#1}$}
\newcommand{\bissec}{bissectrice}
\newcommand{\mediat}{médiatrice}
\newcommand{\ddte}[1]{$[#1)$}

%Figures
\newcommand{\para}{parallélogramme}
\newcommand{\paras}{parallélogrammes}
\newcommand{\myquad}{quadrilatère}
\newcommand{\myquads}{quadrilatères}
\newcommand{\co}{côtés opposés}
\newcommand{\diag}{diagonale}
\newcommand{\diags}{diagonales}
\newcommand{\supp}{supplémentaires}
\newcommand{\car}{carré}
\newcommand{\cars}{carrés}
\newcommand{\rect}{rectangle}
\newcommand{\rects}{rectangles}
\newcommand{\los}{losange}
\newcommand{\loss}{losanges}


%----------------------------------------------------


\usepackage{../../../../pas-math}
\usepackage{../../../../moncours_beamer}

\usepackage{amssymb,amsmath}


\newcommand{\myitem}{\item[\textbullet]}

\graphicspath{{../img/}}

\title[3 : Addition, soustraction, multiplication]{Séquence 3 : Addition, soustraction, multiplication}
%\author{O. FINOT}\institute{Collège S$^t$ Bernard}

%
\AtBeginSection[]
{
	\begin{frame}
		\frametitle{}
		\tableofcontents[currentsection, hideallsubsections]
	\end{frame} 

}
%
%
%\AtBeginSubsection[]
%{
%	\begin{frame}
%		\frametitle{Sommaire}
%		\tableofcontents[currentsection, currentsubsection]
%	\end{frame} 
%}

\begin{document}



\begin{frame}
  \titlepage 
\end{frame}


	

\begin{frame}
	\begin{block}{Objectifs}
		\begin{itemize}
			\item Savoir additionner, soustraire et multiplier des nombres;
			\item Connaitre les propriétés de l’addition, la soustraction et la multiplication;
			\item Calculer astucieusement;
			\item Vérifier si un résultat semble correct avec un ordre de grandeur.
			\item Utiliser les unités de longueur et de masse;
			\item Savoir résoudre des problèmes.
		\end{itemize}
	\end{block}
\end{frame}

\begin{frame}
	\begin{block}{Compétences travaillées}
		
		\begin{itemize}
			
			\item \kw{Calculer}
			\item \kw{Modéliser}
			\item \kw{Raisonner}
			
%			\item \kw{Calculer (Ca1)} : Calculer avec des nombres décimaux et des fractions simples de manière exacte ou approchée, en utilisant des stratégies ou des techniques appropriées (mentalement, en ligne, ou en posant les opérations); 
%			\item \kw{Calculer (Ca2)} : Contrôler la vraisemblance de ses résultats ; 
%			\item \kw{Calculer (Ca3)} :	Utiliser une calculatrice pour trouver ou vérifier un résultat		
		\end{itemize}
	\end{block}
\end{frame}



\section{Additionner et soustraire}




\begin{frame}
	\begin{mydef}
		
		Le résultat d'une addition est une \kword{somme},\pause les nombres utilisés sont des \kword{termes}.\pause
	\end{mydef}
	
	\begin{myex}
		\begin{center}
			\includegraphics<3>[scale=0.9]{somme2}
			\includegraphics<4>[scale=0.9]{somme}
		\end{center}
	\end{myex}
	
\end{frame}

\begin{frame}
	\begin{mydef}
		Une \kword{différence} est le résultat de la soustraction de deux \kword{termes}.\pause
	\end{mydef}
	
	\begin{myex}
		\begin{center}
			\includegraphics<2>[scale=0.9]{difference2}
			\includegraphics<3>[scale=0.9]{difference}
		\end{center}
	\end{myex}
	
\end{frame}

\begin{frame}
		\begin{myprop}
		Dans une addition, \kword{l'ordre des termes n'a pas d'importance}.\pause
	\end{myprop}
	
	\begin{myex}
		\begin{itemize}
			\item $12 + 5 + 8 = $ \pause $ 12 + 8 + 5  = $ \pause $ 25$\pause
			\item $\num{3.5} + 5 + \num{6.5} + 2 = $ \pause $ \num{3.5} + \num{6.5} + 5 + 2 = $ \pause $ 17$
		\end{itemize}
	\end{myex}
\end{frame}

\begin{frame}
	\begin{mymeth}
		Pour avoir rapidement une idée du résultat attendu d'une addition ou d'une soustraction, on peut utiliser un \kword{ordre de grandeur}.\pause		
	\end{mymeth}
	
	\begin{myex}
		Je veux calculer la somme \num{48.7} + \num{97.584} : \pause
		\begin{enumerate}
			\item \num{48.7} est proche de \pause 50 \pause et \num{97.584} de \pause 100 \pause
			\item 50 + 100 =  \pause 150 \pause
			\item Donc cette somme est de l'ordre de 150 (ou voisine de 150).
		\end{enumerate}
	\end{myex}
\end{frame}


\section{Multiplier}


\begin{frame}
	\begin{mydef}
		Un \kword{produit} est le résultat de la \kword{multiplication} de deux \kword{facteurs}.\pause
		
	\end{mydef}
	
	\begin{myex}
		\begin{center}
			\includegraphics<2>[scale=0.7]{produit2}
			\includegraphics<3>[scale=0.7]{produit}
		\end{center}
	\end{myex}
\end{frame}

\begin{frame}
	\begin{myprop}
		Dans une multiplication, \kword{l'ordre des facteurs n'a pas d'importance}.
	\end{myprop}
	
	\begin{myexs}
		\begin{itemize}
			\item $4 \times 2 \times 5 = \pause 2 \times 5 \times 4 = \pause 10 \times 4 \pause = 40$\pause
			\item $\num{3.5} \times \num{2.5} \times 4 \times \num{2} = \pause \num{3.5} \times \num{2} \times 4 \times \num{2.5} = \pause 7 \times 10 = 70$
		\end{itemize}
	\end{myexs}
\end{frame}


\section{Priorité des opérations}


\begin{frame}
	\begin{myprops}
		\begin{itemize}
			\item Dans une expression sans parenthèses, la multiplication est \kword{prioritaire} sur l'addition et la soustraction.\pause
			
			\item Dans une expression avec des parenthèses, on effectue d'abord les calculs \kword{entre parenthèses}.\pause
		\end{itemize}
	\end{myprops}



	\begin{myexs}
		Je calcule les expressions suivantes
		\vspace*{-0.5cm}
		
		\begin{eqnarray*}
			A &=& 2 + 3 \times 4 \\ \pause
			A &=& 2 + 12 \\ \pause
			A &=& 14 \pause
		\end{eqnarray*}
		
			\vspace*{-1cm}
		
		\begin{eqnarray*}
			B &=& (2 + 3) \times 4 \\ \pause
			B &=& 5 \times 4 \\ \pause
			B &=& 20
		\end{eqnarray*}
		
	\end{myexs}
	
\end{frame}
\end{document}