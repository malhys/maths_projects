	\begin{mydef}
	
		Le résultat d'une addition est une \hspace*{3cm}, les nombres utilisés sont des \hspace*{3cm}.
	\end{mydef}

	\begin{myex}
		\begin{center}
			\includegraphics*[scale=0.7]{img/somme2}
		\end{center}
	\end{myex}
		
	\begin{mydef}
			Une \hspace*{4cm} est le résultat de la soustraction de deux \hspace*{3cm}.
		
	\end{mydef}

	\begin{myex}
		\begin{center}
			\includegraphics*[scale=0.7]{img/difference2}
		\end{center}
	\end{myex}


	\begin{myprop}
		Dans une addition, \hspace*{6cm} n'a pas d'importance.
	\end{myprop}

	\begin{myex}
		\begin{itemize}
			\item $12 + 5 + 8 = $ %12 + 8 + 5 = 25$
			\item $\num{3.5} + 5 + \num{6.5} + 2 = $ %\num{3.5} + \num{6.5} + 5 + 2 = 17$
		\end{itemize}
	\end{myex}

	\begin{mymeth}
		Pour avoir rapidement une idée du résultat attendu d'une addition ou d'une soustraction, on peut utiliser un \hspace*{5cm}.		
	\end{mymeth}

	\begin{myex}
		Je veux calculer la somme \num{48.7} + \num{97.584} : 
		\begin{enumerate}
			\item \num{48.7} est proche de \hspace*{2cm} et \num{97.584} de %100
			\item %50 + 100 = 150
			\item Donc cette somme est de l'ordre de \hspace*{2cm} (ou voisine de \hspace*{1.5cm}).
		\end{enumerate}
	\end{myex}