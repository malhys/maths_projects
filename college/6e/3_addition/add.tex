\iftoggle{eleve}{%
	\begin{mydef}
		
		Le résultat \hrulefill 
		
		\vspace*{0.2cm}		
		\hrulefill 
	\end{mydef}
	
	\begin{myex}
		\begin{center}
			\includegraphics*[scale=0.7]{img/somme2}
		\end{center}
	\end{myex}
	
	\begin{mydef}
		Une \hrulefill 
		
		\vspace*{0.2cm}
		\hrulefill
		
	\end{mydef}
	
	\begin{myex}
		\begin{center}
			\includegraphics*[scale=0.7]{img/difference2}
		\end{center}
	\end{myex}
	
	
	\begin{myprop}
		Dans une \hrulefill
		
		\vspace*{0.2cm}
		\hrulefill
	\end{myprop}
	
	\begin{myex}
		\begin{itemize}
			\item $12 + 5 + 8 = $ \hrulefill
			\item $\num{3.5} + 5 + \num{6.5} + 2 = $ \hrulefill % \num{3.5} + \num{6.5} + 5 + 2 = 17$
		\end{itemize}
	\end{myex}
	
	\begin{mymeth}
		Pour avoir rapidement \hrulefill
		
		\vspace*{0.2cm}
		\hrulefill
		
		\vspace*{0.2cm}
		\hrulefill
	\end{mymeth}
	
	\begin{myex}
		Je veux calculer la somme \num{48.7} + \num{97.584} : 
		\begin{enumerate}
			\item \hrulefill 
			\item \hrulefill %50 + 100 = 150
			\item \hrulefill %Donc cette somme est de l'ordre de 150 (ou voisine de 150).
		\end{enumerate}
	\end{myex}
}{%
	\begin{mydef}
	
		Le résultat d'une addition est une \kw{somme}, les nombres utilisés sont des \kw{termes}.
	\end{mydef}

	\begin{myex}
		\begin{center}
			\includegraphics*[scale=0.7]{img/somme}
		\end{center}
	\end{myex}
		
	\begin{mydef}
			Une \kw{différence} est le résultat de la soustraction de deux \kw{termes}.
		
	\end{mydef}

	\begin{myex}
		\begin{center}
			\includegraphics*[scale=0.7]{img/difference}
		\end{center}
	\end{myex}


	\begin{myprop}
		Dans une addition, \kw{l'ordre des termes n'a pas d'importance}.
	\end{myprop}

	\begin{myex}
		\begin{itemize}
			\item $12 + 5 + 8 = 12 + 8 + 5 = 25$
			\item $\num{3.5} + 5 + \num{6.5} + 2 = \num{3.5} + \num{6.5} + 5 + 2 = 17$
		\end{itemize}
	\end{myex}

	\begin{mymeth}
		Pour avoir rapidement une idée du résultat attendu d'une addition ou d'une soustraction, on peut utiliser un \kw{ordre de grandeur}.		
	\end{mymeth}

	\begin{myex}
		Je veux calculer la somme \num{48.7} + \num{97.584} : 
		\begin{enumerate}
			\item \num{48.7} est proche de 50 et \num{97.584} de 100
			\item 50 + 100 = 150
			\item Donc cette somme est de l'ordre de 150 (ou voisine de 150).
		\end{enumerate}
	\end{myex}

}