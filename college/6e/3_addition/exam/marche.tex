\section{Au supermarché}

Joshua achète \num{3.4} kg de poires à \num{1,70} € le kilogramme, un fromage à \num{2.35} € l'unité et un rôti.
Il paie avec 3 billets de 10 €. La caissière lui rend \num{5.25} €.

\begin{questions}
	\question Quel calcul sur une ligne permet de connaître le prix du rôti ?
	\begin{solution}
		Calcul du prix des poires :
		
		\opmul[style=text]{3.4}{1.70}
		
		Il paye \num{5.78} € pour les poires.\\
		
		Calcul du prix payé :
		
		\opmul[style=text]{30}{3}
		
		Il paye 30 €.\\
		
		Calcul du prix des courses :
		\opsub[style=text]{30}{5.25}
		
		Le prix des courses s'élève à \num{24.75} €.\\
		
		L'expression permettant de calculer le prix du rôti est
		
		\begin{equation}
			(10 \times 3 - \num{5.25}) - (\num{3.4} \times \num{1.70} + \num{2.35})
		\end{equation}
		
	\end{solution}
	\question Quel est le prix du rôti ?
	\begin{solution}
		Calcul du prix du rôti :
		
		
		\begin{eqnarray*}
			A &=& (10 \times 3 - \num{5.25}) - (\num{3.4} \times \num{1.70} + \num{2.35}) \\
			A &=& (30 - \num{5.25}) - (\num{5.78} + \num{2.35}) \\
			A &=& \num{24.75} - \num{8.13} \\
			A &=& \num{16.62}
		\end{eqnarray*}
	
		Le rôti coûte \num{16.62} €.
	\end{solution}
\end{questions}