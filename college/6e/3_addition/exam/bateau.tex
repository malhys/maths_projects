\section{Porte conteneurs (3 points)}

Un bateau transporte plusieurs conteneurs identiques, ils ont tous la même masse.
Lorsqu'il quitte le port de Hanovre, sa masse totale (bateau + conteneurs) est de \num{61750} tonnes.
Au port du Havre, il débarque la moitié de ses conteneurs. Après cela, le bateau ne pèse plus que \num{50875} tonnes.

\begin{questions}
	\question[1] Quelle est la masse des conteneurs débarqués au Havre ?
	\begin{solution}
		\begin{equation*}
			\num{61750} - \num{50875} = \num{10875}
		\end{equation*}
		
		La masse des des conteneurs débarqués est \num{10875} tonnes.
	\end{solution}

	\question[1] Quelle est la masse du bateau à vide ?
	
	\begin{solution}
		Les conteneurs débarqués au havre représente la moitié de son chargement, il reste donc une masse équivalente.
		
		\begin{equation*}
			\num{50875} - \num{10875} = \num{40000}
		\end{equation*}
		
		Donc la masse du bateau à vide est \num{40000} tonnes.
	\end{solution}

	\question[1] Convertir  \num{10875} tonnes en kilogrammes ?
\end{questions}