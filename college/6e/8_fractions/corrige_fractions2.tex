\documentclass[12pt,a4paper]{article}


\usepackage[in, plain]{fullpage}
\usepackage{array}
\usepackage{../../../pas-math}

%-------------------------------------------------------------------------------
%          -Packages nécessaires pour écrire en Français et en UTF8-
%-------------------------------------------------------------------------------
\usepackage[utf8]{inputenc}
\usepackage[frenchb]{babel}
\usepackage[T1]{fontenc}
\usepackage{lmodern}
%-------------------------------------------------------------------------------

%-------------------------------------------------------------------------------
%                          -Outils de mise en forme-
%-------------------------------------------------------------------------------
\usepackage{hyperref}
\hypersetup{pdfstartview=XYZ}
\usepackage{enumerate}
\usepackage{graphicx}
\usepackage{multicol}

\usepackage{anysize} %%pour pouvoir mettre les marges qu'on veut
%\marginsize{2.5cm}{2.5cm}{2.5cm}{2.5cm}

\usepackage{indentfirst} %%pour que les premier paragraphes soient aussi indentés
%-------------------------------------------------------------------------------


%-------------------------------------------------------------------------------
%                  -Nécessaires pour écrire des mathématiques-
%-------------------------------------------------------------------------------
\usepackage{amsfonts}
\usepackage{amssymb}
\usepackage{amsmath}
\usepackage{amsthm}
\usepackage{tikz}
%-------------------------------------------------------------------------------

%-------------------------------------------------------------------------------
%                     -Mise en forme d'exercices-
%-------------------------------------------------------------------------------
\newtheoremstyle{exostyle}
{\topsep}% espace avant
{\topsep}% espace apres
{}% Police utilisee par le style de thm
{}% Indentation (vide = aucune, \parindent = indentation paragraphe)
{\bfseries}% Police du titre de thm
{.}% Signe de ponctuation apres le titre du thm
{ }% Espace apres le titre du thm (\newline = linebreak)
{\thmname{#1}\thmnumber{ #2}\thmnote{. \normalfont{\textit{#3}}}}% composants du titre du thm : \thmname = nom du thm, \thmnumber = numéro du thm, \thmnote = sous-titre du thm

\theoremstyle{exostyle}
\newtheorem{exercice}{Exercice}

\newenvironment{questions}{
\begin{enumerate}[\hspace{12pt}\bfseries\itshape a.]}{\end{enumerate}
} %mettre un 1 à la place du a si on veut des numéros au lieu de lettres pour les questions 
%-------------------------------------------------------------------------------



%-------------------------------------------------------------------------------
%                    - Racourcis d'écriture -
%-------------------------------------------------------------------------------

% Angles orientés (couples de vecteurs)
\newcommand{\aopp}[2]{(\vec{#1}, \vec{#2})} %Les deuc vecteurs sont positifs
\newcommand{\aopn}[2]{(\vec{#1}, -\vec{#2})} %Le second vecteur est négatif
\newcommand{\aonp}[2]{(-\vec{#1}, \vec{#2})} %Le premier vecteur est négatif
\newcommand{\aonn}[2]{(-\vec{#1}, -\vec{#2})} %Les deux vecteurs sont négatifs

%Ensembles mathématiques
\newcommand{\naturels}{\mathbb{N}} %Nombres naturels
\newcommand{\relatifs}{\mathbb{Z}} %Nombres relatifs
\newcommand{\rationnels}{\mathbb{Q}} %Nombres rationnels
\newcommand{\reels}{\mathbb{R}} %Nombres réels
\newcommand{\complexes}{\mathbb{C}} %Nombres complexes
%-------------------------------------------------------------------------------



\title{Correction des exercices de la semaine du 18/05}
\date{}

\begin{document}
	
\maketitle

\vspace*{-1cm}

\section*{Exercice 14 page 76}

Pour chaque égalité on multiplie le numérateur et le dénominateur de la première fraction par le même nombre.
\begin{multicols}{2}


\begin{enumerate}
	\item $\dfrac{3}{4} = \dfrac{3 \times 3}{4 \times 3} = \dfrac{\mathbf{9}}{12}$ 
	\item $\dfrac{4}{5} = \dfrac{4 \times 5}{5 \times 5} = \dfrac{\mathbf{20}}{25}$ 
	\item $\dfrac{2}{7} = \dfrac{2 \times 3}{7 \times 3} = \dfrac{6}{\mathbf{21}}$ 
	\item $\dfrac{6}{5} = \dfrac{6 \times 4}{5 \times 4} = \dfrac{24}{\mathbf{20}}$ 
\end{enumerate}

\end{multicols}


\section*{Exercice 15 page 76}

Pour chaque égalité on multiplie le numérateur et le dénominateur d'une des deux fraction par le même nombre pour compléter l'autre.
\begin{multicols}{2}
	
	
	\begin{enumerate}
		\item $\dfrac{7}{11} = \dfrac{7 \times 2}{11 \times 2} = \dfrac{\mathbf{14}}{22}$ 
		\item $\dfrac{12}{5} = \dfrac{12 \times 6}{5 \times 6} = \dfrac{72}{\mathbf{30}}$
		\item $\dfrac{\mathbf{81}}{63} = \dfrac{9 \times 9}{7 \times 9} =\dfrac{9}{7}$ 
		\item $\dfrac{\mathbf{65}}{10} = \dfrac{13 \times 5}{2 \times 5} = \dfrac{13}{2}$ %($10 \div 2 = 5$, $13 \times 5 = 65$)
	\end{enumerate}
	
\end{multicols}

\section*{Exercice 16 page 76}

Pour chaque égalité on multiplie où on divise le numérateur et le dénominateur d'une des deux fraction par le même nombre pour compléter l'autre.
\begin{multicols}{2}
	
	
	\begin{enumerate}
		\item $\dfrac{32}{\mathbf{36}} = \dfrac{8 \times 4}{9 \times 4} = \dfrac{8}{9}$ %($32 \div 8 = 4$, $9 \times 4 = 36$)
		\item $\dfrac{70}{\mathbf{77}} = \dfrac{10 \times 7}{11 \times 7} = \dfrac{10}{11}$ %($70 \div 10 = 7$, $11 \times 7 = 77$)
		\item $\dfrac{36}{42} = \dfrac{36 \div 6}{42 \div 6} = \dfrac{6}{\mathbf{7}}$ %($36 \div 6 = 6$, $42 \div 6 = 7$)
		\item $\dfrac{81}{72} = \dfrac{81 \div 9}{72 \div 9} = \dfrac{\mathbf{9}}{8}$ %($72 \div 8 = 9$, $81 \div 9 = 9$)
	\end{enumerate}
	
\end{multicols}


\section*{Exercice 18 page 76}

Pour simplifier une fraction je cherche tous les diviseurs du numérateur et du dénominateur et je divise par ceux qu'ils ont en commun.
\begin{multicols}{2}
	
	
	\begin{enumerate}
		\item $\dfrac{6}{9} = \dfrac{3 \times 2}{3 \times 3} = \dfrac{2}{3}$ 
		\item $\dfrac{8}{14} = \dfrac{4 \times 2}{7 \times 2} = \dfrac{4}{7}$ 
		\item $\dfrac{25}{15} = \dfrac{5 \times 5}{3 \times 5} = \dfrac{5}{3}$ 
		\item $\dfrac{16}{12} = \dfrac{4 \times 4}{3 \times 4} = \dfrac{4}{3}$ 
	\end{enumerate}
	
\end{multicols}


\section*{Exercice 18 page 76}

Pour simplifier une fraction je cherche tous les diviseurs du numérateur et du dénominateur et je divise par ceux qu'ils ont en commun.
\begin{multicols}{2}
	
	
	\begin{enumerate}
		\item $\dfrac{6}{9} = \dfrac{3 \times 2}{3 \times 3} = \dfrac{2}{3}$ 
		\item $\dfrac{8}{14} = \dfrac{4 \times 2}{7 \times 2} = \dfrac{4}{7}$ 
		\item $\dfrac{25}{15} = \dfrac{5 \times 5}{3 \times 5} = \dfrac{5}{3}$ 
		\item $\dfrac{16}{12} = \dfrac{4 \times 4}{3 \times 4} = \dfrac{4}{3}$ 
	\end{enumerate}
	
\end{multicols}

\section*{Exercice 19 page 76}

Pour simplifier une fraction je cherche tous les diviseurs du numérateur et du dénominateur et je divise par ceux qu'ils ont en commun.
\begin{multicols}{2}
	
	
	\begin{enumerate}
		\item $\dfrac{21}{14} = \dfrac{3 \times 7}{2 \times 7} = \dfrac{3}{2}$ 
		\item $\dfrac{12}{27} = \dfrac{4 \times 3}{9 \times 3} = \dfrac{4}{9}$ 
		\item $\dfrac{36}{42} = \dfrac{6 \times 6}{7 \times 6} = \dfrac{6}{7}$ 
		\item $\dfrac{30}{35} = \dfrac{6 \times 5}{7 \times 5} = \dfrac{6}{7}$ 
	\end{enumerate}
	
\end{multicols}

\section*{Exercice 20 page 76}

Pour simplifier une fraction je cherche tous les diviseurs du numérateur et du dénominateur et je divise par ceux qu'ils ont en commun.
\begin{multicols}{2}
	
	
	\begin{enumerate}
		\item $\dfrac{40}{24} = \dfrac{5 \times 8}{3 \times 8} = \dfrac{5}{3}$ 
		\item $\dfrac{36}{27} = \dfrac{4 \times 9}{3 \times 9} = \dfrac{4}{9}$ 
		\item $\dfrac{36}{60} = \dfrac{3 \times 2 \times 6}{5 \times 2 \times 6} = \dfrac{3}{5}$ 
		\item $\dfrac{45}{75} = \dfrac{3 \times 3 \times 5}{5 \times 3 \times 5} = \dfrac{3}{5}$ 
	\end{enumerate}
	
\end{multicols}


\section*{Exercice 22 page 77}

	\subsection*{1. }
		\begin{multicols}{2}
				
			\begin{enumerate}
				\item Un demi : $\dfrac{1}{2}$
				\item Deux tiers : $\dfrac{2}{3}$
				\item Sept huitièmes : $\dfrac{7}{8}$
				\item Neuf quarts : $\dfrac{9}{4}$
				\item Quinze sixièmes : $\dfrac{15}{6}$
				\item Trois centièmes : $\dfrac{3}{100}$
			\end{enumerate}
		\end{multicols}
	
	\subsection*{2. }
	
		\begin{enumerate}
			
			\begin{multicols}{2}
				\item $4 \times \mathbf{\dfrac{9}{4}} = 9$
				\item $3 \times \mathbf{\dfrac{2}{3}} = 2$
				\item $8 \times \mathbf{\dfrac{7}{8}} = 7$
				\item $\mathbf{\dfrac{3}{100} \times 100} = 3$
			\end{multicols}
		 
			
			\item En multipliant 8 par $\dfrac{1}{2}$, on obtient 4.
			\item En multipliant 6 par $\dfrac{15}{6}$, on obtient 15.
		\end{enumerate}
	
	
\section*{Exercice 26 page 77}

\begin{enumerate}
	\item $\dfrac{4}{3} = \dfrac{\mathbf{12}}{9} = \dfrac{\mathbf{20}}{15} = \dfrac{40}{\mathbf{30}} = \dfrac{48}{\mathbf{36}} = \dfrac{\num{1.2}}{\mathbf{\num{0.9}}} = \dfrac{\mathbf{20}}{15}$
	
	\item $\dfrac{36}{42} = \dfrac{\mathbf{24}}{28} = \dfrac{6}{\mathbf{7}} = \dfrac{\num{2.4}}{\mathbf{\num{2.8}}} = \dfrac{\mathbf{54}}{63} = \dfrac{\mathbf{66}}{77} = \dfrac{30}{\mathbf{35}}$
\end{enumerate}

\section*{Exercice 34 page 78}

\begin{enumerate}
	\item $7 \times \dfrac{15}{5} = 7 \times \dfrac{5 \times 5}{5} = 7 \times 5 \times \dfrac{5}{5} = 7 \times 5  \times 1= 35 $ 
	
	\vspace*{0.35cm}
	
		$7 \times \dfrac{15}{5} = 7 \times \dfrac{\textcolor{red}{\mathbf{3}} \times 5}{5} = 7 \times \textcolor{red}{\mathbf{3}} \times \dfrac{5}{5} = 7 \times \textcolor{red}{\mathbf{3}}  \times 1= \textcolor{red}{\mathbf{21}} $
	
	\vspace*{0.5cm}
	\item $12 \times \dfrac{14}{3} = \dfrac{12 \times 14}{3} = \dfrac{3 \times 2 \times 14}{3} = 2 \times 14 = 28$
	
	
	\vspace*{0.35cm}
	
	 $12 \times \dfrac{14}{3} = \dfrac{12 \times 14}{3} = \dfrac{3 \times \textcolor{red}{\mathbf{4}} \times 14}{3} = \textcolor{red}{\mathbf{4}} \times 14 = \textcolor{red}{\mathbf{56}}$
	
	\vspace*{0.5cm}
	\item $25 \times \dfrac{6}{5} = \dfrac{25 \times 6}{5} = \dfrac{5 \times 5 \times 6}{5} = 5 \times 6 = 30$
\end{enumerate}

\section*{Exercice 35 page 78}

\begin{enumerate}
	\item $27 \times \dfrac{14}{3} = \dfrac{27 \times 14}{3} = \dfrac{3 \times 9 \times 14}{3} = 9 \times 14 = 126$
	
	\vspace*{0.35cm}
	
	$27 \times \dfrac{14}{ \textcolor{red}{\mathbf{9}}} = \dfrac{27 \times 14}{ \textcolor{red}{\mathbf{9}}} = \dfrac{3 \times 9 \times 14}{ \textcolor{red}{\mathbf{9}}} =  \textcolor{red}{\mathbf{3}} \times 14 =  \textcolor{red}{\mathbf{42}}$
	
	\vspace*{0.5cm}
	\item $8 \times \dfrac{16}{8} = 8 \times \dfrac{4 \times 4}{4} = 8 \times 4 \times \dfrac{4}{4} = 8 \times 4 \times 1 = 32$
	
	\vspace*{0.35cm}
	
	$8 \times \dfrac{16}{ \textcolor{red}{\mathbf{4}}} = 8 \times \dfrac{4 \times 4}{4} = 8 \times 4 \times \dfrac{4}{4} = 8 \times 4 \times 1 = 32$
	
	\vspace*{0.5cm}
	\item $66 \times \dfrac{9}{11} = \dfrac{66 \times 9}{11} = \dfrac{6 \times 11 \times 9}{11} = 6 \times 9 = 54$
\end{enumerate}


\section*{Exercice 37 page 78}

\begin{enumerate}
	\item $\dfrac{2}{7} \times 7 = \dfrac{2 \times 7}{7} =  2 \times 1 = 2$ %2 \times \dfrac{7}{7} =
	
	\vspace*{0.5cm}
	\item $9 \times \dfrac{4}{3} = \dfrac{9 \times 4}{3} =  \dfrac{3 \times 3 \times 4}{3} = 3 \times 4 = 12$
	
	\vspace*{0.5cm}
	\item $36 \times \dfrac{7}{6} = \dfrac{36 \times 7}{6} = \dfrac{6 \times 6 \times 7}{6} = 6 \times 7 = 12$
	
	\vspace*{0.35cm}
	
	$36 \times \dfrac{7}{6} = \dfrac{36 \times 7}{6} = \dfrac{6 \times 6 \times 7}{6} = 6 \times 7 = \textcolor{red}{\mathbf{42}}$
	
	\vspace*{0.5cm}
	\item $25 \times \dfrac{12}{5} = \dfrac{25 \times 12}{5} = \dfrac{5 \times 5 \times 12}{5} = 5 \times 12 = 60$
	
	\vspace*{0.5cm}
	\item $12 \times \dfrac{8}{6} = \dfrac{12 \times 8}{6} = \dfrac{2 \times 6 \times 8}{6} = 2 \times 8 = 16$
	
	\vspace*{0.5cm}
	\item $\dfrac{11}{50} \times 100 = \dfrac{11 \times 100}{50} = \dfrac{11 \times 2 \times 50}{50} =  11 \times 50 = 550$ 
	
	\vspace*{0.35cm}
	$\dfrac{11}{50} \times 100 = \dfrac{11 \times 100}{50} = \dfrac{11 \times 2 \times 50}{50} =  11 \times \textcolor{red}{\mathbf{2}} = \textcolor{red}{\mathbf{22}}$ 
\end{enumerate}


\section*{Exercice 40 page 78}

\begin{enumerate}
	\item $8 \times \dfrac{6}{13} = \dfrac{8 \times 6}{13} = \dfrac{48}{13} = 48 \div 13 \approx \num{3.68}$
	
	\vspace*{0.5cm}
	\item $3 \times \dfrac{4}{9} = \dfrac{3 \times 4}{9} = \dfrac{3 \times 4}{3 \times 3} = \dfrac{4}{3} = 4 \div 3  \approx \num{1.33}$
	
	\vspace*{0.5cm}
	\item $2 \times \dfrac{7}{11} = \dfrac{2 \times 7}{11} = \dfrac{14}{11} =  14 \div 11  \approx \num{1.27}$
\end{enumerate}


\section*{Exercice 45 page 79}

Une journée fait 24 heures;

\begin{equation*}
	\dfrac{1}{6} \times 24 = \dfrac{24}{6} = \dfrac{4 \times 6}{6}= 4
\end{equation*}

Elle fait de l'informatique 4 heures par jour.

\newpage

\section*{Exercice 47 page 79}

\begin{equation*}
\dfrac{7}{8} \times 24 = \dfrac{7 \times 24}{8} = \dfrac{7 \times 3 \times 8}{6}= 7 \times 3 = 21
\end{equation*}

Sur les 24 km de son parcours, 21 sont à l'ombre.


\section*{Exercice 49 page 79}

Lisa casse $\frac{4}{10}$ des assiettes qu'elle lave.

\begin{equation*}
\dfrac{4}{10} \times 15 = \dfrac{4 \times 15}{10} = \dfrac{2 \times 2 \times 3 \times 5}{2 \times 5}= 2 \times 3 = 6
\end{equation*}

Elle casse 6 assiettes parmi les 15 qu'elle lave.

\begin{equation*}
	15 - 6 = 9
\end{equation*}

Donc il reste 9 assiettes intactes.

\end{document}