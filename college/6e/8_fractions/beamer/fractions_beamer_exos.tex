\documentclass[xcolor={dvipsnames}]{beamer}
%\usepackage[utf8]{inputenc}
%\usetheme{Madrid}
\usetheme{CambridgeUS}
%\usetheme{Malmoe}
%\usecolortheme{beaver}
\usecolortheme{seahorse}

\input{../../../../utils_maths_beamer}


\usepackage{../../../../pas-math}
\usepackage{../../../../moncours_beamer}

\usepackage{amssymb,amsmath}


\newcommand{\myitem}{\item[\textbullet]}

\graphicspath{{../img/}}

\title{Séquence 8 : Fractions}
\subtitle{Correction des exercices semaine du 18/05}
%\author{O. FINOT}\institute{Collège S$^t$ Bernard}

%
%\AtBeginSection[]
%{
%	\begin{frame}
%		\frametitle{}
%		\tableofcontents[currentsection, hideallsubsections]
%	\end{frame} 
%
%}
%
%
%\AtBeginSubsection[]
%{
%	\begin{frame}
%		\frametitle{Sommaire}
%		\tableofcontents[currentsection, currentsubsection]
%	\end{frame} 
%}

\begin{document}



\begin{frame}
  \titlepage 
\end{frame}

%\section{Corrections}
	

\begin{frame}
	\frametitle{Exercice 31 page 77}
	\framesubtitle{}
	

	
	\begin{block}{{\centering Téléviseur $\frac{4}{3}$}}
		\pause
		\begin{eqnarray*}
			\dfrac{Longueur}{largeur} &= & \dfrac{4}{3} \\ \pause
			\dfrac{96}{?} &= & \dfrac{4}{3} \\ \pause
			\dfrac{24 \times 4}{?} &= & \dfrac{4}{3} \\ \pause
			\dfrac{24 \times 4}{24 \times 3} &= & \dfrac{4}{3} \\ \pause
			\dfrac{96}{72} &= & \dfrac{4}{3}  \pause
		\end{eqnarray*}
		
		Un téléviseur $\dfrac{4}{3}$ de 96 cm de longueur a une largeur de 72 cm. 
	\end{block}


	
\end{frame}


\begin{frame}
	\frametitle{Exercice 31 page 77}

	
	
	\begin{block}{Téléviseur $\frac{16}{9}$}
		\pause
		\begin{eqnarray*}
			\dfrac{Longueur}{largeur} &= & \dfrac{16}{9} \\ \pause
			\dfrac{96}{?} &= & \dfrac{16}{9} \\ \pause
			\dfrac{6 \times 16}{?} &= & \dfrac{16}{9} \\ \pause
			\dfrac{6 \times 16}{6 \times 9} &= & \dfrac{16}{9} \\ \pause
			\dfrac{96}{54} &= & \dfrac{16}{9}  \pause
		\end{eqnarray*}
		
		Un téléviseur $\dfrac{16}{9}$ de 96 cm de longueur a une largeur de 54 cm. 
	\end{block}
	
	
\end{frame}

\begin{frame}
	\frametitle{Exercice 51 page 79}
	Le parcours fait 45 km.
	
	\begin{block}{Coureur 1}
		\begin{eqnarray*}
			d_1 &=& \frac{3}{2} \; de \; 45 \; km \\ \pause
			d_1 &=& \frac{3}{2} \times 45 \\ \pause
			d_1 &=& \frac{3 \times 45}{2} \\ \pause
			d_1 &=& \frac{135}{2} \\ \pause
			d_1 &=& \num{67.5}  \pause
		\end{eqnarray*}
	
	Le premier coureur a terminé la course il a parcouru \num{67.5} km.
	\end{block}
\end{frame}

\begin{frame}
	\frametitle{Exercice 51 page 79}
	Le parcours fait 45 km.
	
	\begin{block}{Coureur 2}
		\begin{eqnarray*}
			d_2 &=& \frac{4}{5} \; de \; 45 \; km \\ \pause
			d_2 &=& \frac{4}{5} \times 45 \\ \pause
			d_2 &=& \frac{4 \times 45}{5} \\ \pause
			d_2 &=& \frac{4 \times 9 \times 5}{5} \\ \pause
			d_2 &=& \num{36}  \pause
		\end{eqnarray*}
		
		Le deuxième coureur a parcouru \num{36} km.
	\end{block}
\end{frame}


\begin{frame}
	\frametitle{Exercice 51 page 79}
	Le parcours fait 45 km.
	
	\begin{block}{Coureur 3}
		\begin{eqnarray*}
			d_3 &=& \frac{7}{9} \; de \; 45 \; km \\ \pause
			d_3 &=& \frac{7}{9} \times 45 \\ \pause
			d_3 &=& \frac{7 \times 45}{9} \\ \pause
			d_3 &=& \frac{7 \times 5 \times 9}{9} \\ \pause
			d_3 &=& \num{35}  \pause
		\end{eqnarray*}
		
		Le troisième coureur a parcouru \num{35} km.
	\end{block}
\end{frame}


\begin{frame}
	\frametitle{Exercice 51 page 79}
		
	\begin{block}{Bilan}
		\begin{itemize}
			\item Le premier coureur a terminé la course il a parcouru \num{67.5} km.
			\item Le deuxième coureur a parcouru \num{36} km.
			\item Le troisième coureur a parcouru \num{35} km. \pause
		\end{itemize} 
		
		

	\end{block}
	
	\vspace*{1cm}
	C'est donc le premier coureur qui est en tête de la course.
\end{frame}


\end{document}