\documentclass[xcolor={dvipsnames}]{beamer}
%\usepackage[utf8]{inputenc}
%\usetheme{Madrid}
\usetheme{CambridgeUS}
%\usetheme{Malmoe}
%\usecolortheme{beaver}
\usecolortheme{seahorse}

\input{../../../../utils_maths_beamer}


\usepackage{../../../../pas-math}
\usepackage{../../../../moncours_beamer}

\usepackage{amssymb,amsmath}


\newcommand{\myitem}{\item[\textbullet]}

\graphicspath{{../img/}}

\title{Séquence 7 : Fractions}
\subtitle{Correction des exercices semaine du 05/04}
%\author{O. FINOT}\institute{Collège S$^t$ Bernard}

%
%\AtBeginSection[]
%{
%	\begin{frame}
%		\frametitle{}
%		\tableofcontents[currentsection, hideallsubsections]
%	\end{frame} 
%
%}
%
%
%\AtBeginSubsection[]
%{
%	\begin{frame}
%		\frametitle{Sommaire}
%		\tableofcontents[currentsection, currentsubsection]
%	\end{frame} 
%}

\begin{document}



\begin{frame}
  \titlepage 
\end{frame}

%\section{Corrections}
	
\begin{frame}
	\frametitle{Exercice 1}
	\begin{enumerate}[a]
		\item $\dfrac{5}{10}$ : \pause  cinq dixièmes
		\item $\dfrac{12}{200}$ : \pause  douze deux-centièmes
		\item $\dfrac{103}{10000}$ : \pause  cent trois dix-millièmes
		\item $\dfrac{5}{2}$ : \pause  cinq demis
		\item $\dfrac{2}{3}$ : \pause  deux tiers
		\item $\dfrac{9}{4}$ : \pause  neuf quarts
		\item $\dfrac{30}{13}$ : \pause  trente treizièmes
	\end{enumerate}
\end{frame}


\begin{frame}
	\frametitle{Exercice 2}
	\begin{enumerate}[a]
		\item douze centièmes: \pause  $\dfrac{12}{100}$
		\item vingt-six millièmes: \pause  $\dfrac{26}{1000}$
		\item seize tiers: \pause  $\dfrac{16}{3}$
		\item trois demis: \pause  $\dfrac{3}{2}$
		\item huit quarts: \pause  $\dfrac{8}{4}$
		\item quatre-vingts neuvièmes: \pause  $\dfrac{80}{9}$
		\item quatre vingt-neuvièmes: \pause  $\dfrac{4}{29}$
	\end{enumerate}
\end{frame}



\begin{frame}
	\frametitle{Exercice 5}
	\begin{enumerate}[a]
		\item \begin{itemize}
			\item {\Large 6 mois représentent \pause \textcolor{blue}{une demi} année.}
			\item {\Large 4 mois représentent \pause \textcolor{blue}{un tiers} d'année.}
			\item {\Large 30 minutes représentent \pause \textcolor{blue}{une demi} heure.}
			\item {\Large 15 minutes représentent \pause \textcolor{blue}{un quart} d'heure.}
			\item {\Large  45 minutes représentent \pause \textcolor{blue}{trois quarts} d'heure.}
		\end{itemize}
		
		\vspace*{2cm}
		
		\item \begin{itemize}
			\item {\Large Un demi-litre de lait représente \pause \textcolor{blue}{50} cL.}
			\item {\Large Une demi douzaine d'\oe ufs représente \pause \textcolor{blue}{6} \oe ufs.}
		\end{itemize}
	\end{enumerate}
\end{frame}

\begin{frame}
	\frametitle{Exercice 6}
	
	Un sac de billes est composé de 5 billes bleues et de 3 billes rouges.
	
	\begin{enumerate}[a]
		\item  $\dfrac{5}{8}$ représente \pause la fraction de billes bleues dans la sac.
		
		\item Si Aïssatou ajoute une bille rouge dans le sac, la fraction sera modifiée au niveau \pause du \textcolor{blue}{dénominateur} $ \left(  \dfrac{5}{9} \right) $.
		
		
		\item Si Aïssatou enlève une bille bleue du sac, la fraction sera modifiée au niveau \pause du \textcolor{blue}{numérateur} et du  \textcolor{blue}{dénominateur} $ \left( \dfrac{4}{7} \right) $.
	\end{enumerate}
 \end{frame}


\begin{frame}
	\frametitle{Exercice 11}
	
	Pour son anniversaire, Léo reçoit trois amis. Sa maman a préparé un gâteau qu'elle coupe en huit parts égales.
	
	
	
	\begin{center}
		\includegraphics[scale=0.08]{11_1}
	\end{center}
\end{frame}


\begin{frame}
	\frametitle{Exercice 11}
	
	Pour son anniversaire, Léo reçoit trois amis. Sa maman a préparé un gâteau qu'elle coupe en huit parts égales.
	
	
	\begin{columns}
		\begin{column}{0.5\textwidth}
			\begin{itemize}
				\item Léo prend une part
			\end{itemize}
		\end{column}
	\begin{column}{0.5\textwidth}
		\begin{center}
			\includegraphics[scale=0.08]{11_2}
		\end{center}
	\end{column}
	\end{columns}
	
\end{frame}

\begin{frame}
	\frametitle{Exercice 11}
	
	Pour son anniversaire, Léo reçoit trois amis. Sa maman a préparé un gâteau qu'elle coupe en huit parts égales.
	
	
	\begin{columns}
		\begin{column}{0.5\textwidth}
			\begin{itemize}
				\item Léo prend une part
				\item Anna prend deux parts 
			\end{itemize}
		\end{column}
		\begin{column}{0.5\textwidth}
			\begin{center}
				\includegraphics[scale=0.08]{11_3}
			\end{center}
		\end{column}
	\end{columns}
	
\end{frame}


\begin{frame}
	\frametitle{Exercice 11}
	
	Pour son anniversaire, Léo reçoit trois amis. Sa maman a préparé un gâteau qu'elle coupe en huit parts égales.
	
	
	\begin{columns}
		\begin{column}{0.5\textwidth}
			\begin{itemize}
				\item Léo prend une part
				\item Anna prend deux parts 
				\item Armand prend deux parts 
			\end{itemize}
		\end{column}
		\begin{column}{0.5\textwidth}
			\begin{center}
				\includegraphics[scale=0.08]{11_4}
			\end{center}
		\end{column}
	\end{columns}
	
\end{frame}

\begin{frame}
	\frametitle{Exercice 11}
	
	Pour son anniversaire, Léo reçoit trois amis. Sa maman a préparé un gâteau qu'elle coupe en huit parts égales.
	
	
	\begin{columns}
		\begin{column}{0.5\textwidth}
			\begin{itemize}
				\item Léo prend une part
				\item Anna prend deux parts 
				\item Armand prend deux parts 
				\item Oscar prend trois parts
			\end{itemize}
		\end{column}
		\begin{column}{0.5\textwidth}
			\begin{center}
				\includegraphics[scale=0.08]{11_5}
			\end{center}
		\end{column}
	\end{columns}
	
\end{frame}

\begin{frame}
	\frametitle{Exercice 11}
	
	Pour son anniversaire, Léo reçoit trois amis. Sa maman a préparé un gâteau qu'elle coupe en huit parts égales.
	
	
	\begin{columns}
		\begin{column}{0.5\textwidth}
			\begin{itemize}
				\item Léo prend une part
				\item Anna prend deux parts 
				\item Armand prend deux parts 
				\item Oscar prend trois parts
			\end{itemize}
		\end{column}
		\begin{column}{0.5\textwidth}
			\begin{center}
				\includegraphics[scale=0.08]{11_5}
			\end{center}
		\end{column}
	\end{columns}
	
	\textcolor{blue}{Tout le gâteau a été mangé il n'en reste plus pour la maman de Léo.}
\end{frame}
%\begin{frame}
%	\frametitle{Exercice 5 }
%
%	
%
%	
%	\begin{block}{{\centering Téléviseur $\frac{4}{3}$}}
%		\pause
%		\begin{eqnarray*}
%			\dfrac{Longueur}{largeur} &= & \dfrac{4}{3} \\ \pause
%			\dfrac{96}{?} &= & \dfrac{4}{3} \\ \pause
%			\dfrac{24 \times 4}{?} &= & \dfrac{4}{3} \\ \pause
%			\dfrac{24 \times 4}{24 \times 3} &= & \dfrac{4}{3} \\ \pause
%			\dfrac{96}{72} &= & \dfrac{4}{3}  \pause
%		\end{eqnarray*}
%		
%		Un téléviseur $\dfrac{4}{3}$ de 96 cm de longueur a une largeur de 72 cm. 
%	\end{block}
%
%
%	
%\end{frame}
%
%
%\begin{frame}
%	\frametitle{Exercice 31 page 77}
%
%	
%	
%	\begin{block}{Téléviseur $\frac{16}{9}$}
%		\pause
%		\begin{eqnarray*}
%			\dfrac{Longueur}{largeur} &= & \dfrac{16}{9} \\ \pause
%			\dfrac{96}{?} &= & \dfrac{16}{9} \\ \pause
%			\dfrac{6 \times 16}{?} &= & \dfrac{16}{9} \\ \pause
%			\dfrac{6 \times 16}{6 \times 9} &= & \dfrac{16}{9} \\ \pause
%			\dfrac{96}{54} &= & \dfrac{16}{9}  \pause
%		\end{eqnarray*}
%		
%		Un téléviseur $\dfrac{16}{9}$ de 96 cm de longueur a une largeur de 54 cm. 
%	\end{block}
%	
%	
%\end{frame}
%
%\begin{frame}
%	\frametitle{Exercice 51 page 79}
%	Le parcours fait 45 km.
%	
%	\begin{block}{Coureur 1}
%		\begin{eqnarray*}
%			d_1 &=& \frac{3}{2} \; de \; 45 \; km \\ \pause
%			d_1 &=& \frac{3}{2} \times 45 \\ \pause
%			d_1 &=& \frac{3 \times 45}{2} \\ \pause
%			d_1 &=& \frac{135}{2} \\ \pause
%			d_1 &=& \num{67.5}  \pause
%		\end{eqnarray*}
%	
%	Le premier coureur a terminé la course il a parcouru \num{67.5} km.
%	\end{block}
%\end{frame}
%
%\begin{frame}
%	\frametitle{Exercice 51 page 79}
%	Le parcours fait 45 km.
%	
%	\begin{block}{Coureur 2}
%		\begin{eqnarray*}
%			d_2 &=& \frac{4}{5} \; de \; 45 \; km \\ \pause
%			d_2 &=& \frac{4}{5} \times 45 \\ \pause
%			d_2 &=& \frac{4 \times 45}{5} \\ \pause
%			d_2 &=& \frac{4 \times 9 \times 5}{5} \\ \pause
%			d_2 &=& \num{36}  \pause
%		\end{eqnarray*}
%		
%		Le deuxième coureur a parcouru \num{36} km.
%	\end{block}
%\end{frame}
%
%
%\begin{frame}
%	\frametitle{Exercice 51 page 79}
%	Le parcours fait 45 km.
%	
%	\begin{block}{Coureur 3}
%		\begin{eqnarray*}
%			d_3 &=& \frac{7}{9} \; de \; 45 \; km \\ \pause
%			d_3 &=& \frac{7}{9} \times 45 \\ \pause
%			d_3 &=& \frac{7 \times 45}{9} \\ \pause
%			d_3 &=& \frac{7 \times 5 \times 9}{9} \\ \pause
%			d_3 &=& \num{35}  \pause
%		\end{eqnarray*}
%		
%		Le troisième coureur a parcouru \num{35} km.
%	\end{block}
%\end{frame}
%
%
%\begin{frame}
%	\frametitle{Exercice 51 page 79}
%		
%	\begin{block}{Bilan}
%		\begin{itemize}
%			\item Le premier coureur a terminé la course il a parcouru \num{67.5} km.
%			\item Le deuxième coureur a parcouru \num{36} km.
%			\item Le troisième coureur a parcouru \num{35} km. \pause
%		\end{itemize} 
%		
%		
%
%	\end{block}
%	
%	\vspace*{1cm}
%	C'est donc le premier coureur qui est en tête de la course.
%\end{frame}
%

\end{document}