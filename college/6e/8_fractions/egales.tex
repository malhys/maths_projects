\begin{myprop}
	
	\iftoggle{eleve}{%
		Une \hrulefill 
		
		\vspace*{0.2cm}
		\hrulefill	
		
		\vspace*{0.2cm}
		\hrulefill
		
		
		\vspace*{2cm}
	
	}{%
		Une fraction ne change pas quand on \kw{multiplie (ou on divise)} le numérateur \kw{et} le dénominateur par un même nombre non nul.
		
		\begin{multicols}{3}
			\begin{equation*}
				\dfrac{a}{b} = \dfrac{a \times k}{b \times k} 
			\end{equation*}
			
			\begin{center}
				ou
			\end{center}
			
			\begin{equation*}
				\dfrac{a}{b} = \dfrac{a \div k}{b \div k} 
			\end{equation*}	
		\end{multicols}	
	}
	
	
\end{myprop}

\begin{myex}
	
	\begin{multicols}{2}
		
	\iftoggle{eleve}{%
		\begin{equation*}
			\dfrac{\num{7}}{\num{5}} = \dfrac{\hspace*{2cm}}{\hspace*{2cm} } = \dfrac{\hspace*{1cm}}{\hspace*{1cm}}
		\end{equation*}
		
		\begin{equation*}
			\dfrac{\num{12}}{\num{27}} = \dfrac{\hspace*{2cm}}{\hspace*{2cm} } = \dfrac{\hspace*{1cm}}{\hspace*{1cm}}
		\end{equation*}
	}{%
		\begin{equation*}
			\dfrac{\num{7}}{\num{5}} = \dfrac{\num{7} \times 10 }{\num{5} \times 10 } = \dfrac{\num{70}}{\num{50}}
		\end{equation*}
	
		\begin{equation*}
			\dfrac{\num{12}}{\num{27}} = \dfrac{\num{12} \div 3 }{\num{27} \div 3 } = \dfrac{\num{4}}{\num{9}} 
		\end{equation*}
	}
	
	
	
	
	\end{multicols}
\end{myex}

\begin{mydef}
	\iftoggle{eleve}{%
		\vspace*{0.2cm}
		\hrulefill
		
		\vspace*{0.2cm}
		\hrulefill
		
		\vspace*{0.2cm}
		\hrulefill
	}{%
		Simplifier une fraction, c'est trouver une autre fraction \kw{égale à la première} avec le numérateur et le dénominateur \kw{les plus petits possibles}.
	}
	
\end{mydef}

\begin{myex}
	\begin{multicols}{2}
		
		\iftoggle{eleve}{%
			\begin{equation*}
				\dfrac{\num{27}}{\num{72}} = \dfrac{\hspace*{2cm}}{\hspace*{2cm} } = \dfrac{\hspace*{1cm}}{\hspace*{1cm}}
			\end{equation*}	
			
			\begin{equation*}
				\dfrac{\num{25}}{\num{100}} = \dfrac{\hspace*{2cm}}{\hspace*{2cm} } = \dfrac{\hspace*{1cm}}{\hspace*{1cm}}
			\end{equation*}	
		\end{multicols}
		}{%
			\begin{equation*}
				\dfrac{\num{27}}{\num{72}} = \dfrac{\num{27} \div 9 }{\num{72} \div 9 } = \dfrac{\num{3}}{\num{8}} 
			\end{equation*}	
			
			\begin{equation*}
				\dfrac{\num{25}}{\num{100}} = \dfrac{\num{25} \div 25 }{\num{100} \div 25 } = \dfrac{\num{1}}{\num{4}} 
			\end{equation*}	
		\end{multicols}
		}
		
\end{myex}

\begin{mymeth}
	Je veux simplifier la fraction $\dfrac{105}{60}$
	
	\begin{enumerate}
		
		\iftoggle{eleve}{%
			\item Je cherche un diviseur commun au numérateur et au dénominateur :
			105 et 60 sont divisibles par 
			\item Je calcule les divisions :
			\begin{equation*}
				\frac{105}{60} = \dfrac{\hspace*{2cm}}{\hspace*{2cm} } = \dfrac{\hspace*{1cm}}{\hspace*{1cm}}
			\end{equation*}
			\item Je recommence si je peux, autant de fois que possible, le numérateur et le dénominateur sont divisibles par 
			
			\vspace*{0.2cm}
			\begin{equation*}
				\dfrac{\hspace*{1cm}}{\hspace*{1cm}} = \dfrac{\hspace*{2cm}}{\hspace*{2cm} } = \dfrac{\hspace*{1cm}}{\hspace*{1cm}}		
			\end{equation*}
			
			\item Si je ne peux pas continuer, j'ai terminé:
			\begin{equation*}
				\frac{105}{60} = \dfrac{\hspace*{1cm}}{\hspace*{1cm}}		
			\end{equation*}
		}{%
			\item Je cherche un diviseur commun au numérateur et au dénominateur :
			105 et 60 sont divisibles par 5.
			\item Je calcule les divisions :
			\begin{equation*}
				\frac{105}{60} = \frac{105 \div 5}{60 \div 5 } = \frac{21}{12}
			\end{equation*}
			\item Je recommence si je peux, autant de fois que possible, le numérateur et le dénominateur sont divisibles par 3.
			\begin{equation*}
				\frac{21}{12} = \frac{21 \div 3}{12 \div 3} = \frac{7}{4}		
			\end{equation*}
			
			\item Si je ne peux pas continuer, j'ai terminé:
			\begin{equation*}
				\frac{105}{60} = \frac{7}{4}		
			\end{equation*}
		}
		
		
	\end{enumerate}

	
\end{mymeth}
	