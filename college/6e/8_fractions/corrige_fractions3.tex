\documentclass[12pt,a4paper]{article}


\usepackage[in, plain]{fullpage}
\usepackage{array}
\usepackage{../../../pas-math}

%-------------------------------------------------------------------------------
%          -Packages nécessaires pour écrire en Français et en UTF8-
%-------------------------------------------------------------------------------
\usepackage[utf8]{inputenc}
\usepackage[frenchb]{babel}
\usepackage[T1]{fontenc}
\usepackage{lmodern}
\usepackage{textcomp}



%-------------------------------------------------------------------------------

%-------------------------------------------------------------------------------
%                          -Outils de mise en forme-
%-------------------------------------------------------------------------------
\usepackage{hyperref}
\hypersetup{pdfstartview=XYZ}
%\usepackage{enumerate}
\usepackage{graphicx}
\usepackage{multicol}
\usepackage{tabularx}
\usepackage{multirow}


\usepackage{anysize} %%pour pouvoir mettre les marges qu'on veut
%\marginsize{2.5cm}{2.5cm}{2.5cm}{2.5cm}

\usepackage{indentfirst} %%pour que les premier paragraphes soient aussi indentés
\usepackage{verbatim}
\usepackage{enumitem}
\usepackage[usenames,dvipsnames,svgnames,table]{xcolor}

\usepackage{variations}

%-------------------------------------------------------------------------------


%-------------------------------------------------------------------------------
%                  -Nécessaires pour écrire des mathématiques-
%-------------------------------------------------------------------------------
\usepackage{amsfonts}
\usepackage{amssymb}
\usepackage{amsmath}
\usepackage{amsthm}
\usepackage{tikz}
\usepackage{xlop}
%-------------------------------------------------------------------------------



%-------------------------------------------------------------------------------


%-------------------------------------------------------------------------------
%                    - Mise en forme avancée
%-------------------------------------------------------------------------------

\usepackage{ifthen}
\usepackage{ifmtarg}


\newcommand{\ifTrue}[2]{\ifthenelse{\equal{#1}{true}}{#2}{$\qquad \qquad$}}

%-------------------------------------------------------------------------------

%-------------------------------------------------------------------------------
%                     -Mise en forme d'exercices-
%-------------------------------------------------------------------------------
%\newtheoremstyle{exostyle}
%{\topsep}% espace avant
%{\topsep}% espace apres
%{}% Police utilisee par le style de thm
%{}% Indentation (vide = aucune, \parindent = indentation paragraphe)
%{\bfseries}% Police du titre de thm
%{.}% Signe de ponctuation apres le titre du thm
%{ }% Espace apres le titre du thm (\newline = linebreak)
%{\thmname{#1}\thmnumber{ #2}\thmnote{. \normalfont{\textit{#3}}}}% composants du titre du thm : \thmname = nom du thm, \thmnumber = numéro du thm, \thmnote = sous-titre du thm

%\theoremstyle{exostyle}
%\newtheorem{exercice}{Exercice}
%
%\newenvironment{questions}{
%\begin{enumerate}[\hspace{12pt}\bfseries\itshape a.]}{\end{enumerate}
%} %mettre un 1 à la place du a si on veut des numéros au lieu de lettres pour les questions 
%-------------------------------------------------------------------------------

%-------------------------------------------------------------------------------
%                    - Mise en forme de tableaux -
%-------------------------------------------------------------------------------

\renewcommand{\arraystretch}{1.7}

\setlength{\tabcolsep}{1.2cm}

%-------------------------------------------------------------------------------



%-------------------------------------------------------------------------------
%                    - Racourcis d'écriture -
%-------------------------------------------------------------------------------

% Angles orientés (couples de vecteurs)
\newcommand{\aopp}[2]{(\vec{#1}, \vec{#2})} %Les deuc vecteurs sont positifs
\newcommand{\aopn}[2]{(\vec{#1}, -\vec{#2})} %Le second vecteur est négatif
\newcommand{\aonp}[2]{(-\vec{#1}, \vec{#2})} %Le premier vecteur est négatif
\newcommand{\aonn}[2]{(-\vec{#1}, -\vec{#2})} %Les deux vecteurs sont négatifs

%Ensembles mathématiques
\newcommand{\naturels}{\mathbb{N}} %Nombres naturels
\newcommand{\relatifs}{\mathbb{Z}} %Nombres relatifs
\newcommand{\rationnels}{\mathbb{Q}} %Nombres rationnels
\newcommand{\reels}{\mathbb{R}} %Nombres réels
\newcommand{\complexes}{\mathbb{C}} %Nombres complexes


%Intégration des parenthèses aux cosinus
\newcommand{\cosP}[1]{\cos\left(#1\right)}
\newcommand{\sinP}[1]{\sin\left(#1\right)}


%Probas stats
\newcommand{\stat}{statistique}
\newcommand{\stats}{statistiques}
%-------------------------------------------------------------------------------

%-------------------------------------------------------------------------------
%                    - Mise en page -
%-------------------------------------------------------------------------------

\newcommand{\twoCol}[1]{\begin{multicols}{2}#1\end{multicols}}


\setenumerate[1]{font=\bfseries,label=\textit{\alph*})}
\setenumerate[2]{font=\bfseries,label=\arabic*)}


%-------------------------------------------------------------------------------
%                    - Elements cours -
%-------------------------------------------------------------------------------




\title{Correction des exercices de la semaine du 25/05}
\date{}

\begin{document}
	
\maketitle

\vspace*{-1cm}

\section*{Exercice 36}

La figure est partagée en 5 colonnes de 3 lignes chacune, il y a donc 15 parties.
De ces 15 parties, 4 sont colorées en rouge et 3 en vert, il y a en tout 7 parties colorées.
On a donc :
	\begin{equation*}
		\dfrac{4}{15} + \dfrac{3}{15} = \dfrac{4+3}{15} = \dfrac{7}{15}
	\end{equation*}


\section*{Exercice 37}

\begin{multicols}{3}
\begin{enumerate}
	\item \begin{equation*}
	\dfrac{1}{5} + \dfrac{3}{5} = \dfrac{1+3}{5} = \dfrac{4}{5}
	\end{equation*}
	
	
	\item \begin{equation*}
	\dfrac{6}{7} - \dfrac{5}{7} = \dfrac{6-5}{7} = \dfrac{2}{7}
	\end{equation*}
	
	
	\item \begin{equation*}
	\dfrac{7}{9} + \dfrac{2}{9} = \dfrac{7+2}{9} = \dfrac{9}{9} = 1
	\end{equation*}
	
\end{enumerate}


\end{multicols}

%
%\section*{Exercice 38}
%
%\begin{multicols}{3}
%	\begin{enumerate}
%		\item \begin{equation*}
%		\dfrac{1}{11} + \dfrac{5}{11} = \dfrac{1+5}{11} = \dfrac{6}{11}
%		\end{equation*}
%		
%		
%		\item \begin{equation*}
%		\dfrac{6}{7} - \dfrac{5}{7} = \dfrac{6-5}{7} = \dfrac{2}{7}
%		\end{equation*}
%		
%		
%		\item \begin{equation*}
%		\dfrac{7}{9} + \dfrac{2}{9} = \dfrac{7+2}{9} = \dfrac{9}{9} = 1
%		\end{equation*}
%		
%	\end{enumerate}
%	
%	
%\end{multicols}

\section*{Exercice 39}

\begin{multicols}{3}
	\begin{enumerate}
		\item \begin{equation*}
		\dfrac{7}{11} + \dfrac{\mathbf{2}}{11} = \dfrac{9}{11}
		\end{equation*}
		
		
		\item \begin{equation*}
		\dfrac{13}{9} - \dfrac{\mathbf{5}}{9} = \dfrac{8}{9}
		\end{equation*}
		
		
		\item \begin{equation*}
		\dfrac{7}{13} - \dfrac{5}{13} = \dfrac{2}{13} 
		\end{equation*}
		
	\end{enumerate}
	
	
\end{multicols}

\newpage
\section*{Exercice 44}

\begin{multicols}{2}
	\begin{enumerate}
		\item \begin{align*}
			A &= \dfrac{4}{7} + \dfrac{3}{4} + \dfrac{2}{7} + \dfrac{5}{4}  + \dfrac{1}{7} \\
			A &= \dfrac{4}{7} + \dfrac{1}{7} + \dfrac{2}{7} + \dfrac{3}{4} + \dfrac{5}{4}\\
			A &= \dfrac{7}{7} + \dfrac{8}{4} \\
			A &= 1 + 2 \\
			A &= 3 \\
		\end{align*}
		
		
		\item \begin{align*}
		B &= \dfrac{5}{12} - \dfrac{5}{3} + \dfrac{23}{12} - \dfrac{2}{3}\\
		B &= \dfrac{5}{12} + \dfrac{23}{12} + \dfrac{5}{3}  + \dfrac{2}{3}\\
		B &= \dfrac{28}{12} + \dfrac{7}{3} \\
		B &= \dfrac{24}{12} + \dfrac{4}{12} - \dfrac{7}{3} \\
		B &= 2 + \dfrac{4 \times 1}{4 \times 3} - \dfrac{7}{3} \\
		B &= 2 + \dfrac{1}{3} - \dfrac{7}{3} \\
		B &= 2 - \dfrac{6}{3} \\
		B &= 2 - 2 \\
		B &= 0 \\
		\end{align*}
		
		
		
	\end{enumerate}
	
	
\end{multicols}




\section*{Exercice 83 page 82}


\begin{equation*}
	\dfrac{5}{8} \times 72 = \dfrac{5 \times 72}{8} = \dfrac{5 \times 9 \times 8}{8} = 5 \times 9 = 45
\end{equation*}

45 des 72 crêpes sont salées.

\begin{equation*}
	72 - 45 = 27
\end{equation*}

Cécile a donc préparé 27 crêpes sucrées.


\section*{Exercice 84 page 82}

Estelle veut préparer 4 quatre quarts :

\begin{equation*}
	4 \times \dfrac{1}{3} = \dfrac{4}{3} = 1 + \dfrac{1}{3}
\end{equation*}

Elle a besoin d'un paquet entier et d'un tiers d'un paquet de farine, elle en a un paquet et demie, c'est bon ($\dfrac{1}{3} < \dfrac{1}{2}$).

\begin{equation*}
4 \times \dfrac{1}{4} = \dfrac{4}{4} = 1 
\end{equation*}

Elle a besoin d'un paquet entier de sucre, elle l'a, c'est bon ($\dfrac{1}{3} < \dfrac{1}{2}$).


\begin{equation*}
4 \times \dfrac{7}{8} = \dfrac{4 \times 7}{8} = \dfrac{4 \times 7}{4 \times 2} = \dfrac{7}{2} = \dfrac{6}{2}+ \dfrac{1}{2} = 3 + \dfrac{1}{2}
\end{equation*}

Elle a besoin 3 livres et demie de beurre, elle en a seulement 3 livres, ce n'est pas assez.


\begin{equation*}
4 \times 3 = 12 
\end{equation*}

Elle a besoin d'une douzaine d'\oe ufs, c'est bon.


Elle ne pourra pas préparer tous les quatre quarts, elle n'a pas assez de beurre.

\section*{Exercice 85 page 82}
 
 L'héritage est partagé entre trois frères.
 
 \begin{equation*}
	 \dfrac{5}{18} \times \num{9000} = \dfrac{5\times \num{9000}}{18} = \dfrac{5 \times 500 \times 2 \times 9}{2 \times 9} = 5 \times 500 = \num{2500}
 \end{equation*}
  Le cadet reçoit $\dfrac{5}{18}$ du total, soit 2500 \$ .
  
  
  \vspace*{1cm}
  \begin{equation*}
		\dfrac{9}{8} \times 2500 = \dfrac{9 \times 25 \times 25 \times 4 }{2 \times 5} = \dfrac{9 \times 625}{2} = \dfrac{5625}{8} = \num{2812.5}
  \end{equation*}
  
  L'ainé reçoit $\dfrac{9}{8}$ de 2500 \$ , soit \num{2812.5} \$ .
  

  \vspace*{1cm}
    
  \begin{equation*}
  	\num{9000} - (\num{2500} + \num{2812.5}) = 9000 - \num{5312.5} = \num{3687.5}
  \end{equation*}
  
  Le dernier reçoit donc \num{3687.5} \$ .
\end{document}