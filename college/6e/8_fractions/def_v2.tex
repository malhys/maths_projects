\begin{mydef}
	$a$ et $b$ sont deux nombres ($b$ $\neq$ 0). Le \hspace{3cm} de $a$ par $b$ se note $a \div b$ ou $\dfrac{a}{b}$, en \hspace{5cm}.
\end{mydef}

\begin{myex}
	%\begin{itemize}
		%\item 
		Le quotient de 5 par 4 est $\dfrac{5}{4}$, c'est le nombre qui multiplié par 4 donne 5. 
		\begin{equation*}
			\dfrac{5}{4} \times 4 = 
		\end{equation*}

		%\item Le quotient de 2 par 3 est $\dfrac{2}{3}$, c'est le nombre qui multiplié par 3 donne 2. $\dfrac{2}{3} \times 3 = 2 $.
	%\end{itemize}
\end{myex}

\begin{mydef}
	Si $a$ et $b$ sont entiers, alors $\dfrac{a}{b}$ est une \hspace{3cm}. $a$ est le \hspace*{3cm} et $b$ est le \hspace*{3cm}.	
	
\end{mydef}

\begin{center}
	\includegraphics*[scale=0.5]{def_2}
\end{center}

\begin{myex}
	$\dfrac{\num{4.2}}{\num{2}}$, $\dfrac{\num{5}}{\num{2.4}}$, $\dfrac{\num{1.3}}{\num{3.7}}$ et $\dfrac{\num{2}}{\num{3}}$ sont toutes des écritures fractionnaires, mais seule \hspace{2cm} est une fraction.
\end{myex}