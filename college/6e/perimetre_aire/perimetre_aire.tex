\documentclass[xcolor=table]{beamer}
%\usepackage[utf8]{inputenc}
%\usetheme{Warsaw}
\usetheme{CambridgeUS}
%\usecolortheme{seahorse}

\input{../../../utils_maths_beamer}

\title{Périmètres et aires}
\author{}\institute{}


\AtBeginSubsection[]
{
	\begin{frame}
		\frametitle{}
		\tableofcontents[currentsection, currentsubsection]
	\end{frame} 
}

\begin{document}
	
	
	
\begin{frame}
	\titlepage
\end{frame}

\section{Périmètre}

\subsection{Définition}

\begin{frame}
\frametitle{ }  
\framesubtitle{ }	

\begin{exampleblock}{Définition}
	Le \bu{périmètre} d'une figure est la \underline{longueur du contour} de cette figure.
\end{exampleblock}

\begin{block}{Exemple}
	\begin{columns}[onlytextwidth]
		\begin{column}{0.465\textwidth}
			\center{\includegraphics[scale=0.55]{./img/aire}}			
		\end{column}
		\begin{column}{0.465\textwidth}
			Le périmètre de cette figure est 16 unités de longueur.			
		\end{column}
	\end{columns}
		
\end{block}	


\end{frame}

\subsection{Unité de longueur}
\begin{frame}
	\frametitle{}  
	\framesubtitle{}	
	
	\begin{exampleblock}{Définition}
		La mesure d'une \bu{longueur} dépend de l'unité choisie.
		L'unité légale de longueur est le \bu{metre} (m).		
	\end{exampleblock}
	
	\begin{alertblock}{Autres unités de longueur}
		\tiny{\begin{tabular}{|c|c|c|c|c|c|c|}
	\hline
		\rowcolor{gray} \multicolumn{3}{|c|}{\textbf{\underline{Multiples} de l'unité}} & \textbf{Unité} & \multicolumn{3}{c|}{\textbf{\underline{Sous-multiples} de l'unité}} \\
	\hline
		\textbf{Kilo}mètre & \textbf{hecto}mètre & \textbf{déca}mètre & \bu{mètre} & \textbf{déci}mètre & \textbf{centi}mètre & \textbf{milli}mètre \\
	\hline
		1 km $=$ 1 000 m & 1hm $=$ 100 m & 1 dam $=$ 10 m & 1m & 1 dm $=$ 0,1 m & 1 cm $=$ 0,01 m & 1 mm $=$ 0,001 m \\
	\hline
	
\end{tabular}}
	\end{alertblock}
		
\end{frame}

\begin{frame}
	\frametitle{}  
	\framesubtitle{}	
	
	\begin{alertblock}{Autres unités de longueur}
		\tiny{\begin{tabular}{|c|c|c|c|c|c|c|}
	\hline
		\rowcolor{gray} \multicolumn{3}{|c|}{\textbf{\underline{Multiples} de l'unité}} & \textbf{Unité} & \multicolumn{3}{c|}{\textbf{\underline{Sous-multiples} de l'unité}} \\
	\hline
		\textbf{Kilo}mètre & \textbf{hecto}mètre & \textbf{déca}mètre & \bu{mètre} & \textbf{déci}mètre & \textbf{centi}mètre & \textbf{milli}mètre \\
	\hline
		1 km $=$ 1 000 m & 1hm $=$ 100 m & 1 dam $=$ 10 m & 1m & 1 dm $=$ 0,1 m & 1 cm $=$ 0,01 m & 1 mm $=$ 0,001 m \\
	\hline
	
\end{tabular}}
	\end{alertblock}
	
	\begin{columns}[onlytextwidth]
		\begin{column}{0.64\textwidth}
			\begin{block}{Exemple}
				On veut calculer le périmètre de la figure ci-contre : 
				
				\begin{small}
					\begin{itemize}
						\item 32 mm $=$ 3,2 cm et 0,31~dm~$=$~3,1~cm
						\item[P=]  AB + BC + CD + DE + EA
						\item[P=] 2 + 3,2 + 2,2 + 3 + 3,1 
						\item[P=] 13,5
						\item[$\rightarrow$] Le périmètre du polygone ABCDE est 13,5 cm.
					\end{itemize}	
				\end{small}
				
				 
				
				
			\end{block}
		\end{column}
		\begin{column}{0.35\textwidth}
			\center{\includegraphics[scale=0.32]{./img/figure}}			
		\end{column}
	\end{columns}
	
\end{frame}

\begin{frame}
	\frametitle{Convertir les unités de longueur}  
	\framesubtitle{À l'aide du tableau de conversion}	
	
	On utilise le tableau ci-dessous :
		\begin{small}
		\begin{center}
			\begin{tabular}{|c|c|c|c|c|c|c|}	
				\hline
					\rowcolor{gray} ~~\textbf{km}~~    &    ~~\textbf{hm}~~ &    ~~\textbf{dam}~~ & ~~\textbf{m}~~ & ~~\textbf{dm}~~ & 	~~\textbf{cm}~~ & ~~\textbf{mm}~~ \\
				\hline
					& & & & & & \\
				\hline	
			\end{tabular}	
		\end{center}
		\end{small}

	
	\begin{block}{Exemple}
		On veut convertir 7,548 hm en m.
		\begin{itemize}
			\item[$\rightarrow$] On met un chiffre par case dans le tableau, en commençant par les unités du nombre de départ. Puis on place la virgule à la nouvelle unité choisie (en ajoutant des zéro si nécessaire)
		\end{itemize}
		
		\begin{small}		
		\begin{center}
			\begin{tabular}{|c|c|c|c|c|c|c|}	
				\hline
				\rowcolor{gray} ~~\textbf{km}~~    &    ~~\textbf{hm}~~ &    ~~\textbf{dam}~~ & ~~\textbf{m}~~ & ~~\textbf{dm}~~ & 	~~\textbf{cm}~~ & ~~\textbf{mm}~~ \\
				\hline
					& 7 & 5 & 4, & 8 & & \\
				\hline	
			\end{tabular}	
		\end{center}
		\end{small}
		
		CONCLUSION : 7,548 hm = 754,8 m.
	\end{block}
	
	
\end{frame}

\begin{frame}
	\frametitle{Convertir les unités de longueur}  
	\framesubtitle{En multipliant ou en divisant directement par 10; 100; 1000 ...}	
	
	\begin{alertblock}{Méthode}
		ON peut convertir directement les unités de longueur à l'aide de multiplications et de divisions par 10; 100; 1000 ...
	\end{alertblock}
	
	\begin{block}{Exemple}
		\begin{itemize}
			\item On veut convertir 32,45 m en cm.
			\item On sait que 1 m $=$ 100 cm.
			\item[$\rightarrow$] 32,45 $\times$ \textbf{100} $=$ 3 245.
		\end{itemize}
		
		Donc 32,45 m $=$ 3 245 cm.		
	\end{block}
	
\end{frame}

\section{Formules de calcul du périmètre}

\subsection{Triangles}

\begin{frame}
	\frametitle{}  
	\framesubtitle{}	
	
	%\begin{alertblock}{Règle générale}
		Le périmètre d'un triangle est égal à la \underline{somme des longueurs de ses} \underline{ trois côtés}.
	%\end{alertblock}
	
	

	\begin{columns}[onlytextwidth]
		\begin{column}{0.48\textwidth}
			\begin{alertblock}{Triangle isocèle}<2->
				\center{\includegraphics[scale=0.45]{./img/iso}}
				\begin{itemize}
					\item Périmètre = 2 x longueur des côtés égaux + longueur de la base
					\item $P = 2 \times\ EF + DF$
				\end{itemize}
				
			\end{alertblock}
		\end{column}
		\begin{column}{0.48\textwidth}
			\begin{alertblock}{Triangle équilatéral}<3->
				\center{\includegraphics[scale=0.45]{./img/equi}}
				\begin{itemize}
					\item Périmètre = 3 x longueur d'un côté
					\item $P = 3 \times\ IJ $
				\end{itemize}
				
			\end{alertblock}
		\end{column}
	\end{columns}
	
	
\end{frame}

\subsection{Quadrilatères}

\begin{frame}
	\frametitle{}  
	\framesubtitle{}	
	
	%\begin{alertblock}{Règle générale}
	Le périmètre d'un quadrilatère est égal à la \underline{somme des longueurs de ses} \underline{quatre côtés}.
	%\end{alertblock}
	
	
	
	\begin{columns}[onlytextwidth]
		\begin{column}{0.3\textwidth}
			\begin{alertblock}{Losange}<2->
				\center{\includegraphics[scale=0.45]{./img/los}}
				\begin{itemize}
					\item Périmètre = 4 x longueur d'un côté
					\item $P = 4 \times\ c$
					\item $P = 4 \times\ AB$
				\end{itemize}
				
			\end{alertblock}
		\end{column}
		\begin{column}{0.35\textwidth}
			\begin{alertblock}{Rectangle}<3->
				\center{\includegraphics[scale=0.45]{./img/rect}}
				\begin{itemize}
					\item Périmètre = 2 $\times$ (longueur $+$ largeur) ou
					\item Périmètre = 2 $\times$ longueur $+$  2 $\times$ largeur)
					\item $P = 2 \times L + 2 \times l$
					\item $P = 2 \times AB + 2 \times BC$
				\end{itemize}
				
			\end{alertblock}
		\end{column}
		\begin{column}{0.25\textwidth}
			\begin{alertblock}{Carré}<4->
				\center{\includegraphics[scale=0.45]{./img/carre}}
				\begin{itemize}
					\item Périmètre = 4 x longueur d'un côté
					\item $P = 3 \times\ c $
					\item $P = 3 \times\ AB $
				\end{itemize}
				
			\end{alertblock}
		\end{column}
	\end{columns}
\end{frame}

\subsection{Cercles}	

\begin{frame}
	\frametitle{}  
	\framesubtitle{}	
	
	\begin{columns}[onlytextwidth]
		\begin{column}{0.3\textwidth}
			\center{\includegraphics[scale=0.5]{./img/cercle1}}
			
		\end{column}
		\begin{column}{0.7\textwidth}
			La longueur d'un cercle est un multiple de son rayon (et de son diamètre).
			La longueur d'un cercle de diamètre $d$ et de rayon $r$, s'obtient avec l'une des deux formules suivantes:
			
			\begin{itemize}
				\item $P = \pi \times d$
				\item $P = 2 \times \pi \times r$
			\end{itemize}
			
			La lettre grecque $\pi$ (pi) désigne un nombre qui n'est pas décimal (On ne le connaît pas exactement).
			On prend généralement 3,14 comme valeur approchée de $\pi$ :
			\begin{itemize}
				\item[$\Rightarrow$] $ \pi \approx 3,14$
			\end{itemize}
		\end{column}
	\end{columns}
	
	\begin{block}{Exemple}
		Si $r = 3 cm$, alors : P = $2 \times \pi \times 3 = \pi \times 6 \approx 18,84$
	\end{block}
\end{frame}
	



\end{document}


\begin{frame}
	\frametitle{}  
	\framesubtitle{}	
	

\end{frame}