\documentclass[xcolor={dvipsnames}]{beamer}
%\usepackage[utf8]{inputenc}
\usetheme{Madrid}
%\usetheme{Malmoe}
\usecolortheme{beaver}
%\usecolortheme{rose}

\input{../../../../utils_maths_beamer}


\usepackage{../../../../pas-math}
\usepackage{../../../../moncours_beamer}

\usepackage{amssymb,amsmath}


\newcommand{\myitem}{\item[\textbullet]}

\graphicspath{{../img/}}

\title{Séquence 5 : Division}
\date{ }
%\author{O. FINOT}\institute{Collège S$^t$ Bernard}

%
\AtBeginSection[]
{
	\begin{frame}
		\frametitle{}
		\tableofcontents[currentsection, hideallsubsections]
	\end{frame} 

}
%
%
\AtBeginSubsection[]
{
	\begin{frame}
		\frametitle{Sommaire}
		\tableofcontents[currentsection, currentsubsection]
	\end{frame} 
}

\begin{document}



\begin{frame}
  \titlepage 
\end{frame}


\begin{frame}{}
	\begin{myobj}
	\begin{itemize}
		\item Je connais et j'utilise le vocabulaire des divisions;
		\item Je sais si un nombre est divisible par un autre ;
		\item Je sais poser et calculer la division d’un nombre entier par un autre ;
		\item Je sais  poser et calculer la division d’un nombre décimal par un nombre entier ;
		\item Je sais résoudre des problèmes en utilisant des additions, soustractions, multiplications et divisions.
		
	\end{itemize}
\end{myobj}

\begin{mycomp}
	\begin{itemize}
		%\begin{multicols}{2}
			
			
			\item \kw{Calculer } %: calculer avec des nombres rationnels, de manière exacte ou approchée, en combinant de façon appropriée le calcul mental, le calcul posé et le calcul instrumenté (calculatrice ou logiciel) ; 
			%\item \kw{Calculer (Ca2) } : contrôler la vraisemblance de ses résultats, notamment en estimant des ordres de grandeur ou en utilisant des encadrements ;
			\item \kw{Modéliser } %: traduire en langage mathématique une situation réelle (par exemple à l'aide d'équations, de fonctions, de configurations géométriques, d'outils statistiques) 
			\item \kw{Raisonner } 
			\item \kw{Représenter}
			\item \kw{Communiquer}
		%\end{multicols}
	\end{itemize}
\end{mycomp}
\end{frame}

\section{Division euclidienne}

\begin{frame}
	\begin{mydef}
		Effectuer la \kword{division euclidienne} d’un nombre entier, appelé \pause  \kword{dividende}, par un nombre entier, différent de zéro, appelé \pause  \kword{diviseur}, c’est trouver deux autres nombres entiers, le \pause \kword{quotient} et le \kword{reste}, tels que : \pause
		
		\begin{equation*}
			diviseur \times quotient + reste = dividende \pause	
		\end{equation*}
	\end{mydef}

	\begin{center}
		$\begin{array}{c|c}
			 & \hspace*{2cm} \\
			\cline{2-2}
			&  \\
			 & \\
		\end{array}$
	\end{center}

\end{frame}


\begin{frame}
	\begin{mydef}
		Effectuer la \kword{division euclidienne} d’un nombre entier, appelé   \kword{dividende}, par un nombre entier, différent de zéro, appelé   \kword{diviseur}, c’est trouver deux autres nombres entiers, le  \kword{quotient} et le \kword{reste}, tels que : 
		
		\begin{equation*}
			diviseur \times quotient + reste = dividende 
		\end{equation*}
	\end{mydef}
	
	\begin{center}
		$\begin{array}{c|c}
			Dividende & Diviseur \\
			\cline{2-2}
			& Quotient \\
			Reste & \\
		\end{array}$
	\end{center}
	
\end{frame}

\section{Multiples et diviseurs}

\subsection{Définition}

\begin{frame}
	\begin{mydefs}
	
			Quand le reste de la division euclidienne du nombre $a$ par le nombre $b$, différent de zéro, est égal à zéro, on dit que : \pause
			
			\begin{itemize}
				\item $a$ est \pause \kword{divisible} par $b$; \pause
				\item $a$ est \pause un \kword{multiple} de $b$; \pause
				\item $b$ est \pause un \kword{diviseur} de $a$. \pause
			\end{itemize}
		
	\end{mydefs}

\begin{myex}
	
	\begin{center}
		\includegraphics[scale=0.3]{../img/div} \pause
	\end{center}
	
	
	936 est divisible par 24 ; 936 est un multiple de 24 ; 24 est un diviseur de 936.\pause
\end{myex}

\begin{myexo}
	\begin{itemize}
		\item Citer 3 multiples de 24 :
		\item Citer tous les diviseurs de 16 :
	\end{itemize}
\end{myexo}

\end{frame}

\subsection{Critères de divisibilité}

\begin{frame}
	\begin{myprops}
		
		
			\begin{itemize}
				\item Un nombre entier est divisible par 2 si \pause \kword{il est pair} (son chiffre des unités est , 2, 4, 6 ou 8); \pause
				\item Un nombre entier est divisible par 5 si \pause son \kword{chiffre des unités est 0 ou 5}; \pause
				\item Un nombre entier est divisible par 10 si \pause son \kword{chiffre des unités est 0};	\pause
				
				\item Un nombre entier est divisible par 3 \pause si \kword{la somme de ses chiffres est divisible par 3}; \pause
				\item Un nombre entier est divisible par 9 si \pause \kword{la somme de ses chiffres est divisible par 9}; \pause
				
				\item Un nombre entier est divisible par 4 si \pause \kword{le nombre formé par ses chiffres des dizaines et ses unités} 
				
				\kword{ est divisible par 4}.
			\end{itemize}
		
	\end{myprops}
\end{frame}

\begin{frame}
	\begin{myexs}
		
			\begin{itemize}
				\item $1250$ est divisible par : \pause 2; 5 et 10. \pause 
				\item $726$ est divisible par : \pause 2 et 3. \pause
				\item $1024$ est divisible par : \pause 2 et 4.\pause
				\item $342$ est divisible par : \pause 2; 3 et 9.
				
			\end{itemize}
	
	\end{myexs}
\end{frame}

\end{document}