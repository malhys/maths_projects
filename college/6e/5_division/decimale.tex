\begin{mydef}
	\iftoggle{eleve}{%
		\vspace*{0.2cm}
		\hrulefill 
		
		\vspace*{0.2cm}
		\hrulefill 
		
		\vspace*{0.2cm}
		\hrulefill 
	}{%
		Effectuer la division décimale d'un nombre décimal par un nombre entier, c'est chercher le \kw{quotient}, tel que : 
		
		\begin{equation*}
			quotient \; \times \; diviseur = \; dividende
		\end{equation*}
	}
	
\end{mydef}

\subsection{Division décimale de deux entiers}

\begin{mymeth}
	\iftoggle{eleve}{%
		\vspace*{0.2cm}
		\hrulefill 
		
		\vspace*{0.2cm}
		\hrulefill 
		
		\vspace*{0.2cm}
		\hrulefill 
		
		\vspace*{0.2cm}
		\hrulefill 
	}{%
		On commence comme une division entière. Quand il n'y a plus de chiffre à abaisser, on ajoute une virgule au quotient et on abaisse des zéros jusqu'à ce que le reste soit égal à zéro (ou qu'on obtienne la valeur approchée voulue).
	}
	
\end{mymeth}

\begin{myexs}
	\begin{multicols}{2}
		\begin{center}
					$\begin{array}{rc|c}
					294 \quad & & 35 \\
					\cline{3-3}
					 &  \\
					 & & \\
					 & & \\
					\end{array}$
			%\opidiv{731}{34}
		\end{center}
		
		
		\begin{center}
			$\begin{array}{rc|c}
			732 \quad & & 5 \\
			\cline{3-3}
			&  \\
			& & \\
			& & \\
			\end{array}$
			
		\end{center}
		
	\end{multicols}
\end{myexs}


\subsection{Division décimale d'un nombre décimal par un entier}

	\begin{mymeth}
		\iftoggle{eleve}{%
			\vspace*{0.2cm}
			\hrulefill 
			
			\vspace*{0.2cm}
			\hrulefill 
			
			\vspace*{0.2cm}
			\hrulefill 
		}{%
			On commence comme pour le cas précédent, mais on met une virgule au quotient dès qu'on arrive à la virgule du diviseur.
		}
		
	\end{mymeth}

	\begin{myexs}
		\begin{multicols}{2}
			\begin{center}
				$\begin{array}{rc|c}
				\num{456.5} \quad & & 25 \\
				\cline{3-3}
				&  \\
				& & \\
				& & \\
				\end{array}$
				%\opidiv{731}{34}
			\end{center}
			
			
			\begin{center}
				$\begin{array}{rc|c}
				\num{102.4} \quad & & 20 \\
				\cline{3-3}
				&  \\
				& & \\
				& & \\
				\end{array}$
				
			\end{center}
			
		\end{multicols}
	\end{myexs}
%\subsection{Division décimale de deux nombres décimaux}
%
%\begin{myprop}
%	\iftoggle{eleve}{%
%		\vspace*{0.2cm}
%		\hrulefill 
%		
%		\vspace*{0.2cm}
%		\hrulefill 
%%		
%%		\vspace*{0.2cm}
%%		\hrulefill 
%	}{%
%		On ne change pas le quotient de deux nombres  quand on multiplie chacun d'eux par un même nombre (10, 100, ...).
%	}
%	
%\end{myprop}
%
%\begin{myex}
%	On veut diviser \num{67.85} par \num{2.3}. \\
%	
%	\iftoggle{eleve}{%
%		\vspace*{0.2cm}
%		\hrulefill 
%		
%		\vspace*{0.5cm}
%		\hrulefill 
%				
%		\vspace*{0.5cm}
%		\hrulefill 
%		
%		\vspace*{0.5cm}
%		\hrulefill 
%		
%		\vspace*{0.7cm}
%		
%		$\begin{array}{rc|c}
%			\hspace*{3cm} \quad & & \hspace*{1cm} \\
%			\cline{3-3}
%			&  \\
%			& & \\
%			& & \\
%		\end{array}$
%	}{%
%		Le diviseur est \num{2.3}, pour qu'il soit entier on le multiplie par 10 : 
%		
%		\begin{equation*}
%			\num{2.3} \times 10 = 23
%		\end{equation*}
%		
%		et on multiplie le dividende par 10, pour ne pas changer la valeur du quotient :
%		
%		\begin{equation*}
%			\num{67.85} \times 10 = \num{678.5}
%		\end{equation*}
%		
%		
%		$\begin{array}{rc|c}
%			\num{678.5} \quad & & 23 \\
%			\cline{3-3}
%			&  \\
%			& & \\
%			& & \\
%		\end{array}$
%	}
%	
%\end{myex}

\subsection{Valeur approchée}

\begin{mydefs}
	
	\iftoggle{eleve}{%
		\begin{itemize}
			\item Une \hrulefill 
			
			\vspace*{0.2cm}
			\hrulefill 
			
			\vspace*{0.2cm}
			\hrulefill 
			
			\item La \hrulefill
			
			\vspace*{0.2cm}
			\hrulefill 
			
			\vspace*{0.2cm}
			\hrulefill 
			
			\vspace*{0.2cm}
			\hrulefill 
			
			\item La \hrulefill
			
			\vspace*{0.2cm}
			\hrulefill 
			
			\vspace*{0.2cm}
			\hrulefill 
			
			\vspace*{0.2cm}
			\hrulefill
		\end{itemize}
	}{%
		\begin{itemize}
			\item Une valeur approchée d'un nombre est un nombre proche de la valeur exacte de ce nombre.
			
			\item La valeur approchée \kw{par défaut} d'un nombre a un rang donné est le nombre du rang voulu \kw{immédiatement inférieur} au nombre de départ.
			
			\item La valeur approchée \kw{par excès} d'un nombre a un rang donné est le nombre du rang voulu \kw{immédiatement supérieur} au nombre de départ.
		\end{itemize}
	}
	
	  
\end{mydefs}

\begin{myexs}
	\begin{itemize}
		\item $\num{42} < \num{42.758} < \num{43}$ :
		
		
		\iftoggle{eleve}{%
			\vspace*{0.2cm}
			\hrulefill 
			
			\vspace*{0.2cm}
			\hrulefill
		}{%
			\num{42} est la valeur approchée par défaut à l'unité près de \num{42.758}. \num{43} est la valeur approchée par excès à l'unité près.
		}
		
		
		
		\vspace*{0.5cm}
		
		\item $\num{185.2} < \num{185.254} < \num{185.3}$ :
		
		\iftoggle{eleve}{%
			\vspace*{0.2cm}
			\hrulefill 
			
			\vspace*{0.2cm}
			\hrulefill
		}{%
			\num{185.2} est la valeur approchée par défaut au dixième près de \num{185.254}. \num{185.2} est la valeur approchée par excès au dixième près.
		}
	\end{itemize}
\end{myexs}

\iftoggle{eleve}{%
	\newpage
}

\subsection{Diviser un nombre par 10 ; 100 ... ou \num{0.1} ; \num{0.01} ...}

\begin{mymeth}
	\iftoggle{eleve}{%
		\vspace*{0.2cm}
		\hrulefill 
		
		\vspace*{0.2cm}
		\hrulefill
	}{%
		Pour diviser un nombre par 10 ; 100 ou 1000 on décale la virgule vers la gauche  de 1 ; 2 ou 3 rangs. 
	}
	
	
\end{mymeth}

\begin{myexs}
	
	\vspace*{-1.2cm}
	\begin{multicols}{3}
		
		\iftoggle{eleve}{%
			\begin{equation*}
				150 \div 10 = 
			\end{equation*}
			
			\begin{equation*}
				\num{254.2} \div 100 =
			\end{equation*}
			
			\begin{equation*}
				\num{78} \div 1000 = 
			\end{equation*}
		}{%
			\begin{equation*}
				150 \div 10 = 15
			\end{equation*}
			
			\begin{equation*}
				\num{254.2} \div 100 = \num{2.542}
			\end{equation*}
			
			\begin{equation*}
				\num{78} \div 1000 = \num{0.078}
			\end{equation*}
		}
		
	\end{multicols}
\end{myexs}

\begin{mymeth}
	\iftoggle{eleve}{%
		\vspace*{0.2cm}
		\hrulefill 
		
		\vspace*{0.2cm}
		\hrulefill
	}{%
		Pour diviser un nombre par \num{0.1} ; \num{0.01} ou \num{0.001} on le multiplie par 10 ; 100 ou 1000. 
	}
	
\end{mymeth}

\begin{myexs}
	\vspace*{-1.2cm}
	\begin{multicols}{3}
		
		\iftoggle{eleve}{%
			\begin{equation*}
				\num{1.85} \div \num{0.1} = 
			\end{equation*}
			
			\begin{equation*}
				\num{254.2} \div \num{0.01} = 
			\end{equation*}
			
			\begin{equation*}
				\num{7} \div \num{0.001} = 
			\end{equation*}
		}{%
			\begin{equation*}
				\num{1.85} \div \num{0.1} = \num{18.5}
			\end{equation*}
			
			\begin{equation*}
				\num{254.2} \div \num{0.01} = \num{25420}
			\end{equation*}
			
			\begin{equation*}
				\num{7} \div \num{0.001} = \num{7000}
			\end{equation*}
		}
		
	\end{multicols}
\end{myexs}