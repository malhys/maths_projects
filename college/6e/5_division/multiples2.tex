\subsection{Définition}

\begin{mydefs}
	Quand le reste de la division euclidienne du nombre $a$ par le nombre $b$, différent de zéro, est égal à zéro, on dit que : 
	
	\begin{itemize}
		\item $a$ est un \hspace*{4cm} par $b$;
		\item $a$ est un \hspace*{4cm} de $b$;
		\item $b$ est un \hspace*{4cm} de $a$.
	\end{itemize}
	
\end{mydefs}

\begin{myex}
	\begin{multicols}{2}
		\begin{center}
			\opidiv{936}{24}
		\end{center}
		
		$24 \times 39 + 0 = 936$
	\end{multicols}

	936 est divisible par 24 ; 936 est un multiple de 24 ; 24 est un diviseur de 936.
\end{myex}


\begin{myexo}
	\begin{itemize}
		\item Citer 3 multiples de 24 :
		\item Citer tous les diviseurs de 16 :
	\end{itemize}
\end{myexo}

\subsection{Critères de divisibilité}

\begin{myprops}
	
	\begin{itemize}
		\item \item Un nombre entier est divisible par 2 si  \\
		\ %(son chiffre des unités est , 2, 4, 6 ou 8);
		\item Un nombre entier est divisible par 5 si \\
		\ 
		\item Un nombre entier est divisible par 10 si son \\
		\ 
	
		\item Un nombre entier est divisible par 3 si \\ 
		\ %\kw{la somme de ses chiffres est divisible par 3};
		\item Un nombre entier est divisible par 9 si \\ 
		\ %si \kw{la somme de ses chiffres est divisible par 9};
		
		\item Un nombre entier est divisible par 4 si \\ 
		\ %si \kw{le nombre formé par ses chiffres des dizaines et des unités est divisible par 4}.
	\end{itemize}
\end{myprops}

\begin{myexs}
	\begin{itemize}
		\item $1250$ est divisible par : %2; 4; 5 et 10. 
		\item $726$ est divisible par : %2 et 3.
		\item $1024$ est divisible par : %2 et 4.
		\item $342$ est divisible par : %2; 3 et 9.
		
	\end{itemize}
\end{myexs}


