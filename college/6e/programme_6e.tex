\subsection{Proportionnalité}\label{ch_6_proba}

\subsubsection*{Pré-requis}

Connaissance des opérations de base pour effectuer les calculs nécessaires. (Chapitres \ref{ch_6_add} et \ref{ch_6_multi})


\subsubsection*{Compétences}

\begin{enumerate}
	
	\item Reconnaître une situation de proportionnalité.
	\item Savoir raisonner dans des situations de proportionnalité
	\begin{itemize}
		\item Utiliser d'un rapport de linéarité (passage d'une colonne à une autre)
		\item Utiliser d'un coefficient de proportionnalité (passage d'une ligne à une autre) 
		\item Utiliser du passage à l'unité (règle de trois)		
	\end{itemize}
	\item Maîtriser la notion d'échelle
	\item Appliquer un taux de pourcentage
\end{enumerate}


\subsection{Nombres entiers, nombres décimaux}\label{ch_6_nombres}

%\subsubsection*{Pré-requis}

%$\varnothing$

\subsubsection*{Compétences}
\begin{enumerate}
	\item Définitions (chiffre, nombre)
	\item Décomposition d'un nombre en "tranches" (unités, dizaines etc.)
	\item \'Ecrire un nombre en toute lettres
	\item Comparer, Ordonner deux nombres
	\item Notion d'ordre croissant et décroissant
	\item Placer des nombres sur une demi-droite graduée
	\item Notion d'abscisse d'un point
	\item Arrondir un nombre décimal		
	
\end{enumerate}


\subsection{Addition et soustractions}\label{ch_6_add}

\subsubsection*{Pré-requis}

Nombres décimaux (Chapitre \ref{ch_6_nombres}).

\subsubsection*{Compétences}

\begin{enumerate}
	\item Maîtriser le vocabulaire de l'addition et de la soustraction (\textbf{somme}, \textbf{termes}, \textbf{différence})
	\item Additionner des nombres décimaux
	\item Soustraire un nombre décimal à un autre
	\item Vérifier le résultat d'un calcul avec un ordre de grandeur
	\item Maitriser les propriétés de l'addition et de la soustraction (commutative ou non)
	\item Calculer des expressions parenthésées 
	
\end{enumerate}


\subsection{Multiplication et division}\label{ch_6_multi}

\subsubsection*{Pré-requis}

Addition (Chapitre \ref{ch_6_add})

\subsubsection*{Compétences}
\begin{enumerate}
	\item Maîtriser le vocabulaire  (produit, facteur, diviseur, dividende, quotient , reste)
	\item Multiplier deux nombres décimaux
	\item Effectuer une division euclidienne
	\item Effectuer une division décimale
	\item Notion de multiple d'un nombre décimal
	\item Critères de divisibilité (par 2; 3; 4; 5; 6; 9)
	\item Choisir la bonne opération suivant le problème à résoudre
\end{enumerate}


\subsection{Fractions}\label{ch_6_frac}

\subsubsection{Pré-requis}
Opérations (Chapitres \ref{ch_6_add} et \ref{ch_6_multi})

\subsubsection*{Compétences}
\begin{enumerate}
	\item Définition (quotient, numérateur, dénominateur)
	\item Représenter une fraction sur un schéma
	\item Placer une fraction sur une demi-droite graduée
	\item Prendre une fraction d'une quantité
	\item Notion de quotients égaux
	\item Simplification de fractions
\end{enumerate}

\subsection{Droites, segments et cercles}\label{ch_6_droites}

\subsubsection*{Compétences}
\begin{enumerate}
	\item Savoir tracer une droite, une demi-droite, un segment et un cercle
	\item Être capable de tracer des droites parallèles, des droites perpendiculaires
	\item Savoir déterminer le milieu d'un segment
	\item Savoir utiliser un compas pour reporter des longueurs
	\item Apprendre à effectuer des démonstrations
\end{enumerate}

\subsection{Angles et Triangles}\label{ch_6_angles}

\subsubsection*{Pré-requis}
Segments (Ch \ref{ch_6_droites})

\subsubsection*{Compétences}

\begin{enumerate}
	\item Savoir mesurer un angle avec un rapporteur
	\item Savoir identifier un angle aigu, droit ou obtus
	\item Savoir ce qu'est la bissectrice d'un angle et comment la tracer
	\item Savoir raisonner avec les angles
	\item Savoir construire les différents types de triangle
	\item Savoir démontrer qu'un triangle est particulier
\end{enumerate}

\subsection{Quadrilatères}\label{ch_6_quad}

\subsubsection*{Pré-requis}
Angles (Ch \ref{ch_6_angles})

\subsubsection*{Compétences}

\begin{enumerate}
	\item Connaître le vocabulaire (angle, angles opposés, sommet, côté, côtés consécutifs, diagonale)
	\item Connaître les quadrilatères particuliers (losange, rectangle, carré)
	\item Savoir manier les définitions et propriétés pour reconnaître certains quadrilatères
\end{enumerate}

\subsection{Symétrie axiale}\label{ch_6_sym}

\subsubsection*{Pré-requis}
 Droites et angles (Ch \ref{ch_6_droites} et \ref{ch_6_angles})
\begin{enumerate}
	\item Savoir tracer le symétrique 
		\begin{itemize}
			\item d'un point
			\item d'un segment
			\item d'une droite
			\item d'un angle
			\item d'une figure
			\item d'un cercle
		\end{itemize}
	\item Connaître les propriétés de la symétrie axiale
	\begin{itemize}
		\item conservation des longueurs
		\item conservation des angles
		%\item conservations du périmètre
		%\item conservation de l'aire
	\end{itemize}
\end{enumerate}

\subsection{Axe de symétrie d'une figure}\label{ch_6_axe}

\subsubsection*{Pré-requis}

Symétrie axiale (ch \ref{ch_6_sym})

\subsection*{Compétences}

\begin{enumerate}
	\item Savoir ce qu'est un axe de symétrie d'une figure
	\item Connaître les axes de symétrie des figures usuelles
	\item Connaître les propriétés dues aux axes de symétrie des figures usuelles
\end{enumerate}

\subsection{Parallélépipède rectangle}\label{ch_6_pave}

\subsubsection*{Pré-requis}

Quadrilatères, périmètres et aire (ch \ref{ch_6_quad} et \ref{ch_6_peri})

\subsubsection*{Compétences}

\begin{enumerate}
	\item Savoir reconnaître un parallélépipède rectangle
	\item Savoir représenter un parallélépipède rectangle en perspective cavalière  
	\item Savoir construire un parallélépipède rectangle
	\item Connaître les unités de volume
	\item Savoir calculer le volume d'un parallélépipède rectangle

\end{enumerate}

\subsection{Périmètre et aire}\label{ch_6_peri}

\subsubsection*{Pré-requis}
 Triangles, quadrilatères, cercles (ch \ref{ch_6_angles}, \ref{ch_6_quad} et \ref{ch_6_angles})
 
 \subsubsection*{Compétences}
\begin{enumerate}
	\item Savoir utiliser les unités de longueur et d'aire
	\item Savoir calculer la circonférence d'un cercle
	\item Être capable de différencier les notions de périmètre et d'aire
	\item Savoir calculer le périmètre et l'aire d'une figure simple
\end{enumerate}

\subsection{Organiser des données}\label{ch_6_data}

\subsubsection*{Pré-requis}
 Nombres (ch \ref{ch_6_nombres})
 
 \subsubsection*{Compétences}
\begin{enumerate}
	\item Être capable de lire et de dresser des tableaux
	\item Savoir lire des 
	\begin{itemize}
		\item diagrammes en bâtons
		\item diagrammes circulaires
		\item graphiques cartésiens
	\end{itemize}
	\item Connaître les unités de temps et de masse
\end{enumerate}