\subsection{Proportionnalité}

\subsubsection*{Pré-requis}
Connaissance des opérations de base pour effectuer les calculs nécessaires.

\subsubsection*{Compétences}

\begin{itemize}
	\item Représenter des données dans un tableau.
	\item Reconnaître une situation de proportionnalité.
		\begin{itemize}
			\item Utiliser d'un rapport de linéarité
			\item Utiliser d'un coefficient de proportionnalité
			\item Utiliser du passage à l'unité (règle de trois)		
		\end{itemize}
	\item Appliquer un taux de pourcentage
\end{itemize}


\subsection{Nombres entiers, nombres décimaux}

\begin{itemize}
	\item Définition
	\item Comparer, Ordonner deux nombres
	\item Placer des nombres sur une demi-droite graduée
	\item Encadrer un nombre, le placer entre deux autres
	\item Valeur approchée décimale
\end{itemize}


\subsection{Addition et soustractions}

\begin{itemize}
	\item Vocabulaire (terme)
	\item Poser une addition
	\item Poser une soustraction
	\item Calculer des expressions parenthésées 
	\item Ordre de grandeur
\end{itemize}


\subsection{Multiplication et division}

\begin{itemize}
	\item Vocabulaire (facteur, diviseur, dividende, quotient , reste)
	\item Poser une multiplication
	\item Poser une division
	\item Critères de divisibilité (par 2; 3; 4; 5; 6; 9)
	\item Choisir le bon opérateur suivant la situation
\end{itemize}


\subsection{Fractions}

\begin{itemize}
	\item Définition
	\item Représenter une fraction sur un schéma
	\item Placer une fraction sur une demi-droite graduée
	\item Prendre une fraction d'une quantité
	\item Notion de quotients égaux
	\item Simplification de fractions
\end{itemize}

\subsection{Droites, segments et cercles}

\begin{itemize}
	\item Définitions d'une droite, d'une demi-droite, d'un segment
	\item Perpendicularité et parallélisme
	\item Notations
	\item Cercle
		\begin{itemize}
			\item Définition
			\item Propriété
		\end{itemize}
\end{itemize}

\subsection{Angles et Triangles}

\begin{itemize}
	\item Définition
	\item Mesurer un angle
	\item Utilisation du rapporteur
	\item Bissectrice d'un angle (Définition, Construction)
	\item Triangle définition
	\item Propriétés
\end{itemize}

\subsection{Quadrilatères}
\begin{itemize}
	\item Définition
	\item Propriétés
	\item Droites remarquables (diagonales)
\end{itemize}

\subsection{Symétrie axiale}

\begin{itemize}
	\item Définition
	\item Symétrique d'un point
	\item Symétrique d'une figure
	\item Propriétés
\end{itemize}

\subsection{Axe de symétrie d'une figure}

\begin{itemize}
	\item Définition
	\item Exemple
\end{itemize}

\subsection{Parallélépipède rectangle}

\begin{itemize}
	\item Perspective
	\item Définition
	\item Construction / Patron
\end{itemize}

\subsection{Périmètre et aire}

\begin{itemize}
	\item Définition de périmètre
	\item Unités de longueur
	\item Calculer un périmètre
	\item Définition Aire
	\item Unité d'aire
\end{itemize}

\subsection{Organiser des données}

\begin{itemize}
	\item Utilisation de tableaux
	\item Diagrammes
	\item Unités de temps
	\item Unités de masse	
\end{itemize}