\subsection{Proportionnalité}\label{ch_6_proba}

\subsubsection*{Pré-requis}
Connaissance des opérations de base pour effectuer les calculs nécessaires.

\subsubsection*{Compétences}

\begin{enumerate}

	\item Reconnaître une situation de proportionnalité.

		\begin{enumerate}
			\item Utiliser d'un rapport de linéarité
			\item Utiliser d'un coefficient de proportionnalité
			\item Utiliser du passage à l'unité (règle de trois)		
		\end{enumerate}
	\item Appliquer un taux de pourcentage
\end{enumerate}


\subsection{Nombres entiers, nombres décimaux}\label{ch_6_nombres}

\subsubsection*{Pré-requis}

$\varnothing$

\subsubsection*{Compétences}
\begin{enumerate}
	\item Définitions
		\begin{enumerate}
			\item Différence entre chiffres et nombres
			\item Décomposition d'un nombre en "tranches"
			\item \'Ecrire un nombre en toute lettres
		\end{enumerate}
	\item Comparer, Ordonner deux nombres
		\begin{enumerate}
			\item Déterminer si deux nombres sont égaux ou si 'lun est supérieur à l'autre
			\item Notion d'ordre croissant et décroissant
		\end{enumerate}
	\item Placer des nombres sur une demi-droite graduée
		\begin{enumerate}
			\item Demi-droite graduée définie par son origine et son unité (longueur reportée régulièrement)
			\item Notion d'abscisse d'un point
		\end{enumerate}
	%\item Encadrer un nombre, le placer entre deux autres
	\item Valeur approchée décimale
		\begin{enumerate}
			\item Troncature
			\item Arrondi
		\end{enumerate}
	
\end{enumerate}


\subsection{Addition et soustractions}\label{ch_6_add}

\subsubsection*{Pré-requis}

Bonne connaissance des nombres.


\begin{enumerate}
	\item Vocabulaire (terme)
	\item Poser une addition
	\item Poser une soustraction
	\item Calculer des expressions parenthésées 
	\item Ordre de grandeur
\end{enumerate}


\subsection{Multiplication et division}\label{ch_6_multi}

\begin{enumerate}
	\item Vocabulaire (facteur, diviseur, dividende, quotient , reste)
	\item Poser une multiplication
	\item Poser une division
	\item Critères de divisibilité (par 2; 3; 4; 5; 6; 9)
	\item Choisir le bon opérateur suivant la situation
\end{enumerate}


\subsection{Fractions}\label{ch_6_frac}

\begin{enumerate}
	\item Définition
	\item Représenter une fraction sur un schéma
	\item Placer une fraction sur une demi-droite graduée
	\item Prendre une fraction d'une quantité
	\item Notion de quotients égaux
	\item Simplification de fractions
\end{enumerate}

\subsection{Droites, segments et cercles}\label{ch_6_droites}

\begin{enumerate}
	\item Définitions d'une droite, d'une demi-droite, d'un segment
	\item Perpendicularité et parallélisme
	\item Notations
	\item Cercle
		\begin{enumerate}
			\item Définition
			\item Propriété
		\end{enumerate}
\end{enumerate}

\subsection{Angles et Triangles}\label{ch_6_angles}

\begin{enumerate}
	\item Définition
	\item Mesurer un angle
	\item Utilisation du rapporteur
	\item Bissectrice d'un angle (Définition, Construction)
	\item Triangle définition
	\item Propriétés
\end{enumerate}

\subsection{Quadrilatères}\label{ch_6_quad}
\begin{enumerate}
	\item Définition
	\item Propriétés
	\item Droites remarquables (diagonales)
\end{enumerate}

\subsection{Symétrie axiale}\label{ch_6_sym}

\begin{enumerate}
	\item Définition
	\item Symétrique d'un point
	\item Symétrique d'une figure
	\item Propriétés
\end{enumerate}

\subsection{Axe de symétrie d'une figure}\label{ch_6_axe}

\begin{enumerate}
	\item Définition
	\item Exemple
\end{enumerate}

\subsection{Parallélépipède rectangle}\label{ch_6_pave}

\begin{enumerate}
	\item Perspective
	\item Définition
	\item Construction / Patron
\end{enumerate}

\subsection{Périmètre et aire}\label{ch_6_peri}

\begin{enumerate}
	\item Définition de périmètre
	\item Unités de longueur
	\item Calculer un périmètre
	\item Définition Aire
	\item Unité d'aire
\end{enumerate}

\subsection{Organiser des données}\label{ch_6_data}

\begin{enumerate}
	\item Utilisation de tableaux
	\item Diagrammes
	\item Unités de temps
	\item Unités de masse	
\end{enumerate}