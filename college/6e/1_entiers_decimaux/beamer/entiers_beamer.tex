\documentclass[xcolor={dvipsnames}]{beamer}
%\usepackage[utf8]{inputenc}
\usetheme{Madrid}
%\usetheme{Malmoe}
\usecolortheme{beaver}
%\usecolortheme{rose}

\input{../../../../utils_maths_beamer}


\usepackage{../../../../pas-math}
\usepackage{../../../../moncours_beamer}

\usepackage{amssymb,amsmath}


\newcommand{\myitem}{\item[\textbullet]}

\graphicspath{{../img/}}

\title{Chapitre 1 : Nombre entiers et décimaux}
%\author{O. FINOT}\institute{Collège S$^t$ Bernard}

%
\AtBeginSection[]
{
	\begin{frame}
		\frametitle{}
		\tableofcontents[currentsection, hideallsubsections]
	\end{frame} 

}
%
%
%\AtBeginSubsection[]
%{
%	\begin{frame}
%		\frametitle{Sommaire}
%		\tableofcontents[currentsection, currentsubsection]
%	\end{frame} 
%}

\begin{document}



\begin{frame}
  \titlepage 
\end{frame}


	
\begin{frame}
	\begin{block}{Objectifs}
		\begin{itemize}
			\item Savoir placer les chiffres d'un nombre jusqu'au milliard et à la 4$^{ème}$ décimale;
			\item Savoir multiplier et diviser un nombre par 10, 100 et 1000;
			\item Connaître les fractions décimales.		
		\end{itemize}
	\end{block}
\end{frame}


\section{\'Ecrire un nombre}




\begin{frame}{}

	\begin{block}{Activité 1 Différentes écritures d'un nombre}
		\begin{enumerate}\pause
			\item 
			\begin{itemize}
				\item 17 et 42 sont des nombres à 2 chiffres;\pause
				\item 128 et 512 sont des nombres à 3 chiffres;\pause
				\item \num{2048} et \num{4096} sont des nombres à 4 chiffres;\pause
				\item \num{16384} et \num{65536} sont des nombres à 5 chiffres.\pause
			\end{itemize}
			
			%\item \'Ecrire les nombres suivants en toutes lettres : \num{32}, \num{128} et \num{1024}. 
			\item Le nombre 25146041337 s'écrit \num{25146041337}.\pause
			\item 
			\begin{itemize}
				\item 3,4 milliers s'écrit \num{3400};\pause
				\item 144,8 millions s'écrit \num{144800000};\pause
				\item 163 milliards s'écrit \num{163000000000}.\pause
			\end{itemize}
			
			\item .
			\begin{itemize}
				\item Cinq cent un millions six cent vingt-deux mille sept cent trente et un s'écrit \num{501622731};\pause
				\item Cinq cent millions s'écrit \num{500000000}.\pause
			\end{itemize}
			
		\end{enumerate}
	\end{block}
\end{frame}

\begin{frame}
	\begin{alertblock}{Définitions}
		\begin{itemize}\pause
			\item Il existe 10 \kword{chiffres} : 0, 1, 2, 3, 4, 5, 6, 7, 8 et 9.\pause
			
			\item On utilise les chiffres pour écrire des \kword{nombres}.\pause
			
			\item Un \kword{nombre décimal} possède une \kword{partie entière} (avant la virgule) et une \kword{partie décimale} (après la virgule).\pause
			
			\item Un nombre décimal où la partie décimale ne contient que des zéros est un \kword{nombre entier}. Dans ce cas la partie décimale n'apparait pas.\pause
		\end{itemize}
	\end{alertblock}

		\begin{center}
			\includegraphics[scale=.8]{tab}
		\end{center}
\end{frame}

\begin{frame}
	\begin{exampleblock}{Exemples}
		\begin{itemize}
			\item Le nombre \num{2048} est un nombre entier composé de 4 chiffres différents.
			\item La partie entière de \num{5239.67}  est \num{5239} et sa partie décimale est 67.	
			
			\item Le nombre \num{124} peut aussi s'écrire \num{124.00}.
		\end{itemize}
	\end{exampleblock}
\end{frame}
\end{document}