\documentclass[12pt,a4paper]{article}

\usepackage[in, plain]{fullpage}
\usepackage{array}
\usepackage{../../../pas-math}
\usepackage{../../../moncours}


%\usepackage{pas-cours}
%-------------------------------------------------------------------------------
%          -Packages nécessaires pour écrire en Français et en UTF8-
%-------------------------------------------------------------------------------
\usepackage[utf8]{inputenc}
\usepackage[frenchb]{babel}
\usepackage[T1]{fontenc}
\usepackage{lmodern}
%-------------------------------------------------------------------------------

%-------------------------------------------------------------------------------
%                          -Outils de mise en forme-
%-------------------------------------------------------------------------------
\usepackage{hyperref}
\hypersetup{pdfstartview=XYZ}
\usepackage{enumerate}
\usepackage{graphicx}
\usepackage{multicol}

\usepackage{anysize} %%pour pouvoir mettre les marges qu'on veut
%\marginsize{2.5cm}{2.5cm}{2.5cm}{2.5cm}

\usepackage{indentfirst} %%pour que les premier paragraphes soient aussi indentés
%-------------------------------------------------------------------------------


%-------------------------------------------------------------------------------
%                  -Nécessaires pour écrire des mathématiques-
%-------------------------------------------------------------------------------
\usepackage{amsfonts}
\usepackage{amssymb}
\usepackage{amsmath}
\usepackage{amsthm}
\usepackage{tikz}
%-------------------------------------------------------------------------------

%-------------------------------------------------------------------------------
%                     -Mise en forme d'exercices-
%-------------------------------------------------------------------------------
\newtheoremstyle{exostyle}
{\topsep}% espace avant
{\topsep}% espace apres
{}% Police utilisee par le style de thm
{}% Indentation (vide = aucune, \parindent = indentation paragraphe)
{\bfseries}% Police du titre de thm
{.}% Signe de ponctuation apres le titre du thm
{ }% Espace apres le titre du thm (\newline = linebreak)
{\thmname{#1}\thmnumber{ #2}\thmnote{. \normalfont{\textit{#3}}}}% composants du titre du thm : \thmname = nom du thm, \thmnumber = numéro du thm, \thmnote = sous-titre du thm

\theoremstyle{exostyle}
\newtheorem{exercice}{Exercice}

\newenvironment{questions}{
\begin{enumerate}[\hspace{12pt}\bfseries\itshape a.]}{\end{enumerate}
} %mettre un 1 à la place du a si on veut des numéros au lieu de lettres pour les questions 
%-------------------------------------------------------------------------------



%-------------------------------------------------------------------------------
%                    - Racourcis d'écriture -
%-------------------------------------------------------------------------------

% Angles orientés (couples de vecteurs)
\newcommand{\aopp}[2]{(\vec{#1}, \vec{#2})} %Les deuc vecteurs sont positifs
\newcommand{\aopn}[2]{(\vec{#1}, -\vec{#2})} %Le second vecteur est négatif
\newcommand{\aonp}[2]{(-\vec{#1}, \vec{#2})} %Le premier vecteur est négatif
\newcommand{\aonn}[2]{(-\vec{#1}, -\vec{#2})} %Les deux vecteurs sont négatifs

%Ensembles mathématiques
\newcommand{\naturels}{\mathbb{N}} %Nombres naturels
\newcommand{\relatifs}{\mathbb{Z}} %Nombres relatifs
\newcommand{\rationnels}{\mathbb{Q}} %Nombres rationnels
\newcommand{\reels}{\mathbb{R}} %Nombres réels
\newcommand{\complexes}{\mathbb{C}} %Nombres complexes
%-------------------------------------------------------------------------------




%\makeatletter
%\renewcommand*{\@seccntformat}[1]{\csname the#1\endcsname\hspace{0.1cm}}
%\makeatother


%\author{Olivier FINOT}
\date{}
\title{}

\lhead{CH1 : Nombres entiers et décimaux}
\rhead{O. FINOT}
%
%\rfoot{Page \thepage}
\begin{document}
%\maketitle

\chap[num=1, color=red]{Nombres entiers et décimaux}{Olivier FINOT, \today }

\begin{myobj}
	\begin{itemize}
		\item Reconnaître un segment, une demie-droite, une droite et savoir les tracer;
		\item Tracer avec l’équerre la droite perpendiculaire à une droite donnée passant par un point donné;
		\item Tracer avec la règle et l’équerre la droite parallèle à une droite donnée passant par un point donné;
		\item Déterminer la distance entre deux points, entre un point et une droite;
		\item Savoir coder et lire une figure.
	\end{itemize}
\end{myobj}

\begin{mycomp}
	\begin{itemize}
		\item \kw{Modéliser} 
		\item \kw{Représenter} 
		\item \kw{Raisonner} 
		\item \kw{Communiquer}
		
	\end{itemize}
\end{mycomp}

\section{\'Ecrire un nombre}

\begin{myact}{1 Différentes écritures des nombres}

	\label{act:nbres}
	
	\begin{enumerate}
		\item Donner deux nombres à 2, 3, 4 et 5 chiffres.
		
		%\item \'Ecrire les nombres suivants en toutes lettres : \num{32}, \num{128} et \num{1024}. 
		\item \'Ecrire  le nombre 25146041337 en séparant les classes.
		\item La planète Mars a un rayon d'environ \textbf{3,4 milliers} de km, une superficie d'environ \textbf{144,8 millions} de $km^2$ et un volume d'environ \textbf{163 milliards} de $km^3$.
		
		\'Ecrire les trois nombres en gras en utilisant que des chiffres.
		
		\item Lire le texte ci-dessous, puis écrire en chiffres les nombres en gras.
		
			\begin{center}
				\includegraphics[scale=1.1]{img/act1}
			\end{center}
	\end{enumerate}
\end{myact}

\begin{myactrep}{1 Différentes écritures des nombres}
	
	\begin{enumerate}
		\item 
			\begin{itemize}
				\item 17 et 42 sont des nombres à 2 chiffres;
				\item 128 et 512 sont des nombres à 3 chiffres;
				\item \num{2048} et \num{4096} sont des nombres à 4 chiffres;
				\item \num{16384} et \num{65536} sont des nombres à 5 chiffres.
			\end{itemize}
		
		%\item \'Ecrire les nombres suivants en toutes lettres : \num{32}, \num{128} et \num{1024}. 
		\item Le nombre 25146041337 s'écrit \num{25146041337}.
		\item 
			\begin{itemize}
				\item 3,4 milliers s'écrit \num{3400};
				\item 144,8 millions s'écrit \num{144800000};
				\item 163 milliards s'écrit \num{163000000000}.
			\end{itemize}
		
		\item .
			\begin{itemize}
				\item Cinq cent un millions six cent vingt-deux mille sept cent trente et un s'écrit \num{501622731};
				\item Cinq cent millions s'écrit \num{500000000}.
			\end{itemize}

	\end{enumerate}
\end{myactrep}


\begin{mydefs}
	\begin{itemize}
		\item Il existe 10 \kw{chiffres} : 0, 1, 2, 3, 4, 5, 6, 7, 8 et 9.
		
		\item On utilise les chiffres pour écrire des \kw{nombres}.
		
		\item Un \kw{nombre décimal} possède une \kw{partie entière} (avant la virgule) et une \kw{partie décimale} (après la virgule).
		
		\item Un nombre décimal où la partie décimale ne contient que des zéros est un \kw{nombre entier}. Dans ce cas la partie décimale n'apparait pas.
	\end{itemize}
\end{mydefs}

\begin{center}
	\includegraphics[scale=1]{img/tab}
\end{center}

\begin{myexs}
	\begin{itemize}
		\item Le nombre \num{2048} est un nombre entier composé de 4 chiffres différents.
		\item La partie entière de \num{5239.67}  est \num{5239} et sa partie décimale est 67.	
		
		\item Le nombre \num{124} peut aussi s'écrire \num{124.00}.
	\end{itemize}
\end{myexs}

\begin{myexos}
	\begin{itemize}
		\item Exercices 1 et 2 page 16 : identifier les chiffres d'un rang donné;
		\item \Exo{4}{16} : décomposition d'un nombre;
		\item \Exo{7}{16} : \'ecriture décimale d'un nombre donné en toutes lettres;
		\item Exercices 8 ,9 et 13 page 17 : problèmes identifier des nombres selon des critères donnés.
		\item \Exo{11}{17} : regroupement des chiffres d'un nombre en classes.
		\item \Exo{14}{17} : enlever les zéros inutiles.
	\end{itemize}
	
\end{myexos}

\section{Nombres et classement}

\begin{myact}{2 Classement}
	Activité 3 page 13
\end{myact}

\begin{myactrep}{2 Classement}
	\begin{enumerate}
		\item Le colis le plus lourd est celui qui a une masse de \num{15.3} kg et \num{13.999} kg pour le plus léger.
		\item On a donc :
		
			\num{13.999} < \num{14.15} < \num{14.509} < \num{14.575} < \num{14.59} <  \num{14.805} < \num{15.29} < \num{15.3}
	\end{enumerate}
\end{myactrep}

\begin{mydefs}
	\begin{itemize}
		\item \kw{Comparer} des nombres, c'est dire si un est plus petit ou plus grand que l'autre ou s'ils sont égaux.
		
		\item Ranger des nombres du plus petit au plus grand, c'est les classer par \kw{ordre croissant}.
		
		\item Ranger des nombres du plus grand au plus petit, c'est les classer par \kw{ordre décroissant}.
		
		\item \kw{Encadrer} un nombre, c'est trouver un nombre plus petit \textbf{et} un nombre plus grand que ce nombre.
		
		\item \kw{Intercaler} un nombre entre deux autres, c'est un nombre compris entre ces deux nombres.
	\end{itemize}
\end{mydefs}

\begin{myexs}
	\begin{itemize}
		\item 42 < 128, \pause se lit <<42 est inférieur à (ou plus petit que) 128>>;\pause
		\item 1337 < 1024,\pause se lit <<\num{1337} est supérieur à (ou plus grand que) \num{1024}>>;\pause
		\item 2 < \num{3.2} < \num{6.4} < \num{25.6} : ces nombres sont rangés dans l'ordre \pause croissant;\pause
		\item 123 > \num{45.6} > \num{7.89} > \num{5} : ces nombres sont rangés dans l'ordre \pause décroissant;\pause
		\item Encadrement de 21 à l'unité près : \pause 20 < 21 < 22 ;\pause
		\item Encadrement de \num{21.987} au centième près : \pause \num{21.977} < \num{21.987} < \num{21.997} ;\pause
	\end{itemize}
\end{myexs}

\begin{myexos}
	\begin{itemize}
		\item \Exo{16}{18} : Intercaler des nombres;
		\item Exercices 17 et 18 page 18 : Comparer des nombres;
		\item \Exo{19}{18} : Ordre croissant
		\item \Exo{20}{18} : Ordre décroissant
		\item Exercices 23 et 24 page 18 : Encadrer des nombres;
		\item \Exo{25}{19} : Comparer des nombres;
		\item \Exo{26}{19} : Ordre croissant;
		\item \Exo{27}{19} : Ordre décroissant.
	\end{itemize}
\end{myexos}



\section{Multiplier et diviser par 10, 100, 1000}

\begin{mymeth}
	Pour multiplier un nombre par 10, 100 ou 1000 :
	\begin{enumerate}
		\item on repère la virgule;
		\item on la décale vers la droite d'un rang ($\times 10$) , de deux rangs ($\times 10$) ou de trois ($\times 10$);
		\item on rajoute des zéros si besoin entre le chiffre le plus à droite et la virgule.
	\end{enumerate}
\end{mymeth}

	\begin{myexs}
		\begin{multicols}{2}
			\begin{itemize}
				\item $\num{25.26} \times 10 = \num{252.6}$ 
				\item $\num{25.26} \times 100 = \num{2526.0} = \num{24526}$
				%\item $\num{245.26} \times 1000 = \num{245260}$
				\item $\num{285} \times 10 = \num{285.0} \times \num{10}= \num{2850}$ 
				\item $\num{285} \times 1000 = \num{285000}$ 
			\end{itemize}	
		\end{multicols}
		
	\end{myexs}

	\begin{mymeth}
		Pour diviser un nombre par 10, 100 ou 1000 :
		\begin{enumerate}
			\item on repère la virgule;
			\item on la décale vers la gauche d'un rang ($\times 10$) , de deux rangs ($\times 10$) ou de trois ($\times 10$);
			\item on rajoute des zéros si besoin entre la virgule et le chiffre le plus à gauche.
		\end{enumerate}
	\end{mymeth}


	\begin{myexs}
		\begin{multicols}{2}
			\begin{itemize}
				\item $\num{25.26} \div 10 = \num{2.526}$ 
				\item $\num{25.26} \div 1000 = \num{0.02526} $
				%\item $\num{245.26} \times 1000 = \num{245260}$
				\item $\num{285} \div 10 = \num{28.5} $ 
				\item $\num{285} \div 1000 = \num{0.285}$ 
			\end{itemize}	
		\end{multicols}
		
	\end{myexs}

	\begin{myprops}
		\begin{itemize}
			\item Multiplier un nombre par \num{0.1}, \num{0.01} ou \num{0.001} revient à le diviser par 10, 100 ou 1000.
			
			\item A l'inverse, diviser un nombre par \num{0.1}, \num{0.01} ou \num{0.001} revient à le multiplier par 10, 100 ou 1000.
		\end{itemize}
	\end{myprops}

	\begin{myexs}
		%\begin{multicols}{2}
			\begin{itemize}
				\item $\num{45.78} \times \num{0.1} = \num{45.78} \div \num{10} = \num{4.578}$ 
				\item $\num{45.78} \times \num{0.01} = \num{45.78} \div \num{100} = \num{0.4578}$ 
				\item $\num{45.78} \div \num{0.1} = \num{45.78} \times \num{10} = \num{457.8}$ 
				\item $\num{45.78} \div \num{0.001} = \num{45.78} \times \num{1000} = \num{45780}$ 
			\end{itemize}	
		%\end{multicols}
		
	\end{myexs}

	\begin{myexos}
		\begin{itemize}
			\item \Exo{5}{16} : Décomposition d'un nombre et multiplications par dizaines;
			\item \Exo{15}{17} : Nombre d'unités de dizaines etc. complètes
			
		\end{itemize}
	\end{myexos}
\section{Fractions décimales}

\begin{myact}{3}
	Activité 2 page 12
\end{myact}


\begin{myactrep}{2 page 12}
	\begin{enumerate}
		\item Je convertis les fractions en nombre décimal :
		
		$ \dfrac{3}{100} = 3 \div 100 = \num{0.03}$ ; $\dfrac{91}{100} = 91 \div 100 = \num{0.91}$; $\dfrac{956}{100} = 956 \div 100 = \num{9.56}$; $\dfrac{18}{1000} = 18 \div 1000 = \num{0.018}$
		
		J'additionne ces nombres aux temps et j'obtiens :
		
		\begin{tabular}{|@{\ }c@{\ }|@{\ }c@{\ }|@{\ }c@{\ }|@{\ }c@{\ }|@{\ }c@{\ }|@{\ }c@{\ }|@{\ }c@{\ }|@{\ }c@{\ }|@{\ }c@{\ }|}
			\hline
			Appel à      & Léa         & Chloé       & Djamila      & Sarah        & Marine      & Sophiane    & Cindy        & Charlotte    \\ \hline
			Temps (en s) & \num{19.98} & \num{20.03} & \num{29.690} & \num{19.893} & \num{19.91} & \num{28.56} & \num{20.018} & \num{19.935} \\ \hline
		\end{tabular}
	
	C'est donc avec Djamila qu'elle a passé le plus de temps et avec Sarah le moins.
	
		\item Je classe les appels téléphoniques du plus court au plus long :
		
		Sarah, Marine, Charlotte, Léa, Cindy, Chloé, Sophiane, Djamila.
	\end{enumerate}
\end{myactrep}
\end{document}