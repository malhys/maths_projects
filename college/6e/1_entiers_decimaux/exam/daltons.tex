\section{Les Dalton (3 points bonus)}

Les tailles des Dalton (en mètres) sont données ci dessous :

\begin{itemize}
	\item $1 + \dfrac{11}{10} + \dfrac{3}{100}$;
	\item $\dfrac{167}{100}$;
	\item $1 + \dfrac{9}{10} + \dfrac{3}{100}$;
	\item Une unité et quatre dixièmes.
\end{itemize}

\begin{questions}
	\question[3] Retrouve la taille de chaque frère et écris-la sous la forme d'un nombre décimal, en utilisant les informations suivantes :
	
	\begin{itemize}
		\item William et Joe sont plus petits qu'Averell.
		\item Jack est plus grand que William qui est plus grand que Joe.
		\item Averell est plus grand que Jack.
	\end{itemize}

	\begin{solution}
		J'écris les tailles sous forme de nombres décimaux :
		\begin{itemize}
			\item $1 + \dfrac{11}{10} + \dfrac{3}{100}$ = 1 + \num{1.1} + \num{0.03} = \num{2.13};
			\item $\dfrac{167}{100}$ = \num{1.67};
			\item $1 + \dfrac{9}{10} + \dfrac{3}{100}$ = 1 + \num{0.9} + \num{0.03} = \num{1.93};
			\item Une unité et quatre dixièmes = \num{1.40}.
		\end{itemize}
	
		Les frères Dalton classés du plus petit au plus grand sont :
		Joe (\num{1.40}  m), William (\num{1.67} m), Jack (\num{1.93} m) zt Averell (\num{2.13} m).
	\end{solution}
\end{questions}