\section{Combinaison d'un cadenas (4 points)}

%illustration phare 80 page 24

Marie à oublié la combinaison de son cadenas à 3 chiffres. Elle se souvient seulement que cette combinaison est composée des chiffres 8 ; 5 et 0.

\begin{questions}
	\question[2] \'Ecrire toutes les combinaisons possibles.
	\begin{solution}
		Les combinaisons possibles sont 850; 805; 508; 580; 058 et 085
	\end{solution}
	\question[2] \'Ecrire en toutes lettres chacun des nombres de la question précédente.
	\begin{solution}
		Ces nombres sont :
		\begin{itemize}
			\item huit-cent-cinq;
			\item huit-cent-cinq;
			\item cinq-cent-huit;
			\item cinq-cent-quatre-vingts;
			\item cinquante-huit;
			\item quatre-vingt-cinq.
		\end{itemize}
	\end{solution}
\end{questions}