
\subsection{Multiplier}

\begin{mymeth}
	Pour multiplier un nombre par 10, 100 ou 1000 :
	\begin{enumerate}
		\item on repère la virgule;
		\item on la décale vers la droite d'un rang ($\times 10$) , de deux rangs ($\times 10$) ou de trois ($\times 10$);
		\item on rajoute des zéros si besoin entre le chiffre le plus à droite et la virgule.
	\end{enumerate}
\end{mymeth}

	\begin{myexs}
		\begin{multicols}{2}
			\begin{itemize}
				\item $\num{25.26} \times 10 = \num{252.6}$ 
				\item $\num{25.26} \times 100 = \num{2526.0} = \num{24526}$
				%\item $\num{245.26} \times 1000 = \num{245260}$
				\item $\num{285} \times 10 = \num{285.0} \times \num{10}= \num{2850}$ 
				\item $\num{285} \times 1000 = \num{285000}$ 
			\end{itemize}	
		\end{multicols}
		
	\end{myexs}


\subsection{Diviser}

	\begin{mymeth}
		Pour diviser un nombre par 10, 100 ou 1000 :
		\begin{enumerate}
			\item on repère la virgule;
			\item on la décale vers la gauche d'un rang ($\times 10$) , de deux rangs ($\times 10$) ou de trois ($\times 10$);
			\item on rajoute des zéros si besoin entre la virgule et le chiffre le plus à gauche.
		\end{enumerate}
	\end{mymeth}


	\begin{myexs}
		\begin{multicols}{2}
			\begin{itemize}
				\item $\num{25.26} \div 10 = \num{2.526}$ 
				\item $\num{25.26} \div 1000 = \num{0.02526} $
				%\item $\num{245.26} \times 1000 = \num{245260}$
				\item $\num{285} \div 10 = \num{28.5} $ 
				\item $\num{285} \div 1000 = \num{0.285}$ 
			\end{itemize}	
		\end{multicols}
		
	\end{myexs}

	\begin{myprops}
		\begin{itemize}
			\item Multiplier un nombre par \num{0.1}, \num{0.01} ou \num{0.001} revient à le diviser par 10, 100 ou 1000.
			
			\item A l'inverse, diviser un nombre par \num{0.1}, \num{0.01} ou \num{0.001} revient à le multiplier par 10, 100 ou 1000.
		\end{itemize}
	\end{myprops}

	\begin{myexs}
		%\begin{multicols}{2}
			\begin{itemize}
				\item $\num{45.78} \times \num{0.1} = \num{45.78} \div \num{10} = \num{4.578}$ 
				\item $\num{45.78} \times \num{0.01} = \num{45.78} \div \num{100} = \num{0.4578}$ 
				\item $\num{45.78} \div \num{0.1} = \num{45.78} \times \num{10} = \num{457.8}$ 
				\item $\num{45.78} \div \num{0.001} = \num{45.78} \times \num{1000} = \num{45780}$ 
			\end{itemize}	
		%\end{multicols}
		
	\end{myexs}

	\begin{myexos}
		\begin{itemize}
			\item \Exo{5}{16} : Décomposition d'un nombre et multiplications par dizaines;
			\item \Exo{15}{17} : Nombre d'unités de dizaines etc. complètes
			
		\end{itemize}
	\end{myexos}

\subsection{Fractions décimales}



\begin{myact}{3}
	Activité 2 page 12
\end{myact}


\begin{myactrep}{2 page 12}
	\begin{enumerate}
		\item Je convertis les fractions en nombre décimal :
		
		$ \dfrac{3}{100} = 3 \div 100 = \num{0.03}$ ; $\dfrac{91}{100} = 91 \div 100 = \num{0.91}$; $\dfrac{956}{100} = 956 \div 100 = \num{9.56}$; $\dfrac{18}{1000} = 18 \div 1000 = \num{0.018}$
		
		J'additionne ces nombres aux temps et j'obtiens :
		
		\begin{tabular}{|@{\ }c@{\ }|@{\ }c@{\ }|@{\ }c@{\ }|@{\ }c@{\ }|@{\ }c@{\ }|@{\ }c@{\ }|@{\ }c@{\ }|@{\ }c@{\ }|@{\ }c@{\ }|}
			\hline
			Appel à      & Léa         & Chloé       & Djamila      & Sarah        & Marine      & Sophiane    & Cindy        & Charlotte    \\ \hline
			Temps (en s) & \num{19.98} & \num{20.03} & \num{29.690} & \num{19.893} & \num{19.91} & \num{28.56} & \num{20.018} & \num{19.935} \\ \hline
		\end{tabular}
		
		C'est donc avec Djamila qu'elle a passé le plus de temps et avec Sarah le moins.
		
		\item Je classe les appels téléphoniques du plus court au plus long :
		
		Sarah, Marine, Charlotte, Léa, Cindy, Chloé, Sophiane, Djamila.
	\end{enumerate}
\end{myactrep}


