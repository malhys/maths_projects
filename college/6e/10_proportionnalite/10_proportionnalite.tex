\documentclass[12pt,a4paper]{article}

\usepackage[in, plain]{fullpage}
\usepackage{array}
\usepackage{../../../pas-math}
\usepackage{../../../moncours}


%\usepackage{pas-cours}
%-------------------------------------------------------------------------------
%          -Packages nécessaires pour écrire en Français et en UTF8-
%-------------------------------------------------------------------------------
\usepackage[utf8]{inputenc}
\usepackage[frenchb]{babel}
\usepackage[T1]{fontenc}
\usepackage{lmodern}
%-------------------------------------------------------------------------------

%-------------------------------------------------------------------------------
%                          -Outils de mise en forme-
%-------------------------------------------------------------------------------
\usepackage{hyperref}
\hypersetup{pdfstartview=XYZ}
\usepackage{enumerate}
\usepackage{graphicx}
\usepackage{multicol}

\usepackage{anysize} %%pour pouvoir mettre les marges qu'on veut
%\marginsize{2.5cm}{2.5cm}{2.5cm}{2.5cm}

\usepackage{indentfirst} %%pour que les premier paragraphes soient aussi indentés
%-------------------------------------------------------------------------------


%-------------------------------------------------------------------------------
%                  -Nécessaires pour écrire des mathématiques-
%-------------------------------------------------------------------------------
\usepackage{amsfonts}
\usepackage{amssymb}
\usepackage{amsmath}
\usepackage{amsthm}
\usepackage{tikz}
%-------------------------------------------------------------------------------

%-------------------------------------------------------------------------------
%                     -Mise en forme d'exercices-
%-------------------------------------------------------------------------------
\newtheoremstyle{exostyle}
{\topsep}% espace avant
{\topsep}% espace apres
{}% Police utilisee par le style de thm
{}% Indentation (vide = aucune, \parindent = indentation paragraphe)
{\bfseries}% Police du titre de thm
{.}% Signe de ponctuation apres le titre du thm
{ }% Espace apres le titre du thm (\newline = linebreak)
{\thmname{#1}\thmnumber{ #2}\thmnote{. \normalfont{\textit{#3}}}}% composants du titre du thm : \thmname = nom du thm, \thmnumber = numéro du thm, \thmnote = sous-titre du thm

\theoremstyle{exostyle}
\newtheorem{exercice}{Exercice}

\newenvironment{questions}{
\begin{enumerate}[\hspace{12pt}\bfseries\itshape a.]}{\end{enumerate}
} %mettre un 1 à la place du a si on veut des numéros au lieu de lettres pour les questions 
%-------------------------------------------------------------------------------



%-------------------------------------------------------------------------------
%                    - Racourcis d'écriture -
%-------------------------------------------------------------------------------

% Angles orientés (couples de vecteurs)
\newcommand{\aopp}[2]{(\vec{#1}, \vec{#2})} %Les deuc vecteurs sont positifs
\newcommand{\aopn}[2]{(\vec{#1}, -\vec{#2})} %Le second vecteur est négatif
\newcommand{\aonp}[2]{(-\vec{#1}, \vec{#2})} %Le premier vecteur est négatif
\newcommand{\aonn}[2]{(-\vec{#1}, -\vec{#2})} %Les deux vecteurs sont négatifs

%Ensembles mathématiques
\newcommand{\naturels}{\mathbb{N}} %Nombres naturels
\newcommand{\relatifs}{\mathbb{Z}} %Nombres relatifs
\newcommand{\rationnels}{\mathbb{Q}} %Nombres rationnels
\newcommand{\reels}{\mathbb{R}} %Nombres réels
\newcommand{\complexes}{\mathbb{C}} %Nombres complexes
%-------------------------------------------------------------------------------




%\makeatletter
%\renewcommand*{\@seccntformat}[1]{\csname the#1\endcsname\hspace{0.1cm}}
%\makeatother


%\author{Olivier FINOT}
\date{}
\title{}

\graphicspath{{./img/}}

\lhead{Seq 3: Fractions}
\rhead{O. FINOT}
%
%\rfoot{Page \thepage}
\begin{document}
%\maketitle




\chap[num=10, color=red]{Proportionnalité}{}

\begin{myobj}
	\begin{itemize}
		\item Reconnaître un segment, une demie-droite, une droite et savoir les tracer;
		\item Tracer avec l’équerre la droite perpendiculaire à une droite donnée passant par un point donné;
		\item Tracer avec la règle et l’équerre la droite parallèle à une droite donnée passant par un point donné;
		\item Déterminer la distance entre deux points, entre un point et une droite;
		\item Savoir coder et lire une figure.
	\end{itemize}
\end{myobj}

\begin{mycomp}
	\begin{itemize}
		\item \kw{Modéliser} 
		\item \kw{Représenter} 
		\item \kw{Raisonner} 
		\item \kw{Communiquer}
		
	\end{itemize}
\end{mycomp}


\section{Grandeurs proportionnelles}

\begin{mydef}
	Deux grandeurs sont \kw{proportionnelles} lorsqu'on peut calculer les valeurs de l'une en multipliant les valeurs de l'autre par un même nombre non nul.
	
	ce nombre est appelé \kw{coefficient de proportionnalité}.
\end{mydef}


\begin{mymeth}
	Pour identifier une situation de proportionnalité, on calcule les quotients des nombres de la seconde ligne par les nombres de la première ligne.
\end{mymeth}

\begin{myex}
	
		On s'intéresse à la distance parcourue à vélo par Aurélie pendant trois jours.
		
		\begin{center}
			\includegraphics[scale=0.5]{tab1}
		\end{center}
	

	$42 \div 2 = 63 \div 3 = 105 \div 5 = 21$, ici les grandeurs <<temps>> et <<distance parcourue>> sont proportionnelles. Chaque heure elle parcoure 21 km, 21 est le coefficient de proportionnalité.

\end{myex}	

\begin{myex}	
		Dans ce tableau on a reporté le nombre de cotés de certains polygones et leur nombre de diagonales.
		
		\begin{center}
			\includegraphics[scale=0.5]{tab2}
		\end{center}
	
		$2 \div 4 = \num{0.5}$, $5 \div 5 = 1$, donc le nombre de côtés d'un polygone n'est pas proportionnel à son nombre de diagonales.	
	
\end{myex}

\section{Compléter un tableau de proportionnalité}

\begin{myex}
	On veut remplir le tableau de proportionnalité suivant :
	
	\begin{center}
		\includegraphics[scale=0.5]{tab3_1}
	\end{center}
\end{myex}


\subsection{Par passage à l'unité}


\begin{mymeth}
	En 4 heures, nous parcourons 10 km.
	
	En 1 heure, nous parcourrons donc 4 fois moins de distance à savoir $10 \div 4 = \num{2.5}$ km.
	
	En 6 heures, nous parcourrons donc 6 fois plus de temps qu’en 1 heure à savoir $\num{2.5} \times 6 = 15 $km.
	
	En résumé :
	
	
	\begin{center}
		\includegraphics[scale=0.5]{tab3_2}
	\end{center}
\end{mymeth}

\subsection{Avec le coefficient multiplicateur}

\begin{mymeth}
	On cherche par quel nombre on multiplie 4 pour obtenir 10. $4 \times ...= 10$.
	
	C’est le nombre \num{2.5} ($10 \div 4$). $6 \times \num{2.5} = 15$.
	
	
	\begin{center}
		\includegraphics[scale=0.5]{tab3_3}
	\end{center}
\end{mymeth}


\subsection{En utilisant les propriétés de la proportionnalité}

\begin{myprop}
	Dans un tableau de proportionnalité, on peut :
		\begin{itemize}
			\item multiplier/diviser une colonne par un nombre;
			\item ajouter/soustraire des colonnes entre elles.
		\end{itemize}
	
	\begin{center}
		\includegraphics[scale=0.5]{tab3_4}
	\end{center}
\end{myprop}

\newpage

\section{Pourcentages}

%\subsection{Appliquer un pourcentage}

\begin{mydef}
	Un pourcentage traduit une situation de proportionnalité. 

	Un pourcentage est une proportion exprimée sur un total de 100 (de dénominateur égal à 100).
	
\end{mydef}

\begin{myex}
	<<Dans une confiture, il y a 60 \% de fruits>>
	\begin{itemize}
		\item La masse de fruits est proportionnelle à la masse totale de confiture.
		\item[$\Rightarrow$] Il y a 60g de fruits pour 100g de confiture.
	\end{itemize}
\end{myex}


\begin{myprop}
	$P$ est un nombre positif.
	
	Pour calculer $P\% $ d'une quantité, on multiplie cette quantité par $\frac{P}{100}$.
\end{myprop}


\begin{myex}
	Calculer $20 \% $ de 50 revient à multiplier 50 par $\frac{20}{100}$ :
	
	\begin{equation*}
		50 \times \dfrac{20}{100} = 50 \times \num{0.2} = 10
	\end{equation*}
	
	
	$20 \% $ de 50  vaut 10.
\end{myex}

%\subsection{Calculer un taux de pourcentage}
%
%
%\begin{myex}
%	Dans un collège, il y a 800 élèves et 200 sont externes. Quel est le pourcentage d'externes ?\\
%	
%	
%		\begin{tabular}{|l|l|l|}
%			\hline
%			Nombre d'externes & 200 & $P$ \\ \hline
%			Nombre d'élèves   & 800 & 100 \\ \hline
%		\end{tabular}
%	
%	\vspace*{0.5cm}
%
%	
%	Ce tableau est un tableau de proportionnalité. Le coefficient de proportionnalité est 4 ($800 \div 200$).
%	
%	Calcul de $P$ : $100 \div 4 = 25$.\\
%	
%	 Il y a $25 \%$ d'externes.
%\end{myex}
%\section{Notion d'échelle}
%
%\begin{mydef}
%	\begin{itemize}
%		\item 	Sur un plan à \kw{l'échelle}, les longueurs sur le plan sont proportionnelles aux longueurs dans la réalité.
%		
%		\item  L'échelle d'un plan est  est le quotient de la longueur sur le plan par la longueur réelle correspondante, lorsque ces longueurs sont exprimées dans la même unité.
%	\end{itemize}
%
%\end{mydef}
%
%\begin{myexs}
%	\begin{enumerate}
%		\item Un plan est à l'échelle $ 1 / \num{2000}$. Cela signifie que 1 cm sur le plan représente 20 m (\num{2000} cm) dans la réalité. Les longueurs du plan sont 2000 fois plus petites que les longueurs réelles.
%		
%		
%		\item Un schéma est à l'échelle 50. Cela signifie que 1 cm sur le schéma représente \num{0.02} cm dans la réalité. Les longueurs du plan sont 50 fois plus grandes que les longueurs réelles.
%		
%		
%		\item Sur une carte, 3 cm représentent  12 km dans la réalité. Quelle est l'échelle de la carte ?
%		
%		12 km = \num{1200000} cm.
%		\begin{equation*}
%		\dfrac{3}{\num{1200000}} = \dfrac{1}{\num{400000}}
%		\end{equation*}
%		
%		L'échelle de cette carte est $1 / \num{400000}$.
%	\end{enumerate}
%\end{myexs}
%
%\begin{myrems}
%	\begin{itemize}
%		\item Une échelle n'a pas d'unité.
%		\item L'échelle d'un plan est est le nombre par lequel on multiplie les longueurs réelles pour obtenir les longueurs sur le plan, dans la même unité.
%	\end{itemize}
%\end{myrems}
\end{document}

