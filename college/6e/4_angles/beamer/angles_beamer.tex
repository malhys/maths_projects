\documentclass[xcolor={dvipsnames}]{beamer}
%\usepackage[utf8]{inputenc}
\usetheme{Madrid}
%\usetheme{Malmoe}
\usecolortheme{beaver}
%\usecolortheme{rose}

\input{../../../../utils_maths_beamer}


\usepackage{../../../../pas-math}
\usepackage{../../../../moncours_beamer}

\usepackage{amssymb,amsmath}


\newcommand{\myitem}{\item[\textbullet]}

\graphicspath{{../img/}}

\title{Séquence 4 : Angles}
\date{ }
%\author{O. FINOT}\institute{Collège S$^t$ Bernard}

%
\AtBeginSection[]
{
	\begin{frame}
		\frametitle{}
		\tableofcontents[currentsection, hideallsubsections]
	\end{frame} 

}
%
%
%\AtBeginSubsection[]
%{
%	\begin{frame}
%		\frametitle{Sommaire}
%		\tableofcontents[currentsection, currentsubsection]
%	\end{frame} 
%}

\begin{document}



\begin{frame}
  \titlepage 
\end{frame}


\begin{frame}{}
	\begin{myobj}
	\begin{itemize}
		\item Reconnaître un segment, une demie-droite, une droite et savoir les tracer;
		\item Tracer avec l’équerre la droite perpendiculaire à une droite donnée passant par un point donné;
		\item Tracer avec la règle et l’équerre la droite parallèle à une droite donnée passant par un point donné;
		\item Déterminer la distance entre deux points, entre un point et une droite;
		\item Savoir coder et lire une figure.
	\end{itemize}
\end{myobj}

\begin{mycomp}
	\begin{itemize}
		\item \kw{Modéliser} 
		\item \kw{Représenter} 
		\item \kw{Raisonner} 
		\item \kw{Communiquer}
		
	\end{itemize}
\end{mycomp}
\end{frame}

\section{Définir et nommer un angle}




\begin{frame}{}

	\begin{mydef}
		Un angle est défini par \kword{deux demi-droites de même origine}. \pause Les demis droites sont les \kword{cotés} de l'angle \pause et leur origine est son \kword{sommet}.\pause
	\end{mydef}
	
	
	\begin{myex}
			
			\begin{center}
				\includegraphics<3-5>[scale=0.17]{ex1}
				\includegraphics<6->[scale=0.18]{ex1_1}
			\end{center}
			
			Cet angle est défini par les demi-droites \pause$[BA)$ et $[BC)$. $[BA)$ et $[BC)$ sont  \pause ses cotés et $B$ est son sommet.\pause
			On le note $\widehat{ABC}$ (le sommet de l'angle est toujours au milieu).
			
			
			
		
	\end{myex}
\end{frame}


\end{document}