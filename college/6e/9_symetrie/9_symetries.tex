\documentclass[12pt,a4paper]{article}

\usepackage[in, plain]{fullpage}
\usepackage{array}
\usepackage{../../../pas-math}
\usepackage{../../../moncours}


%\usepackage{pas-cours}
%-------------------------------------------------------------------------------
%          -Packages nécessaires pour écrire en Français et en UTF8-
%-------------------------------------------------------------------------------
\usepackage[utf8]{inputenc}
\usepackage[frenchb]{babel}
\usepackage[T1]{fontenc}
\usepackage{lmodern}
%-------------------------------------------------------------------------------

%-------------------------------------------------------------------------------
%                          -Outils de mise en forme-
%-------------------------------------------------------------------------------
\usepackage{hyperref}
\hypersetup{pdfstartview=XYZ}
\usepackage{enumerate}
\usepackage{graphicx}
\usepackage{multicol}

\usepackage{anysize} %%pour pouvoir mettre les marges qu'on veut
%\marginsize{2.5cm}{2.5cm}{2.5cm}{2.5cm}

\usepackage{indentfirst} %%pour que les premier paragraphes soient aussi indentés
%-------------------------------------------------------------------------------


%-------------------------------------------------------------------------------
%                  -Nécessaires pour écrire des mathématiques-
%-------------------------------------------------------------------------------
\usepackage{amsfonts}
\usepackage{amssymb}
\usepackage{amsmath}
\usepackage{amsthm}
\usepackage{tikz}
%-------------------------------------------------------------------------------

%-------------------------------------------------------------------------------
%                     -Mise en forme d'exercices-
%-------------------------------------------------------------------------------
\newtheoremstyle{exostyle}
{\topsep}% espace avant
{\topsep}% espace apres
{}% Police utilisee par le style de thm
{}% Indentation (vide = aucune, \parindent = indentation paragraphe)
{\bfseries}% Police du titre de thm
{.}% Signe de ponctuation apres le titre du thm
{ }% Espace apres le titre du thm (\newline = linebreak)
{\thmname{#1}\thmnumber{ #2}\thmnote{. \normalfont{\textit{#3}}}}% composants du titre du thm : \thmname = nom du thm, \thmnumber = numéro du thm, \thmnote = sous-titre du thm

\theoremstyle{exostyle}
\newtheorem{exercice}{Exercice}

\newenvironment{questions}{
\begin{enumerate}[\hspace{12pt}\bfseries\itshape a.]}{\end{enumerate}
} %mettre un 1 à la place du a si on veut des numéros au lieu de lettres pour les questions 
%-------------------------------------------------------------------------------



%-------------------------------------------------------------------------------
%                    - Racourcis d'écriture -
%-------------------------------------------------------------------------------

% Angles orientés (couples de vecteurs)
\newcommand{\aopp}[2]{(\vec{#1}, \vec{#2})} %Les deuc vecteurs sont positifs
\newcommand{\aopn}[2]{(\vec{#1}, -\vec{#2})} %Le second vecteur est négatif
\newcommand{\aonp}[2]{(-\vec{#1}, \vec{#2})} %Le premier vecteur est négatif
\newcommand{\aonn}[2]{(-\vec{#1}, -\vec{#2})} %Les deux vecteurs sont négatifs

%Ensembles mathématiques
\newcommand{\naturels}{\mathbb{N}} %Nombres naturels
\newcommand{\relatifs}{\mathbb{Z}} %Nombres relatifs
\newcommand{\rationnels}{\mathbb{Q}} %Nombres rationnels
\newcommand{\reels}{\mathbb{R}} %Nombres réels
\newcommand{\complexes}{\mathbb{C}} %Nombres complexes
%-------------------------------------------------------------------------------




%\makeatletter
%\renewcommand*{\@seccntformat}[1]{\csname the#1\endcsname\hspace{0.1cm}}
%\makeatother


%\author{Olivier FINOT}
\date{}
\title{}

\graphicspath{{./img/}}

\lhead{Seq 2: Symétries}
\rhead{O. FINOT}
%
%\rfoot{Page \thepage}
\begin{document}
%\maketitle




\chap[num=9, color=red]{Symétrie axiale}{\today }

\begin{myobj}
	\begin{itemize}
		\item Reconnaître un segment, une demie-droite, une droite et savoir les tracer;
		\item Tracer avec l’équerre la droite perpendiculaire à une droite donnée passant par un point donné;
		\item Tracer avec la règle et l’équerre la droite parallèle à une droite donnée passant par un point donné;
		\item Déterminer la distance entre deux points, entre un point et une droite;
		\item Savoir coder et lire une figure.
	\end{itemize}
\end{myobj}

\begin{mycomp}
	\begin{itemize}
		\item \kw{Modéliser} 
		\item \kw{Représenter} 
		\item \kw{Raisonner} 
		\item \kw{Communiquer}
		
	\end{itemize}
\end{mycomp}

\section{Médiatrice}


\begin{mydef}
	La médiatrice d'un segment est la droite \kw{perpendiculaire à ce segment} et qui \kw{passe par son milieu}.
\end{mydef}


\begin{myex}
	La droite $(d)$ est la médiatrice du segment $[AB]$.
	
	\begin{center}
		\includegraphics[scale=0.15]{med1}
	\end{center}
\end{myex}

\begin{myprops}
	\begin{itemize}
		\item \textbf{Si} un point appartient à la médiatrice d'un segment, \textbf{alors} ce point est à la même distance des extrémités de ce segment.
		\item \textbf{Si} un point est à la même distance des extrémités d'un segment, \textbf{alors} il appartient à la médiatrice de ce segment.
	\end{itemize}
\end{myprops}


\begin{myexs}
	\begin{multicols}{2}
		
		\begin{enumerate}
			
		
		\item  Le point $D$ appartient à la médiatrice $(d)$ du segment $[AB]$, donc $AD=BD$.
		
		\vspace*{0.5cm}
		\begin{center}
			\includegraphics[scale=0.15]{med2}
		\end{center}
	
		
		\item On a $EG=GF$, $EH=HF$ et $EI=IF$, donc les points $G$, $H$ et $I$ appartiennent tous à la médiatrice du segment $[EF]$.
		
		\begin{center}
			\includegraphics[scale=0.13]{med3}
		\end{center}
	
	\end{enumerate}
	\end{multicols}


	
\end{myexs}

\begin{mymeth}
Pour tracer la médiatrice d'un segment $[AB]$ au compas et à la règle non graduée :

\begin{enumerate}
	\item choisir un écartement plus grand que la moitié du segment;
	\item placer la pointe du compas en $A$ et tracer un arc de cercle;\label{step1}
	\item en gardent le même écartement, placer la pointe du compas en $B$;
	\item tracer un arc de cercle qui coupe le premier;
	\item placer le point $I$ à l'intersection;\label{step2}
	\item refaire les étapes \ref{step1} à \ref{step2} avec un autre écartement en nommant le point $J$;
	\item tracer la droite $(IJ)$ médiatrice du segment $[AB]$.
	
\end{enumerate}
\end{mymeth}

\newpage
\section{Symétrique  d'un point par rapport à une droite}
	
	\begin{myprop}
		Si le point $B$ est le symétrique du point $A$ par rapport à une droite $(d)$, alors la droite $(d)$ est la médiatrice du segment $[AB]$.
	\end{myprop}

	\begin{mymeth}
		Pour tracer le symétrique d'un point $A$ par rapport à une droite $(d)$ :
		
		
		\begin{enumerate}
		%	\begin{multicols}{3}
				
			
			\item  On fixe un écartement de notre compas suffisamment grand. On pique en A et on trace deux arcs de cercle qui coupe la droite $(d)$ en deux endroits.
			
			\begin{center}
				\includegraphics[scale=0.2]{meth1}
			\end{center}
		
		
			\item Toujours avec le même écartement. On pique au niveau de la première intersection et on créé un arc de cercle de l'autre côté de la droite $(d)$.
			
			\begin{center}
				\includegraphics[scale=0.2]{meth2}
			\end{center}
			\item Toujours avec le même écartement. On pique au niveau de la 2$^e$ intersection et on créé un arc de cercle de l'autre côté de la droite $(d)$ qui va couper le dernier arc de cercle tracé, le point d'intersection est le symétrie de $A$.
			
			\begin{center}
				\includegraphics[scale=0.2]{meth3}
			\end{center}
			
		%	\end{multicols}
		\end{enumerate}
	\end{mymeth}
	
\section{Symétrique  d'une figure par rapport à une droite}

	\begin{mydef}
		Deux figures $F_1$ et $F_2$ sont symétriques par rapport à la droite $(d)$ si \kw{par pliage le long de la droite $(d)$ les figures se superposent}.
		
		
		\begin{center}
			\includegraphics[scale=0.35]{def}
		\end{center}
	\end{mydef}

\section{Propriétés de la symétrie}

	\begin{myprops}
		\begin{itemize}
			\item Si des points sont alignés, alors leurs symétriques par rapport à une droite sont \kw{aussi alignés}.
			\item Si deux segments sont symétriques par rapport à une droite, alors ils ont la \kw{même longueur}.
			%\item Si deux angles sont symétriques par rapport à une droite, alors ils ont la \kw{même mesure}.
			\item Si deux cercles sont symétriques par rapport à une droite, alors ils ont le \kw{même rayon} et leurs \kw{centres sont symétriques}.
		\end{itemize}
		
	\end{myprops}


	\begin{myexs}
		\begin{itemize}
			\item Le symétrique de la droite $(d_1)$ par rapport à la droite $(d)$ est la droite $(d_2)$.
			\begin{center}
				\includegraphics[scale=0.6]{prop1}
			\end{center}
		
			\item Les segment $[IJ]$ et $[I'J']$ sont symétriques par rapport à la droite $(d)$, ils ont la même longueur.
			\begin{center}
				\includegraphics[scale=0.4]{prop2}
			\end{center}
		
		
			\item $O'$ est le symétrique de $O$ par rapport à la droite $(d)$. Le symétrique du cercle de centre $O$ est de rayon $r$ est le cercle de centre $O'$ et de rayon $r$.
			\begin{center}
				\includegraphics[scale=0.4]{prop3}
			\end{center}
		\end{itemize}
	\end{myexs}
\end{document}