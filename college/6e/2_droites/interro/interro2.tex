\documentclass[a4paper,12pt]{exam}
\printanswers % pour imprimer les réponses (corrigé)
% \noprintanswers % Pour ne pas imprimer les réponses (énoncé)
\addpoints % Pour compter les points
% \noaddpoints % pour ne pas compter les points
\qformat{\textbf{Question\ \thequestion  } \hfill }
%\qformat{\textbf{Question\ \thequestion}\quad(\thepoints)\hfill} % Pour définir le style des questions (facultatif)
\usepackage{color} % définit une nouvelle couleur
\shadedsolutions % définit le style des réponses
% \framedsolutions % définit le style des réponses
\definecolor{SolutionColor}{rgb}{0.8,0.9,1} % bleu ciel
\renewcommand{\solutiontitle}{\noindent\textbf{Solution:}\par\noindent} % Définit le titre des solutions




\makeatletter

\def\maketitle{{\centering%
	\par{\huge\textbf{\@title}}%
	\par{\@date}%
	\par}}

\renewcommand{\thesection}{Exercice \arabic{section} }

\renewcommand{\thesubsection}{Partie \Alph{subsection}}   



\makeatother

\lhead{NOM Pr\'enom :}
\rhead{Classe : }
\cfoot{\thepage / \pageref{LastPage}}


%\usepackage{../../pas-math}
%\usepackage{../../moncours}


%\usepackage{pas-cours}
%-------------------------------------------------------------------------------
%          -Packages nécessaires pour écrire en Français et en UTF8-
%-------------------------------------------------------------------------------
\usepackage[utf8]{inputenc}
\usepackage[frenchb]{babel}
\usepackage[T1]{fontenc}
\usepackage{lmodern}
%-------------------------------------------------------------------------------

%-------------------------------------------------------------------------------
%                          -Outils de mise en forme-
%-------------------------------------------------------------------------------
\usepackage{hyperref}
\hypersetup{pdfstartview=XYZ}
\usepackage{enumerate}
\usepackage{graphicx}
\usepackage{multicol}

\usepackage{anysize} %%pour pouvoir mettre les marges qu'on veut
%\marginsize{2.5cm}{2.5cm}{2.5cm}{2.5cm}

\usepackage{indentfirst} %%pour que les premier paragraphes soient aussi indentés
%-------------------------------------------------------------------------------


%-------------------------------------------------------------------------------
%                  -Nécessaires pour écrire des mathématiques-
%-------------------------------------------------------------------------------
\usepackage{amsfonts}
\usepackage{amssymb}
\usepackage{amsmath}
\usepackage{amsthm}
\usepackage{tikz}
%-------------------------------------------------------------------------------

%-------------------------------------------------------------------------------
%                     -Mise en forme d'exercices-
%-------------------------------------------------------------------------------
\newtheoremstyle{exostyle}
{\topsep}% espace avant
{\topsep}% espace apres
{}% Police utilisee par le style de thm
{}% Indentation (vide = aucune, \parindent = indentation paragraphe)
{\bfseries}% Police du titre de thm
{.}% Signe de ponctuation apres le titre du thm
{ }% Espace apres le titre du thm (\newline = linebreak)
{\thmname{#1}\thmnumber{ #2}\thmnote{. \normalfont{\textit{#3}}}}% composants du titre du thm : \thmname = nom du thm, \thmnumber = numéro du thm, \thmnote = sous-titre du thm

\theoremstyle{exostyle}
\newtheorem{exercice}{Exercice}

\newenvironment{questions}{
\begin{enumerate}[\hspace{12pt}\bfseries\itshape a.]}{\end{enumerate}
} %mettre un 1 à la place du a si on veut des numéros au lieu de lettres pour les questions 
%-------------------------------------------------------------------------------



%-------------------------------------------------------------------------------
%                    - Racourcis d'écriture -
%-------------------------------------------------------------------------------

% Angles orientés (couples de vecteurs)
\newcommand{\aopp}[2]{(\vec{#1}, \vec{#2})} %Les deuc vecteurs sont positifs
\newcommand{\aopn}[2]{(\vec{#1}, -\vec{#2})} %Le second vecteur est négatif
\newcommand{\aonp}[2]{(-\vec{#1}, \vec{#2})} %Le premier vecteur est négatif
\newcommand{\aonn}[2]{(-\vec{#1}, -\vec{#2})} %Les deux vecteurs sont négatifs

%Ensembles mathématiques
\newcommand{\naturels}{\mathbb{N}} %Nombres naturels
\newcommand{\relatifs}{\mathbb{Z}} %Nombres relatifs
\newcommand{\rationnels}{\mathbb{Q}} %Nombres rationnels
\newcommand{\reels}{\mathbb{R}} %Nombres réels
\newcommand{\complexes}{\mathbb{C}} %Nombres complexes
%-------------------------------------------------------------------------------




%\usepackage{fullpage}
\author{\ }
\date{14 octobre 2020}
\title{Interrogation numéro 1}


\begin{document}
%	\usepackage{fancyhdr}
%	
%	\pagestyle{fancy}
%	\fancyhf{}
	%\rhead{Share\LaTeX}

\maketitle
\section{Points et droites}

\begin{center}
	\includegraphics[scale=0.2]{img/droite}	
\end{center}

\begin{questions}
	\question Compléter les expressions suivantes avec $\in$ ou $\notin$ :
	
	\begin{multicols}{4}
		\begin{itemize}
			
			\item E ......... $[DB]$
			\item B ......... $[CD]$
			\item E ......... $[AC]$			
			\item A ......... $(CD)$
			\item D ......... $[AE)$
			\item B ......... $[DC)$
			\item A ......... $[CB)$
			\item D ......... $(AB)$
		\end{itemize}
	\end{multicols}

	\question Sur la figure placer les points $F$, $G$, $H$ et $I$ tels que :
	
	\begin{itemize}
		\item $F \in (ED)$.
		\item $G \in [CB] $ et $G \notin [ED)$.		
		\item $H \notin [EB) $ et $H \notin [AC)$.
		\item $I \in [CD] $ et $I \notin [CE]$.
		
	\end{itemize}
\end{questions}



\newpage


\section{Tracer}

Sur la figure ci-dessous, tracer :

\begin{itemize}
	\item $[AC]$
	\item $(DB)$
	\item La droite $(d_3)$ perpendiculaire à $(d_2)$ passant par $D$.
	\item La droite $(d_4)$ parallèle à $(d_1)$ passant par $B$. 
	\item $M$ le point d'intersection de $(d_3)$ et $(d_4)$.
	\item $[ME)$

\end{itemize}

\vspace*{3cm}
\begin{center}
	\includegraphics[scale=0.2]{img/fig}	
\end{center}

%\maketitle

%{\Large 	\includegraphics[scale=0.4]{competences}

\section{Calculer}
Calculer les expressions suivantes en détaillant tous les calculs:
\begin{questions}
	
	
	\question[3]  $A = 44 - 37 + 28 - 15$
	
	\fillwithdottedlines{6cm}
	{\LARGE \begin{solution}
		\begin{eqnarray*}
		A &=& 44 - 37 + 28 - 15 \\
		A &=& 7 + 28 - 15 \\
		A &=& 35 -15 \\
		A &=& 20
		\end{eqnarray*}
	\end{solution}}
	
	
	
\question[3]  $B = (7 - 5) \times (8 + 2)$
	
	\fillwithdottedlines{6cm}
	{\LARGE \begin{solution}
		\begin{eqnarray*}
		B &=& (7 - 5) \times (8 + 2)\\
		B &=& 2 \times 10 \\			
		B &=& 20
		\end{eqnarray*}
	\end{solution}}
	
	\newpage
	 \question[4]  $C = 44 - (37 + 28) - 15$
	
	\fillwithdottedlines{5cm}
	{\LARGE \begin{solution}
		\begin{eqnarray*}
		C &=&  44 - (37 + 28) - 15\\
		C &=&  44 - 65 - 15\\
		C &=& (- 11) - 15\\
		C &=& - 26
		\end{eqnarray*}
	\end{solution}
	}
	
	
	\question[4]  $D = (50 + (13 - 1) \times 2) + 6$
	
	\fillwithdottedlines{5cm}
	{\LARGE \begin{solution}
		\begin{eqnarray*}
		D &=&  (50 + (13 - 1) \times 2) + 6\\
		D &=&  (50 + 12 \times 2) + 6\\
		D &=& (50 + 24) + 6\\
		D &=& 74 + 6 \\
		D &=& 82 
		\end{eqnarray*}
	\end{solution}}
	
	
%	{\LARGE \question[2]  $E = (19 + 7 \times 2) - 4$}
%	
%	%\fillwithdottedlines{6cm}
%	{\LARGE \begin{solution}
%		\begin{eqnarray*}
%		E &=&  (19 + 7 \times 2) - 4\\
%		E &=&  (19 + 14) - 4\\
%		E &=& 33 - 4 \\
%		E &=& 29 
%		\end{eqnarray*}
%	\end{solution}}
\end{questions}


\section{Traduire}
Traduire les expressions suivantes par une phrase, sans faire les calculs.
\begin{questions}
	
	\question[3]  $E = 63 + 45 \times 7$
	\fillwithdottedlines{3cm}
	
	
	\question[3]  $F = (73 - 5) \times (24 \div 9)$
	\fillwithdottedlines{3cm}
\end{questions}
%}


\label{LastPage}
\end{document}