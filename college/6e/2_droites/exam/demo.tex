\section{Démonstrations (6 points)}

A partir de la figure ci-dessous :

\begin{center}
	\includegraphics*[scale=0.15]{img/figure2}	
\end{center}

\begin{questions}
	\question 
		\begin{parts}
			\part[1] Citer deux droites pour lesquelles on peut justifier qu'elles sont parallèles.
			\begin{solution}
				Les droites $(d_3)$ et $(d_4)$ sont parallèles.
			\end{solution}
			\part[2] Rédiger la démonstration.
			\begin{solution}
				\textbf{On sait que} $(d_3) \bot (d_2)$ et $(d_4) \bot (d_2)$.\\
				\textbf{Or} si deux droites sont parallèles à une même troisième droite, alors elles sont parallèles.\\
				\textbf{Donc} $(d_3) // (d_4)$.
			\end{solution}
		\end{parts}
	
	\question Dans cette question, on a : $(d_1) // (d_6)$
		\begin{parts}
			\part[1] Citer deux droites pour lesquelles on peut justifier qu'elles sont perpendiculaires.
			\begin{solution}
				Les droites $(d_6)$ et $(d_7)$ sont perpendiculaires.
			\end{solution}
			\part[2] Rédiger la démonstration.
			\begin{solution}
				\textbf{On sait que} $(d_1) \bot (d_7)$ et $(d_1) // (d_6)$.\\
				\textbf{Or} si deux droites sont parallèles , alors toute perpendiculaire à l'une est perpendiculaire à l'autre.\\
				\textbf{Donc} $(d_6) // (d_7)$.
			\end{solution}
		\end{parts}
\end{questions}