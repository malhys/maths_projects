\documentclass[xcolor={dvipsnames}]{beamer}
%\usepackage[utf8]{inputenc}
\usetheme{Madrid}
%\usetheme{Malmoe}
\usecolortheme{beaver}
%\usecolortheme{rose}

\input{../../../../utils_maths_beamer}


\usepackage{../../../../pas-math}
\usepackage{../../../../moncours_beamer}

\usepackage{amssymb,amsmath}


\newcommand{\myitem}{\item[\textbullet]}

\graphicspath{{../img/}}

\title{Séquence 2 : Droites, segments et codage}
%\author{O. FINOT}\institute{Collège S$^t$ Bernard}

%
\AtBeginSection[]
{
	\begin{frame}
		\frametitle{}
		\tableofcontents[currentsection, hideallsubsections]
	\end{frame} 

}
%
%
%\AtBeginSubsection[]
%{
%	\begin{frame}
%		\frametitle{Sommaire}
%		\tableofcontents[currentsection, currentsubsection]
%	\end{frame} 
%}

\begin{document}



\begin{frame}
  \titlepage 
\end{frame}


\begin{frame}{}
	\begin{myobj}
	\begin{itemize}
		\item Reconnaître un segment, une demie-droite, une droite et savoir les tracer;
		\item Tracer avec l’équerre la droite perpendiculaire à une droite donnée passant par un point donné;
		\item Tracer avec la règle et l’équerre la droite parallèle à une droite donnée passant par un point donné;
		\item Déterminer la distance entre deux points, entre un point et une droite;
		\item Savoir coder et lire une figure.
	\end{itemize}
\end{myobj}

\begin{mycomp}
	\begin{itemize}
		\item \kw{Modéliser} 
		\item \kw{Représenter} 
		\item \kw{Raisonner} 
		\item \kw{Communiquer}
		
	\end{itemize}
\end{mycomp}
\end{frame}

\section{Droites}




\begin{frame}{}

	\begin{mydef}
		Une \kword{droite} est un objet géométrique formé de \kword{points alignés}. Une droite est illimitée des deux cotés.\pause
	\end{mydef}
	
	\begin{myprops}
		\begin{itemize}
			\item Une droite qui passe par deux points $A$ et $B$, se note $(AB)$ ou $(BA)$;
			\item Si un point $C$ appartient à la droite $(AB)$, on note $C \in (AB)$.
			\item Si il n'appartient pas à la droite $(AB)$, on note $C \notin (AB)$.\pause
		\end{itemize}
	\end{myprops}
	
	\begin{myex}
		Les points $M$, $R$ et $A$ sont alignés.
		\begin{center}
			\includegraphics[scale=0.15]{../img/droite1}
		\end{center}
		
		\begin{itemize}
			\item La droite $(d)$ passant par les points $M$ et $R$ se note 
			\item Le point A appartient à la droite $(MR)$, on note :
			\item Le point S n'appartient pas à la droite $(MR)$, on note :
		\end{itemize}
	\end{myex}
\end{frame}

\begin{frame}
	\begin{mydef}
		Une \kword{demi-droite} est une portion de droite limitée d'un seul côté par un point, son \kword{origine}.\pause
	\end{mydef}
	
	\begin{myprop}
		La demi-droite d'origine $A$ et passant par $B$ se note $[AB)$.\pause
	\end{myprop}
	
	\begin{myex}
		\begin{center}
			\includegraphics[scale=0.55]{../img/demi-droite}
		\end{center}
		
		La demi droite 
	\end{myex}
\end{frame}

\begin{frame}
	\begin{mydef}
		Un \kword{segment} est une portion de droite limitée par deux points : ses \kword{extrémités}.
	\end{mydef}
	
	
	\begin{myprop}
		Le segment d'extrémités $A$ et $B$ se note $[AB]$ ou $[BA]$.
	\end{myprop}
	
	\begin{myex}
		\begin{center}
			\includegraphics[scale=0.55]{../img/segment}
		\end{center}
		
		Le segment 
	\end{myex}
\end{frame}

\section{Sécantes, perpendiculaires et parallèles}

\begin{frame}
	\begin{mydef}
			Deux droites sont \kword{sécantes} \pause si elles n'ont qu'un seul point commun : \pause leur \kword{point d'intersection}.\pause
		
	\end{mydef}


	\begin{myex}
		
			\begin{center}
				\includegraphics[scale=0.2]{sec}\pause
			\end{center}
		
			Les droites $(d)$ et $(d')$ sont sécantes en $O$ qui est leur point d'intersection.
		
			
	\end{myex}
\end{frame}

\begin{frame}
	\begin{mydef}
		Deux droites $(d_1)$ et $(d_2)$ sont \pause \kword{perpendiculaires} si elles se coupent en formant \kword{quatre angles droits}. \pause On note $(d_1) \perp (d_2)$.\pause
		
	\end{mydef}
	
	
	\begin{myex}
		
		\begin{center}
			\includegraphics[scale=0.2]{perp} \pause
		\end{center}
		
		Les droites $(d_1)$ et $(d_2)$ sont perpendiculaires en $A$. 
		
		
	\end{myex}
\end{frame}

\begin{frame}
	\begin{mydef}
		Deux droites $(d_3)$ et $(d_4)$ \pause  qui ne sont pas sécantes sont \kword{parallèles}. \pause On note $(d_3) // (d_4)$..\pause
		
	\end{mydef}
	
	
	\begin{myex}
		
		\begin{center}
			\includegraphics[scale=0.2]{para1}\pause
		\end{center}
		
		Les droites $(d_3)$ et $(d_4)$ sont parallèles. \pause Même en les prolongeant à l'infini elles ne se rencontreront jamais.
		
		
	\end{myex}
\end{frame}

\section{Longueurs et codages}


\begin{frame}
	\begin{mydefs}
		La mesure d'un segment \pause (distance entre ses deux extrémités) \pause est sa \kword{longueur}. \pause
		
		La longueur d'un segment $[AB]$, \pause se note $AB$ ou $BA$. \pause
	\end{mydefs}
	
		
	\begin{myex}
		
		\begin{center}
			\includegraphics[scale=0.6]{../img/lgr}\pause
		\end{center}
		
		La longueur du segment $[AB]$ est de \num{3.5} cm, \pause on note $AB =\num{3.5}$ cm.
	\end{myex}
\end{frame}


\begin{frame}
	\begin{mydef}
		Le \kword{milieu} d'un segment \pause est le point qui appartient au segment \kword{et} \pause qui est à égale distance de ses extrémités.\pause
	\end{mydef}
	
	\begin{myrem}
		Des segments de même longueur sont codés de façon identique.\pause
	\end{myrem}
	
	\begin{myex}
		\begin{center}
			\includegraphics[scale=0.2]{../img/milieu}
		\end{center}
		
		On a : \pause $M \in [AB]$ et $AM = MB$, \pause donc le point $M$ est le milieu du segment $[AB]$. \pause On a ainsi $AM = AB \div 2$. 
	\end{myex}
\end{frame}


\begin{frame}
	\begin{mydef}
		La distance d’un point à une droite \pause est la longueur du plus court chemin \pause entre ce point et la droite. \pause		
	\end{mydef}

	\begin{alertblock}{Propriété}
			La distance d’un point $A$ à une droite $(d)$ \pause  est la longueur du segment $[AH]$, avec \pause $H$ le pied de la perpendiculaire à $(d)$ passant par $A$. \pause
	\end{alertblock}

	\begin{center}
		\includegraphics[scale=0.25]{lgr_droite}
	\end{center}
\end{frame}

\section{Utiliser les propriétés des droites}

\begin{frame}
	\begin{myprop}
		\kword{Si} deux droites sont perpendiculaires à \pause une même troisième droite, \kword{alors} \pause ces deux droites sont parallèles.\pause
	\end{myprop}

	\begin{myex}
		\begin{center}
			\includegraphics[scale=0.6]{para2}
		\end{center}
	
	\kword{On sait que} $(d_1)$ et $(d_2)$ \pause sont toutes deux perpendiculaires à $(D)$.\\ 
	\kword{Donc} \pause $(d_1)$ et $(d_2)$ sont parallèles.
	\end{myex}
\end{frame}

\begin{frame}
	\begin{myprop}
		\kword{Si} deux droites sont parallèles, \kword{alors} \pause toute perpendiculaire à l’une \pause est perpendiculaire à l’autre. \pause 
	\end{myprop}
	
	
	\begin{myex}
		%\begin{multicols}{2}
		\begin{center}
			\includegraphics[scale=0.56]{para2}
		\end{center}
		
		\kword{On sait que} $(d_1)$ est parallèle à $(d_2)$ et \pause $(d_1)$ est perpendiculaire à  $(D)$\\ 
		\kword{Donc} \pause $(d_2)$ est perpendiculaire à $(D)$.
		
		
		%\end{multicols}
		
	\end{myex}
\end{frame}

\begin{frame}
	\begin{myprop}
			\kword{Si} deux droites sont parallèles à une même troisième, \kword{alors} ces deux droites sont parallèles entre elles.
	\end{myprop}
	
	
	\begin{myex}
		%\begin{multicols}{2}
			\begin{center}
				\includegraphics[scale=0.16]{para3}
			\end{center}
			
			\kword{On sait que} $(d_1)$ et $(d_2)$ sont toutes deux perpendiculaires à $(D)$.\\
			\kword{Donc} $(d_1)$ et $(d_2)$ sont parallèles.
			
			
		%\end{multicols}
		
	\end{myex}
\end{frame}
\end{document}