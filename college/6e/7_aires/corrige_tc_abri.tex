\documentclass[12pt,a4paper]{article}


\usepackage[in, plain]{fullpage}
\usepackage{array}
\usepackage{../../../pas-math}

%-------------------------------------------------------------------------------
%          -Packages nécessaires pour écrire en Français et en UTF8-
%-------------------------------------------------------------------------------
\usepackage[utf8]{inputenc}
\usepackage[frenchb]{babel}
\usepackage[T1]{fontenc}
\usepackage{lmodern}
\usepackage{textcomp}



%-------------------------------------------------------------------------------

%-------------------------------------------------------------------------------
%                          -Outils de mise en forme-
%-------------------------------------------------------------------------------
\usepackage{hyperref}
\hypersetup{pdfstartview=XYZ}
%\usepackage{enumerate}
\usepackage{graphicx}
\usepackage{multicol}
\usepackage{tabularx}
\usepackage{multirow}


\usepackage{anysize} %%pour pouvoir mettre les marges qu'on veut
%\marginsize{2.5cm}{2.5cm}{2.5cm}{2.5cm}

\usepackage{indentfirst} %%pour que les premier paragraphes soient aussi indentés
\usepackage{verbatim}
\usepackage{enumitem}
\usepackage[usenames,dvipsnames,svgnames,table]{xcolor}

\usepackage{variations}

%-------------------------------------------------------------------------------


%-------------------------------------------------------------------------------
%                  -Nécessaires pour écrire des mathématiques-
%-------------------------------------------------------------------------------
\usepackage{amsfonts}
\usepackage{amssymb}
\usepackage{amsmath}
\usepackage{amsthm}
\usepackage{tikz}
\usepackage{xlop}
%-------------------------------------------------------------------------------



%-------------------------------------------------------------------------------


%-------------------------------------------------------------------------------
%                    - Mise en forme avancée
%-------------------------------------------------------------------------------

\usepackage{ifthen}
\usepackage{ifmtarg}


\newcommand{\ifTrue}[2]{\ifthenelse{\equal{#1}{true}}{#2}{$\qquad \qquad$}}

%-------------------------------------------------------------------------------

%-------------------------------------------------------------------------------
%                     -Mise en forme d'exercices-
%-------------------------------------------------------------------------------
%\newtheoremstyle{exostyle}
%{\topsep}% espace avant
%{\topsep}% espace apres
%{}% Police utilisee par le style de thm
%{}% Indentation (vide = aucune, \parindent = indentation paragraphe)
%{\bfseries}% Police du titre de thm
%{.}% Signe de ponctuation apres le titre du thm
%{ }% Espace apres le titre du thm (\newline = linebreak)
%{\thmname{#1}\thmnumber{ #2}\thmnote{. \normalfont{\textit{#3}}}}% composants du titre du thm : \thmname = nom du thm, \thmnumber = numéro du thm, \thmnote = sous-titre du thm

%\theoremstyle{exostyle}
%\newtheorem{exercice}{Exercice}
%
%\newenvironment{questions}{
%\begin{enumerate}[\hspace{12pt}\bfseries\itshape a.]}{\end{enumerate}
%} %mettre un 1 à la place du a si on veut des numéros au lieu de lettres pour les questions 
%-------------------------------------------------------------------------------

%-------------------------------------------------------------------------------
%                    - Mise en forme de tableaux -
%-------------------------------------------------------------------------------

\renewcommand{\arraystretch}{1.7}

\setlength{\tabcolsep}{1.2cm}

%-------------------------------------------------------------------------------



%-------------------------------------------------------------------------------
%                    - Racourcis d'écriture -
%-------------------------------------------------------------------------------

% Angles orientés (couples de vecteurs)
\newcommand{\aopp}[2]{(\vec{#1}, \vec{#2})} %Les deuc vecteurs sont positifs
\newcommand{\aopn}[2]{(\vec{#1}, -\vec{#2})} %Le second vecteur est négatif
\newcommand{\aonp}[2]{(-\vec{#1}, \vec{#2})} %Le premier vecteur est négatif
\newcommand{\aonn}[2]{(-\vec{#1}, -\vec{#2})} %Les deux vecteurs sont négatifs

%Ensembles mathématiques
\newcommand{\naturels}{\mathbb{N}} %Nombres naturels
\newcommand{\relatifs}{\mathbb{Z}} %Nombres relatifs
\newcommand{\rationnels}{\mathbb{Q}} %Nombres rationnels
\newcommand{\reels}{\mathbb{R}} %Nombres réels
\newcommand{\complexes}{\mathbb{C}} %Nombres complexes


%Intégration des parenthèses aux cosinus
\newcommand{\cosP}[1]{\cos\left(#1\right)}
\newcommand{\sinP}[1]{\sin\left(#1\right)}


%Probas stats
\newcommand{\stat}{statistique}
\newcommand{\stats}{statistiques}
%-------------------------------------------------------------------------------

%-------------------------------------------------------------------------------
%                    - Mise en page -
%-------------------------------------------------------------------------------

\newcommand{\twoCol}[1]{\begin{multicols}{2}#1\end{multicols}}


\setenumerate[1]{font=\bfseries,label=\textit{\alph*})}
\setenumerate[2]{font=\bfseries,label=\arabic*)}


%-------------------------------------------------------------------------------
%                    - Elements cours -
%-------------------------------------------------------------------------------




\title{Correction de la tâche complexe 62}
\date{}

\begin{document}
	
\maketitle


\section{Démarches administratives}

	\begin{align*}
		3980 \: mm & =  \num{3.98} \: m \\
		5280 \: mm & =  \num{5.28} \: m 		
	\end{align*}

	L'abri de jardin fait donc \num{5.28} $m$ par \num{3.98} $m$.
	
	\begin{equation*}
		\num{5.28} \times \num{3.98} = \num{21.0144}
	\end{equation*}
	
	La surface au sol de l'abri de jardin est d'environ 21 $m^2$. Il fait plus de 20 $m^2$ donc M. DURIN devra demander un permis de construire.


\section{Lasure}

	\subsection*{Côtés}
	
		Les cotés de l'abri sont des rectangles de longueur \num{5.28} m et de largeur \num{2.295} m.
		
		\begin{equation*}
			\num{5.28} \times \num{2.295} = \num{11.1176}
		\end{equation*}
		
		Un côté de l'abri a une surface de \num{12.1176} $m^2$.
		
				
		
	\subsection*{Faces avant et arrière}
	
				
		Les faces avant et arrières sont composées d'un rectangle de \num{3.98} m par  \num{2.295} m et d'un triangle de base \num{3.98} m et de hauteur {0.608} m (\num{2.903} - \num{2.295}).
		
		\begin{equation*}
		\num{3.98} \times \num{2.295} = \num{9.1341}
		\end{equation*}
		
		La surface du rectangle est donc \num{9.1341} $m^2$.
		
		
		\begin{equation*}
		\num{3.98} \times \num{0.068} \div 2 = \num{1.20992}
		\end{equation*}
		
		La surface du triangle est donc \num{1.20992} $m^2$.
		
		\begin{equation*}
			\num{9.1341} + \num{1.20992} = \num{10.34402}
		\end{equation*}
		
		La surface des faces avant et arrière est \num{10.34402} $m^2$.
		
%	\subsection*{Porte}
%			
%		\begin{equation*}
%			\num{2.4} \times \num{2.132} = \num{5.1168}
%		\end{equation*}
%		
%		La porte a une surface de \num{5.1168}$m^2$.
		
	\subsection*{Total}
	
		\begin{equation*}
			\num{12.1176} \times 2 + \num{10.34402} \times 2  = \num{38.80644}
		\end{equation*}
		
		L'ensemble des murs de l'abri représentent une surface d'environ 39 $m^2$.
		
		\begin{equation*}
			39 \times 4 = 156
		\end{equation*}
		
		Chaque mur devra être lasuré deux fois de chaque côté, la surface à couvrir est donc de 156 $m^2$.
		
		\begin{equation*}
			156 \div 12 = \num{13}
		\end{equation*}
		
		Il faudra donc \num{12,7} l de lasure pour l'abri.
			
		\begin{equation*}
			\num{13} \div \num{2.5} \approx \num{5.2} 
		\end{equation*}
		
		\textbf{Donc M. DURIN aura besoin de 6 pots de lasure pour son abri. Il dépensera 303 € (\num{50.50} $\times$ 6)}.
\end{document}