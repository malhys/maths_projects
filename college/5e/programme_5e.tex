\subsection{Proportionnalité}\label{ch_5_proba}
\begin{enumerate}
	\item Rappels
	\item Règle de trois
	\item Pourcentage
	\item \'Echelle
\end{enumerate}

\subsection{Statistiques}\label{ch_5_stats}
\begin{enumerate}
	\item Effectifs
	\item Fréquence
	\item Classes
	\item Tableau de données
\end{enumerate}

\subsection{Calcul littéral}\label{ch_5_lit}
\begin{enumerate}
	\item Utiliser une expression littérale
\end{enumerate}

\subsection{Opérations sur les nombres}\label{ch_5_op}

\begin{enumerate}
	\item Priorité des opérateurs
	\item Enchaîner des opérations
	\item Distributivité sur des exemples numériques et littéraux
	\item Diviser par un nombre décimal
\end{enumerate}

\subsection{Fractions}\label{ch_5_frac}
\begin{enumerate}
	\item Rappels
	\item Utiliser des fractions (fréquence, proportionnalité)
	\item Additionner et soustraire
		\begin{enumerate}
			\item Avec un même dénominateur
			\item Avec des dénominateurs multiples l'un de l'autre
		\end{enumerate}
	\item Multiplier des fractions
\end{enumerate}

\subsection{Nombres relatifs}\label{ch_5_rels}
\begin{enumerate}
	\item Définition d'un nombre relatif
	\item Nombres opposés
	\item Ordonner des nombres relatifs
	\item Placer des nombres relatifs sur une droite graduée
	\item Se repérer dans le plan	
\end{enumerate}

\subsection{Addition et soustraction de nombres relatifs}\label{ch_5_add_rels}
\begin{enumerate}
	\item Règles d'addition
	\item Règles de soustraction
	\item Distance entre deux points
	\item Expressions algébriques
\end{enumerate}

\subsection{\'Equations}\label{ch_5_eq}
\begin{enumerate}
	\item Introduction
\end{enumerate}

\subsection{Parallélogramme}\label{ch_5_para}
\begin{enumerate}
	\item Définition
	\item Propriétés
	\item Démontrer qu'un quadrilatère est un parallélogramme
\end{enumerate}

\subsection{Parallélogrammes particuliers}\label{ch_5_para2}
\begin{enumerate}
	\item Rectangle
		\begin{itemize}
			\item Définition
			\item Propriétés
			\item Prouver qu'un quadrilatère est un rectangle
		\end{itemize}
	\item Losange
		\begin{itemize}
			\item Définition
			\item Propriétés
			\item Prouver qu'un quadrilatère est un losange
		\end{itemize}
	\item Carré
		\begin{itemize}
			\item Définition
			\item Propriétés
			\item Prouver qu'un quadrilatère est un carré
		\end{itemize}
\end{enumerate}

\subsection{Triangle}\label{ch_5_tri}
	\begin{enumerate}
		\item Propriétés usuelles
		\item Mesure des angles d'un triangle
		\item Construire un triangle
		\item Droites remarquables (médiane, hauteur)
		\item Cercle circonscrit à un triangle
	\end{enumerate}
	
\subsection{Symétrie centrale}\label{ch_5_sym}	
\begin{enumerate}
	\item Rappels sur la symétrie axiale
	\item Définition
	\item Symétrique d'un point
	\item Symétrique d'un segment
	\item Symétrique d'une droite
	\item Symétrique d'une figure
	\item Symétrique d'un cercle
\end{enumerate}

\subsection{Prismes et cylindres de révolution}\label{ch_5_prismes}
\begin{enumerate}
	\item Rappels sur la perspective
	\item Fabriquer un prisme droit ou un cylindre de révolution
	\item Calculer l'aire
	\item Calculer le volume
\end{enumerate}