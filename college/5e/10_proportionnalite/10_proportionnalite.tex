\documentclass[12pt,a4paper]{article}

\usepackage[in, plain]{fullpage}
\usepackage{array}
%\usepackage{../../../pas-math}
\usepackage{../../../moncours2}




\date{}
\title{\textcircled{{\normalsize{6}}} Proportionnalité}

\graphicspath{{./img/}}


%\toggletrue{eleve}
%\toggletrue{dys}

%\rfoot{Page \thepage}
\begin{document}

\maketitle






\begin{myobj}
	\begin{itemize}
		\item Reconnaître un segment, une demie-droite, une droite et savoir les tracer;
		\item Tracer avec l’équerre la droite perpendiculaire à une droite donnée passant par un point donné;
		\item Tracer avec la règle et l’équerre la droite parallèle à une droite donnée passant par un point donné;
		\item Déterminer la distance entre deux points, entre un point et une droite;
		\item Savoir coder et lire une figure.
	\end{itemize}
\end{myobj}

\begin{mycomp}
	\begin{itemize}
		\item \kw{Modéliser} 
		\item \kw{Représenter} 
		\item \kw{Raisonner} 
		\item \kw{Communiquer}
		
	\end{itemize}
\end{mycomp}


%\section{Rappel}

\section{Tableau de proportionnalité}

\begin{mydefs}
	
	\iftoggle{eleve}{%
		\begin{itemize}
			\item Un \hrulefill
			
			\vspace*{0.2cm}
			\hrulefill
			
			\vspace*{0.2cm}
			\hrulefill
			
			\vspace*{0.2cm}
			\hrulefill
			
			
			\item \vspace*{0.2cm}
			\hrulefill
		\end{itemize}
	}{%
		\begin{itemize}
			\item Un tableau a deux lignes est un \kw{tableau de proportionnalité} si on peut calculer les nombres de la deuxième lignes sont obtenues en multipliant ceux de la première \kw{par un même nombre}.
			
			\item Ce nombre est le \kw{coefficient de proportionnalité}.
		\end{itemize}
	}
	
	
	
\end{mydefs}


\begin{mymeth}
	
	\iftoggle{eleve}{%
		Pour \hrulefill
		
		\vspace*{0.2cm}
		\hrulefill
		
		\vspace*{0.2cm}
		\hrulefill
		
		
		

	}{%
		Pour identifier une situation de proportionnalité, on calcule les quotients des nombres de la seconde ligne par ceux de la première ligne. 
		Il y a proportionnalité si c'est toujours le même.
	}
	
	
	
\end{mymeth}

\begin{myex}
	
		Ce tableau présente le prix de différentes masses de cerises :
		
	
	
\iftoggle{eleve}{%
	
	\begin{center}
		\includegraphics[scale=0.6]{tb_prop1-2}
	\end{center}

	\vspace*{0.2cm}
	\hrulefill
	
	\vspace*{0.2cm}
	\hrulefill
	
	\vspace*{0.2cm}
	\hrulefill
}{%
	\begin{center}
		\includegraphics[scale=0.6]{tb_prop1}
	\end{center}

	$\num{1,35} \div \num{0.5} = \num{2,70} \div 1 = \num{5.40} \div 2 = \num{13,50} \div 5 = \num{2.70} $, ce tableau est un tableau de proportionnalité.
	
	Le coefficient de proportionnalité est \num{2.70}.
	
}
	
\end{myex}	

%\begin{myex}	
%		Dans ce tableau on a reporté le nombre de cotés de certains polygones et leur nombre de diagonales.
%		
%		\begin{center}
%			\includegraphics[scale=0.5]{tab2}
%		\end{center}
%	
%		$2 \div 4 = \num{0.5}$, $5 \div 5 = 1$, donc le nombre de côtés d'un polygone n'est pas proportionnel à son nombre de diagonales.	
%	
%\end{myex}
%
\section{Compléter un tableau de proportionnalité}

\begin{mymeth}
	On veut remplir le tableau de proportionnalité suivant :
	
	\begin{center}
		\includegraphics[scale=0.5]{tab3_1}
	\end{center}
\end{mymeth}

\vspace*{-0.5cm}

\subsection{Avec le coefficient de proportionnalité}

%\begin{mymeth}

\iftoggle{eleve}{%
	
	\vspace*{0.2cm} 
	\hrulefill
	
	
	\vspace*{0.2cm} 
	\hrulefill
	
	
	\begin{center}
		\includegraphics[scale=0.5]{tab3_3-2}
	\end{center}
}{%
	On calcule le coefficient : $10 \div 4 = \num{2.5} $.
	
	Donc  $6 \times \num{2.5} = 15$.
	
	
	\begin{center}
		\includegraphics[scale=0.5]{tab3_3}
	\end{center}
}
	
%\end{mymeth}



\vspace*{-0.5cm}

\subsection{En utilisant les propriétés de la proportionnalité}

\begin{myprop}
	
	\iftoggle{eleve}{%
		\vspace*{0.2cm} 
		\hrulefill
		
		\begin{itemize}
			\item \vspace*{0.2cm} 
			\hrulefill
			
			\item 
			\vspace*{0.2cm} 
			\hrulefill
		\end{itemize}
	}{%
		Dans un tableau de proportionnalité, on peut :
		\begin{itemize}
			\item multiplier/diviser une colonne par un nombre;
			\item ajouter/soustraire des colonnes entre elles.
		\end{itemize}
	}
	
	
	
\end{myprop}
%

%\begin{mymeth}
\iftoggle{eleve}{%
	\vspace*{0.2cm} 
	\hrulefill
	
	\vspace*{0.2cm} 
	\hrulefill
	\begin{center}
		\includegraphics[scale=0.35]{tab3_4-2}
	\end{center}
}{%
	On parcourt 10 km en 4 heures et 15 en 6 heures.
	
	Donc en 10 heures on parcourt 25 km (10 + 15) .
	\begin{center}
		\includegraphics[scale=0.35]{tab3_4}
	\end{center}
}
	
%\end{mymeth}

\vspace*{-0.5cm}

\subsection{Par passage à l'unité}


%\begin{mymeth}

\iftoggle{eleve}{%
	\vspace*{0.2cm} 
	\hrulefill
	
	\vspace*{0.2cm} 
	\hrulefill
	
	\vspace*{0.2cm} 
	\hrulefill
	
	
	
	\begin{center}
		\includegraphics[scale=0.5]{tab3_2-2}
	\end{center}
}{%
	En 4 heures, nous parcourons 10 km.
	
	En 1 heure, nous parcourrons donc $10 \div 4 = \num{2.5}$ km.
	
	En 6 heures, nous parcourrons $\num{2.5} \times 6 = 15 $km.
	
	
	
	\begin{center}
		\includegraphics[scale=0.5]{tab3_2}
	\end{center}
}

%\end{mymeth}
\section{Pourcentages}



\begin{mydef}
	Un pourcentage traduit une situation de proportionnalité. 

	Un pourcentage est une proportion exprimée sur un total de 100 (de dénominateur égal à 100).
	
\end{mydef}


\begin{myex}
	<<Dans une confiture, il y a 60 \% de fruits>>
	\begin{itemize}
		\item La masse de fruits est proportionnelle à la masse totale de confiture.
		\item[$\Rightarrow$] Il y a 60g de fruits pour 100g de confiture.
	\end{itemize}
\end{myex}

\subsection{Appliquer un pourcentage}

\begin{myprop}
	$P$ est un nombre positif.
	
	Pour calculer $P\% $ d'une quantité, on multiplie cette quantité par $\frac{P}{100}$.
\end{myprop}


\begin{myex}
	Calculer $20 \% $ de 50 revient à multiplier 50 par $\dfrac{20}{100}$ :
	
	\begin{equation*}
		50 \times \dfrac{20}{100} = 50 \times \num{0.2} = 10
	\end{equation*}
	
	
	$20 \% $ de 50  vaut 10.
\end{myex}


\subsection{Calculer un taux de pourcentage}


\begin{myex}
	Dans un collège, il y a 800 élèves et 200 sont externes. Quel est le pourcentage d'externes ?\\
	
	
	\begin{tabular}{|l|l|l|}
		\hline
		Nombre d'externes & 200 & $P$ \\ \hline
		Nombre d'élèves   & 800 & 100 \\ \hline
	\end{tabular}
	
	\vspace*{0.5cm}
	
	
	Ce tableau est un tableau de proportionnalité. 
	
	Pour passer de 800 à 100 je divise par 8 ($800 \div 100 = 8$).
	
	Calcul de $P$ : $200 \div 8 = 25$.\\
	
	Il y a $25 \%$ d'élèves externes.
\end{myex}



\section{Notion d'échelle}

\begin{mydef}
	\begin{itemize}
		\item 	Sur un plan à \kw{l'échelle}, les longueurs sur le plan sont proportionnelles aux longueurs dans la réalité.
		
		\item  L'échelle d'un plan est  est le quotient de la longueur sur le plan par la longueur réelle correspondante, lorsque ces longueurs sont exprimées dans la même unité.
	\end{itemize}

\end{mydef}

\begin{myexs}
	\begin{enumerate}
		\item Un plan est à l'échelle $ 1 / \num{2000}$. Cela signifie que 1 cm sur le plan représente 20 m (\num{2000} cm) dans la réalité. Les longueurs du plan sont 2000 fois plus petites que les longueurs réelles.
		
		
		\item Un schéma est à l'échelle 50. Cela signifie que 1 cm sur le schéma représente \num{0.02} cm dans la réalité. Les longueurs du plan sont 50 fois plus grandes que les longueurs réelles.
		
		
		\item Sur une carte, 3 cm représentent  12 km dans la réalité. Quelle est l'échelle de la carte ?
		
		12 km = \num{1200000} cm.
		\begin{equation*}
		\dfrac{3}{\num{1200000}} = \dfrac{1}{\num{400000}}
		\end{equation*}
		
		L'échelle de cette carte est $1 / \num{400000}$.
	\end{enumerate}
\end{myexs}

\begin{myrem}
	\begin{itemize}
		\item Une échelle n'a pas d'unité.
		%\item L'échelle d'un plan est est le nombre par lequel on multiplie les longueurs réelles pour obtenir les longueurs sur le plan, dans la même unité.
	\end{itemize}
\end{myrem}

\newpage
%
%\section{Ratio}
%
%\begin{mydefs}
%	
%	\begin{itemize}
%		\item On peut partager une quantité selon un ratio.
%		
%		\item On dit que deux nombres $a$ et $b$ non nuls sont dans le ratio $3:4$ si 
%		
%		\begin{equation*}
%			\dfrac{a}{3} = \dfrac{b}{4}
%		\end{equation*}
%	
%		\item On dit que trois nombres $a$ , $b$ et $c$ non nuls sont dans le ratio $2:3:7$ si 
%		
%		\begin{equation*}
%			\dfrac{a}{2} = \dfrac{b}{3} = \dfrac{b}{7}
%		\end{equation*}
%	
%		\begin{center}
%			\includegraphics[scale=0.6]{ratio_def}
%		\end{center}
%	\end{itemize}
%	
%	
%\end{mydefs}
%
%\begin{myex}
%	On partage un sac de bonbons entre Maroi et Esteban dans un ration $3:4$ (<<trois pour quatre>>). Cela veut dire que Maroi reçoit 3 bonbons quand Esteban en reçoit 4. C'est un partage inégal.
%	
%	Si la poche contient 21 bonbons distribués en trois tours :
%	
%	\begin{center}
%		\includegraphics[scale=0.6]{ratio_ex}
%	\end{center}
%\end{myex}

\end{document}

