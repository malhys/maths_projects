
\section{\'Echelle}

Déterminer l’échelle utilisée dans chaque situation. Justifier la réponse.

\begin{questions}
	\question Sur une carte routière, la distance entre deux villes est de 15 cm. En réalité, cette distance est de 300 km.
	\question Sur la maquette d’un building, la flèche de l’immeuble mesure 12 cm. En réalité, elle mesure 36 m.
	
	%\question Sur le plan d’une halle des sports, les gradins ont une longueur de 82,5 cm. En réalité, ils mesurent 55 m.
	
	\question Une Tour Eiffel en modèle réduit mesure 18 cm de haut. En réalité, elle mesure 324 m (antennes de télévision incluses).
\end{questions}

