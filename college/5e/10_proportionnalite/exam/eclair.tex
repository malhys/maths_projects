\section{Bonus : \'Eclair et tonnerre}

Le son de déplace toujours à la même vitesse. Dans l'air, à ma température de 20°C, sa vitesse est de 340 m/s (mètres par seconde).

\begin{questions}
	\question Quelle distance peut parcourir le son en 5 secondes ? En une minute ?
	
	\question Le grand père de Lisa lui dit :
	
	<<Les jours d'orage, lorsque tu vois un éclair dans le ciel, tu peux savoir à quelle distance de toi est tombée la foudre :
	
	\begin{itemize}
		\item dès que tu vois l'éclair, tu comptes le nombre de secondes que met le bruit du tonnerre pour arriver jusqu'à toi;
		
		\item tu divises le nombre de secondes par 3;
		
		\item tu obtiens la distance en kilomètres qui te sépare du point d'impact de la foudre.>>
		 
	\end{itemize}

	Que penser de la méthode du grand père de Lisa ?
\end{questions}