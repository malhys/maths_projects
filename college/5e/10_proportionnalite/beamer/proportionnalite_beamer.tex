\documentclass[xcolor={dvipsnames}]{beamer}
%\usepackage[utf8]{inputenc}
\usetheme{Madrid}
%\usetheme{Malmoe}
\usecolortheme{beaver}
%\usecolortheme{rose}

\input{../../../../utils_maths_beamer}


\usepackage{../../../../pas-math}
\usepackage{../../../../moncours_beamer}

\usepackage{amssymb,amsmath}


\newcommand{\myitem}{\item[\textbullet]}

\graphicspath{{../img/}}

\title{Séquence 3 : Fractions}
%\author{O. FINOT}\institute{Collège S$^t$ Bernard}

%
\AtBeginSection[]
{
	\begin{frame}
		\frametitle{}
		\tableofcontents[currentsection, hideallsubsections]
	\end{frame} 
	
}
%
%
\AtBeginSubsection[]
{
	\begin{frame}
		\frametitle{Sommaire}
		\tableofcontents[currentsection, currentsubsection]
	\end{frame} 
}

\begin{document}
	
	
	
	\begin{frame}
		\titlepage 
	\end{frame}
	
	
	%	
	%
	%\begin{frame}
	%	\begin{block}{Objectifs}
	%		\begin{itemize}
	%			
	%			\item Savoir si deux fractions sont égales
	%			\item Savoir si un nombre est divisible par un autre
	%			\item Identifier un nombre premier
	%			\item Décomposer un nombre en produit de facteurs premiers
	%			\item Simplifier une fraction
	%			\item Comparer des fractions
	%			\item Additionner et soustraire des fractions dont les dénominateurs sont des multiples l’un de l’autre
	%			
	%			\end{itemize}
	%	\end{block}
	%\end{frame}
	%
	%\begin{frame}
	%	\begin{block}{Compétences travaillées}
	%		\begin{itemize}
	%			\item \kw{Représenter (Re2)} :  produire et utiliser plusieurs représentations d’un nombre;
	%			\item \kw{Calculer (Ca1)} :  calculer avec des nombres rationnels, de manière exacte ou approchée en combinant astucieusement le calcul mental, le calcul posé et le calcul instrumenté ;
	%			\item \kw{Raisonner (Ra1)} :  résoudre des problèmes impliquant des grandeurs variées : mobiliser les connaissances nécessaires, analyser et exploiter ses erreurs, mettre à l’essai plusieurs solutions.		
	%		\end{itemize}
	%	\end{block}
	%\end{frame}
	
	
	
	\section{Tableau de proportionnalité}
	
	\begin{frame}
		\begin{block}{Objectifs}
			\begin{itemize}
				\item reconnaître un tableau de proportionnalité
				\item Calculer un coefficient de proportionnalité
			\end{itemize}
		\end{block}
	
		\begin{block}{Compétence}
			\textbf{Modéliser} : J'identifie une situation de proportionnalité et je l'utilise pour résoudre un problème 
		\end{block}
	\end{frame}
	
	
	\begin{frame}
		\begin{mydefs}
		
			\begin{itemize}
				\item Un tableau a deux lignes est un \kword{tableau de proportionnalité} si on peut calculer les nombres de la deuxième lignes sont obtenues en multipliant ceux de la première \kword{par un même nombre}. \pause
				
				\item Ce nombre est le \kword{coefficient de proportionnalité}.\pause
			\end{itemize}			
			
		\end{mydefs}
	
		\begin{block}{Méthode} 
			Pour identifier une situation de proportionnalité, on calcule les quotients des nombres de la seconde ligne par ceux de la première ligne.\pause
			Il y a proportionnalité si c'est toujours le même.
		\end{block}
	\end{frame}

	\begin{frame}
		\begin{exampleblock}{Exemple}
			Ce tableau présente le prix de différentes masses de cerises :
			\begin{center}
				
				\includegraphics[scale=0.5]{tb_prop1-2}
			\end{center}
		\end{exampleblock}
	\end{frame}

	\begin{frame}
		\begin{exampleblock}{Exemple}
			Ce tableau présente le prix de différentes masses de cerises :
			\begin{center}
				
				\includegraphics[scale=0.5]{tb_prop1}
			\end{center}
		
		$\num{1,35} \div \num{0.5} = \num{2,70} \div 1 = \num{5.40} \div 2 = \num{13,50} \div 5 = \num{2.70} $, ce tableau est un tableau de proportionnalité. \pause
		
		Le coefficient de proportionnalité est \num{2.70}.
		\end{exampleblock}
	\end{frame}

\section{Compléter un tableau de proportionnalité}

	\begin{frame}
	\begin{block}{Objectif}
		\begin{itemize}
			\item Savoir compléter un tableau de proportionnalité.
		\end{itemize}
	\end{block}
	
	\begin{block}{Compétence}
		\textbf{Modéliser} : J'identifie une situation de proportionnalité et je l'utilise pour résoudre un problème 
	\end{block}
\end{frame}

\begin{frame}
	\begin{mymeth}
		On veut remplir le tableau de proportionnalité suivant :
		
		\begin{center}
			\includegraphics[scale=0.5]{tab3_1}
		\end{center}
	\end{mymeth}
		
\end{frame}

\subsection{Avec le coefficient de proportionnalité}


\begin{frame}
	
	On calcule le coefficient : \pause $10 \div 4 = \num{2.5} $. \pause
	
	Donc  $6 \times \num{2.5} = 15$. \pause
	
	
	\begin{center}
		\includegraphics[scale=0.5]{tab3_3}
	\end{center}
	
\end{frame}

\subsection{En utilisant les propriétés de la proportionnalité}

\begin{frame}
	

\begin{myprop}
	
	Dans un tableau de proportionnalité, on peut : \pause
	\begin{itemize}
		\item multiplier/diviser une colonne par un nombre; \pause 
		\item ajouter/soustraire des colonnes entre elles. \pause
	\end{itemize}	
	
	
\end{myprop}

\vspace*{.75cm}

On parcourt 10 km en 4 heures et 15 en 6 heures. \pause

Donc en 10 heures on parcourt 25 km (10 + 15).\pause

\vspace*{.75cm}

\begin{center}
	\includegraphics[scale=0.35]{tab3_4}
\end{center}

\end{frame}

\begin{frame}
	En 4 heures, nous parcourons 10 km. \pause
	
	En 1 heure, nous parcourrons donc $10 \div 4 = \num{2.5}$ km. \pause
	
	En 6 heures, nous parcourrons $\num{2.5} \times 6 = 15 $km. \pause
	
	
	\begin{center}
		\includegraphics[scale=0.5]{tab3_2}
	\end{center}

\end{frame}
\end{document}