%\subsection{Definition}

\begin{mydef}
	
	\iftoggle{eleve}{%
		Effectuer la \hrulefill 
		
		\vspace*{0.2cm}
		\hrulefill 
		
		\vspace*{0.2cm}
		\hrulefill 
		
		
		\vspace*{0.2cm}
		\hrulefill 
		
		
		\vspace*{0.2cm}
		\hrulefill 
		
		
		\vspace*{0.2cm}
		\hrulefill 
	}{%
		Effectuer la \kw{division euclidienne} d’un nombre entier, appelé \kw{dividende}, par un nombre entier, différent de zéro, appelé \kw{diviseur}, c’est trouver deux autres nombres entiers, le \kw{quotient} et le \kw{reste}, tels que : 
		
		\begin{equation*}
			diviseur \times quotient + reste = dividende	
		\end{equation*}
	}
	

	
\end{mydef}


\iftoggle{eleve}{%
	\begin{center}
		$\begin{array}{c|c}
			 &  \hspace*{2cm} \\
			\cline{2-2}
			&  \\
			 & \\
		\end{array}$
	\end{center}
}{%
	\begin{center}
		$\begin{array}{c|c}
			Dividende & Diviseur \\
			\cline{2-2}
			& Quotient \\
			Reste & \\
		\end{array}$
	\end{center}
}


%\begin{mywarning}
%	On ne peut pas diviser par 0.
%\end{mywarning}

%\subsection{Technique de division}

%\begin{mymeth}
%\vspace*{-1cm}	

%\begin{mymethname}{Technique de division}
%\begin{multicols}{2}
%	\begin{center}
%%		$\begin{array}{rc|c}
%%		731 & & 34 \\
%%		\cline{3-3}
%%		51& & 21 \\
%%		17 & & \\
%%		\end{array}$
%		\opidiv{731}{34}
%	\end{center}
%
%Conclusion : $34 \times 21 + 17 = 731$
%\end{multicols}
%
%
%Le diviseur 34 a deux chiffres, on commence la division avec les deux premiers chiffres du dividende $73 > 34 $; 
%combien de fois 34 dans 73 : 2 fois ; on soustrait $2 \times 34$ à 73.
%
%On abaisse le 1. 
%Combien de fois 34 dans 51 : 1 fois ; 
%on soustrait $1 \times 34$ à 51 ; $17 < 34$ ; on a fini.\\
%
%
%\textbf{\underline{Vérification:}}\\
%Il existe un moyen pour vérifier si une division euclidienne est juste. 
%Pour cela, on multiplie le quotient par le diviseur, puis on ajoute le reste. 
%Si le résultat obtenu est égal au dividende, la division est juste.
%\end{mymethname}


\begin{myexs}
	Poser et vérifier les divisions euclidiennes suivantes :  $653 \div 7$ et $73 \div 5$
	
	\vspace*{4cm} 
\end{myexs}

%\begin{mypb}
%	Dans une classe de 26 élèves, on forme des équipes de volley-ball (6 joueurs par équipe).
%	
%	\begin{enumerate}
%		\item Combien d’équipes forme-t-on ?
%		\item Combien y-a-t-il de remplaçants ?	
%	\end{enumerate}
%
%\vspace*{5cm} 	
%\end{mypb}