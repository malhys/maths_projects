\begin{mydef}
	Effectuer la division décimale d'un nombre décimal par un nombre entier, c'est chercher le \hspace*{4cm}, tel que : 
	
	\begin{equation*}
%	quotient \; \times \; diviseur = \; dividende
	\end{equation*}
\end{mydef}

\subsection{Division décimale de deux entiers}

\begin{mymeth}
	On commence comme une division euclidienne. Quand il n'y a plus de chiffre à abaisser, pn ajoute une virgule au quotient et on abaisse des zéros jusqu'à ce que le reste soit égal à zéro (ou qu'on obtienne la valeur approchée demandée).
\end{mymeth}

\begin{myexs}
	\begin{multicols}{2}
		\begin{center}
					$\begin{array}{rc|c}
					294 \quad & & 35 \\
					\cline{3-3}
					 &  \\
					 & & \\
					 & & \\
					\end{array}$
			%\opidiv{731}{34}
		\end{center}
		
		
		\begin{center}
			$\begin{array}{rc|c}
			732 \quad & & 5 \\
			\cline{3-3}
			&  \\
			& & \\
			& & \\
			\end{array}$
			
		\end{center}
		
	\end{multicols}
\end{myexs}


\subsection{Division décimale d'un nombre décimal par un entier}

	\begin{mymeth}
		On commence comme pour le cas précédent, mais on met une virgule au quotient dès qu'on arrive à la virgule du diviseur.
	\end{mymeth}

	\begin{myexs}
		\begin{multicols}{2}
			\begin{center}
				$\begin{array}{rc|c}
				\num{456.5} \quad & & 25 \\
				\cline{3-3}
				&  \\
				& & \\
				& & \\
				\end{array}$
				%\opidiv{731}{34}
			\end{center}
			
			
			\begin{center}
				$\begin{array}{rc|c}
				\num{102.4} \quad & & 20 \\
				\cline{3-3}
				&  \\
				& & \\
				& & \\
				\end{array}$
				
			\end{center}
			
		\end{multicols}
	\end{myexs}
\subsection{Division décimale de deux nombres décimaux}

\begin{myprop}
	On ne change pas le quotient de deux nombres décimaux quand on multiplie chacun d'eux par un même nombre (10, 100, ...) pour obtenir un diviseur entier.
\end{myprop}

\begin{myex}
	Diviser \num{67.85} par \num{2.3}. \\
	
	Le diviseur est égal à \num{2.3}, pour qu'il soit entier on le multiplie par 10 : 
	
		\begin{equation*}
			\num{2.3} \times 10 = 23
		\end{equation*}
		
	et on multiplie le dividende par 10, pour ne pas changer la valeur du quotient :
	
		\begin{equation*}
			\num{67.85} \times 10 = \num{678.5}
		\end{equation*}
		
		
		$\begin{array}{rc|c}
		\num{678.5} \quad & & 23 \\
		\cline{3-3}
		&  \\
		& & \\
		& & \\
		\end{array}$
\end{myex}