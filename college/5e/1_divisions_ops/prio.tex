\begin{myprop}
	\iftoggle{eleve}{%
		Dans une expression numérique \hrulefill 
		
		\vspace*{0.2cm}
		\hrulefill 
		
		\vspace*{0.2cm}
		\hrulefill 
	}{%
		Dans une expression numérique sans parenthèse, on effectue les multiplications et les divisions \kw{avant} les additions et les soustractions.
	}
	
\end{myprop}

\begin{myex}
	
	\begin{multicols}{2}
		\begin{align*}
			A &= \num{13.8} - \num{1.25} \times 10 \\
			A &= \num{13.8} - \num{12.5} \\
			A &= \num{1.3}
		\end{align*}
		
		\begin{align*}
			B &= \num{1.7} + \num{9} \div 2 \\
			B &= \num{13.8} + \num{4.5} \\
			B &= \num{6.2}
		\end{align*}	
	\end{multicols}
	
\end{myex}

\begin{myprop}
	\iftoggle{eleve}{%
		Dans une expression numérique \hrulefill 
		
		\vspace*{0.2cm}
		\hrulefill 
		
		\vspace*{0.2cm}
		\hrulefill 
	}{%
		Dans une expression numérique qui contient des parenthèses, on effectue 	\kw{d'abord les calculs entre parenthèses}.
	}
\end{myprop}

\begin{myex}
	\begin{align*}
		C &= (4+5) \times (10-7) \\
		C &= 9 \times 3 \\
		C &= 27
	\end{align*}
\end{myex}

\begin{myprop}
	\iftoggle{eleve}{%
		Dans une expression numérique \hrulefill 
		
		\vspace*{0.2cm}
		\hrulefill 
		
		\vspace*{0.2cm}
		\hrulefill 
	}{%
		Dans une expression numérique qui contient uniquement des additions et des soustractions, on effectue les calculs dans l'ordre de lecture.
	}
\end{myprop}

\begin{myex}
	\begin{align*}
		D &= 2 + 8 - 3 + 7 -5 \\
		D &= 10 - 3 + 7 -5 \\
		D &= 7 + 7 -5 \\
		D &= 14 -5 \\
		D &= 9
	\end{align*}
\end{myex}
