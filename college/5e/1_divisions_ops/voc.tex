%iftoggle{eleve}{%
%	
%	\begin{mydef}
%		Une \kw{expression numérique} est 
%	\end{mydef}
%	
%	\begin{myex}
%		Je calcule l'expression $A = (3 \times (7 - 3))  + 1$ :
%		
%		\vspace*{1cm}
%		\ligne[24pt]{4}
%		
%	\end{myex}
%	
%}{%
%	\begin{mydef}
%		Une \kw{expression numérique} est une suite de calculs.
%	\end{mydef}
%	
%	\begin{myex}
%		Je calcule l'expression $C = (3 \times (7 - 3))  + 1$ :
%		
%		\vspace*{-0.5cm}
%		
%		\begin{eqnarray*}
%			C & = & (3 \times (7-3))  + 1   \\
%			C & = & (3 \times 4)+ 1  \\
%			C & = & 12  + 1 \\
%			C & = & 13
%		\end{eqnarray*}
%	\end{myex}
%}

\iftoggle{eleve}{%
	
	\begin{mydefs}
		
		\begin{itemize}
			\item Le résultat \hrulefill 
						
				\vspace*{0.2cm}
				\hrulefill 
			
			\item Une \hrulefill 
			
			\vspace*{0.2cm}
			\hrulefill 
			
			\item Un \hrulefill 
			
			\vspace*{0.2cm}
			\hrulefill 
			
			%\item Le résultat de la \kw{division} d'un 
			%\vspace*{1.5cm}
		\end{itemize}
		
		
	\end{mydefs}
	
}{%
	
	\begin{mydefs}
		
		\begin{itemize}
			\item Le résultat d'une \kw{addition} est une \kw{somme}, les nombres utilisés sont des \kw{termes}.
			
			\item Une \kw{différence} est le résultat de la \kw{soustraction} de deux \kw{termes}.
			
			\item Un \kw{produit} est le résultat de la \kw{multiplication} de deux \kw{facteurs}.
			
			%\item Le résultat de la \kw{division} d'un \kw{dividende} par un \kw{diviseur} est un \kw{quotient}.
		\end{itemize}
		
		
	\end{mydefs}
}
%\newpage



\begin{myexs}
	
	\iftoggle{eleve}{%
		\begin{center}
			
			\vspace*{1.5cm}
			%\begin{multicols}{2}
			\includegraphics*[scale=0.8]{img/somme2}
			
			\vspace*{1.5cm}
			
			\includegraphics*[scale=0.8]{img/difference2}
			
			\vspace*{1.5cm}
			
			\includegraphics*[scale=0.8]{img/produit2}			
			
			%	\includegraphics*[scale=0.7]{img/somme}
			
			%\end{multicols}
			
		\end{center}
	}{
		\begin{center}
			
			%\begin{multicols}{2}
			\includegraphics*[scale=0.7]{img/somme}
			
			\vspace*{0.5cm}
			
			\includegraphics*[scale=0.7]{img/difference}
			
			\vspace*{0.5cm}
			
			\includegraphics*[scale=0.7]{img/produit}			
			
			%	\includegraphics*[scale=0.7]{img/somme}
			
			%\end{multicols}
			
		\end{center}
	}
	
	
	
	
%	
%	\iftoggle{eleve}{%
%		\begin{itemize}
%			\item L'expression $5 + 3 \times 4$ est 
%			
%			\vspace*{1.5cm}
%			
%			\item L'expression $(2 + 3 ) \times 4$ est 
%			
%			\vspace*{1.5cm}
%			
%			%		\item $ 3 \times (4 + 1)$ est 
%			%		
%			%		\vspace*{1.5cm}
%			%		
%			%		\item $ 3 \times 4 + 1$ est 
%			%		
%			%		\vspace*{1.5cm}
%			
%			\item $\dfrac{19 - 3}{2 \times 4}$ est 
%			
%			\vspace*{1.5cm}
%		\end{itemize}
%	}{%
%		
%		
%		\begin{itemize}
%			\item L'expression $5 + 3 \times 4$ est la somme de  et du produit de 3 par 4.
%			
%			\item L'expression $(2 + 3 ) \times 4$ est le produit de la somme de 2 et 3 par 4.
%			
%			%		\item $ 3 \times (4 + 1)$ est le produit de 3 par la somme de 4 et 1.
%			%		
%			%		\item $ 3 \times 4 + 1$ est la somme du produit de 1 par 3 et  4.
%			
%			\item $\dfrac{19 - 3}{2 \times 4}$ est le quotient de la différence entre  19 et 3 par le produit de 2 par 4.
%		\end{itemize}
%		
%		
%	}
%	
\end{myexs}	