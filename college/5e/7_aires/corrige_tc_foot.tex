\documentclass[12pt,a4paper]{article}


\usepackage[in, plain]{fullpage}
\usepackage{array}
\usepackage{../../../pas-math}

%-------------------------------------------------------------------------------
%          -Packages nécessaires pour écrire en Français et en UTF8-
%-------------------------------------------------------------------------------
\usepackage[utf8]{inputenc}
\usepackage[frenchb]{babel}
\usepackage[T1]{fontenc}
\usepackage{lmodern}
%-------------------------------------------------------------------------------

%-------------------------------------------------------------------------------
%                          -Outils de mise en forme-
%-------------------------------------------------------------------------------
\usepackage{hyperref}
\hypersetup{pdfstartview=XYZ}
\usepackage{enumerate}
\usepackage{graphicx}
\usepackage{multicol}

\usepackage{anysize} %%pour pouvoir mettre les marges qu'on veut
%\marginsize{2.5cm}{2.5cm}{2.5cm}{2.5cm}

\usepackage{indentfirst} %%pour que les premier paragraphes soient aussi indentés
%-------------------------------------------------------------------------------


%-------------------------------------------------------------------------------
%                  -Nécessaires pour écrire des mathématiques-
%-------------------------------------------------------------------------------
\usepackage{amsfonts}
\usepackage{amssymb}
\usepackage{amsmath}
\usepackage{amsthm}
\usepackage{tikz}
%-------------------------------------------------------------------------------

%-------------------------------------------------------------------------------
%                     -Mise en forme d'exercices-
%-------------------------------------------------------------------------------
\newtheoremstyle{exostyle}
{\topsep}% espace avant
{\topsep}% espace apres
{}% Police utilisee par le style de thm
{}% Indentation (vide = aucune, \parindent = indentation paragraphe)
{\bfseries}% Police du titre de thm
{.}% Signe de ponctuation apres le titre du thm
{ }% Espace apres le titre du thm (\newline = linebreak)
{\thmname{#1}\thmnumber{ #2}\thmnote{. \normalfont{\textit{#3}}}}% composants du titre du thm : \thmname = nom du thm, \thmnumber = numéro du thm, \thmnote = sous-titre du thm

\theoremstyle{exostyle}
\newtheorem{exercice}{Exercice}

\newenvironment{questions}{
\begin{enumerate}[\hspace{12pt}\bfseries\itshape a.]}{\end{enumerate}
} %mettre un 1 à la place du a si on veut des numéros au lieu de lettres pour les questions 
%-------------------------------------------------------------------------------



%-------------------------------------------------------------------------------
%                    - Racourcis d'écriture -
%-------------------------------------------------------------------------------

% Angles orientés (couples de vecteurs)
\newcommand{\aopp}[2]{(\vec{#1}, \vec{#2})} %Les deuc vecteurs sont positifs
\newcommand{\aopn}[2]{(\vec{#1}, -\vec{#2})} %Le second vecteur est négatif
\newcommand{\aonp}[2]{(-\vec{#1}, \vec{#2})} %Le premier vecteur est négatif
\newcommand{\aonn}[2]{(-\vec{#1}, -\vec{#2})} %Les deux vecteurs sont négatifs

%Ensembles mathématiques
\newcommand{\naturels}{\mathbb{N}} %Nombres naturels
\newcommand{\relatifs}{\mathbb{Z}} %Nombres relatifs
\newcommand{\rationnels}{\mathbb{Q}} %Nombres rationnels
\newcommand{\reels}{\mathbb{R}} %Nombres réels
\newcommand{\complexes}{\mathbb{C}} %Nombres complexes
%-------------------------------------------------------------------------------



\title{Correction de la tâche complexe 1 page 238}
\date{}

\begin{document}
	
\maketitle


\section{Marquage au sol}

%Calcul de la longueur des marquages au sol.

\subsection*{Bords du terrain et ligne médiane}

\begin{equation*}
	120 \times 2 + 90 \times 3 = 510
\end{equation*}

Les bords du terrain et la ligne médiane font 510 mètres.


\subsection*{Surface de réparation}

\begin{equation*}
	\num{16.5} \times 2 + 7.32 = \num{40.32}
\end{equation*}

La longueur de la surface de réparation est \num{40.32} mètres.

\begin{equation*}
	\num{9.15} \times 2 \times \pi \times 106 \div 360 \approx \num{16.93}
\end{equation*}

L'arc de cercle mesure environ 16,93 mètres

\begin{equation*}
	\num{40.32} + \num{16.93} + 16.5\times 2 = 90,25	
\end{equation*}

Les marquages de la surface de réparation font environ \num{90,25} mètres.

\subsection*{Zone de but}

\begin{equation*}
\num{7.32} + \num{5.5} \times 2 + \num{5.5} \times 2= 29,32	
\end{equation*}

Les marquages de la zone de but font \num{29.32} mètres.

\subsection*{Rond central et points de corner}

\begin{equation*}
	\num{9.15} \times 2 \times \pi \approx \num{57.49}
\end{equation*}

Le rond central a un périmètre d'environ \num{57.49} mètres.

\begin{equation*}
	\num{1} \times 2 \times \pi \approx \num{6.28}
\end{equation*}

Les points de corner forment un cercle d'environ \num{6.28} mètres de périmètre.

\subsection*{Total}

\begin{equation*}
	\num{510} + \num{90.25} \times 2 + \num{29.32} \times 2 + \num{57.49} + \num{6.28} = \num{812.91}
\end{equation*}

Les marquages font environ \num{813} mètres.


\begin{equation*}
813 \times \num{0.10} = \num{81.3}
\end{equation*}

Il y a environ \num{81.3} mètres carrés de marquage sur le terrain.

\begin{equation*}
52 \div 3 \approx \num{17.3}
\end{equation*}

Le marquage doit être refait 17 ou 18 fois chaque année.


\begin{equation*}
	\num{81.3} \times 18 = \num{1463.4}
\end{equation*}

Une année correspond au maximum à  \num{1463.4} mètres carrés de marquage.

\begin{equation*}
\num{1463.4} \div 50 = \num{29.268}
\end{equation*}

\textbf{Il faudra donc au plus \num{29.268} litres de peinture pour entretenir le terrain pendant un an, soit 2 bidons de 15 litres.
}

\section{Consommation d'eau}

\begin{equation*}
	120 \times 90 = \num{10800}
\end{equation*}

La surface du terrain est de \num{10800} mètres carrés.


\begin{equation*}
	\num{10800} \times 30 = \num{324000}
\end{equation*}

Il faut \num{324000} litres d'eau pour arroser le terrain.

\begin{equation*}
	365 \div 5 = 73
\end{equation*}

Il a 73 arrosages par an.

\begin{equation*}
	\num{324000} \times 73 = \num{23652000}
\end{equation*}

Il faut donc \num{23652000} de litres d'eau par an, soit \num{23652} mètres cubes (car 1 $m^3$ = 1000 l).
\end{document}