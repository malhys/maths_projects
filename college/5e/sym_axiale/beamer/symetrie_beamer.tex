\documentclass[xcolor={dvipsnames}]{beamer}
%\usepackage[utf8]{inputenc}
%\usetheme{Madrid}
\usetheme{CambridgeUS}
%\usetheme{Malmoe}
%\usecolortheme{beaver}
\usecolortheme{seahorse}

\input{../../../../utils_maths_beamer}


\usepackage{../../../../pas-math}
\usepackage{../../../../moncours_beamer}

\usepackage{amssymb,amsmath}


\newcommand{\myitem}{\item[\textbullet]}

\graphicspath{{../img/}}

\title{Séquence 6 : Symétrie axiale}
%\subtitle{Correction des exercices semaine du 18/05}
%\author{O. FINOT}\institute{Collège S$^t$ Bernard}

%
%\AtBeginSection[]
%{
%	\begin{frame}
%		\frametitle{}
%		\tableofcontents[currentsection, hideallsubsections]
%	\end{frame} 
%
%}
%
%
%\AtBeginSubsection[]
%{
%	\begin{frame}
%		\frametitle{Sommaire}
%		\tableofcontents[currentsection, currentsubsection]
%	\end{frame} 
%}

\begin{document}



\begin{frame}
  \titlepage 
\end{frame}

\section{Symétrique d'une figure par rapport à une droite}
	

\section{Médiatrice}

\begin{frame}
	\begin{mydef}
		
		\begin{itemize}
			\item La \kword{médiatrice} d'un segment est son axe de symétrie. \pause
			\item C'est la droite \kword{perpendiculaire à ce segment} qui \kword{passe par son milieu}.\pause 
		\end{itemize}
		
	\end{mydef}


\begin{myex}
	

	La droite $(d)$ est la médiatrice du segment $[AB]$.

	\begin{center}
		\includegraphics[scale=0.15]{med1}
	\end{center}
\end{myex}

\end{frame}

\begin{frame}
	\begin{myprops}
		\begin{itemize}
				\item \textbf{Si} un point appartient à la médiatrice d'un segment, \textbf{alors} ce point est à la même distance des extrémités de ce segment. \pause
				\item \textbf{Si} un point est à la même distance des extrémités d'un segment, \textbf{alors} il appartient à la médiatrice de ce segment. \pause			
		\end{itemize}
	\end{myprops}


\begin{myexs}
		\begin{enumerate}
			\item  Le point $D$ appartient à la médiatrice $(d)$ du segment $[AB]$, donc $AD=BD$.
			\vspace*{0.5cm}
			\begin{center}
				\includegraphics[scale=0.1]{med2}
			\end{center}

			
		\end{enumerate}
	
	
\end{myexs}

\end{frame}

\begin{frame}
	\begin{myprops}
		\begin{itemize}
			\item \textbf{Si} un point appartient à la médiatrice d'un segment, \textbf{alors} ce point est à la même distance des extrémités de ce segment. 
			\item \textbf{Si} un point est à la même distance des extrémités d'un segment, \textbf{alors} il appartient à la médiatrice de ce segment. 			
		\end{itemize}
	\end{myprops}
	
	
	\begin{myexs}
		\begin{enumerate}
						
			\item On a $EG=GF$, $EH=HF$ et $EI=IF$, donc les points $G$, $H$ et $I$ appartiennent tous à la médiatrice du segment $[EF]$.
			
			\begin{center}
				\includegraphics[scale=0.1]{med3}
			\end{center}
			
		\end{enumerate}
		
		
	\end{myexs}
	
\end{frame}

\section{Propriétés de la symétrie}


\begin{frame}
	\begin{myprops}
		\begin{itemize}
				\item Si des points sont alignés, alors leurs symétriques par rapport à une droite sont \kword{aussi alignés}.\pause
				\item Si deux segments sont symétriques par rapport à une droite, alors ils ont la \kword{même longueur}.\pause
				%\item Si deux angles sont symétriques par rapport à une droite, alors ils ont la \kw{même mesure}.
				\item Si deux cercles sont symétriques par rapport à une droite, alors ils ont le \kword{même rayon} et leurs \kword{centres sont symétriques}.\pause
		\end{itemize}
		
	\end{myprops}

	\begin{myexs}
		Le symétrique de la droite $(d_1)$ par rapport à $(d)$ est la droite $(d_2)$.
	

		\begin{center}
			\includegraphics[scale=0.4]{prop1}
		\end{center}
	\end{myexs}
\end{frame}

\begin{frame}
	\begin{myprops}
		\begin{itemize}
			\item Si des points sont alignés, alors leurs symétriques par rapport à une droite sont \kword{aussi alignés}.
			\item Si deux segments sont symétriques par rapport à une droite, alors ils ont la \kword{même longueur}.
			%\item Si deux angles sont symétriques par rapport à une droite, alors ils ont la \kw{même mesure}.
			\item Si deux cercles sont symétriques par rapport à une droite, alors ils ont le \kword{même rayon} et leurs \kword{centres sont symétriques}.
		\end{itemize}
		
	\end{myprops}
	
	\begin{myexs}
		$[IJ]$ et $[I'J']$ sont symétriques par rapport à la droite $(d)$, donc $IJ = I'J'$.
	
		\begin{center}
			\includegraphics[scale=0.3]{prop2}
		\end{center}
	\end{myexs}
\end{frame}


\begin{frame}
	\begin{myprops}
		\begin{itemize}
			\item Si des points sont alignés, alors leurs symétriques par rapport à une droite sont \kword{aussi alignés}.
			\item Si deux segments sont symétriques par rapport à une droite, alors ils ont la \kword{même longueur}.
			%\item Si deux angles sont symétriques par rapport à une droite, alors ils ont la \kw{même mesure}.
			\item Si deux cercles sont symétriques par rapport à une droite, alors ils ont le \kword{même rayon} et leurs \kword{centres sont symétriques}.
		\end{itemize}
		
	\end{myprops}
	
	\begin{myexs}
		Le symétrique du cercle de centre $O$ et de rayon $r$ est le cercle de centre $O'$ et de rayon $r$.
		
		\begin{center}
			\includegraphics[scale=0.25]{prop3}
		\end{center}
	\end{myexs}
\end{frame}
\end{document}