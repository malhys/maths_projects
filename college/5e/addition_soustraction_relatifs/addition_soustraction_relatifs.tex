\documentclass{beamer}
%\usepackage[utf8]{inputenc}
\usetheme{Warsaw}
\usecolortheme{seahorse}

\input{../../../utils_maths_beamer}

\title{Nombres Relatifs : Addition et Soustraction}
\author{}\institute{}


\AtBeginSubsection[]
{
	\begin{frame}
		\frametitle{Sommaire}
		\tableofcontents[currentsection, currentsubsection]
	\end{frame} 
}

\begin{document}
	
	
	
\begin{frame}
	\titlepage
\end{frame}

\section{Addition}

\subsection{Règles de Calcul}


\begin{frame}
\frametitle{Addition de deux nombres de même signe}  
\framesubtitle{Règle}	
	
\begin{block}{Propriété 1}
	Pour additionner deux nombres de même signe :
	\begin{itemize}
		\item On garde le signe commun aux deux nombres.
		\item On additionne les distances à zéro.
	\end{itemize}
\end{block}	

\end{frame}

\begin{frame}
	\frametitle{Addition de deux nombres de même signe}  
	\framesubtitle{Exemples}	
	
	On veut calculer la \underline{somme} de $ +2,4 $ et de $ +1,3 $ et la \underline{somme} de $ -13 $ et $ -9 $.\pause
	
	\begin{exampleblock}{$ (+2,4) + (+1,3) $}
		\begin{itemize} %\pause
			\item Le signe commun est $ \ll \textbf{+} \gg $. %\pause
			\item La somme des distances à zéro est égale à 3,7 
			car $ 2,4 + 1,3 = 3,7 $. %\pause
			\item[$\Rightarrow$] Donc \textbf{(+2,4) + (+1,3) = +3,7}.\pause
		\end{itemize}
	\end{exampleblock}
	
	\begin{exampleblock}{$ (-13) + (-9) $}
		\begin{itemize}
			\item Le signe commun est $ \ll \textbf{-} \gg $.
			\item La somme des distances à zéro est égale à 22 
			car $ 13 + 9 = 22 $.
			\item[$\Rightarrow$] Donc \textbf{(-13) + (-9) = -22}.
		\end{itemize}
	\end{exampleblock}
\end{frame}

\begin{frame}
	\frametitle{Addition de deux nombres de signes contraires}  
	\framesubtitle{Règle}	
	
	\begin{block}{Propriété 2}
		Pour additionner deux nombres de signes contraires :
		\begin{itemize}
			\item On garde le signe du nombre qui a \textbf{la plus grande distance à zéro}. 
			\item On soustrait les distances à zéro des deux nombres. 
		\end{itemize}
	\end{block}	
	
\end{frame}

\begin{frame}
	\frametitle{Addition de deux nombres de signes contraires}  
	\framesubtitle{Exemples}	
	
	On veut calculer la \underline{somme} de $ +13 $ et de $ -7 $ et la \underline{somme} de $ -4,5 $ et $ +3,1 $.\pause
	
	\begin{exampleblock}{$ (+13) + (-7) $}
		\begin{itemize} %\pause
			\item Le signe de la somme est $ \ll \textbf{+} \gg $ car $\textbf{13 > 7}$. %\pause
			\item La différence des distances à zéro est égale à 6 
			car $ 13 - 7 = 6 $. %\pause
			\item[$\Rightarrow$] Donc \textbf{(+13) + (-7) = +6}.\pause
		\end{itemize}
	\end{exampleblock}
	
	\begin{exampleblock}{$ (-4,5) + (+3,1) $}
		\begin{itemize}
			\item Le signe de la somme est $ \ll \textbf{-} \gg $ car $\textbf{4,5 > 3,1}$.
			\item La différence entre les distances à zéro est égale à 1,4 
			car $ 4,5 - 3,1 = 1,4 $.
			\item[$\Rightarrow$] Donc \textbf{(-4,5) + (+3,1) = -1,5}.
		\end{itemize}
	\end{exampleblock}
\end{frame}


\begin{frame}
	\frametitle{Addition de deux nombres relatifs}  
	\framesubtitle{Application}	
	
	Exercices :
	\begin{itemize}
		\item 18 p 99
		\item 19 p 99
		\item 20 p 99
	\end{itemize}
\end{frame}


\subsection{Cas particulier}
\begin{frame}
	\frametitle{Cas particulier pour l'addition de nombres relatifs}  
	\framesubtitle{ \ }	
	
	\begin{block}{Propriétés}
		\begin{itemize}
			\item La somme de deux nombres opposés est égale à zéro.
			\item Deux nombres dont la somme est égale à zéro sont opposés.\pause
		\end{itemize}
	\end{block}
	
	\begin{exampleblock}{Exemples}
		\begin{itemize}
			\item $(-3,8)$  et $(+3,8)$ sont opposés donc $(+3,8) + (-3,8) = 0$.
			\item $-(\frac{12}{4}) + (+3) = 0$ donc $-\frac{12}{4}$  et $+3$ sont opposés.
		\end{itemize}
	\end{exampleblock}
	
\end{frame}

\begin{frame}
	\frametitle{Application}  
	\framesubtitle{ \ }	
	
	Exercices :
	\begin{itemize}
		\item 38 p 101
		\item 39 p 101
		\item 73 p 104
	\end{itemize}
\end{frame}

\section{Soustraction de nombres relatifs}

\subsection{Différence de deux nombres relatifs}

\begin{frame}
	\frametitle{Calcul d'une différence de deux nombres relatifs}  
	\framesubtitle{\ }	
	
	\begin{block}{Règle}
		Pour \textbf{\underline{soustraire}} un nombre relatif, on \underline{ajoute son opposé}.
	\end{block}
	

		\begin{columns}[onlytextwidth]
			\begin{column}{0.465\textwidth}
				\begin{exampleblock}{Exemple 1}
					\begin{itemize}
						\item $ (-7) - (\textbf{+4}) $
						\item[$=$] $ (-7) + (\textbf{-4}) $
						\item[$=$] $ -11$
						\item[$\rightarrow$] Pour soustraire \textbf{+4},\\  on ajoute \textbf{-4}\pause
					\end{itemize}
				\end{exampleblock}
			\end{column}
			\begin{column}{0.465\textwidth}
				\begin{exampleblock}{Exemple 2}
					\begin{itemize}
						\item $ (-2,5) - (\textbf{-3,1}) $
						\item[$=$] $ (-2,5) + (\textbf{+3,1}) $
						\item[$=$] $ +0,6$
						\item[$\rightarrow$] Pour soustraire \textbf{-3,1},\\  on ajoute \textbf{+3,1}.
					\end{itemize}
				\end{exampleblock}
			\end{column}
		\end{columns}
	
\end{frame}

\begin{frame}
	\frametitle{Différence de deux nombres relatifs}  
	\framesubtitle{Application }	
	
	Exercices :
	\begin{itemize}
		\item 21 p 99
		\item 22 p 99
		\item 23 p 99
		\item 42 p 101
		\item 43 p 101
	\end{itemize}
\end{frame}

\subsection{Distance de deux points sur une droite graduée}

\begin{frame}
	\frametitle{Distance de deux points}  
	\framesubtitle{\ }	
	
	\begin{block}{Définition}
		
	\begin{itemize}
		

	\item Sur une droite graduée, la \underline{\textbf{distance}} entre deux points est égale à la \underline{\textbf{différence}} entre l'abscisse la plus grande et l'abscisse la plus petite.\pause
	
		
	\item Sur une droite, on considère deux points $ A $ et $ B $ d'abscisses respectives $ a $ et $ b $. Alors la distance $AB$ entre $A$ et $B$ est :
	
	\begin{columns}[onlytextwidth]
		\begin{column}{0.465\textwidth}
			\center{\textbf{$AB = b -a $ si $ b > a$} }
			\center{\includegraphics[scale=0.7]{./img/axe1}}
		\end{column}
		\begin{column}{0.465\textwidth}
			\center{\textbf{$AB = b -a $ si $ b > a$} }
			\center{\includegraphics[scale=0.7]{./img/axe2}}
		\end{column}
	\end{columns}
	

	\end{itemize}	
	\end{block}
	
\end{frame}

\begin{frame}
	\frametitle{Distance de deux points}  
	\framesubtitle{Exemple}	
	
	On veut calculer les distances $AB$ et $BC$ :
	\center{\includegraphics[scale=0.7]{./img/axe3}}\pause

	\begin{exampleblock}{Distance AB}
				
		\begin{itemize}
			\item L'abscisse du point A est +2 et celle du point B est -3.
			\item On a : +2 > -3.
			\item La distance AB est donc égale à la différence entre l'abscisse de A et l'abscisse de B :
			\item[$\Rightarrow $] $AB = (+2) - (-3) = (+2) + (+3) = +5$
		\end{itemize}
	\end{exampleblock}
		
\end{frame}

\begin{frame}
	\frametitle{Distance de deux points}  
	\framesubtitle{Exemple}	
	
	On veut calculer les distances $AB$ et $BC$ :
	\center{\includegraphics[scale=0.7]{./img/axe3}}
	

	\begin{exampleblock}{Distance BC}
		
		\begin{itemize}
			\item L'abscisse du point B est -3 et celle du point C est -0,5.
			\item On a : -0,5 > -3.
			\item La distance BC est donc égale à la différence entre l'abscisse de C et l'abscisse de B :
			\item[$\Rightarrow $] $BC = (-0,5) - (-3) = (-0,5) + (+3) = +2,5$
		\end{itemize}
	\end{exampleblock}
\end{frame}

\subsection{Expression algébrique}

\begin{frame}
	\frametitle{bla}  
	\framesubtitle{}	
	
	bla
\end{frame}

\end{document}


\begin{frame}
	\frametitle{}  
	\framesubtitle{}	
	
\end{frame}