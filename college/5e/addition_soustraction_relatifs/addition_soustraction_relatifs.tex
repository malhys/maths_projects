\documentclass{beamer}
%\usepackage[utf8]{inputenc}
\usetheme{CambridgeUS}

%-------------------------------------------------------------------------------
%          -Packages nécessaires pour écrire en Français et en UTF8-
%-------------------------------------------------------------------------------
\usepackage[utf8]{inputenc}
\usepackage[frenchb]{babel}
\usepackage[T1]{fontenc}
\usepackage{lmodern}
\usepackage{textcomp}

%-------------------------------------------------------------------------------

%-------------------------------------------------------------------------------
%                          -Outils de mise en forme-
%-------------------------------------------------------------------------------
\usepackage{hyperref}
\hypersetup{pdfstartview=XYZ}
\usepackage{enumerate}
\usepackage{graphicx}
%\usepackage{multicol}
%\usepackage{tabularx}

%\usepackage{anysize} %%pour pouvoir mettre les marges qu'on veut
%\marginsize{2.5cm}{2.5cm}{2.5cm}{2.5cm}

\usepackage{indentfirst} %%pour que les premier paragraphes soient aussi indentés
\usepackage{verbatim}
%\usepackage[table]{xcolor}  
%\usepackage{multirow}
\usepackage{ulem}
%-------------------------------------------------------------------------------


%-------------------------------------------------------------------------------
%                  -Nécessaires pour écrire des mathématiques-
%-------------------------------------------------------------------------------
\usepackage{amsfonts}
\usepackage{amssymb}
\usepackage{amsmath}
\usepackage{amsthm}
\usepackage{tikz}
\usepackage{xlop}
\usepackage[output-decimal-marker={,}]{siunitx}
%-------------------------------------------------------------------------------


%-------------------------------------------------------------------------------
%                    - Mise en forme 
%-------------------------------------------------------------------------------

\newcommand{\bu}[1]{\underline{\textbf{#1}}}


\usepackage{ifthen}


\newcommand{\ifTrue}[2]{\ifthenelse{\equal{#1}{true}}{#2}{$\qquad \qquad$}}

\newcommand{\kword}[1]{\textcolor{red}{\underline{#1}}}


%-------------------------------------------------------------------------------



%-------------------------------------------------------------------------------
%                    - Racourcis d'écriture -
%-------------------------------------------------------------------------------

% Angles orientés (couples de vecteurs)
\newcommand{\aopp}[2]{(\vec{#1}, \vec{#2})} %Les deuc vecteurs sont positifs
\newcommand{\aopn}[2]{(\vec{#1}, -\vec{#2})} %Le second vecteur est négatif
\newcommand{\aonp}[2]{(-\vec{#1}, \vec{#2})} %Le premier vecteur est négatif
\newcommand{\aonn}[2]{(-\vec{#1}, -\vec{#2})} %Les deux vecteurs sont négatifs

%Ensembles mathématiques
\newcommand{\naturels}{\mathbb{N}} %Nombres naturels
\newcommand{\relatifs}{\mathbb{Z}} %Nombres relatifs
\newcommand{\rationnels}{\mathbb{Q}} %Nombres rationnels
\newcommand{\reels}{\mathbb{R}} %Nombres réels
\newcommand{\complexes}{\mathbb{C}} %Nombres complexes


%Intégration des parenthèses aux cosinus
\newcommand{\cosP}[1]{\cos\left(#1\right)}
\newcommand{\sinP}[1]{\sin\left(#1\right)}

%Fractions
\newcommand{\myfrac}[2]{{\LARGE $\frac{#1}{#2}$}}

%Vocabulaire courrant
\newcommand{\cad}{c'est-à-dire}

%Droites
\newcommand{\dte}[1]{droite $(#1)$}
\newcommand{\fig}[1]{figure $#1$}
\newcommand{\sym}{symétrique}
\newcommand{\syms}{symétriques}
\newcommand{\asym}{axe de symétrie}
\newcommand{\asyms}{axes de symétrie}
\newcommand{\seg}[1]{$[#1]$}
\newcommand{\monAngle}[1]{$\widehat{#1}$}
\newcommand{\bissec}{bissectrice}
\newcommand{\mediat}{médiatrice}
\newcommand{\ddte}[1]{$[#1)$}

%Figures
\newcommand{\para}{parallélogramme}
\newcommand{\paras}{parallélogrammes}
\newcommand{\myquad}{quadrilatère}
\newcommand{\myquads}{quadrilatères}
\newcommand{\co}{côtés opposés}
\newcommand{\diag}{diagonale}
\newcommand{\diags}{diagonales}
\newcommand{\supp}{supplémentaires}
\newcommand{\car}{carré}
\newcommand{\cars}{carrés}
\newcommand{\rect}{rectangle}
\newcommand{\rects}{rectangles}
\newcommand{\los}{losange}
\newcommand{\loss}{losanges}


%----------------------------------------------------

\title{Nombres Relatifs : Addition et Soustraction}
\author{}\institute{}


\AtBeginSection[]
{
	\begin{frame}
		\frametitle{Sommaire}
		\tableofcontents[currentsection, hideothersubsections]
	\end{frame} 
}

\begin{document}
	
	
	
\begin{frame}
	\titlepage
\end{frame}

\section{Addition}

\subsection{Règles de Calcul}


\begin{frame}
\frametitle{Addition de deux nombres de même signe}  
\framesubtitle{Règle}	
	
\begin{block}{Propriété}
	Pour additionner deux nombre de même signe :
	\begin{itemize}
		\item On garde le signe commun aux deux nombres.
		\item On additionne les distances à zéro.
	\end{itemize}
\end{block}	

\end{frame}

\begin{frame}
	\frametitle{Addition de deux nombres de même signe}  
	\framesubtitle{Exemples}	
	
	On veut calculer la \underline{somme} de $ +2,4 $ et de $ +1,3 $ et la \underline{somme} de $ -13 $ et $ -9 $.\pause
	
	\begin{exampleblock}{$ (+2,4) + (+1,3) $}
		\begin{itemize}\pause
			\item Le signe commun est $ \ll \textbf{+} \gg $. \pause
			\item La somme des distances à zéro est égale à 3,7 
			car $ 2,4 + 1,3 = 3,7 $. \pause
			\item[$\Rightarrow$] Donc \textbf{(+2,4) + (+1,3) = +3,7}.\pause
		\end{itemize}
	\end{exampleblock}
	
	\begin{exampleblock}{$ (-13) + (-9) $}
		\begin{itemize}
			\item Le signe commun est $ \ll \textbf{-} \gg $.
			\item La somme des distances à zéro est égale à 22 
			car $ 13 + 9 = 22 $.
			\item[$\Rightarrow$] Donc \textbf{(-13) + (-9) = -22}.
		\end{itemize}
	\end{exampleblock}
\end{frame}

\begin{frame}
	\frametitle{Addition de deux nombres de signes contraires}  
	\framesubtitle{Règle}	
	
	\begin{block}{Propriété}
		Pour additionner deux nombre de même signe :
		\begin{itemize}
			\item On garde le signe du nombre qui a \textbf{la plus grande distance à zéro}. 
			\item On soustrait les distances à zéro des deux nombres. 
		\end{itemize}
	\end{block}	
	
\end{frame}

\begin{frame}
	\frametitle{Addition de deux nombres de signes contraires}  
	\framesubtitle{Exemples}	
	
	On veut calculer la \underline{somme} de $ +13 $ et de $ -7 $ et la \underline{somme} de $ -4,5 $ et $ +3,1 $.\pause
	
	\begin{exampleblock}{$ (+13) + (-7) $}
		\begin{itemize}\pause
			\item Le signe de la somme est $ \ll \textbf{+} \gg $ car $\textbf{13 > 7}$.\pause
			\item La différence des distances à zéro est égale à 6 
			car $ 13 + 7 = 6 $.\pause
			\item[$\Rightarrow$] Donc \textbf{(+13) + (-7) = -6}.\pause
		\end{itemize}
	\end{exampleblock}
	
	\begin{exampleblock}{$ (-4,5) + (+3,1) $}
		\begin{itemize}
			\item Le signe de la somme est $ \ll \textbf{-} \gg $ car $\textbf{4,5 > 3,1}$.
			\item La différence entre les distances à zéro est égale à 1,4 
			car $ 4,5 - 3,1 = 1,4 $.
			\item[$\Rightarrow$] Donc \textbf{(-4,5) + (+3,1) = -1,5}.
		\end{itemize}
	\end{exampleblock}
\end{frame}


\begin{frame}
	\frametitle{Addition de deux nombres relatifs}  
	\framesubtitle{Application}	
	
	Exercices :
	\begin{itemize}
		\item 18 p 99
		\item 19 p 99
		\item 20 p 99
	\end{itemize}
\end{frame}


\subsection{Cas particuliers}
\begin{frame}
	\frametitle{Cas particuliers pour l'addition de nombres relatifs}  
	\framesubtitle{ \ }	
	
	\begin{block}{Propriété}
		\begin{itemize}
			\item La somme de deux nombres opposés est égale à zéro.
			\item Deux nombres dont la somme est égale à zéro sont opposés.\pause
		\end{itemize}
	\end{block}
	
	\begin{exampleblock}{Exemples}
		\begin{itemize}
			\item $(-3,8)$  et $(+3,8)$ sont opposés donc $(+3,8) + (-3,8) = 0$.
			\item $-(\frac{12}{4}) + (+3) = 0$ donc $-\frac{12}{4}$  et $+3$ sont opposés.
		\end{itemize}
	\end{exampleblock}
	
\end{frame}

\begin{frame}
	\frametitle{Application}  
	\framesubtitle{ \ }	
	
	Exercices :
	\begin{itemize}
		\item 38 p 101
		\item 39 p 101
		\item 73 p 104
	\end{itemize}
\end{frame}

\end{document}


\begin{frame}
	\frametitle{}  
	\framesubtitle{}	
	
\end{frame}