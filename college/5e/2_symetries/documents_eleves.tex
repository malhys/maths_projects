\documentclass[12pt,a4paper]{article}

%\usepackage[left=1.5cm,right=1.5cm,top=1cm,bottom=2cm]{geometry}
\usepackage[in, plain]{fullpage}
\usepackage{array}
\usepackage{../../../pas-math}
\usepackage{../../../moncours}


%\usepackage{pas-cours}
%-------------------------------------------------------------------------------
%          -Packages nécessaires pour écrire en Français et en UTF8-
%-------------------------------------------------------------------------------
\usepackage[utf8]{inputenc}
\usepackage[frenchb]{babel}
\usepackage[T1]{fontenc}
\usepackage{lmodern}
%-------------------------------------------------------------------------------

%-------------------------------------------------------------------------------
%                          -Outils de mise en forme-
%-------------------------------------------------------------------------------
\usepackage{hyperref}
\hypersetup{pdfstartview=XYZ}
\usepackage{enumerate}
\usepackage{graphicx}
\usepackage{multicol}

\usepackage{anysize} %%pour pouvoir mettre les marges qu'on veut
%\marginsize{2.5cm}{2.5cm}{2.5cm}{2.5cm}

\usepackage{indentfirst} %%pour que les premier paragraphes soient aussi indentés
%-------------------------------------------------------------------------------


%-------------------------------------------------------------------------------
%                  -Nécessaires pour écrire des mathématiques-
%-------------------------------------------------------------------------------
\usepackage{amsfonts}
\usepackage{amssymb}
\usepackage{amsmath}
\usepackage{amsthm}
\usepackage{tikz}
%-------------------------------------------------------------------------------

%-------------------------------------------------------------------------------
%                     -Mise en forme d'exercices-
%-------------------------------------------------------------------------------
\newtheoremstyle{exostyle}
{\topsep}% espace avant
{\topsep}% espace apres
{}% Police utilisee par le style de thm
{}% Indentation (vide = aucune, \parindent = indentation paragraphe)
{\bfseries}% Police du titre de thm
{.}% Signe de ponctuation apres le titre du thm
{ }% Espace apres le titre du thm (\newline = linebreak)
{\thmname{#1}\thmnumber{ #2}\thmnote{. \normalfont{\textit{#3}}}}% composants du titre du thm : \thmname = nom du thm, \thmnumber = numéro du thm, \thmnote = sous-titre du thm

\theoremstyle{exostyle}
\newtheorem{exercice}{Exercice}

\newenvironment{questions}{
\begin{enumerate}[\hspace{12pt}\bfseries\itshape a.]}{\end{enumerate}
} %mettre un 1 à la place du a si on veut des numéros au lieu de lettres pour les questions 
%-------------------------------------------------------------------------------



%-------------------------------------------------------------------------------
%                    - Racourcis d'écriture -
%-------------------------------------------------------------------------------

% Angles orientés (couples de vecteurs)
\newcommand{\aopp}[2]{(\vec{#1}, \vec{#2})} %Les deuc vecteurs sont positifs
\newcommand{\aopn}[2]{(\vec{#1}, -\vec{#2})} %Le second vecteur est négatif
\newcommand{\aonp}[2]{(-\vec{#1}, \vec{#2})} %Le premier vecteur est négatif
\newcommand{\aonn}[2]{(-\vec{#1}, -\vec{#2})} %Les deux vecteurs sont négatifs

%Ensembles mathématiques
\newcommand{\naturels}{\mathbb{N}} %Nombres naturels
\newcommand{\relatifs}{\mathbb{Z}} %Nombres relatifs
\newcommand{\rationnels}{\mathbb{Q}} %Nombres rationnels
\newcommand{\reels}{\mathbb{R}} %Nombres réels
\newcommand{\complexes}{\mathbb{C}} %Nombres complexes
%-------------------------------------------------------------------------------




%\makeatletter
%\renewcommand*{\@seccntformat}[1]{\csname the#1\endcsname\hspace{0.1cm}}
%\makeatother


%\author{Olivier FINOT}
\date{}
\title{}

%\newcommand{\disp}{false}
\graphicspath{{./img/}}
%
%\rfoot{Page \thepage}
\begin{document}
%\maketitle

\begin{myobj}
	\begin{itemize}
		\item Reconnaître un segment, une demie-droite, une droite et savoir les tracer;
		\item Tracer avec l’équerre la droite perpendiculaire à une droite donnée passant par un point donné;
		\item Tracer avec la règle et l’équerre la droite parallèle à une droite donnée passant par un point donné;
		\item Déterminer la distance entre deux points, entre un point et une droite;
		\item Savoir coder et lire une figure.
	\end{itemize}
\end{myobj}

\begin{mycomp}
	\begin{itemize}
		\item \kw{Modéliser} 
		\item \kw{Représenter} 
		\item \kw{Raisonner} 
		\item \kw{Communiquer}
		
	\end{itemize}
\end{mycomp}

\begin{myobj}
	\begin{itemize}
		\item Reconnaître un segment, une demie-droite, une droite et savoir les tracer;
		\item Tracer avec l’équerre la droite perpendiculaire à une droite donnée passant par un point donné;
		\item Tracer avec la règle et l’équerre la droite parallèle à une droite donnée passant par un point donné;
		\item Déterminer la distance entre deux points, entre un point et une droite;
		\item Savoir coder et lire une figure.
	\end{itemize}
\end{myobj}

\begin{mycomp}
	\begin{itemize}
		\item \kw{Modéliser} 
		\item \kw{Représenter} 
		\item \kw{Raisonner} 
		\item \kw{Communiquer}
		
	\end{itemize}
\end{mycomp}
%\vspace*{-1cm}

<<<<<<< HEAD
%\begin{myact}
%	\noindent Carla veut construire le symétrique de la droite $(AB)$ par rapport à une droite $(d)$. 
%	Malheureusement la droite $(d)$ a été effacée. Il reste quand même le point $A'$ symétrique de $A$ par rapport à $(d)$.\\
%	
%	\noindent Comment peut-elle faire ?
%	
%	\begin{center}
%		\includegraphics[scale=0.22]{act1}
%	\end{center}
%\end{myact}
%
%
%\begin{center}
%	\includegraphics[scale=.8]{fig1}
%\end{center}


%\begin{center}
%	\includegraphics[scale=.8]{fig1}
%\end{center}
%
%
%\begin{center}
%	\includegraphics[scale=.8]{fig1}
%\end{center}
%
%
%\begin{center}
%	\includegraphics[scale=.8]{fig1}
%\end{center}

%%%%%%%%%%%%%%%%%%%%%%%%%%%%%%%%%%%%%%%%%%%%%%%%%%%%%%%%%%%%%%%%
%Section 2
%%%%%%%%%%%%%%%%%%%%%%%%%%%%%%%%%%%%%%%%%%%%%%%%%%%%%%%%%%%%%%%

\begin{center}
	\includegraphics[scale=.75]{fig2}
\end{center}

\begin{center}
	\includegraphics[scale=.75]{fig2}
\end{center}

\begin{center}
	\includegraphics[scale=.75]{fig2}
\end{center}

%%%%%%%%%%%%%%%%%%%%%%%%%%%%%%%%%%%%%%%%%%%%%%%%%%%%%%%%%%%
%Section 3
%%%%%%%%%%%%%%%%%%%%%%%%%%%%%%%%%%%%%%%%%%%%%%%%%%%%%%%%%%%%


\begin{myexs}
	\begin{multicols}{2}
		\begin{center}
			\includegraphics[scale=0.1]{sym_droites1}
		\end{center}
		
		\begin{itemize}
			\item Les points $A$, $B$ et $C$ sont alignés, donc $A'$, $B'$ et $C'$ leur symétriques par rapport à la droite $(e)$ sont aussi alignés.
		\end{itemize}	
		
		\begin{center}
			\includegraphics[scale=0.2]{sym_droites2}
		\end{center}
		
		\begin{itemize}
			\item Les points $A$, $B$ et $C$ sont alignés, donc $A'$, $B'$ et $C'$ leur symétriques par rapport à la droite $(e)$ sont aussi alignés.
			\item La droite $(AB)$ est parallèle à la droite $(A'B')$.
		\end{itemize}
	\end{multicols}
\end{myexs}

\begin{myex}
	\begin{center}
		\includegraphics[scale=0.2]{sym_seg}
	\end{center}
	
	Le segment $[A'B']$ est le symétrique du segment $[AB]$ par rapport à la droite $(d)$ et $[A'_1B'_1]$ le symétrique de $[AB]$ par rapport au point $O$. 
	Ils ont tous la même longueur
	
	
\end{myex}	

\begin{myex}
	\begin{center}
		\includegraphics[scale=0.2]{sym_figures}
	\end{center}
	
	La figure $F2$ est le symétrique de $F1$ par rapport à la droite $(d)$; $F3$ est le symétrique de $F1$ par rapport au point O.
	Elles ont le même périmètre, la même aire et leurs angles ont la même mesure.
\end{myex}



\begin{myex}
	\begin{center}
		\includegraphics[scale=0.2]{sym_figures}
	\end{center}
	
	La figure $F2$ est le symétrique de $F1$ par rapport à la droite $(d)$; $F3$ est le symétrique de $F1$ par rapport au point O.
	Elles ont le même périmètre, la même aire et leurs angles ont la même mesure.
\end{myex}

%%%%%%%%%%%%%%%%%%%%%%%%%%%%%%%%%%%%%%%%%%%%%%%%%%%%%%%%%%%%%%%%%%%%%%%%%%%
%Section 4
%%%%%%%%%%%%%%%%%%%%%%%%%%%%%%%%%%%%%%%%%%%%%%%%%%%%%%%%%%%%%%%%%%%%%%%

\begin{myexs}
	\begin{center}
		\includegraphics[scale=0.7]{axes}
	\end{center}
\end{myexs}

\begin{myexs}
	\begin{center}
		\includegraphics[scale=0.6]{centres}
	\end{center}
\end{myexs}

\begin{myapp}
	Dire si les panneaux suivants ont un axe et / ou un centre de symétrie.
	
	\begin{center}
		\includegraphics[scale=0.43]{app}
	\end{center}
\end{myapp}
\end{document}