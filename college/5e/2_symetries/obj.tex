\begin{myobj}
	\begin{itemize}
		\item Construire le symétrique d’un point par rapport à un autre à la main où à l’aide d’un logiciel;
		\item Construire le symétrique d’une figure par rapport à un point;
		\item Utiliser les propriétés de la symétrie axiale ou centrale;
		\item Identifier des symétries dans des figures.		
	\end{itemize}
\end{myobj}

\begin{mycomp}
	\begin{itemize}
		\item \kw{Chercher (Ch2)} :  s’engager    dans    une    démarche    scientifique, observer, questionner, manipuler, expérimenter (sur une feuille de papier, avec des objets, à l’aide de logiciels), émettre des hypothèses, chercher des exemples ou des contre-exemples, simplifier ou particulariser une situation, émettre une conjecture ;
		\item \kw{Raisonner (Ra3)} :  démontrer : utiliser un raisonnement logique et des règles établies (propriétés, théorèmes, formules) pour parvenir à une conclusion ;
		\item \kw{Communiquer (Co2)} :  expliquer à l’oral ou à l’écrit (sa démarche, son raisonnement, un calcul, un protocole   de   construction   géométrique, un algorithme), comprendre les explications d’un autre et argumenter dans l’échange ; 
		
	\end{itemize}
\end{mycomp}