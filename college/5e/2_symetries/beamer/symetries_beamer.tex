\documentclass[xcolor={dvipsnames}]{beamer}
%\usepackage[utf8]{inputenc}
\usetheme{Madrid}
%\usetheme{Malmoe}
\usecolortheme{beaver}
%\usecolortheme{rose}

\input{../../../../utils_maths_beamer}


\usepackage{../../../../pas-math}
\usepackage{../../../../moncours_beamer}

\usepackage{amssymb,amsmath}


\newcommand{\myitem}{\item[\textbullet]}

\graphicspath{{../img/}}

\title{Séquence 2 : Symétries}
%\author{O. FINOT}\institute{Collège S$^t$ Bernard}

%
\AtBeginSection[]
{
	\begin{frame}
		\frametitle{}
		\tableofcontents[currentsection, hideallsubsections]
	\end{frame} 

}
%
%
%\AtBeginSubsection[]
%{
%	\begin{frame}
%		\frametitle{Sommaire}
%		\tableofcontents[currentsection, currentsubsection]
%	\end{frame} 
%}

\begin{document}



\begin{frame}
  \titlepage 
\end{frame}


	



\section{Symétrie axiale}




\begin{frame}{}
	\begin{mydef}
		Deux figures sont \kword{symétriques par rapport à une droite $(d)$} si elles se superposent quand on plie le long de cette droite. 
		La droite $(d)$ est appelée \kword{axe de symétrie}.	
	
	\end{mydef}

	\begin{myex}
		\begin{center}
			\includegraphics[scale=.5]{fig1}
		\end{center}	
	\end{myex}


\end{frame}

\begin{frame}
	
	\begin{columns}
		\begin{column}{0.35\textwidth}
			%\begin{left}
				\includegraphics[scale=0.18]{def}
			%\end{left}
		\end{column}
	
		\begin{column}{0.65\textwidth}
			\begin{myprops}
				Soit $(d)$ une droite :
				\begin{itemize}
					\item Si un point $A$ n'appartient pas à la droite $(d)$, alors son symétrique par rapport à la droite $(d)$ est le point $A'$ tel que \kword{$(d)$ est la médiatrice du segment $[AA']$}.
					\item Si un point $B$ appartient à la droite $(d)$, alors son symétrique par rapport à la droite $(d)$ est \kword{lui même}.
				\end{itemize}
			\end{myprops}
		\end{column}
	\end{columns}
	
	
\end{frame}

\section{Symétrie centrale}


\begin{frame}
	\begin{mydef}
		
		Deux figures sont \kword{symétriques par rapport à un point $O$} si elles se superposent lorsqu'on effectue un demi-tour autour du point $O$. Le point $O$ est appelé \kword{centre de symétrie}.\pause
		
	\end{mydef}
	
	\begin{myex}
		\begin{center}
			\includegraphics[scale=.5]{fig2}
		\end{center}
	\end{myex}
\end{frame}

\section{Identifier un axe ou un centre de symétrie}

\begin{frame}
	\begin{mydef}
		
		Si une figure et son symétrique \pause par rapport à une droite $(d)$ sont confondus, alors \pause $(d)$ est un \kword{axe de symétrie} de la figure.\pause 
		
		
	\end{mydef}
	
	\begin{myexs}
		\begin{center}
			\includegraphics[scale=0.65]{axes}
		\end{center}
	\end{myexs}
\end{frame}


\begin{frame}
	\begin{mydef}
		Si une figure et son symétrique \pause par rapport à un point $O$ sont confondus, alors \pause $O$ est un \kword{centre de symétrie} de la figure.\pause
		
	\end{mydef}
	
	\begin{myexs}
		\begin{center}
			\includegraphics[scale=0.5]{centres}
		\end{center}
	\end{myexs}
\end{frame}
\section{Propriétés de la symétrie}

\begin{frame}
	\begin{myprops}
		\begin{itemize}
			\item Le symétrique d'une droite par rapport à une droite ou un point est une autre droite. La symétrie \kword{conserve l'alignement}.
			\item Si deux droites sont \kword{symétriques par rapport à un point} alors elles sont \kword{parallèles}.
		\end{itemize}
	\end{myprops}
\end{frame}

\begin{frame}
	\begin{myexs}
		\begin{columns}
			\begin{column}{0.5\textwidth}
				
			
			\begin{center}
				\includegraphics[scale=0.11]{sym_droites1}
			\end{center}
			
			\begin{itemize}
				\item Les points $A$, $B$ et $C$ sont alignés, donc $A'$, $B'$ et $C'$ leur symétriques par rapport à la droite $(e)$ sont \pause aussi alignés.
			\end{itemize}	
			
			\end{column}
		
			\begin{column}{0.5\textwidth}
			\begin{center}
				\includegraphics[scale=0.12]{sym_droites2}\pause
			\end{center}
			
			\begin{itemize}
				\item Les points $A$, $B$ et $C$ sont alignés, donc $A'$, $B'$ et $C'$ leur symétriques par rapport à la droite $(e)$ sont aussi alignés.\pause
				\item La droite $(AB)$ est parallèle à la droite $(A'B')$.
			\end{itemize}
			\end{column}
		\end{columns}
	\end{myexs}
\end{frame}

\begin{frame}
	\begin{myprop}
		Le symétrique d'un segment par rapport à une droite ou un point est un segment de \kword{même longueur}. 
	\end{myprop}
	
	\begin{myex}
		\begin{center}
			\includegraphics[scale=0.1]{sym_seg}
		\end{center}
		
		Le segment $[A'B']$ est le symétrique du segment $[AB]$ par rapport à la droite $(d)$ et $[A'_1B'_1]$ le symétrique de $[AB]$ par rapport au point $O$. \pause
		Ils ont tous la même longueur
		
		
	\end{myex}
\end{frame}

\begin{frame}
	\begin{myprop}
		Le symétrique d'une figure par rapport à une droite ou un point est une figure de même forme. La symétrie \kword{conserve les angles}, \kword{les périmètres et les aires}.\pause
	\end{myprop}
	
	\begin{myex}
		\begin{center}
			\includegraphics[scale=0.1]{sym_figures}
		\end{center}
		
		La figure $F2$ est le symétrique de $F1$ par rapport à la droite $(d)$; $F3$ est le symétrique de $F1$ par rapport au point O.\pause
		Elles ont le même périmètre, la même aire et leurs angles ont la même mesure.
	\end{myex}
\end{frame}
\end{document}