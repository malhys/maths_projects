\documentclass[xcolor={dvipsnames}]{beamer}
%\usepackage[utf8]{inputenc}
\usetheme{Madrid}
%\usetheme{Malmoe}
\usecolortheme{beaver}
%\usecolortheme{rose}

\input{../../../../utils_maths_beamer}


\usepackage{../../../../pas-math}
\usepackage{../../../../moncours_beamer}

\usepackage{amssymb,amsmath}


\newcommand{\myitem}{\item[\textbullet]}

\graphicspath{{../img/}}

\title{Séquence 2 : Symétries}
%\author{O. FINOT}\institute{Collège S$^t$ Bernard}

%
\AtBeginSection[]
{
	\begin{frame}
		\frametitle{}
		\tableofcontents[currentsection, hideallsubsections]
	\end{frame} 

}
%
%
%\AtBeginSubsection[]
%{
%	\begin{frame}
%		\frametitle{Sommaire}
%		\tableofcontents[currentsection, currentsubsection]
%	\end{frame} 
%}

\begin{document}



\begin{frame}
  \titlepage 
\end{frame}


	



\section{Symétrie axiale}




\begin{frame}{}
	\begin{mydef}
		Deux figures sont \kword{symétriques par rapport à une droite $(d)$} si elles se superposent quand on plie le long de cette droite. 
		La droite $(d)$ est appelée \kword{axe de symétrie}.
		%	\begin{itemiz\begin{mydef}
		Deux figures sont \kword{symétriques par rapport à une droite $(d)$} si elles se superposent quand on plie le long de cette droite. 
		La droite $(d)$ est appelée \kword{axe de symétrie}.\pause
	
	\end{mydef}

	\begin{myex}
		\begin{center}
			\includegraphics[scale=.5]{fig1}
		\end{center}	
	\end{myex}


\end{frame}

\begin{frame}
	
	\begin{columns}
		\begin{column}{0.35\textwidth}
			%\begin{left}
				\includegraphics[scale=0.18]{def}
			%\end{left}
		\end{column}
	
		\begin{column}{0.65\textwidth}
			\begin{myprops}
				Soit $(d)$ une droite :\pause
				\begin{itemize}
					\item Si un point $A$ n'appartient pas à la droite $(d)$, alors son symétrique par rapport à la droite $(d)$ est le point $A'$ tel que \kword{$(d)$ est la médiatrice du segment $[AA']$}.\pause
					\item Si un point $B$ appartient à la droite $(d)$, alors son symétrique par rapport à la droite $(d)$ est \kword{lui même}.
				\end{itemize}
			\end{myprops}
		\end{column}
	\end{columns}
	
	
\end{frame}

\section{Symétrie centrale}

\begin{frame}
	\begin{mydef}
		
		Deux figures sont \kword{symétriques par rapport à un point $O$} si elles se superposent lorsqu'on effectue un demi-tour autour du point $O$. Le point $O$ est appelé \kword{centre de symétrie}.\pause
		
	\end{mydef}
	
	\begin{myex}
		\begin{center}
			\includegraphics[scale=.5]{fig2}
		\end{center}
	\end{myex}
\end{frame}
\end{document}