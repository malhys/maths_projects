\documentclass[xcolor={dvipsnames}]{beamer}
%\usepackage[utf8]{inputenc}
\usetheme{Madrid}
%\usetheme{Malmoe}
\usecolortheme{beaver}
%\usecolortheme{rose}

\input{../../../../utils_maths_beamer}


\usepackage{../../../../pas-math}
\usepackage{../../../../moncours_beamer}

\usepackage{amssymb,amsmath}


\newcommand{\myitem}{\item[\textbullet]}

\graphicspath{{../img/}}

\title{Chapitre 1 : Enchainement d'opérations}
%\author{O. FINOT}\institute{Collège S$^t$ Bernard}

%
\AtBeginSection[]
{
	\begin{frame}
		\frametitle{}
		\tableofcontents[currentsection, hideallsubsections]
	\end{frame} 

}
%
%
%\AtBeginSubsection[]
%{
%	\begin{frame}
%		\frametitle{Sommaire}
%		\tableofcontents[currentsection, currentsubsection]
%	\end{frame} 
%}

\begin{document}



\begin{frame}
  \titlepage 
\end{frame}


	



\section{Priorités des opérations}




\begin{frame}{}

	\begin{block}{Activité 1 Découvrir les priorités des opérations (1 p 38)}
		\begin{enumerate}\pause
			\item Tom a fait les calculs dans l'ordre ($8 + 2 = 10$, $10 \times 3 = 30$), et Alice a commencé par la multiplication ($2 \times 3 = 6$, $8 + 6 = 14$).\pause
			
			\item Une calculatrice scientifique donne le résultat 14, c'est donc Alice qui a raison.\pause
			
			\item On a :\pause
			\begin{columns}
				\begin{column}{0.25\textwidth}
					\begin{itemize}
						\myitem $A = 22 $;\pause
						\myitem $B = 13 $;\pause
						\myitem $C = 22 $;\pause
					\end{itemize}
				\end{column}
				\begin{column}{0.25\textwidth}
					\begin{itemize}
						\myitem $D = 22 $;\pause
						\myitem $E = 25 $;\pause
						\myitem $F = 14 $;\pause
					\end{itemize}
				\end{column}
				\begin{column}{0.25\textwidth}
					\begin{itemize}
						\myitem $G = 20 $;\pause
						\myitem $H = 9 $;\pause
						\myitem $I = 12 $;\pause
					\end{itemize}
				\end{column}
				\begin{column}{0.25\textwidth}
					\begin{itemize}
						\myitem $J = 21 $;\pause
						\myitem $K = 136 $;\pause
						\myitem $L = 4 $.
					\end{itemize}
				\end{column}
			\end{columns}
			
			\item Pour calculer une expression qui contient plusieurs opérations, on calcule les multiplications et les divisions avant les additions et les soustractions.
		\end{enumerate}
	\end{block}
\end{frame}

\begin{frame}{}
	\begin{alertblock}{Propriétés}
		\begin{itemize}\pause
			
			\myitem Dans une expression numérique sans parenthèses, on effectue :
			\begin{enumerate}
				\item D'abord les multiplications et les divisions, de gauche à droite;
				\item Puis les additions et les soustractions, également de gauche à droite.
			\end{enumerate}\pause
			
			\myitem On dit que la multiplication et la division sont \kword{prioritaires} sur l'addition et la soustraction. \pause
			
			\myitem Dans une expression sans parenthèses qui contient uniquement des additions ou uniquement des multiplications, on effectue les calculs dans l'ordre que l'on veut. L'addition et la multiplication sont \kword{commutatives}.\pause
			
		\end{itemize}
	\end{alertblock}
\end{frame}

\begin{frame}
	\begin{exampleblock}{Exemples}
		\begin{itemize}
			\myitem Je calcule l'expression $A= 20 - 2 \times 3 + 12 \div 6$ :\pause
			
			\vspace*{-0.5cm}
			
			\begin{eqnarray*}
				A &=& 20 - 2 \times 3 + 12 \div 6 \\ \pause
				A &=& 20 - 6 + 12 \div 6 \text{ (je commence par la multiplication) }\\ \pause
				A &=& 20 - 6 + 2 \text{ (ensuite la division)} \\ \pause
				A &=& 14 + 2 \text{ (puis le reste des opérations de gauche à droite)}\\ \pause
				A &=& 16 \pause
			\end{eqnarray*}
			
			\myitem Je calcule l'expression $B= 12 + 3 +8$ de trois façons différentes : \pause
			
			\vspace*{-0.5cm}
			\begin{columns}
				\begin{column}{0.33\textwidth}
					\begin{eqnarray*}
						B &=& \underline{12 + 3} + 8 \\
						B &=& 15 + 8 \\
						B &=& 23
					\end{eqnarray*}
				\end{column}\pause
				
				\begin{column}{0.33\textwidth}
					\begin{eqnarray*}
						B &=& 12 + \underline{3 + 8} \\
						B &=& 12 + 11 \\
						B &=& 23
					\end{eqnarray*}
				\end{column}\pause
				
				
				\begin{column}{0.33\textwidth}
					\begin{eqnarray*}
						B &=& \underline{12 + 8} + 3 \\
						B &=& 20 + 3 \\
						B &=& 23
					\end{eqnarray*}	
				\end{column}
				
			\end{columns}
			
		\end{itemize}
	\end{exampleblock}
\end{frame}

\section{Calculer une expression}


\begin{frame}
	\begin{alertblock}{Propriétés}
		\begin{itemize}
			\item Dans une expression numérique qui contient des parenthèses, on calcule :
			\begin{enumerate}
				\item d'abord les opérations entre parenthèses;\pause
				\item puis on calcule l'expression sans parenthèses obtenue.\pause
			\end{enumerate}
			
			\item Si l'expression contient des parenthèses imbriquées, on commence par celles qui sont le plus à l'intérieur.
		\end{itemize}
	\end{alertblock}

\begin{myex}
	Je calcule l'expression $C = (3 \times (7 - 3))  + 1$ :\pause
	
	\vspace*{-0.5cm}
	
	\begin{eqnarray*}
		C & = & (3  \times   \underbrace{(7-3)})  + 1 \\ \pause %\text{ (on commence par la parenthèse intérieure)} \\
		C & = & \quad \: \underbrace{(3 \times 4)}+ 1 \\ \pause%\text{ (puis l'autre)} \\
		C & = & \qquad \underbrace{12  + 1} \\ \pause%\text{ (enfin on calcule le reste de l'expression)}\\
		C & = & \qquad \quad 13
	\end{eqnarray*}
\end{myex}
\end{frame}

\section{Vocabulaire}


\begin{frame}
	\begin{alertblock}{Définition}
		Le résultat d'une \kword{addition} est une \kword{somme}, les nombres utilisés sont des \kword{termes}.\pause
	\end{alertblock}

	\begin{exampleblock}{Example}
		\begin{center}
			\includegraphics<2>[scale=0.8]{somme2}
			\includegraphics<3>[scale=0.8]{somme}
		\end{center}
	\end{exampleblock}

\end{frame}

\begin{frame}
	\begin{alertblock}{Définition}
		Une \kword{différence} est le résultat de la \kword{soustraction} de deux \kword{termes}.\pause
	\end{alertblock}
	
	\begin{exampleblock}{Example}
		\begin{center}
			\includegraphics<2>[scale=0.8]{difference2}
			\includegraphics<3>[scale=0.8]{difference}
		\end{center}
	\end{exampleblock}
	
\end{frame}


\begin{frame}
	\begin{alertblock}{Définition}
		Un \kword{produit} est le résultat de la \kword{multiplication} de deux \kword{facteurs}.\pause
	\end{alertblock}
	
	\begin{exampleblock}{Example}
		\begin{center}
			\includegraphics<2>[scale=0.8]{produit2}
			\includegraphics<3>[scale=0.8]{produit}
		\end{center}
	\end{exampleblock}
	
\end{frame}


\begin{frame}
	\begin{alertblock}{Définition}
		Le résultat de la \kword{division} d'un \kword{dividende} par un \kword{diviseur} est un \kword{quotient}.\pause
	\end{alertblock}
	
	\begin{exampleblock}{Example}
		\begin{center}
			\includegraphics<2>[scale=0.8]{quotient2}
			\includegraphics<3>[scale=0.8]{quotient}
		\end{center}
	\end{exampleblock}
	
\end{frame}

\begin{frame}
	\begin{myexs}
		\begin{itemize}
			\item L'expression $5 + 3 \times 4$ est \pause une somme, car la dernière opération effectuées est une addition.\pause
			
			\item L'expression $(2 + 3 ) \times 4$ est \pause un produit, car la dernière opération effectuées est une multiplication.\pause
			
			\item $ 3 \times (4 + 1)$ est \pause le produit de 3 par la somme de 4 et 1.\pause
			
			\item $ 3 \times 4 + 1$ est \pause la somme du produit de 1 par 3 et  4.\pause
			
			\item $(19 - 3) \div (2 \times 4)$ est \pause le quotient de la différence entre  19 et 3 par le produit de 2 par 4.
		\end{itemize}
	\end{myexs}
\end{frame}
\end{document}