\documentclass[12pt,a4paper]{extarticle}

%\usepackage[in, plain]{fullpage}
\usepackage{array}
%\usepackage{../../../pas-math}
%\usepackage{../../../moncours2}


%\usepackage{pas-cours}
%-------------------------------------------------------------------------------
%          -Packages nécessaires pour écrire en Français et en UTF8-
%-------------------------------------------------------------------------------
\usepackage[utf8]{inputenc}
\usepackage[frenchb]{babel}
\usepackage[T1]{fontenc}
\usepackage{lmodern}
%-------------------------------------------------------------------------------

%-------------------------------------------------------------------------------
%                          -Outils de mise en forme-
%-------------------------------------------------------------------------------
\usepackage{hyperref}
\hypersetup{pdfstartview=XYZ}
\usepackage{enumerate}
\usepackage{graphicx}
\usepackage{multicol}

\usepackage{anysize} %%pour pouvoir mettre les marges qu'on veut
%\marginsize{2.5cm}{2.5cm}{2.5cm}{2.5cm}

\usepackage{indentfirst} %%pour que les premier paragraphes soient aussi indentés
%-------------------------------------------------------------------------------


%-------------------------------------------------------------------------------
%                  -Nécessaires pour écrire des mathématiques-
%-------------------------------------------------------------------------------
\usepackage{amsfonts}
\usepackage{amssymb}
\usepackage{amsmath}
\usepackage{amsthm}
\usepackage{tikz}
%-------------------------------------------------------------------------------

%-------------------------------------------------------------------------------
%                     -Mise en forme d'exercices-
%-------------------------------------------------------------------------------
\newtheoremstyle{exostyle}
{\topsep}% espace avant
{\topsep}% espace apres
{}% Police utilisee par le style de thm
{}% Indentation (vide = aucune, \parindent = indentation paragraphe)
{\bfseries}% Police du titre de thm
{.}% Signe de ponctuation apres le titre du thm
{ }% Espace apres le titre du thm (\newline = linebreak)
{\thmname{#1}\thmnumber{ #2}\thmnote{. \normalfont{\textit{#3}}}}% composants du titre du thm : \thmname = nom du thm, \thmnumber = numéro du thm, \thmnote = sous-titre du thm

\theoremstyle{exostyle}
\newtheorem{exercice}{Exercice}

\newenvironment{questions}{
\begin{enumerate}[\hspace{12pt}\bfseries\itshape a.]}{\end{enumerate}
} %mettre un 1 à la place du a si on veut des numéros au lieu de lettres pour les questions 
%-------------------------------------------------------------------------------



%-------------------------------------------------------------------------------
%                    - Racourcis d'écriture -
%-------------------------------------------------------------------------------

% Angles orientés (couples de vecteurs)
\newcommand{\aopp}[2]{(\vec{#1}, \vec{#2})} %Les deuc vecteurs sont positifs
\newcommand{\aopn}[2]{(\vec{#1}, -\vec{#2})} %Le second vecteur est négatif
\newcommand{\aonp}[2]{(-\vec{#1}, \vec{#2})} %Le premier vecteur est négatif
\newcommand{\aonn}[2]{(-\vec{#1}, -\vec{#2})} %Les deux vecteurs sont négatifs

%Ensembles mathématiques
\newcommand{\naturels}{\mathbb{N}} %Nombres naturels
\newcommand{\relatifs}{\mathbb{Z}} %Nombres relatifs
\newcommand{\rationnels}{\mathbb{Q}} %Nombres rationnels
\newcommand{\reels}{\mathbb{R}} %Nombres réels
\newcommand{\complexes}{\mathbb{C}} %Nombres complexes
%-------------------------------------------------------------------------------





%\makeatletter
%\renewcommand*{\@seccntformat}[1]{\csname the#1\endcsname\hspace{0.1cm}}
%\makeatother

%\toggletrue{eleve}
%\toggletrue{dys}

\date{}
\title{Correction des exercices sur les expressions numériques}

\renewcommand{\labelitemi}{∙}
%\rfoot{Page \thepage}

\begin{document}
\maketitle

%\setenumerate[1]{label=\textbf{\arabic*)}}

%\chap[num=1, color=red]{Enchainement d'opérations}{Olivier FINOT, \today }

 \vspace*{-1cm}
\section*{Exercice 18}
 \vspace*{-1cm}
 
 
\begin{multicols}{3}
	\begin{eqnarray*}
		A &=&  5 + 8 - 4 \times 3 \\
		A &=& 5 + 8 - 12 \\
		A &=& 13 - 12 \\
		A &=& 1
	\end{eqnarray*}

	\begin{eqnarray*}
		B &=& 36 \div 6 + 7 \times 6 \\
		B &=& 6 + 42 \\
		B &=& 48 
	\end{eqnarray*}

	\begin{eqnarray*}
		C &=& 4 + 63 \div 9 + 2 \\
		C &=& 4 + 7 + 2 \\
		C &=& 11 + 2 \\
		C &=& 13
	\end{eqnarray*}

	\begin{eqnarray*}
		D &=&  81 - 11 \times 6 \div 3 \\
		D &=& 81 - 66 \div 3 \\
		D &=& 81 - 22 \\
		D &=& 59
	\end{eqnarray*}

	\begin{eqnarray*}
		E &=& 40 \div 8 + 8 \times 8 \\
		E &=& 5 + 64 \\
		E &=& 69
	\end{eqnarray*}

	\begin{eqnarray*}
		F &=&  12 \times 6 \div 8 \times 7 \\
		F &=& 72 \div 8 \times 7 \\
		F &=& 9 \times 7 \\
		F &=& 63
	\end{eqnarray*}
\end{multicols}

\section*{Exercice 19}
 \vspace*{-1cm}
 
\begin{multicols}{3}
	\begin{eqnarray*}
		A &=&  (1 + 4 \times 8) + 2 \\
		A &=& (1 + 32) + 2 \\
		A &=& 33 + 2 \\
		A &=& 35
	\end{eqnarray*}
	
	\begin{eqnarray*}
		B &=& 72 \div (16 \div 2) \\
		B &=& 72 \div 8 \\
		B &=& 9 
	\end{eqnarray*}
	
	\begin{eqnarray*}
		C &=& 7 \times 6 + (18 \div 9) \\
		C &=& 7 \times 6 + 2 \\
		C &=& 42 + 2 \\
		C &=& 44
	\end{eqnarray*}
	
	\begin{eqnarray*}
		D &=&  20 - (8 \times 4 - 20) \\
		D &=& 20 - (32 - 20) \\
		D &=& 20 - 12 \\
		D &=& 8
	\end{eqnarray*}
	
	\begin{eqnarray*}
		E &=& 35 \div 7 \times (47 - 12) \\
		E &=& 35 \div 7 \times 35 \\
		E &=& 5 \times 35 \\
		E &=& 175
	\end{eqnarray*}
	
	\begin{eqnarray*}
		F &=&  (15 + 2) \times 3 + 4 \\
		F &=& 17 \times 3 + 4 \\
		F &=& 51 + 4 \\
		F &=& 55
	\end{eqnarray*}
\end{multicols}

\newpage

\section*{Exercice 25}
\vspace*{-1cm}

\begin{multicols}{2}
	\begin{eqnarray*}
		A &=&  8 + (7 + 13) \div 4 \\
		A &=& 8 + 20 \div 4 \\
		A &=& 8 + 5 \\
		A &=& 13
	\end{eqnarray*}
	
	\begin{eqnarray*}
		B &=& 7 \times 3 - (6 + 63 \div 7) \\
		B &=& 7 \times 3 - (6 + 9)\\
		B &=& 7 \times 3 - 15 \\
		B &=& 21 - 15\\
		B &=& 6 
	\end{eqnarray*}
	
	\begin{eqnarray*}
		C &=& 80 - (80 - (3 \times (5 -2))) \\
		C &=& 80 - (80 - (3 \times 3)) \\
		C &=& 80 - (80 - 9) \\
		C &=& 80 - 71 \\
		C &=& 9 \\		
	\end{eqnarray*}
	
	\begin{eqnarray*}
		D &=&  (5 \times 6 + ((9 - 7) \times 4)) \div 2 \\
		D &=& (5 \times 6 + (2 \times 4)) \div 2 \\
		D &=& (5 \times 6 + 8) \div 2 \\
		D &=& (30 + 8) \div 2 \\
		D &=& 38 \div 2 \\
		D &=& 19
	\end{eqnarray*}
\end{multicols}


\section*{Exercice 31}
\vspace*{-1cm}

\begin{multicols}{2}
	\begin{eqnarray*}
		A &=& (35 + (9 \div 3)) - 2\\
		A &=& (35 + 3) - 2\\
		A &=& 38 - 2 \\
		A &=& 36
	\end{eqnarray*}
	
	\begin{eqnarray*}
		B &=& ((8 + 2 \times 4) \div 2) \times 3 \\
		B &=& ((8 + 8) \div 2) \times 3 \\
		B &=& (16 \div 2) \times 3 \\
		B &=& 8 \times 3 \\
		B &=& 24 
	\end{eqnarray*}
	
	\begin{eqnarray*}
		C &=& ((2+3)\times 2) - 3 \\
		C &=& (5\times 2) - 3 \\
		C &=& 10 - 3 \\
		C &=& 7
	\end{eqnarray*}
	
	\begin{eqnarray*}
		D &=& (12 - 11 + (10 - 9)) \times (8-7) \\
		D &=& (12 - 11 + 1) \times (8-7) \\
		D &=& (1 + 1) \times (8-7) \\
		D &=& 2 \times 2 \\
		D &=& 4
	\end{eqnarray*}
\end{multicols}

\section*{Exercice 34}

\begin{enumerate}[label = \alph*.]
	\item La différence entre 9 et 7.
	\item Le produit de 61 par 11.
	\item La somme de 36 et du quotient de 12 par 6.
	\item Le quotient  de la différence entre 56 par 3.
	\item La somme de la différence entre 31 et 23 et 4.
\end{enumerate}

\section*{Exercice 35}

%\begin{multicols}{2}
	\begin{enumerate}[label = \alph*.]
	\item La somme de 8 et du produit de 9 par 4.
	
	\begin{eqnarray*}
		A &=& 8 + 9 \times 4 \\
		A &=& 8 + 36 \\		
		A &=& 44
	\end{eqnarray*}


	\item La somme du quotient de 14 par 7 et de 3.
	
	\begin{eqnarray*}
		B &=& 14 \div 7 + 3 \\
		B &=& 2 + 3 \\		
		B &=& 5
	\end{eqnarray*}

	\item Le produit de la différence entre 23 et 17 par la somme de 7 et 3.
	\begin{eqnarray*}
		C &=& (23 - 17) \times (7 + 3) \\
		C &=& 6 \times 10 \\		
		C &=& 60
	\end{eqnarray*}
	
	\item La somme de 6 et de la différence entre 12 et le produit de 5 par 2.
	\begin{eqnarray*}
		D &=& 6 + (12 - 5 \times 2) \\
		D &=& 6 + (12 - 10) \\		
		D &=& 6 + 2 \\
		D &=& 8
	\end{eqnarray*}

\end{enumerate}
%\end{multicols}


\section*{Exercice 36}

\begin{enumerate}[label = \alph*.]
	\item Somme de 15 et de 7 : $15 + 7$
	\item Différence entre 15 et 7 : $15 - 7$
	\item Produit de 4 par la différence entre 15 et 9 : $4 \times (15 - 9)$
	\item Quotient de 15 par 7 : $15 \div 7$
\end{enumerate}

\section*{Exercice 38}

\begin{multicols}{4}
	\begin{enumerate}[label=\alph*.]
	\item \begin{eqnarray*}
		A &=& 5 \times 7 \\
		A &=& 35
	\end{eqnarray*}

	\item \begin{eqnarray*}
		B &=& 43 - 32 \\
		B &=& 11
	\end{eqnarray*}

	\item \begin{eqnarray*}
		C &=& 8 + 35 \\
		C &=& 33
	\end{eqnarray*}

	\item \begin{eqnarray*}
		D &=& 36 \div 4 \\
		D &=& 9
	\end{eqnarray*}
\end{enumerate}
\end{multicols}

\section*{Exercice 39}
	\begin{multicols}{3}
		\begin{enumerate}[label=\alph*.]
			
		\item \begin{eqnarray*}
			A &=& 7 \times (8 - 4) \\
			A &=& 7 \times 4 \\
			A &=& 28
		\end{eqnarray*}
		
		\item \begin{eqnarray*}
			B &=& (27 \div 9) + (12 + 2)\\
			B &=& 3 + 14\\
			B &=& 17
		\end{eqnarray*}
		
		\item \begin{eqnarray*}
			C &=& (5 \times (3 + 2)) - 10 \\
			C &=& (5 \times 5) - 10 \\
			C &=& 25 - 10 \\
			C &=& 15
		\end{eqnarray*}
		
	\end{enumerate}
	\end{multicols}

\section*{Exercice 40}

	\begin{multicols}{2}
		\begin{enumerate}[label=\alph*.]
			
			\item \begin{eqnarray*}
				A &=& 6 \times 4  - 8\\
				A &=& 24 - 8 \\
				A &=& 16
			\end{eqnarray*}
			
			\item \begin{eqnarray*}
				B &=& 7 - (4 - 2)\\
				B &=& 7 - 2\\
				B &=& 5
			\end{eqnarray*}
			
			\item \begin{eqnarray*}
				C &=& (7 + 4) \times (25 \div 5) \\
				C &=& 11 \times 5 \\
				C &=& 55
			\end{eqnarray*}
		
			\item \begin{eqnarray*}
				D &=& (8 \times 3) \div (5 + 1) \\
				D &=& 12 \div 6 \\
				D &=& 2
			\end{eqnarray*}
			
		\end{enumerate}
	\end{multicols}
\end{document}