\section{Sucreries (3 points bonus)}

Magali achète 7 paquets de gâteaux à \num{1.50} € pièce et 14 sucettes à \num{0.50} € pièce. Elle a payé avec un billet de 20 €.

\begin{questions}
	\question[\half] Que représente le calcul $14 \times \num{0.50}$ ?
	\begin{solution}
		Ce calcul représente le prix d'achat des 14 sucettes.
	\end{solution}

	\question[\half] Que représente le calcul $7 \times \num{1.50}$ ?
	\begin{solution}
		Ce calcul représente le prix d'achat des 7  paquets de gâteaux.
	\end{solution}
	
	\question[1] En n'utilisant que les nombres écrits dans l'énoncé écrire l'expression permettant de calculer la monnaie que la caissière lui rendra. (le résultat n'est pas attendu)
	\begin{solution}
		La monnaie que le caissière lui rendra est obtenue par l'expression suivante : $A = 20 - (7 \times \num{1.50} + 14 \times \num{0.50})$
	\end{solution}
		
	\question[1] Effectuer le calcul et conclure. 
	\begin{solution}
		\begin{eqnarray*}
			A &=& 20 - (7 \times \num{1.50} + 14 \times \num{0.50})\\
			A &=& 20 - (\num{10.50} + 7)\\
			A &=& 20 - \num{17.50}\\
			A &=& \num{2.50}
		\end{eqnarray*}
	
	La caissière lui rendra \num{2.50} €.
	\end{solution}
	
\end{questions}