\section{Problèmes (7 points)}

Pour chaque problème, écrire une expression qui permet de trouver la réponse puis le résoudre en concluant par une phrase.

\begin{questions}
	\question[1] Chloé achète trois livres à \num{5.20} € l'unité et un CD à \num{19.80} €. Elle a payé avec un billet de 50 €. 
	
	Quelle somme lui a-t-on rendue à la caisse ?
	
	\question[1\half] Pour récompenser les vainqueurs du cross du collège, le F.S.E. a acheté 8 coupes à 24 € l'unité et 16 médailles à \num{4.20} € l'unité.
	
	Quelle est la dépense totale du F.S.E. ?
	
	\question[1\half] Daniel a gagné \num{4630} € aux courses. Il décide de donner 400 € à l'association du Téléthon, de conserver la moitié du reste pour se payer un voyage, puis de distribuer la somme restante en parts égales à ses cinq petits enfants.
	
	Quelle somme reçoit chacun de ses petits enfants ?
	
	\question[1\half] Hassan a économisé \num{84.70} €. Il s'achète une raquette de tennis à \num{49.50} € et offre la moitié de la somme restante à son jeune frère. 
	
	Quelle somme lui reste-t-il ?
	
	\question[1\half] Emma a acheté trois livres identiques et a payé 36 € en tout. Vincent qui avait 150 €, achète un de ces livres. 
	
	Quelle somme reste-t-il à Vincent ?
	
%	\question[2] Théo doit lire un livre de 150 pages. Le lundi il lit 36 pages. Il le termine en lisant le même nombre de pages chacun des trois jours suivants. Combien de pages a-t-il lu chacun de ces trois jours ?
%	\begin{solution}
%		Le nombre de pages lues chacun de ces trois jours est obtenu en calculant l'expression $(150 - 36) \div 3 $.
%		
%		\begin{eqnarray*}
%			A &=& (150 - 36) \div 3 \\
%			A &=& 114 \div 3 \\
%			A &=& 38 \\
%		\end{eqnarray*}
%		
%		Il  a lu 38 pages.
%	\end{solution}
\end{questions}