\section{Expressions (5 points)}

Pour chacune des deux situations suivantes, écrire une seule expression permettant de répondre à la question posée :

\begin{questions}
	\question[1\half] Emma a acheté trois livres identiques et a payé 36 €. Vincent qui avait 150 €, achète un de ces livres. Quelle somme reste-t-il à Vincent ?
	
	\begin{solution}
		La somme qui reste à Vincent est obtenue en calculant l'expression $150 - 36 \div 3 $.
		
		\begin{eqnarray*}
		 	A &=& 150 - 36 \div 3 \\
		 	A &=& 150 - 12 \\
		 	A &=& 138 \\
		\end{eqnarray*}
	
		Il  lui reste 138 euros.
	\end{solution}
	
	\question[1\half] Dans une planche de 150 cm de long, Paul découpe trois morceaux de 36 cm de long. Quelle longueur reste-t-il ?
	\begin{solution}
		La longueur de planche restante est obtenue en calculant l'expression $150 - 36 \times 3 $.
		
		\begin{eqnarray*}
			A &=& 150 - 36 \times 3 \\
			A &=& 150 - 108 \\
			A &=& 42 \\
		\end{eqnarray*}
		
		Il  reste 42 cm.
	\end{solution}
	
	\question[2] Théo doit lire un livre de 150 pages. Le lundi il lit 36 pages. Il le termine en lisant le même nombre de pages chacun des trois jours suivants. Combien de pages a-t-il lu chacun de ces trois jours ?
	\begin{solution}
		Le nombre de pages lues chacun de ces trois jours est obtenu en calculant l'expression $(150 - 36) \div 3 $.
		
		\begin{eqnarray*}
			A &=& (150 - 36) \div 3 \\
			A &=& 114 \div 3 \\
			A &=& 38 \\
		\end{eqnarray*}
		
		Il  a lu 38 pages.
	\end{solution}
\end{questions}