\item Dans une expression numérique sans parenthèses, on effectue :
			\begin{enumerate}
				\item D'abord les multiplications et les divisions, de gauche à droite;
				\item Puis les additions et les soustractions, également de gauche à droite.
			\end{enumerate}		
		
		\item On dit que la multiplication et la division sont \kw{prioritaires} sur l'addition et la soustraction. 
		
		\item Dans une expression sans parenthèses qui contient uniquement des additions ou uniquement des multiplications, on effectue les calculs dans l'ordre que l'on veut. L'addition et la multiplication sont \kw{commutatives}.
		
	\end{itemize}


\begin{myexs}
	\begin{itemize}
		\item Je calcule l'expression $A= 20 - 2 \times 3 + 12 \div 6$
		
		\vspace*{-0.5cm}
		
		\begin{eqnarray*}
			A &=& 20 - 2 \times 3 + 12 \div 6 \\
			A &=& 20 - 6 + 12 \div 6 \text{ (je commence par la multiplication) }\\
			A &=& 20 - 6 + 2 \text{ (ensuite la division)} \\
			A &=& 14 + 2 \text{ (puis le reste des opérations de gauche à droite)}\\
			A &=& 16
		\end{eqnarray*}
		
		\item Je calcule l'expression $B= 12 + 3 +8$ de trois façons différentes :
		
		\vspace*{-0.5cm}
		\begin{multicols}{3}
			\begin{eqnarray*}
				B &=& \kw{12 + 3} + 8 \\
				B &=& 15 + 8 \\
				B &=& 23
			\end{eqnarray*}
			
			\begin{eqnarray*}
				B &=& 12 + \kw{3 + 8} \\
				B &=& 12 + 11 \\
				B &=& 23
			\end{eqnarray*}
			
			\begin{eqnarray*}
				B &=& \kw{12 + 8} + 3 \\
				B &=& 20 + 3 \\
				B &=& 23
			\end{eqnarray*}
		\end{multicols}
		
	\end{itemize}
\end{myexs}
