\documentclass[12pt,a4paper]{extarticle}

%\usepackage[in, plain]{fullpage}
\usepackage{array}
%\usepackage{../../../pas-math}
\usepackage{../../../moncours2}


%\usepackage{pas-cours}
%%-------------------------------------------------------------------------------
%          -Packages nécessaires pour écrire en Français et en UTF8-
%-------------------------------------------------------------------------------
\usepackage[utf8]{inputenc}
\usepackage[frenchb]{babel}
\usepackage[T1]{fontenc}
\usepackage{lmodern}
\usepackage{textcomp}



%-------------------------------------------------------------------------------

%-------------------------------------------------------------------------------
%                          -Outils de mise en forme-
%-------------------------------------------------------------------------------
\usepackage{hyperref}
\hypersetup{pdfstartview=XYZ}
%\usepackage{enumerate}
\usepackage{graphicx}
\usepackage{multicol}
\usepackage{tabularx}
\usepackage{multirow}


\usepackage{anysize} %%pour pouvoir mettre les marges qu'on veut
%\marginsize{2.5cm}{2.5cm}{2.5cm}{2.5cm}

\usepackage{indentfirst} %%pour que les premier paragraphes soient aussi indentés
\usepackage{verbatim}
\usepackage{enumitem}
\usepackage[usenames,dvipsnames,svgnames,table]{xcolor}

\usepackage{variations}

%-------------------------------------------------------------------------------


%-------------------------------------------------------------------------------
%                  -Nécessaires pour écrire des mathématiques-
%-------------------------------------------------------------------------------
\usepackage{amsfonts}
\usepackage{amssymb}
\usepackage{amsmath}
\usepackage{amsthm}
\usepackage{tikz}
\usepackage{xlop}
%-------------------------------------------------------------------------------



%-------------------------------------------------------------------------------


%-------------------------------------------------------------------------------
%                    - Mise en forme avancée
%-------------------------------------------------------------------------------

\usepackage{ifthen}
\usepackage{ifmtarg}


\newcommand{\ifTrue}[2]{\ifthenelse{\equal{#1}{true}}{#2}{$\qquad \qquad$}}

%-------------------------------------------------------------------------------

%-------------------------------------------------------------------------------
%                     -Mise en forme d'exercices-
%-------------------------------------------------------------------------------
%\newtheoremstyle{exostyle}
%{\topsep}% espace avant
%{\topsep}% espace apres
%{}% Police utilisee par le style de thm
%{}% Indentation (vide = aucune, \parindent = indentation paragraphe)
%{\bfseries}% Police du titre de thm
%{.}% Signe de ponctuation apres le titre du thm
%{ }% Espace apres le titre du thm (\newline = linebreak)
%{\thmname{#1}\thmnumber{ #2}\thmnote{. \normalfont{\textit{#3}}}}% composants du titre du thm : \thmname = nom du thm, \thmnumber = numéro du thm, \thmnote = sous-titre du thm

%\theoremstyle{exostyle}
%\newtheorem{exercice}{Exercice}
%
%\newenvironment{questions}{
%\begin{enumerate}[\hspace{12pt}\bfseries\itshape a.]}{\end{enumerate}
%} %mettre un 1 à la place du a si on veut des numéros au lieu de lettres pour les questions 
%-------------------------------------------------------------------------------

%-------------------------------------------------------------------------------
%                    - Mise en forme de tableaux -
%-------------------------------------------------------------------------------

\renewcommand{\arraystretch}{1.7}

\setlength{\tabcolsep}{1.2cm}

%-------------------------------------------------------------------------------



%-------------------------------------------------------------------------------
%                    - Racourcis d'écriture -
%-------------------------------------------------------------------------------

% Angles orientés (couples de vecteurs)
\newcommand{\aopp}[2]{(\vec{#1}, \vec{#2})} %Les deuc vecteurs sont positifs
\newcommand{\aopn}[2]{(\vec{#1}, -\vec{#2})} %Le second vecteur est négatif
\newcommand{\aonp}[2]{(-\vec{#1}, \vec{#2})} %Le premier vecteur est négatif
\newcommand{\aonn}[2]{(-\vec{#1}, -\vec{#2})} %Les deux vecteurs sont négatifs

%Ensembles mathématiques
\newcommand{\naturels}{\mathbb{N}} %Nombres naturels
\newcommand{\relatifs}{\mathbb{Z}} %Nombres relatifs
\newcommand{\rationnels}{\mathbb{Q}} %Nombres rationnels
\newcommand{\reels}{\mathbb{R}} %Nombres réels
\newcommand{\complexes}{\mathbb{C}} %Nombres complexes


%Intégration des parenthèses aux cosinus
\newcommand{\cosP}[1]{\cos\left(#1\right)}
\newcommand{\sinP}[1]{\sin\left(#1\right)}


%Probas stats
\newcommand{\stat}{statistique}
\newcommand{\stats}{statistiques}
%-------------------------------------------------------------------------------

%-------------------------------------------------------------------------------
%                    - Mise en page -
%-------------------------------------------------------------------------------

\newcommand{\twoCol}[1]{\begin{multicols}{2}#1\end{multicols}}


\setenumerate[1]{font=\bfseries,label=\textit{\alph*})}
\setenumerate[2]{font=\bfseries,label=\arabic*)}


%-------------------------------------------------------------------------------
%                    - Elements cours -
%-------------------------------------------------------------------------------





%\makeatletter
%\renewcommand*{\@seccntformat}[1]{\csname the#1\endcsname\hspace{0.1cm}}
%\makeatother

%\toggletrue{eleve}
%\toggletrue{dys}

\date{}
\title{\textcircled{{\normalsize{1}}}Calculs et priorités}

\renewcommand{\labelitemi}{∙}
%\rfoot{Page \thepage}

\begin{document}
\maketitle

\setenumerate[1]{label=\textbf{\arabic*)}}

%\chap[num=1, color=red]{Enchainement d'opérations}{Olivier FINOT, \today }

\vspace*{-0.5cm}
\begin{myobj}
	\begin{itemize}
		
		\item Construire le symétrique d’un point ou d'une figure par rapport à une droite à la main où à l’aide d’un logiciel;
		\item Construire le symétrique d’un point ou d'une figure par rapport à un point, à la main où à l’aide d’un logiciel;
		\item Utiliser les propriétés de la symétrie axiale ou centrale;
		\item Identifier des symétries dans des figures.		
	\end{itemize}
\end{myobj}

\begin{mycomp}
	\begin{itemize}
		\item \kw{Chercher (Ch2)} :  s’engager    dans    une    démarche    scientifique, observer, questionner, manipuler, expérimenter (sur une feuille de papier, avec des objets, à l’aide de logiciels), émettre des hypothèses, chercher des exemples ou des contre-exemples, simplifier ou particulariser une situation, émettre une conjecture ;
		\item \kw{Raisonner (Ra3)} :  démontrer : utiliser un raisonnement logique et des règles établies (propriétés, théorèmes, formules) pour parvenir à une conclusion ;
		\item \kw{Communiquer (Co2)} :  expliquer à l’oral ou à l’écrit (sa démarche, son raisonnement, un calcul, un protocole   de   construction   géométrique, un algorithme), comprendre les explications d’un autre et argumenter dans l’échange ; 
		
	\end{itemize}
\end{mycomp}



\vspace*{-0.5cm}

\section{Priorités des opérations}

%\begin{myact}{1 Découvrir les priorités des opérations}
%
%	Activité 1 page 38
%\end{myact}
%
%\begin{myactrep}{1 Découvrir les priorités des opérations}
%	
%	\begin{enumerate}
%		\item Tom a fait les calculs dans l'ordre ($8 + 2 = 10$, $10 \times 3 = 30$), et Alice a commencé par la multiplication ($2 \times 3 = 6$, $8 + 6 = 14$).
%		
%		\item Une calculatrice scientifique donne le résultat 14, c'est donc Alice qui a raison.
%		
%		\item On a :
%		
%		\begin{multicols}{4}
%			\begin{itemize}
%				\item $A = 22 $;
%				\item $B = 13 $;
%				\item $C = 22 $;
%				\item $D = 22 $;
%				\item $E = 25 $;
%				\item $F = 14 $;
%				\item $G = 20 $;
%				\item $H = 9 $;
%				\item $I = 12 $;
%				\item $J = 21 $;
%				\item $K = 136 $;
%				\item $L = 4 $.
%			\end{itemize}
%		\end{multicols}
%	
%		\item 
%		
%		\item Pour calculer une expression qui contient plusieurs opérations, on calcule les multiplications et les divisions avant les additions et les soustractions.
%	\end{enumerate}
%\end{myactrep}

\begin{myprop}
	Dans une suite d'additions ou de multiplications, l'ordre des calculs n'a pas d'importance.		
\end{myprop}


\begin{myexs}
	Calculer $2 + \num{3.4} + 8 + \num{6.6} + 5$ et $\num{2.5} \times 5 \times 2$.\\
	
	\iftoggle{eleve}{%
		
			\vspace*{2cm}
		
	}{%
		\begin{itemize}[label=\textbullet]
			\item $2 + \num{3.4} + 8 + \num{6.6} + 5 = 2 + 8 + \num{3.4} + \num{6.6} + 5 = 25$ \\
			
			
			\item $\num{2.5} \times 5 \times 2 = 5 \times \num{2.5}  \times 2 = 25$ 
		\end{itemize}
		
		
		
		
	}
	
\end{myexs}

%\vspace*{-0.5cm}

\begin{myprop}
	Dans une suite de calculs qui contient uniquement des additions et des soustractions on effectue les calculs dans l'ordre d'écriture (de gauche à droite).
\end{myprop}


\newpage

\begin{myexs}
	Calculer $2 + 8 - 3 + 7 - 5$ et $\num{2.5} \times 10 \div 5 \times 2$.\\
	
	\iftoggle{eleve}{%
		
		\vspace*{2cm}
		
	}{%
		\begin{itemize}[label=\textbullet]
			 \item $2 + 8 - 3 + 7 - 5 = 10 - 3 + 7 - 5 = 7 + 7 - 5 = 14 - 5 = 9$ \\
			
			
			 \item $ \num{2.5} \times 10 \div 5 \times 2 = 25 \div 5 \times 2 = 5 \times 2 = 10 $
		\end{itemize}
		
		
		
		
	}

\end{myexs}



\begin{myprop}
	Dans une suite de calculs sans parenthèses on effectue les multiplications et les divisions avant les additions et les soustractions
\end{myprop}

\begin{myexs}
	Calculer $4 + 5 \times 3$ et $3 + 8 \div 2 - 2 \times 2$.\\
	
	\iftoggle{eleve}{%
		
		\vspace*{2cm}
		
	}{%
		\begin{itemize}[label=\textbullet]
			\item $4 + 5 \times 3 = 4 + 15 = 19$ \\
			
			
			\item $ 3 + 8 \div 2 - 2 \times 2 = 3 + 4 - 4 = 3 $
		\end{itemize}
		
		
		
		
	}
\end{myexs}



\begin{myprop}
	Dans une suite de calculs on effectue d'abord les calculs entre parenthèses. On commence toujours par les parenthèses les plus à l'intérieur.
\end{myprop}

\begin{myexs}
	Calculer $(4 + 5) \times 3$ et $(3 + 8 \div (6 - 2)) \times 2$.\\
	
	\iftoggle{eleve}{%
		
		\vspace*{2cm}
		
	}{%
		\begin{itemize}[label=\textbullet]
			\item $(4 + 5) \times 3 = 9 \times 3 = 27$ \\
			
			
			\item $ (3 + 8 \div (6 - 2)) \times 2 = (3 + 8 \div 4) \times 2 = (3 + 2) \times 2 = 5 \times 2 = 10$
		\end{itemize}	
		
	}
\end{myexs}
%\begin{myexos}
%	\begin{itemize}
%		\item \exo{1}{42} : Calcul mental
%		\item Exercices 2 - 5 page 42 : Calcul d'expressions diverses
%		\item Exercices 9-10 page 43 : Expressions à compléter
%		\item \Exo{11}{43} : traduire un problème en expression et la calculer
%		\item \exo{12}{43} : première approche calcul littéral
%		\item \exo{15}{43} : problème
%	\end{itemize}
%\end{myexos}


%\section{Calculer une expression}
%
%
%\begin{myact}{Expression avec des parenthèses}
%	(Activité au tableau)
%	
%	\begin{enumerate}
%		\item Calculer les expression suivantes sans calculatrice
%		
%		\begin{multicols}{2}
%			\begin{itemize}
%				\item $(10 - 2)  \times 2$
%				\item $ 10 - (2 \times 2 )$
%				\item $ (3 -2) \times (3 + 2)$
%				\item $ 3 - 2 \times (3 + 2)$
%			\end{itemize}
%		\end{multicols}
%	
%		\item En utilisant uniquement 4 fois le chiffre 4 des opérations $(+, -, \times , \div)$ et des parenthèses :
%			\begin{enumerate}
%				\item trouver 0;
%				\item trouver tous les nombres entiers de 0 à 9 inclus;
%				\item obtenir 0 du plus grand nombre de façons possibles;
%				\item trouver le plus possible de nombres entiers inférieurs à 100.
%			\end{enumerate}
%	\end{enumerate}
%\end{myact}
%
%%\begin{myactrep}
%%	\begin{enumerate}
%%		\item 
%%			\begin{enumerate}
%%				\item On a $4 + 4 + 4 - 4 = 8$.
%%				
%%				\item 
%%			\end{enumerate}
%%	\end{enumerate}
%%\end{myactrep}
%
%\begin{myprop}
%	\begin{itemize}
%		\item Dans une expression numérique qui contient des parenthèses, on calcule :
%		\begin{enumerate}
%			\item d'abord les calculs entre parenthèses;
%			\item puis on calcule l'expression sans parenthèses obtenue
%		\end{enumerate}
%	
%		\item Si l'expression contient des parenthèses imbriquées, on commence par celles qui sont le plus à l'intérieur.
%	\end{itemize}
%	
%	
%\end{myprop}


%\begin{myexos}
%	\begin{itemize}
%		\item \Exo{16}{44} : Calcul mental
%		\item Exercices 17, 19 page 44 : Calcul d'expressions diverses
%		\item \Exo{16}{44} : Placer des parenthèses pour obtenir le bon résultat  
%		\item \Exo{18}{44} : faire le lien entre plusieurs écritures d'une même expression 
%		\item \Exo{21}{45} : passer d'une situation à une expression numérique
%		\item \Exo{23}{45} : Le compte est bon
%		\item \Exo{26}{45} : problème
%		\item \Exo{27}{45} : problème
%	\end{itemize}
%\end{myexos}

\newpage

\section{Vocabulaire}

\iftoggle{eleve}{%
	
	\begin{mydef}
		Une \kw{expression numérique} est 
	\end{mydef}
	
	\begin{myex}
		Je calcule l'expression $A = (3 \times (7 - 3))  + 1$ :
		
		\vspace*{1cm}
		\ligne[24pt]{4}
		
	\end{myex}
	
}{%
	\begin{mydef}
		Une \kw{expression numérique} est une suite de calculs.
	\end{mydef}
	
	\begin{myex}
		Je calcule l'expression $C = (3 \times (7 - 3))  + 1$ :
		
		\vspace*{-0.5cm}
		
		\begin{eqnarray*}
			C & = & (3 \times (7-3))  + 1   \\
			C & = & (3 \times 4)+ 1  \\
			C & = & 12  + 1 \\
			C & = & 13
		\end{eqnarray*}
	\end{myex}
}

\iftoggle{eleve}{%
	
	\begin{mydefs}
		
		\begin{itemize}
			\item Le résultat d'une \kw{addition} est une 
			\vspace*{1.5cm}
			
			\item Une \kw{différence} est le résultat de 
			\vspace*{1.5cm}
			
			\item Un \kw{produit} est le résultat de la 
			\vspace*{1.5cm}
			
			\item Le résultat de la \kw{division} d'un 
			\vspace*{1.5cm}
		\end{itemize}
		
		
	\end{mydefs}
	
}{%

	\begin{mydefs}
		
		\begin{itemize}
			\item Le résultat d'une \kw{addition} est une \kw{somme}, les nombres utilisés sont des \kw{termes}.
			
			\item Une \kw{différence} est le résultat de la \kw{soustraction} de deux \kw{termes}.
			
			\item Un \kw{produit} est le résultat de la \kw{multiplication} de deux \kw{facteurs}.
			
			\item Le résultat de la \kw{division} d'un \kw{dividende} par un \kw{diviseur} est un \kw{quotient}.
		\end{itemize}
	
		
	\end{mydefs}
}
%\newpage



	\begin{myexs}
		\begin{center}
			
			\begin{multicols}{2}
				\includegraphics*[scale=0.7]{img/somme}
				
				\includegraphics*[scale=0.7]{img/difference}
				
				\includegraphics*[scale=0.7]{img/produit}			
				
				\includegraphics*[scale=0.7]{img/somme}
				
			\end{multicols}
			
		\end{center}
	
		
	
\iftoggle{eleve}{%
	\begin{itemize}
		\item L'expression $5 + 3 \times 4$ est 
		
		\vspace*{1.5cm}
		
		\item L'expression $(2 + 3 ) \times 4$ est 
		
		\vspace*{1.5cm}
		
%		\item $ 3 \times (4 + 1)$ est 
%		
%		\vspace*{1.5cm}
%		
%		\item $ 3 \times 4 + 1$ est 
%		
%		\vspace*{1.5cm}
		
		\item $\dfrac{19 - 3}{2 \times 4}$ est 
		
		\vspace*{1.5cm}
	\end{itemize}
}{%


	\begin{itemize}
		\item L'expression $5 + 3 \times 4$ est la somme de  et du produit de 3 par 4.
		
		\item L'expression $(2 + 3 ) \times 4$ est le produit de la somme de 2 et 3 par 4.
		
%		\item $ 3 \times (4 + 1)$ est le produit de 3 par la somme de 4 et 1.
%		
%		\item $ 3 \times 4 + 1$ est la somme du produit de 1 par 3 et  4.
		
		\item $\dfrac{19 - 3}{2 \times 4}$ est le quotient de la différence entre  19 et 3 par le produit de 2 par 4.
	\end{itemize}


}

\end{myexs}		
%\begin{myexos}
%	\begin{itemize}
%		\item Exercices 28-30 page 46 : identifier un type d'expression.
%		\item \Exo{31}{46} : Description d'une expression.
%		\item \Exo{32}{46} : Lien entre description et expression.
%		\item \Exo{33}{46} : Donner l'expression décrite.
%		\item Exercices 34 - 35 page 47 : Arbre à calcul <-> expression
%		\item \Exo{36}{47} : Vrai ou faux
%		\item \Exo{37}{47} : traduire la description d'un calcul en expression (Intéressant pour faire le lien avec algo).
%	\end{itemize}
%\end{myexos}
\end{document}