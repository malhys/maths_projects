\documentclass[12pt,a4paper]{article}

\usepackage[in, plain]{fullpage}
\usepackage{array}
%\usepackage{../../../pas-math}
\usepackage{../../../moncours2}

%\makeatletter
%\renewcommand*{\@seccntformat}[1]{\csname the#1\endcsname\hspace{0.1cm}}
%\makeatother


%\author{Olivier FINOT}
\date{}
\title{\textcircled{{\normalsize{8}}} Calcul littéral}

\graphicspath{{./img/}}
%
%\rfoot{Page \thepage}
\begin{document}
%\maketitle






\begin{myobj}
	\begin{itemize}
		
		\item Construire le symétrique d’un point ou d'une figure par rapport à une droite à la main où à l’aide d’un logiciel;
		\item Construire le symétrique d’un point ou d'une figure par rapport à un point, à la main où à l’aide d’un logiciel;
		\item Utiliser les propriétés de la symétrie axiale ou centrale;
		\item Identifier des symétries dans des figures.		
	\end{itemize}
\end{myobj}

\begin{mycomp}
	\begin{itemize}
		\item \kw{Chercher (Ch2)} :  s’engager    dans    une    démarche    scientifique, observer, questionner, manipuler, expérimenter (sur une feuille de papier, avec des objets, à l’aide de logiciels), émettre des hypothèses, chercher des exemples ou des contre-exemples, simplifier ou particulariser une situation, émettre une conjecture ;
		\item \kw{Raisonner (Ra3)} :  démontrer : utiliser un raisonnement logique et des règles établies (propriétés, théorèmes, formules) pour parvenir à une conclusion ;
		\item \kw{Communiquer (Co2)} :  expliquer à l’oral ou à l’écrit (sa démarche, son raisonnement, un calcul, un protocole   de   construction   géométrique, un algorithme), comprendre les explications d’un autre et argumenter dans l’échange ; 
		
	\end{itemize}
\end{mycomp}




\section{Expression littérale}


\begin{mydef}
	Une \kw{expression littérale} est une expression numérique qui \kw{contient une ou plusieurs lettres}. Ces lettres désignent des nombres, ce sont des \kw{inconnues}.
\end{mydef}

\begin{myexs}
	Les formules de calcul de périmètres et d'aires sont des exemples d'expressions littérales :
	\begin{itemize}
		\item Aire d'un rectangle : $L \times l$;
		\item périmètre d'un cercle : $2 \times r \times \pi$ (ici $\pi$ n'est pas une inconnue qui désigne un nombre, c'est un nombre).
	\end{itemize}
\end{myexs}

\section{Simplifications d'écritures}

\subsection{Symbole de multiplication}
\begin{mymeth}
	Par convention, on simplifie l'écriture d'une expression littérale en supprimant au maximum les symboles <<$\times$>> :
	\begin{itemize}
		\item $3 \times a $ s'écrit $3a$;
		\item $a \times b $ s'écrit $ab$;
		\item $4 \times (a - 2)$ s'écrit $4 \times (a - 2)$ (se lit 4 \textit{facteur} de a moins 2)
		\item $15 + 4 \times a$ s'écrit $15 + 4a$
	\end{itemize}
\end{mymeth}

\begin{myrem}
	
	\begin{itemize}
		\item $2 \times 3$ ne s'écrit pas $23$;
		\item On écrit $2a$, on n'écrit pas $a2$;
		\item[$\rightarrow$] \kw{Le nombre s'écrit toujours devant la lettre.} 
	\end{itemize}
\end{myrem}


\subsection{Nombres au carré et au cube}

\begin{mymeth}
	
	\begin{itemize}
		\item $3 \times 3$ s'écrit $3^2$;
		\item $6 \times 6$ s'écrit $6^2$;
		\item $5 \times 5 \times 5$ s'écrit $5^3$;
	\end{itemize}
	
	\vspace*{0.25cm}
	
	\begin{itemize}
		
		\item $x \times x$ s'écrit $x^2$ (et se lit << $x$ au carré>>);
		\item $x \times x \times x$ s'écrit $x^3$ (et se lit << $x$ au cube>>).
	\end{itemize}
\end{mymeth}

\begin{myexs}
	\begin{eqnarray*}
		a \times 4 \times 2 \times a &=& 4 \times 2 \times  a \times a \\
		a \times 4 \times 2 \times a &=& 8a^2
	\end{eqnarray*}
	
	\begin{eqnarray*}
		a \times 2 \times 3 \times a \times b &=& 2 \times 3 \times a \times a \times b\\
		a \times 2 \times 3 \times a \times b &=& 6a^2b
	\end{eqnarray*}
	
	
	\begin{eqnarray*}
		a \times 2 \times a \times (7 - 1) &=& 2 \times (7 - 1) \times a \times a\\
		a \times 2 \times a \times (7 - 1) &=& 12a^2
	\end{eqnarray*}


	\begin{eqnarray*}
		a \times 2 \times (2 + b + 3) \times a &=& 2 \times a \times a \times (2 + 3 + b)\\
		a \times 2 \times (2 + b + 3) \times a &=& 2a(5+b)
	\end{eqnarray*}
\end{myexs}

\newpage

\section{Valeur d'une expression littérale}

	\begin{mydef}
		\kw{Calculer la valeur} d'une expression littérale c'est assigner une valeur à chaque lettre pour pouvoir effectuer les calculs.
	\end{mydef}

	\begin{myrems}
		\begin{itemize}
			\item Lorsque qu'une même lettre est présente plusieurs fois dans la même expression, elle désigne toujours le même nombre.
			\item Quand on veut calculer une expression littérale, il est indispensable d'écrire les symboles de multiplications sous entendues. Quand on multiplie deux nombres, le symbole <<$\times$>> doit être présent.
		\end{itemize}
	\end{myrems}


	\begin{myex}
		Calculer la valeur de $5x^2 + 3(x-1)+4y^3 $, avec $x=4$ et $y=10$ :
		
		\begin{align*}
			A &= 5 \times x \times x + 3 \times (x - 1) + 4 \times y \times y \times y \\ %& \text{On ajoute des '$\times$' sous entendus}\\
			A &= 5 \times 4 \times 4 + 3 \times (4 - 1) + 4 \times 10 \times 10 \times 10 \\% & \text{On remplace les <<$x$>> par 4 et les <<$y$>> par 10}\\
			A &= 80 + 3 \times 3 + \num{4000} \\ %& \text{On effectue les calculs}\\
			A &= 80 + 9 + \num{4000} \\
			A &= \num{4089}
		\end{align*}
	\end{myex}



\section{Vérifier une égalité}

\begin{mydef}
	Une \kw{égalité} est composée de deux membres séparés par le symbole $=$.
	Une égalité est \kw{vraie} lorsque ses deux membres ont la même valeur.
\end{mydef}


\begin{myexs}
	\begin{enumerate}
		\item $4 \times 10 = 100 - 60$ est une \kw{égalité vraie} car $4 \times 10 = 40 $ et $100 - 60 = 40$.
		
		
		\item $4 \times 10 = 40 + 3$ est une \kw{égalité fausse} car $4 \times 10 = 40 $ et $40 + 3 = 43$.
	\end{enumerate}
\end{myexs}

\begin{myprops}
	\begin{itemize}
		\item Une égalité d'expressions littérales est vraie si elle est \kw{vraie pour toutes les valeurs attribuées aux lettres}.
		
		\item Il suffit de trouver un \kw{contre-exemple} pour montrer que deux expressions ne sont \kw{pas égales}.	
	\end{itemize}	
\end{myprops}


\begin{mymeth}
	Pour Vérifier que deux expressions littérales sont égales :
	
	\begin{enumerate}
		\item On choisit une valeur à attribuer à chaque inconnue.
		\item On calcule les valeurs des deux expressions.
		\item Si \kw{les valeurs sont différentes}, alors on a terminé : \kw{l'égalité est fausse}.
		\item Si les valeurs sont les mêmes, on simplifie les deux expressions.
		\item Si les deux expressions simplifiées sont les mêmes, alors \kw{l'égalité est vraie}.
	\end{enumerate} 
\end{mymeth}

\begin{myex}
	
		On veut vérifier l'égalité : $2 + 3x = 5x$ :
			\begin{enumerate}
				\item Je choisis $x=0$.
				\item 
				
					\begin{multicols}{2}
						\begin{align*}
							2 + 3x &= 2 + 3 \times x \\
							2 + 3x &= 2 + 3 \times 0 \\
							2 + 3x &= 2 + 0 \\
							2 + 3x &= 2
						\end{align*}
						
						
						\begin{align*}
							5x &= 5 \times x \\
							5x &= 5 \times 0 \\
							5x &= 0 \\							
						\end{align*}
					\end{multicols}
				
				\item Les valeurs sont différentes donc \kw{l'égalité est fausse}.
			\end{enumerate}
	
\end{myex}


\begin{myex}
	
	On veut vérifier l'égalité : $2 + 4x + 3 = \num{1.5} \times x \times 2 + x + 5$ :
	\begin{enumerate}
		\item Je choisis $x=0$.
		\item \ 
		
		\vspace*{-2cm}		
		
		\begin{multicols}{2}
			\begin{align*}
			A &= 2 + 4x + 3  \\
			A &= 2 + 4 \times x + 3  \\
			A  &= 2 + 4 \times 0 + 3  \\
			A  &= 2 + 0 + 3  \\
			A  &= 5
			\end{align*}
			
			
			\begin{align*}
			B &= \num{1.5} \times x \times 2 + x + 5 \\
			B &= \num{1.5} \times 0 \times 2 + 0 + 5 \\
			B &= 0 + 0 + 5 \\
			B &= 5 \\							
			\end{align*}
		\end{multicols}
		
		\item Les valeurs sont égales, donc \kw{l'égalité est vraie pour $x=0$}.
		
		\item \begin{multicols}{2}
			\begin{align*}
				A &= 2 + 4x + 3  \\
				A &= 2 + 3 + 4x  \\
				A  &= 5 + 4x  \\
				A  &= 4x + 5  \\	
			\end{align*}
			
			
			\begin{align*}
				B &= \num{1.5} \times x \times 2 + x + 5 \\
				B &= \num{1.5} \times 2 \times x + x + 5 \\
				B &= 3x + x + 5 \\
				B &= (x + x + x) + x + 5 \\						
				B &= 4x + 5 
			\end{align*}
		\end{multicols}
	
		\item Les deux expressions simplifiées sont les mêmes, donc \kw{l'égalité est vraie (pour tout $x$)}.
	\end{enumerate}
	
\end{myex}
\end{document}

