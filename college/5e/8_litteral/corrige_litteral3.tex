\documentclass[12pt,a4paper]{article}


\usepackage[in, plain]{fullpage}
\usepackage{array}
\usepackage{../../../pas-math}

%-------------------------------------------------------------------------------
%          -Packages nécessaires pour écrire en Français et en UTF8-
%-------------------------------------------------------------------------------
\usepackage[utf8]{inputenc}
\usepackage[frenchb]{babel}
\usepackage[T1]{fontenc}
\usepackage{lmodern}
%-------------------------------------------------------------------------------

%-------------------------------------------------------------------------------
%                          -Outils de mise en forme-
%-------------------------------------------------------------------------------
\usepackage{hyperref}
\hypersetup{pdfstartview=XYZ}
\usepackage{enumerate}
\usepackage{graphicx}
\usepackage{multicol}

\usepackage{anysize} %%pour pouvoir mettre les marges qu'on veut
%\marginsize{2.5cm}{2.5cm}{2.5cm}{2.5cm}

\usepackage{indentfirst} %%pour que les premier paragraphes soient aussi indentés
%-------------------------------------------------------------------------------


%-------------------------------------------------------------------------------
%                  -Nécessaires pour écrire des mathématiques-
%-------------------------------------------------------------------------------
\usepackage{amsfonts}
\usepackage{amssymb}
\usepackage{amsmath}
\usepackage{amsthm}
\usepackage{tikz}
%-------------------------------------------------------------------------------

%-------------------------------------------------------------------------------
%                     -Mise en forme d'exercices-
%-------------------------------------------------------------------------------
\newtheoremstyle{exostyle}
{\topsep}% espace avant
{\topsep}% espace apres
{}% Police utilisee par le style de thm
{}% Indentation (vide = aucune, \parindent = indentation paragraphe)
{\bfseries}% Police du titre de thm
{.}% Signe de ponctuation apres le titre du thm
{ }% Espace apres le titre du thm (\newline = linebreak)
{\thmname{#1}\thmnumber{ #2}\thmnote{. \normalfont{\textit{#3}}}}% composants du titre du thm : \thmname = nom du thm, \thmnumber = numéro du thm, \thmnote = sous-titre du thm

\theoremstyle{exostyle}
\newtheorem{exercice}{Exercice}

\newenvironment{questions}{
\begin{enumerate}[\hspace{12pt}\bfseries\itshape a.]}{\end{enumerate}
} %mettre un 1 à la place du a si on veut des numéros au lieu de lettres pour les questions 
%-------------------------------------------------------------------------------



%-------------------------------------------------------------------------------
%                    - Racourcis d'écriture -
%-------------------------------------------------------------------------------

% Angles orientés (couples de vecteurs)
\newcommand{\aopp}[2]{(\vec{#1}, \vec{#2})} %Les deuc vecteurs sont positifs
\newcommand{\aopn}[2]{(\vec{#1}, -\vec{#2})} %Le second vecteur est négatif
\newcommand{\aonp}[2]{(-\vec{#1}, \vec{#2})} %Le premier vecteur est négatif
\newcommand{\aonn}[2]{(-\vec{#1}, -\vec{#2})} %Les deux vecteurs sont négatifs

%Ensembles mathématiques
\newcommand{\naturels}{\mathbb{N}} %Nombres naturels
\newcommand{\relatifs}{\mathbb{Z}} %Nombres relatifs
\newcommand{\rationnels}{\mathbb{Q}} %Nombres rationnels
\newcommand{\reels}{\mathbb{R}} %Nombres réels
\newcommand{\complexes}{\mathbb{C}} %Nombres complexes
%-------------------------------------------------------------------------------



\title{Correction des exercices de la semaine du 18/05}
\date{}

\begin{document}
	
\maketitle

\vspace*{-1cm}

\section*{Exercice 38 page 107}


\begin{enumerate}
	
%	\begin{multicols}{2}
		
		
		
		
	
	\item \begin{equation*}
		(a+b)^2 = a^2+b^2
	\end{equation*}
	
	on prend $a = 1$ et $b = 1$
	
	\begin{multicols}{2}
		\begin{align*}
		(a+b)^2 &= (1 + 1)^2 \\
		(a+b)^2 &= 2^2 \\
		(a+b)^2 &= 4 
		\end{align*} 
		
		
		\begin{align*}
			a^2 + b^2 &= 1^2 + 1^2\\
			a^2 + b^2 &= 1+1 \\
			a^2 + b^2 &= 2 
		\end{align*}
		
	\end{multicols}
	
	L'égalité est fausse.
	
	
	\item \begin{equation*}
	x^2 = 2x
	\end{equation*}
	
	on prend $x = 1$ 
	
	\begin{multicols}{2}
		\begin{align*}
			x^2 &= 1^2 \\
			x^2 &= 1 \\
		\end{align*} 
		
		
		\begin{align*}
			2x &= 2 \times x \\
			2x &= 2 \times 1 \\
			2x &= 2 
		\end{align*}
		
	\end{multicols}
	
	L'égalité est fausse.
	
	
	\item \begin{equation*}
	5 + 3x = 8x
	\end{equation*}
	
	on prend $x = 0$ 
	
	\begin{multicols}{2}
		\begin{align*}
			5 + 3x &= 5 + 3 \times 0 \\
			5 + 3x &= 5 +0 \\
			5 + 3x &= 5
		\end{align*} 
		
		
		\begin{align*}
			8x &= 8 \times 0 \\
			8x &= 0 \\
		\end{align*}
		
	\end{multicols}
	
	L'égalité est fausse.
	
	\newpage
	
	\item \begin{equation*}
	3x + x = 4x
	\end{equation*}
	
	on prend $x = 0$ 
	
	\begin{multicols}{2}
		\begin{align*}
		3x + x &= 3 \times 0 +  0 \\
		3x + x &= 0 +0 \\
		3x + x &= 0
		\end{align*} 
		
		
		\begin{align*}
			4x &= 4 \times 0 \\
			4x &= 0 \\
		\end{align*}
		
	\end{multicols}
	
	L'égalité est vraie pour $x=0$.
	
	Je simplifie le membre de gauche :
	
	\begin{align*}
		3x + x &=  3 \times x + 1 \times x \\
		3x + x &=  (3 + 1) \times x \\
		3x + x &=  4 \times x \\
		3x + x &=  4x 
	\end{align*}.
	
	Donc l'égalité est vraie pour tout x.
	
	
	
	\item \begin{equation*}
	3x \times 2x = 6x^2
	\end{equation*}
	
	on prend $x = 1$ 
	
	\begin{multicols}{2}
		\begin{align*}
		3x \times 2x &= 3 \times 1 \times 2 \times 1 \\
		3x \times 2x &= 3 \times 2\\
		3x \times 2x &= 6
		\end{align*} 
		
		
		\begin{align*}
			6x^2 &= 6 \times x \times x \\
			6x^2 &= 6 \times 1 \times 1 \\
			6x^2 &= 6\\
		\end{align*}
		
	\end{multicols}
	
	L'égalité est vraie pour $x=1$.
	
	Je simplifie le membre de gauche :
	
	\begin{align*}
		3x \times 2x &=  3 \times x \times  2 \times x \\
		3x \times 2x &=  3 \times 2 \times  x \times x \\
		3x \times 2x &=  6 \times  x \times x \\
		3x \times 2x &=  6x^2 \\
	\end{align*}.
	
	Donc l'égalité est vraie pour tout x.
%	\end{multicols}

	\newpage
	
	\item \begin{equation*}
		2y^2 = (2y)^2 
	\end{equation*}
	
	on prend $y = 2$ 
	
	\begin{multicols}{2}
		\begin{align*}
			2y^2 &= 2 \times 2 \times 2 \\
			2y^2 &= 8 \\			
		\end{align*} 
		
		
		\begin{align*}
			(2y)^2 &= (2 \times 2)^2 \\
			(2y)^2 &= 4^2 \\
			(2y)^2 &= 16 
		\end{align*}
		
	\end{multicols}
	
	L'égalité est fausse.
\end{enumerate}

\section*{Exercice 63 page 110}

Traduction des programmes de calcul en expressions littérales :

1$^{er}$ programme :
\begin{itemize}
	\item Choisir un nombre : $x$
	\item Multiplier par \num{0.4} : $\num{0.4}x$
	\item Ajouter \num{1.4} : $\num{0.4}x + \num{1.4}$
	\item Multiplier par 5 : $5(\num{0.4}x + \num{1.4})$
	\item Soustraire le double du nombre choisi : $5(\num{0.4}x + \num{1.4}) - 2x$
\end{itemize}


2$^{e}$ programme :
\begin{itemize}
	\item Choisir un nombre : $x$
	\item Calculer son carré : $x^2$
	\item Ajouter \num{11} : $x^2 + 11$
	\item Soustraire 6 fois le nombre choisi : $x^2 + 11 - 6x$
	\item Multiplier par le nombre choisi : $x(x^2 + 11 - 6x)$
	\item Ajouter 1 : $2(x^2 + 11 - 6x) + 1$
\end{itemize}



\vspace*{1cm}


On prend $x=1$


\begin{multicols}{2}
	\begin{align*}
	A &= 5(\num{0.4}x + \num{1.4}) - 2x\\
	A &= 5(\num{0.4}\times 1 + \num{1.4}) - 2 \times 1\\
	A &= 5(\num{0.4} + \num{1.4}) - 2 \\
	A &= 5 \times \num{1.8} - 2 \\
	A &= 9 - 2 \\
	A &= 7 
	\end{align*}
	
	\begin{align*}
	B &= x(x^2 + 11 - 6x) + 1\\
	B &= 1 \times (1 \times 1 + 11 - 6 \times 1) + 1 \\
	B &= 1 \times (1 + 11 - 6) + 1 \\
	B &= 1 \times 6 + 1 \\
	B &= 6 +1 \\
	B &= 7 
	\end{align*}	
\end{multicols}


\vspace*{1cm}

\newpage
On prend $x=2$


\begin{multicols}{2}
	\begin{align*}
	A &= 5(\num{0.4}x + \num{1.4}) - 2x\\
	A &= 5(\num{0.4}\times 2 + \num{1.4}) - 2 \times 2\\
	A &= 5(\num{0.8} + \num{1.4}) - 4 \\
	A &= 5 \times \num{2.2} - 4 \\
	A &= 11 - 4 \\
	A &= 7 
	\end{align*}
	
	\begin{align*}
	B &= x(x^2 + 11 - 6x) + 1\\
	B &= 2 \times (2 \times 2 + 11 - 6 \times 2) + 1 \\
	B &= 2 \times (4 + 11 - 12) + 1 \\
	B &= 2 \times 3 + 1 \\
	B &= 6+1 \\
	B &= 7 
	\end{align*}	
\end{multicols}

Le calculs de Julie sont corrects.


\vspace*{1cm}


On prend $x=3$


\begin{multicols}{2}
	\begin{align*}
	A &= 5(\num{0.4}x + \num{1.4}) - 2x\\
	A &= 5(\num{0.4}\times 3 + \num{1.4}) - 2 \times 3\\
	A &= 5(\num{1.2} + \num{1.4}) - 6 \\
	A &= 5 \times \num{2.6} - 6 \\
	A &= 13 - 6 \\
	A &= 7 
	\end{align*}
	
	\begin{align*}
	B &= x(x^2 + 11 - 6x) + 1\\
	B &= 3 \times (3 \times 3 + 11 - 6 \times 3) + 1 \\
	B &= 3 \times (9 + 11 - 18) + 1 \\
	B &= 3 \times 2 + 1 \\
	B &= 6+1 \\
	B &= 7 \\
	\end{align*}	
\end{multicols}

Cela fonctionne aussi avec $x=3$

Si on choisi de prendre $x=0$, on obtient $A=7$ et $B=1$, donc l'affirmation de Julie est fausse.

\section*{Exercice 65 page 110}

7 est un nombre entier, on prend $x=7$

\begin{align*}
	x^2 + \dfrac{120}{x} &= x^2 + \dfrac{120}{x} \\
	x^2 + \dfrac{120}{x} &= 7^2 + \dfrac{120}{7} \\
	x^2 + \dfrac{120}{x} &\approx 49 + \num{17.14} \\
	x^2 + \dfrac{120}{x} &\approx \num{56.14}
\end{align*}

Donc l'affirmation est fausse.

\section*{Exercice 68 page 110}


\subsection*{1.}

L'expression littérale correspondant au programme de calcul est  : $ 2x + 4y$

\subsection*{2.}
Si le second nombre est le double du premier, on a $y=2x$, donc

\begin{align*}
 2x + 4y = 2x + 4 \times 2x \\
 2x + 4y = 2x + 8x \\
 2x + 4y = 10x \\
\end{align*}

Dans ce cas, le résultat sera 10 fois le premier nombre.

\subsection*{3.}

\begin{multicols}{2}
	\begin{itemize}
		\item Choisir deux nombres
		\item multiplier le premier par 3
		\item multiplier le second par 2
		\item ajouter les deux résultats.
	\end{itemize}


	\begin{itemize}
		\item Choisir deux nombres
		\item ajouter 12 au second 
		\item multiplier le résultat par le premier nombre.
	\end{itemize}
\end{multicols}


\section*{Exercice 69 page 111}

\subsection*{1.}

L'expression littérale correspondant au programme de calcul est  : $ 4(x+3) \times (3x -2)$

\subsection*{2.}
En testant plusieurs valeurs on trouve $x=6$

\newpage

\subsection*{3.}

\begin{multicols}{2}
	Expression 1
	\begin{itemize}
		\item Choisir un nombre
		\item le multiplier par 2
		\item ajouter 5
	\end{itemize}
	
	\vspace*{0.5cm}
		
	Expression 2
	\begin{itemize}
		\item Choisir un nombre
		\item le multiplier par 5
		\item ajouter 2
	\end{itemize}

	\vspace*{0.5cm}
	
	Expression 3
	\begin{itemize}
		\item Choisir un nombre
		\item ajouter 2
		\item multiplier le résultat par 5
	\end{itemize}
	
	\vspace*{0.5cm}
	
	Expression 4
	\begin{itemize}
		\item Choisir un nombre
		\item le multiplier par 2
		\item multiplier le nombre de départ par 5
		\item ajouter les deux résultats;
	\end{itemize}
\end{multicols}


\section*{Exercice 71 page 111}

\subsection*{1.}

Pour construire 5 maisons il lui faudra 21 allumettes.

\subsection*{2.}

Pour construire 10 maisons il lui faudra 41 allumettes, et 60 pour 15 maisons.

\subsection*{3.}

Pour construire \num{1345} maisons il lui faudra \num{5381} allumettes.

\subsection*{4.}

La formule pour trouver le nombre d'allumettes est $4N +1$ avec $N$ le nombre de maisons.


\subsection*{5.}

\begin{equation*}
	560 - 1 = 559
\end{equation*}

\begin{equation*}
559 \div 4 = \num{139.75}
\end{equation*}

Avec 560 allumettes on peut faire 139 maisons.

%\section*{Exercice 80 page 113}
\end{document}