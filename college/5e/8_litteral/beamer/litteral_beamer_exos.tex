\documentclass[xcolor={dvipsnames}]{beamer}
%\usepackage[utf8]{inputenc}
%\usetheme{Madrid}
\usetheme{CambridgeUS}
%\usetheme{Malmoe}
%\usecolortheme{beaver}
\usecolortheme{seahorse}

\input{../../../../utils_maths_beamer}


\usepackage{../../../../pas-math}
\usepackage{../../../../moncours_beamer}

\usepackage{amssymb,amsmath}


\newcommand{\myitem}{\item[\textbullet]}

\graphicspath{{../img/}}

\title{Séquence 8 : Expressions littérales}
\subtitle{Correction des exercices semaine du 18/05}
%\author{O. FINOT}\institute{Collège S$^t$ Bernard}

%
%\AtBeginSection[]
%{
%	\begin{frame}
%		\frametitle{}
%		\tableofcontents[currentsection, hideallsubsections]
%	\end{frame} 
%
%}
%
%
%\AtBeginSubsection[]
%{
%	\begin{frame}
%		\frametitle{Sommaire}
%		\tableofcontents[currentsection, currentsubsection]
%	\end{frame} 
%}

\begin{document}



%\begin{frame}
%  \titlepage 
%\end{frame}


	

\begin{frame}
	\frametitle{Exercice 25 page 105}
	
	\begin{block}{Aridité d'une région}
		\begin{equation*}
			{\LARGE I = \dfrac{P}{T + 10}\pause}
		\end{equation*}
		
		\begin{itemize}
			\item $I$ : indice d'aridité \pause
			\item $P$ : précipitations en mm \pause
			\item $T$ : température moyenne.
		\end{itemize}
	\end{block}
	
	
\end{frame}


\begin{frame}
	\frametitle{Exercice 25 page 105}
	\framesubtitle{Aridité de la Bretagne}
	
	\begin{block}{Calcul de l'indice d'aridité}
		\begin{eqnarray*}
			I_{Bretagne} &=& \dfrac{P}{T + 10} \\ \pause
			I_{Bretagne} &=& \dfrac{\num{1130}}{14 + 10} \\ \pause
			I_{Bretagne} &=& \dfrac{\num{1130}}{24} \\ \pause
			%I_{Bretagne} &=& \dfrac{\num{565}}{12} \\ \pause
			I_{Bretagne} &\approx& \num{47} \pause			
		\end{eqnarray*}
		
		
	\end{block}
	
	\begin{alertblock}{Conclusion}
		$I_{Bretagne} \ge 30$, donc la Bretagne est une région humide.
	\end{alertblock}
\end{frame}


\begin{frame}
	\frametitle{Exercice 25 page 105}
	\framesubtitle{Aridité de la Corse}
	
	\begin{block}{Calcul de l'indice d'aridité}
		\begin{eqnarray*}
			I_{Corse} &=& \dfrac{P}{T + 10} \\ \pause
			I_{Corse} &=& \dfrac{\num{659}}{20 + 10} \\ \pause
			I_{Corse} &=& \dfrac{\num{659}}{30} \\ \pause
			%I_{Bretagne} &=& \dfrac{\num{565}}{12} \\ \pause
			I_{Corse} &\approx& \num{22} \pause			
		\end{eqnarray*}
		
		
	\end{block}
	
	\begin{alertblock}{Conclusion}
		$20 \le I_{Corse} < 30$, donc la Corse est une région demi-humide.
	\end{alertblock}
\end{frame}


\begin{frame}
	\frametitle{Exercice 25 page 105}
	\framesubtitle{Aridité de Bardenas}
	
	\begin{block}{Calcul de l'indice d'aridité}
		\begin{eqnarray*}
			I_{Bardenas} &=& \dfrac{P}{T + 10} \\ \pause
			I_{Bardenas} &=& \dfrac{\num{410}}{15 + 10} \\ \pause
			I_{Bardenas} &=& \dfrac{\num{410}}{25} \\ \pause
			%I_{Bretagne} &=& \dfrac{\num{565}}{12} \\ \pause
			I_{Bardenas} &=& \num{16.4} \pause			
		\end{eqnarray*}
		
		
	\end{block}
	
	\begin{alertblock}{Conclusion}
		$10 \le I_{Bardenas} < 20$, donc la région de Bardenas est demi-aride.
	\end{alertblock}
\end{frame}



\begin{frame}
	\frametitle{Exercice 35 page 107}
	\framesubtitle{Programme de calcul}
	
	\begin{block}{\'Eric}
		\begin{enumerate}
			\item on choisit un nombre \pause : $N$ \pause
			\item on le multiplie par 8 \pause : $8 \times N$ \pause
			\item on soustrait 15 \pause : $8 \times N - 15$ \pause
			\item[$\Rightarrow$] $8N - 15$ \pause
		\end{enumerate}
	\end{block}
	
	\begin{block}{\'Angélique}
		\begin{enumerate}
			\item on choisit un nombre \pause : $N$ \pause
			\item on le multiplie par 4 \pause : $4 \times N$ \pause
			\item on ajoute 6 \pause : $4 \times N - 6$ \pause
			\item[$\Rightarrow$] $4N + 6$ 
		\end{enumerate}
	\end{block}
\end{frame}



\begin{frame}
	\frametitle{Exercice 35 page 107}
	\framesubtitle{Vérification de l'égalité}
	
	\begin{block}{$N = 0$}
		
		\begin{columns}
			\begin{column}{0.5\textwidth}
				\begin{eqnarray*}
					8N - 15 &=& \pause 8 \times 0 - 15 \\ \pause
					8N - 15 &=& 0 - 15 \\ \pause
					8N - 15 &=& -15 \pause
				\end{eqnarray*}
			\end{column}
		
		
			\begin{column}{0.5\textwidth}
				\begin{eqnarray*}
					4N + 6 &=& \pause 4 \times 0 + 6 \\ \pause
					4N + 6 &=& 0 + 6 \\ \pause
					4N + 6 &=& 6 \pause
				\end{eqnarray*}
			\end{column}
		\end{columns}
		
	\end{block}

	\begin{alertblock}{Conclusion}
		L'égalité est fausse.
	\end{alertblock}
\end{frame}


\begin{frame}
	\frametitle{Exercice 35 page 107}
	\framesubtitle{Trouver le nombre choisi}
	
	\begin{block}{$N = 4$}
		
		\begin{columns}
			\begin{column}{0.5\textwidth}
				\begin{eqnarray*}
					8N - 15 &=&  8 \times 4 - 15 \\ 
					8N - 15 &=& 32 - 15 \\ 
					8N - 15 &=& 17 
				\end{eqnarray*}
			\end{column}
			
			
			\begin{column}{0.5\textwidth}
				\begin{eqnarray*}
					4N + 6 &=&  4 \times 4 + 6 \\ 
					4N + 6 &=& 16 + 6 \\ 
					4N + 6 &=& 22 \pause
				\end{eqnarray*}
			\end{column}
		\end{columns}
		
	\end{block}


	\begin{block}{$N = 5$}
		
		\begin{columns}
			\begin{column}{0.5\textwidth}
				\begin{eqnarray*}
					8N - 15 &=&  8 \times 5 - 15 \\ 
					8N - 15 &=& 40 - 15 \\ 
					8N - 15 &=& 25 
				\end{eqnarray*}
			\end{column}
			
			
			\begin{column}{0.5\textwidth}
				\begin{eqnarray*}
					4N + 6 &=&  4 \times 5 + 6 \\ 
					4N + 6 &=& 20 + 6 \\ 
					4N + 6 &=& 26 
				\end{eqnarray*}
			\end{column}
		\end{columns}
		
	\end{block}

\end{frame}
\begin{frame}
	\frametitle{Exercice 35 page 107}
	\framesubtitle{Trouver le nombre choisi}
	
	\begin{block}{$N = 6$}
		
		\begin{columns}
			\begin{column}{0.5\textwidth}
				\begin{eqnarray*}
					8N - 15 &=&  8 \times 6 - 15 \\ 
					8N - 15 &=& 48 - 15 \\ 
					8N - 15 &=& 33 
				\end{eqnarray*}
			\end{column}
			
			
			\begin{column}{0.5\textwidth}
				\begin{eqnarray*}
					4N + 6 &=&  4 \times 6 + 6 \\ 
					4N + 6 &=& 24 + 6 \\ 
					4N + 6 &=& 30 \pause
				\end{eqnarray*}
			\end{column}
		\end{columns}
		
	\end{block}


	\begin{block}{$N = 7$}
		
		\begin{columns}
			\begin{column}{0.5\textwidth}
				\begin{eqnarray*}
					8N - 15 &=&  8 \times 7 - 15 \\ 
					8N - 15 &=& 56 - 15 \\ 
					8N - 15 &=& 41 
				\end{eqnarray*}
			\end{column}
			
			
			\begin{column}{0.5\textwidth}
				\begin{eqnarray*}
					4N + 6 &=&  4 \times 7 + 6 \\ 
					4N + 6 &=& 28 + 6 \\ 
					4N + 6 &=& 34 
				\end{eqnarray*}
			\end{column}
		\end{columns}
		
	\end{block}


\end{frame}



\begin{frame}
	\frametitle{Exercice 35 page 107}
	\framesubtitle{Trouver le nombre choisi}
	
	
	\begin{block}{$N = 5$}
		
		\begin{columns}
			\begin{column}{0.5\textwidth}
				\begin{eqnarray*}
					8N - 15 &=&  8 \times 5 - 15 \\ 
					8N - 15 &=& 40 - 15 \\ 
					8N - 15 &=& 25 
				\end{eqnarray*}
			\end{column}
			
			
			\begin{column}{0.5\textwidth}
				\begin{eqnarray*}
					4N + 6 &=&  4 \times 5 + 6 \\ 
					4N + 6 &=& 20 + 6 \\ 
					4N + 6 &=& 26 
				\end{eqnarray*}
			\end{column}
		\end{columns}
		
	\end{block}
	
	\begin{block}{$N = 6$}
		
		\begin{columns}
			\begin{column}{0.5\textwidth}
				\begin{eqnarray*}
					8N - 15 &=&  8 \times 6 - 15 \\ 
					8N - 15 &=& 48 - 15 \\ 
					8N - 15 &=& 33 
				\end{eqnarray*}
			\end{column}
			
			
			\begin{column}{0.5\textwidth}
				\begin{eqnarray*}
					4N + 6 &=&  4 \times 6 + 6 \\ 
					4N + 6 &=& 24 + 6 \\ 
					4N + 6 &=& 30 
				\end{eqnarray*}
			\end{column}
		\end{columns}
		
	\end{block}
	
\end{frame}

\begin{frame}
	\frametitle{Exercice 35 page 107}
	%\framesubtitle{Trouver le nombre choisi}
	
	\begin{block}{$N = \num{5.5}$}
		
		\begin{columns}
			\begin{column}{0.5\textwidth}
				\begin{eqnarray*}
					8N - 15 &=&  8 \times \num{5.5} - 15 \\ 
					8N - 15 &=& 44 - 15 \\ 
					8N - 15 &=& 29 \pause
				\end{eqnarray*}
			\end{column}
			
			
			\begin{column}{0.5\textwidth}
				\begin{eqnarray*}
					4N + 6 &=&  4 \times \num{5.5}6 + 6 \\ 
					4N + 6 &=& 22 + 6 \\ 
					4N + 6 &=& 28 \pause
				\end{eqnarray*}
			\end{column}
		\end{columns}
		
	\end{block}
	
	
	\begin{block}{$N = \num{5.25}$}
		
		\begin{columns}
			\begin{column}{0.5\textwidth}
				\begin{eqnarray*}
					8N - 15 &=&  8 \times \num{5.25} - 15 \\ 
					8N - 15 &=& 42 - 15 \\ 
					8N - 15 &=& 27 \pause
				\end{eqnarray*}
			\end{column}
			
			
			\begin{column}{0.5\textwidth}
				\begin{eqnarray*}
					4N + 6 &=&  4 \times \num{5.25} + 6 \\ 
					4N + 6 &=& 21 + 6 \\ 
					4N + 6 &=& 27 \pause 
				\end{eqnarray*}
			\end{column}
		\end{columns}
		
	\end{block}
	
	\begin{alertblock}{Conclusion}
		Ils ont choisi le nombre \num{5.25}.
	\end{alertblock}
	
\end{frame}

\begin{frame}
\frametitle{Exercice 35 page 107}
\framesubtitle{Résolution d'équation}

	Je cherche la valeur de $N$ telle que les deux expressions soient égales. \pause
	
	\begin{eqnarray*}
		8N - 15 &=& 4N + 6 \\ \pause %& On pose l'équation \pause 
		8N - 15 + 15 &=& 4N + 6 + 15 \\ \pause %& 
		8N + 0 &=& 4N + 21 \\ \pause %& 
		8N - 4N &=& 4N + 21 -4N \\ \pause %& 
		8N - 4N &=& 4N -4N + 21 \\ \pause %& 
		4N &=& 0N + 21 \\ \pause %& 
		4N &=& 21 \\ \pause %& 
		N &=& \frac{21}{4} \\ \pause %& 
		N &=& \num{5.25}
	\end{eqnarray*}
	
\end{frame}
\end{document}