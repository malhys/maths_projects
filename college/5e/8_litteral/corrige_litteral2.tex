\documentclass[12pt,a4paper]{article}


\usepackage[in, plain]{fullpage}
\usepackage{array}
\usepackage{../../../pas-math}

%-------------------------------------------------------------------------------
%          -Packages nécessaires pour écrire en Français et en UTF8-
%-------------------------------------------------------------------------------
\usepackage[utf8]{inputenc}
\usepackage[frenchb]{babel}
\usepackage[T1]{fontenc}
\usepackage{lmodern}
\usepackage{textcomp}



%-------------------------------------------------------------------------------

%-------------------------------------------------------------------------------
%                          -Outils de mise en forme-
%-------------------------------------------------------------------------------
\usepackage{hyperref}
\hypersetup{pdfstartview=XYZ}
%\usepackage{enumerate}
\usepackage{graphicx}
\usepackage{multicol}
\usepackage{tabularx}
\usepackage{multirow}


\usepackage{anysize} %%pour pouvoir mettre les marges qu'on veut
%\marginsize{2.5cm}{2.5cm}{2.5cm}{2.5cm}

\usepackage{indentfirst} %%pour que les premier paragraphes soient aussi indentés
\usepackage{verbatim}
\usepackage{enumitem}
\usepackage[usenames,dvipsnames,svgnames,table]{xcolor}

\usepackage{variations}

%-------------------------------------------------------------------------------


%-------------------------------------------------------------------------------
%                  -Nécessaires pour écrire des mathématiques-
%-------------------------------------------------------------------------------
\usepackage{amsfonts}
\usepackage{amssymb}
\usepackage{amsmath}
\usepackage{amsthm}
\usepackage{tikz}
\usepackage{xlop}
%-------------------------------------------------------------------------------



%-------------------------------------------------------------------------------


%-------------------------------------------------------------------------------
%                    - Mise en forme avancée
%-------------------------------------------------------------------------------

\usepackage{ifthen}
\usepackage{ifmtarg}


\newcommand{\ifTrue}[2]{\ifthenelse{\equal{#1}{true}}{#2}{$\qquad \qquad$}}

%-------------------------------------------------------------------------------

%-------------------------------------------------------------------------------
%                     -Mise en forme d'exercices-
%-------------------------------------------------------------------------------
%\newtheoremstyle{exostyle}
%{\topsep}% espace avant
%{\topsep}% espace apres
%{}% Police utilisee par le style de thm
%{}% Indentation (vide = aucune, \parindent = indentation paragraphe)
%{\bfseries}% Police du titre de thm
%{.}% Signe de ponctuation apres le titre du thm
%{ }% Espace apres le titre du thm (\newline = linebreak)
%{\thmname{#1}\thmnumber{ #2}\thmnote{. \normalfont{\textit{#3}}}}% composants du titre du thm : \thmname = nom du thm, \thmnumber = numéro du thm, \thmnote = sous-titre du thm

%\theoremstyle{exostyle}
%\newtheorem{exercice}{Exercice}
%
%\newenvironment{questions}{
%\begin{enumerate}[\hspace{12pt}\bfseries\itshape a.]}{\end{enumerate}
%} %mettre un 1 à la place du a si on veut des numéros au lieu de lettres pour les questions 
%-------------------------------------------------------------------------------

%-------------------------------------------------------------------------------
%                    - Mise en forme de tableaux -
%-------------------------------------------------------------------------------

\renewcommand{\arraystretch}{1.7}

\setlength{\tabcolsep}{1.2cm}

%-------------------------------------------------------------------------------



%-------------------------------------------------------------------------------
%                    - Racourcis d'écriture -
%-------------------------------------------------------------------------------

% Angles orientés (couples de vecteurs)
\newcommand{\aopp}[2]{(\vec{#1}, \vec{#2})} %Les deuc vecteurs sont positifs
\newcommand{\aopn}[2]{(\vec{#1}, -\vec{#2})} %Le second vecteur est négatif
\newcommand{\aonp}[2]{(-\vec{#1}, \vec{#2})} %Le premier vecteur est négatif
\newcommand{\aonn}[2]{(-\vec{#1}, -\vec{#2})} %Les deux vecteurs sont négatifs

%Ensembles mathématiques
\newcommand{\naturels}{\mathbb{N}} %Nombres naturels
\newcommand{\relatifs}{\mathbb{Z}} %Nombres relatifs
\newcommand{\rationnels}{\mathbb{Q}} %Nombres rationnels
\newcommand{\reels}{\mathbb{R}} %Nombres réels
\newcommand{\complexes}{\mathbb{C}} %Nombres complexes


%Intégration des parenthèses aux cosinus
\newcommand{\cosP}[1]{\cos\left(#1\right)}
\newcommand{\sinP}[1]{\sin\left(#1\right)}


%Probas stats
\newcommand{\stat}{statistique}
\newcommand{\stats}{statistiques}
%-------------------------------------------------------------------------------

%-------------------------------------------------------------------------------
%                    - Mise en page -
%-------------------------------------------------------------------------------

\newcommand{\twoCol}[1]{\begin{multicols}{2}#1\end{multicols}}


\setenumerate[1]{font=\bfseries,label=\textit{\alph*})}
\setenumerate[2]{font=\bfseries,label=\arabic*)}


%-------------------------------------------------------------------------------
%                    - Elements cours -
%-------------------------------------------------------------------------------




\title{Correction des exercices de la semaine du 18/05}
\date{}

\begin{document}
	
\maketitle

\vspace*{-1cm}

\section*{Exercice 13 page 104}


On veut calculer l'expression $A=5 + 3 \times x$
\begin{enumerate}
	
	\begin{multicols}{2}
		\item pour $x=1$
		
		\begin{align*}
		A &= 5 + 3 \times x \\
		A &= 5 + 3 \times 1 \\
		A &= 5 + 3  \\
		A &= 8
		\end{align*}
		
		\item pour $x=4$
		
		\begin{align*}
		A &= 5 + 3 \times x \\
		A &= 5 + 3 \times 4 \\
		A &= 5 + 12  \\
		A &= 17
		\end{align*}
	\end{multicols}
	
	
\end{enumerate}


\section*{Exercice 14 page 104}


On veut calculer l'expression $B=3 \times (n + 2)$
\begin{enumerate}
	
	\begin{multicols}{2}
		\item pour $n=3$
		
		\begin{align*}
		B &=3 \times (n + 2) \\
		B &=3 \times (3 + 2) \\
		B &=3 \times 5 \\
		B &= 15
		\end{align*}
		
		\item pour $n=9$
		
		\begin{align*}
		B &=3 \times (n + 2) \\
		B &=3 \times (9 + 2) \\
		B &=3 \times 11 \\
		B &= 33
		\end{align*}
	\end{multicols}
	
	
\end{enumerate}

\section*{Exercice 16 page 104}


On veut calculer l'expression $C=3x+ 5x +4$
\begin{enumerate}
	
	\begin{multicols}{3}
		\item pour $x=2$
		
		\begin{align*}
		C &=3 \times x + 5 \times x + 4 \\
		C &=3 \times 2 + 5 \times 2 + 4 \\
		C &= 6 + 10 + 4 \\
		C &= 20
		\end{align*}
		
		\item pour $x=17$
		
		\begin{align*}
		C &=3 \times x + 5 \times x + 4 \\
		C &=3 \times 17 + 5 \times 17 + 4 \\
		C &= 51 + 85 + 4 \\
		C &= 140		
		\end{align*}
		
		
		\item pour $x=\num{2.6}$
		
		\begin{align*}
		C &=3 \times x + 5 \times x + 4 \\
		C &=3 \times \num{2.6} + 5 \times \num{2.6} + 4 \\
		C &= \num{7.8} + 13 + 4 \\
		C &= \num{24.8}		
		\end{align*}
	\end{multicols}
	
	
\end{enumerate}


\section*{Exercice 18 page 104}

On a $x=7$ et $y=3$

\begin{enumerate}

	\begin{multicols}{2}
	
		\item 
			\begin{align*}
				A &= x^2 + y^2\\
				A &= x \times x + y \times y \\
				A &= 7 \times 7 + 3 \times 3 \\
				A &= 49 + 9\\
				A &= 58
			\end{align*}
			
		\item 
			\begin{align*}
				B &= 4xy + x + y\\
				B &= 4 \times x \times y + x + y \\
				B &= 4 \times 7 \times 3+ 7 + 3\\
				B &= 84 + 7 + 3\\
				B &= 94
			\end{align*}
			
	\end{multicols}	

	\begin{multicols}{2}
	
		\item 
			\begin{align*}
				C &= \dfrac{x + y}{xy}\\
				C &= \dfrac{x + y}{x \times y}\\
				C &= \dfrac{7 + 3}{7 \times 3}\\
				C &= \dfrac{10}{21}				
			\end{align*}
			
		\item
			\begin{align*}
				D &= (3x + 1)(12 - 2y)\\
				D &= (3 \times x + 1) \times (12 - 2 \times y)\\
				D &= (3 \times 7 + 1) \times (12 - 2 \times 3)\\
				D &= (21 + 1) \times (12 - 6)\\
				D &= 22 \times 6\\
				D &= 132\\
			\end{align*}

	\end{multicols}
\end{enumerate}

\section*{Exercice 22 page 105}

\subsection*{Calcul de la ration journalière de Clément}

\num{4.72} kg = \num{4720} g

Clément pèse 4720 grammes.

\vspace*{-0.5cm}

\begin{align*}
	RJ &= \dfrac{M}{10} + 250 \\
	RJ &= \dfrac{4720}{10} + 250 \\
	RJ &= 472 + 250 \\
	RJ &= 722
\end{align*}

Sa ration journalière est 722 grammes de lait.

\newpage

\subsection*{Calcul de la ration journalière de Benoît}

\num{3.54} kg = \num{3540} g

Clément pèse 4720 grammes.

\vspace*{-0.5cm}

\begin{align*}
	RJ &= \dfrac{M}{10} + 250 \\
	RJ &= \dfrac{3540}{10} + 250 \\
	RJ &= 354 + 250 \\
	RJ &= 604
\end{align*}

Sa ration journalière est 604 grammes de lait.


\subsection*{Calcul de la ration journalière d'Aminata}

\num{7.25} kg = \num{7250} g

Aminata pèse 7250 grammes.

\vspace*{-0.5cm}

\begin{align*}
RJ &= \dfrac{M}{10} + 250 \\
RJ &= \dfrac{7250}{10} + 250 \\
RJ &= 725 + 250 \\
RJ &= 975
\end{align*}

Sa ration journalière est 975 grammes de lait.

\section*{Exercice 24 page 105}

La cellule $A2$ contient le nombre 3.

On a donc :

\vspace*{-1cm}

\begin{align*}
	B2 &= (4 \times A2 + 7) \times A2\\
	B2 &= (4 \times 3 + 7) \times 3\\
	B2 &= (12 + 7) \times 3\\
	B2 &= 19 \times 3\\
	B2 &= 57\\
\end{align*}

\vspace*{-0.5cm}

Le nombre affiché dans la cellule $B2$ sera 57.

\newpage

\section*{Exercice 28 page 106}

\begin{enumerate}
	\item On a $x=1$ :
	
	\vspace*{-1cm}
	
	\begin{multicols}{2}
		\begin{align*}
			31 - x &= 31 - 1 \\
			31 - x &= 30 %\\
		\end{align*}
		
		\begin{align*}
			20 + x &= 20 + 1 \\
			20 + x &= 21 %\\
		\end{align*}
	\end{multicols}
	
%	\vspace*{-1cm}
	
	Donc l'égalité est fausse pour $x=1$.
	
	\item On a $x=2$ :
	
	\vspace*{-1cm}
	
	\begin{multicols}{2}
		\begin{align*}
			31 - x &= 31 - 2 \\
			31 - x &= 29 %\\
		\end{align*}
		
		\begin{align*}
			20 + x &= 20 + 2 \\
			20 + x &= 22 %\\
		\end{align*}
	\end{multicols}
	
%	\vspace*{-1cm}
	
	Donc l'égalité est fausse pour $x=2$.
	
	
	\item On a $x=3$ :
	
	\vspace*{-1cm}
	
	\begin{multicols}{2}
		\begin{align*}
			31 - x &= 31 - 2 \\
			31 - x &= 28 %\\
		\end{align*}
			
		\begin{align*}
			20 + x &= 20 + 3 \\
			20 + x &= 23 %\\
		\end{align*}
	\end{multicols}

%	\vspace*{-1cm}
	
	Donc l'égalité est fausse pour $x=3$.
	
	
	
\end{enumerate}


\section*{Exercice 29 page 106}

On pose $x=3$ et $y=5$.

\begin{enumerate}
	\item 
	

	
	\begin{multicols}{2}
		\begin{align*}
			5x + 4y &= 5 \times x + 4 \times y \\
			5x + 4y &= 5 \times 3 + 4 \times 5 \\
			5x + 4y &= 15 + 20 \\
			5x + 4y &= 35 
		\end{align*}
		
		\begin{align*}
			40 - y &= 40 - 2 \\
			40 - y &= 38 
		\end{align*}
	\end{multicols}
	
	Donc l'égalité est fausse pour ces valeurs.
	
	
	\item 
	\begin{multicols}{2}
		\begin{align*}
			6xy - 2y &= 6 \times x \times y - 2 \times y \\
			6xy - 2y &= 6 \times 3 \times 5 - 2 \times 5 \\
			6xy - 2y &= 90 - 10 \\
			6xy - 2y &= 80 
		\end{align*}
		
		\begin{align*}
			(5x+1)y &= (5 \times x+1) \times y \\
			(5x+1)y &= (5 \times 3+1) \times 5 \\
			(5x+1)y &= (15 +1) \times 5 \\
			(5x+1)y &= 16 \times 5 \\
			(5x+1)y &= 80 
		\end{align*}
	\end{multicols}
	
	Donc l'égalité est vraie pour $x=3$ et $y=5$
	
	
	\newpage
	
	\item 
	\begin{multicols}{2}
		\begin{align*}
			x + y &= 3 + 5 \\
			x + y &= 8 \\
		\end{align*}
		
		\begin{align*}
			4(y - x) &= 4 \times (y - x) \\
			4(y - x) &= 4 \times (5 - 3) \\
			4(y - x) &= 4 \times 2 \\
			4(y - x) &= 80 
		\end{align*}
	\end{multicols}
	
	Donc l'égalité est vraie pour $x=3$ et $y=5$
	
	.
\end{enumerate}
	
	
\section*{Exercice 30 page 106}

\begin{enumerate}
	\item On pose $x= 0$
	
	\begin{align*}
		6 \times x - 6 &= 6 \times 0 - 6 \\
		6 \times x - 6 &= 0 - 6 \\
		6 \times x - 6 &= - 6 
	\end{align*}
	
	$-6 \neq 0$, donc l'égalité n'est pas toujours vraie.
	
	\item On pose $x= 0$
		\begin{multicols}{2}
			\begin{align*}
				4(x+1) &= 4  \times (x+1) \\
				4(x+1) &= 4  \times (0+1) \\
				4(x+1) &= 4  \times 1 \\
				4(x+1) &= 4
			\end{align*}
			
			\begin{align*}
				4x + 1 &= 4  \times x + 1 \\
				4x+1 &= 4  \times 0 +1 \\
				4x+1 &= 0 + 1 \\
				4x+1 &= 1
			\end{align*}
		
	\end{multicols}
	
	
	$4 \neq 1$, donc l'égalité n'est pas toujours vraie.
	
	
	
	\item On pose $x= 1$
	\begin{multicols}{2}
		\begin{align*}
			2x + 3x &= 2 \times x + 3 \times x \\
			2x + 3x &= 2 \times 1 + 3 \times 1 \\
			2x + 3x &= 2 + 3  \\
			2x + 3x &= 5
		\end{align*}
		
		\begin{align*}
			6x^2 &= 6  \times x \times x \\
			6x^2 &= 6  \times 1 \times 1 \\
			6x^2 &= 6 \\
		\end{align*}
		
	\end{multicols}
	
	
	$5 \neq 6$, donc l'égalité n'est pas toujours vraie.
\end{enumerate}


\newpage

\section*{Exercice 31 page 106}


\begin{enumerate}
	\item On a $x= 1$
	
	\begin{multicols}{2}
		\begin{align*}
			A &= 2 \times x +1 +x + x\\
			A &= 2 \times 1 +1 +1 + 1\\
			A &= 2 +1 +1 + 1\\
			A &= 5
		\end{align*}
		
		\begin{align*}
			B &= 2 + x \times x + 2\\
			B &= 2 + 1 \times 1 + 2\\
			B &= 2 + 1 + 2\\
			B &= 5
		\end{align*}
	
		
	\end{multicols}
	
	
	\item On a $x= 3$
	
	\begin{multicols}{2}
		\begin{align*}
			A &= 2 \times x +1 +x + x\\
			A &= 2 \times 3 +1 +3 + 3\\
			A &= 6 +1 +3 + 3\\
			A &= 13
		\end{align*}
		
		\begin{align*}
			B &= 2 + x \times x + 2\\
			B &= 2 + 3 \times 3 + 2\\
			B &= 2 + 9 + 2\\
			B &= 13
		\end{align*}
		
		
	\end{multicols}
	
	\item On pose $x= 0$
	
	\begin{multicols}{2}
		\begin{align*}
			A &= 2 \times x +1 +x + x\\
			A &= 2 \times 0 +1 +0 + 0\\
			A &= 0 +1 +0 + 0\\
			A &= 1
		\end{align*}
		
		\begin{align*}
			B &= 2 + x \times x + 2\\
			B &= 2 + 0 \times 0 + 2\\
			B &= 2 + 0 + 2\\
			B &= 4
		\end{align*}
		
		
	\end{multicols}
	L'égalité n'est pas toujours vraie, $x=0$ est un conte-exemple.
\end{enumerate}

\section*{Exercice 33 page 107}

On pose $x=2$

\begin{align*}
	(x +3)(x+2) &= (x + 3) \times (x - 2)\\
	(x +3)(x+2) &= (2 + 3) \times (2 - 2)\\
	(x +3)(x+2) &= 5 \times 0\\
	(x +3)(x+2) &= 0\\
\end{align*}

%\begin{multicols}{3}
	\begin{align*}
		x^2 + 3x -2 &= x \times x + 3 \times x -2 \\
		x^2 + 3x -2 &= 2 \times 2 + 3 \times 2 -2 \\
		x^2 + 3x -2 &= 4 + 6 -2 \\
		x^2 + 3x -2 &= 8		
	\end{align*}
	
	Donc $(x +3)(x+2) \neq x^2 + 3x -2$.
	
	
	\begin{align*}
		2x + 1 &= 2 \times x + 1 \\
		2x + 1 &= 2 \times 2 + 1 \\
		2x + 1 &= 4 + 1 \\
		2x + 1 &= 5
	\end{align*}
	
	Donc $(x +3)(x+2) \neq 2x+1$.
	
	\begin{align*}
		x^2 - 6 &= x \times x - 6 \\
		x^2 - 6 &= 2 \times 2 - 6 \\
		x^2 - 6 &= 4 - 6 \\
		x^2 - 6 &= -2
	\end{align*}
	
	Donc $(x +3)(x+2) \neq x^2-6$.
	
	Par élimination, on a $ (x +3)(x+2) = x^2 +x -6$.
	
%\end{multicols}

\section*{Exercice 34 page 107}


	
	\subsection*{1.}
	On a $x=2$ :
	
	
	\begin{multicols}{2}
		\begin{align}
			3x+1 = 3 \times x + 1\\
			3x+1 = 3 \times 2 + 1\\
			3x+1 = 6 + 1\\
			3x+1 = 7
		\end{align}
		
		\begin{align}
			2x+3 = 2 \times x + 3\\
			2x+3 = 2 \times 2 + 3\\
			3x+1 = 4 + 3\\
			3x+1 = 7
		\end{align}
	\end{multicols}

	\textbf{Je sais que}, pour $x=2$, deux cotés consécutifs du rectangle ont une longueur de 7.
	
	\textbf{Or} si un rectangle a deux cotés consécutifs de même longueur, alors c'est un carré.
	
	\textbf{Donc} pour $x=2$ ce rectangle est un carré.
	
	\newpage
	
	\subsection*{2.}
	
	On prend $x=0$ :
	
	
	\begin{multicols}{2}
		\begin{align}
			3x+1 = 3 \times x + 1\\
			3x+1 = 3 \times 0 + 1\\
			3x+1 = 0 + 1\\
			3x+1 = 1
		\end{align}
		
		\begin{align}
			2x+3 = 2 \times x + 3\\
			2x+3 = 2 \times 0 + 3\\
			3x+1 = 0 + 3\\
			3x+1 = 3
		\end{align}
	\end{multicols}

	$x=0$ est un contre-exemple, l'égalité n'est pas toujours vraie. Donc ce rectangle n'est pas un carré pour toutes les valeurs de $x$.

\end{document}