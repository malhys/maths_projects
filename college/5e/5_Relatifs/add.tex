\subsection{Additionner deux nombres relatifs}

\begin{myprop}
	\iftoggle{eleve}{%
		Si deux nombres relatifs ont \hrulefill 
		
		\vspace*{0.2cm}
		
		\hrulefill
		\begin{itemize}
			\item \hrulefill
			\item \hrulefill
			\vspace*{0.2cm}
			
			\hrulefill
		\end{itemize}
	}{%
		Si deux nombres relatifs ont \kw{le même signe}, alors leur somme a :
		\begin{itemize}
			\item \kw{le même signe};
			\item pour distance à zéro, \kw{la somme} de leurs distances à zéro.
		\end{itemize}
	}
\end{myprop}

\begin{myexs}
	\begin{multicols}{2}
			\iftoggle{eleve}{%
				On veut calculer (+\num{2.4}) + (+\num{5.2}) :
				
				Les deux nombres sont positifs :
				\begin{itemize}
					\item leur somme est positive;
					\item on ajoute les distances à zéro \\ $\num{2.4} + \num{5.2} = \num{7.6}$
					\item[$\Rightarrow$] (+\num{2.4}) + (+\num{5.2}) = (+\num{7.6})
				\end{itemize} 
				
				
				On veut calculer (-\num{4.6}) + (-\num{3.7}) :
				
				\hrulefill :
				\begin{itemize}
					\item \hrulefill;
					\item \hrulefill \\ \hrulefill
					\item[$\Rightarrow$] \hrulefill
				\end{itemize} 
					
			}{%
		
			On veut calculer (+\num{2.4}) + (+\num{5.2}) :
			
			Les deux nombres sont positifs :
			\begin{itemize}
				\item leur somme est positive;
				\item on ajoute les distances à zéro \\ $\num{2.4} + \num{5.2} = \num{7.6}$
				\item[$\Rightarrow$] (+\num{2.4}) + (+\num{5.2}) = (+\num{7.6})
			\end{itemize} 
			
			
			On veut calculer (-\num{4.6}) + (-\num{3.7}) :
			
			Les deux nombres sont négatifs :
			\begin{itemize}
				\item leur somme est négative;
				\item on ajoute les distances à zéro \\ $\num{4.6} + \num{3.7} = \num{8.3}$
				\item[$\Rightarrow$] (-\num{4.6}) + (-\num{3.7}) = (-\num{8.3})
			\end{itemize} 
		}
	\end{multicols}
\end{myexs}

\begin{myprop}
	\iftoggle{eleve}{%
		Si deux nombres relatifs ont \hrulefill
		
		\vspace*{0.2cm}
		\hrulefill
		\begin{itemize}
			\item \hrulefill
			\vspace*{0.2cm}
			
			\hrulefill
			
			\item \hrulefill
			\vspace*{0.2cm}
			
			\hrulefill
		\end{itemize}
	}{%
	
		Si deux nombres relatifs ont \kw{des signes différents}, alors leur somme a :
		\begin{itemize}
			\item le signe du nombre qui à \kw{la plus grande distance à zéro};
			\item pour distance à zéro, \kw{la différence} de leurs distances à zéro.
		\end{itemize}
	}
\end{myprop}

\begin{myexs}
	%\begin{multicols}{2}
		On veut calculer (-\num{2.4}) + (+\num{5.2}) :
		
		Les deux nombres sont de signe différents :
		\begin{itemize}
			\item (+ \num{5.2}) a la plus grande distance à zéro, leur somme est positive;
			\item on soustrait les distances à zéro \\ $\num{5.2} - \num{2.4} = \num{2.8}$
			\item[$\Rightarrow$] (-\num{2.4}) + (+\num{5.2}) = (+\num{2.8})
		\end{itemize} 
		
		\vspace*{1cm}
		
		
		\iftoggle{eleve}{%
			\newpage
			On veut calculer (-\num{4.6}) + (+\num{3.7}) :
			
			\hrulefill
			\begin{itemize}
				\item \hrulefill
				
				\vspace*{0.2cm}
				
				\hrulefill
				\item \hrulefill \\ \hrulefill
				\item[$\Rightarrow$] \hrulefill
			\end{itemize} 
		}{%
			On veut calculer (-\num{4.6}) + (+\num{3.7}) :
			
			Les deux nombres sont de signe différents :
			\begin{itemize}
				\item (- \num{4.6}) a la plus grande distance à zéro, leur somme est négative;
				\item on soustrait les distances à zéro \\ $\num{4.6} - \num{3.7} = \num{0.9}$
				\item[$\Rightarrow$] (-\num{4.6}) + (-\num{3.7}) = (-\num{8.3})
			\end{itemize} 
		}
	%\end{multicols}
\end{myexs}



\begin{myprop}
	\iftoggle{eleve}{%
		La \hrulefill
	}{%
		La \kw{somme de deux nombres opposés} est égale à 0.
	}
	

\end{myprop}

\begin{myexs}

	\begin{multicols}{2}
		\iftoggle{eleve}{%
			\begin{equation*}
				(+ 25) + (-25) = 
			\end{equation*}
			
			\begin{equation*}
				(- \num{4.2}) + (+ \num{4.2}) = 
			\end{equation*}
		}{%
		\begin{equation*}
			(+ 25) + (-25) = 0
		\end{equation*}
		
		\begin{equation*}
			(- \num{4.2}) + (+ \num{4.2}) = 0
		\end{equation*}
		}
	\end{multicols}


\end{myexs}

\subsection{Soustraire deux nombres relatifs}

\begin{myprop}
	\iftoggle{eleve}{%
		Pour soustraire \hrulefill
	}{%
		Pour soustraire un nombre relatif, on \kw{ajoute son opposé}.
	}
	
\end{myprop}

\begin{myexs}
	%\begin{multicols}{2}
	On veut calculer $A = (-\num{5}) - (+2)$ :
	
	Pour soustraire $(+2)$, on ajoute son opposé : $(-2)$ :
	\begin{eqnarray*}
		A &=& (-5) - (+2) \\
		A &=& (-5) + (-2) \\
		A &=& (- 7)
	\end{eqnarray*} 
	
	\vspace*{1cm}
	
	On veut calculer $B = (+\num{3}) - (-\num{6.2})$ :
	
	\iftoggle{eleve}{%
		\vspace*{0.2cm}
		Pour soustraire $(-\num{6.2})$, \hrulefill
		\begin{eqnarray*}
			B &=& \hrulefill \\
			B &=& \hrulefill \\
			B &=& \hrulefill
		\end{eqnarray*}  
	}{%
		Pour soustraire $(-\num{6.2})$, on ajoute son opposé : $(+\num{6.2})$ :
		\begin{eqnarray*}
			B &=& (+\num{3}) - (-\num{6.2}) \\
			B &=& (+\num{3}) + (+\num{6.2}) \\
			B &=& (+ \num{9.2})
		\end{eqnarray*}  
	}
	
	%\end{multicols}
\end{myexs}
