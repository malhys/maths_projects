\begin{myobj}
	\begin{itemize}
		
		\item Savoir ce qu’est un nombre relatif et connaître le vocabulaire associé.
		\item Savoir comparer des nombres relatifs.
		\item Savoir additionner et soustraire des nombres relatifs.
		\item Savoir sur repérer sur un axe ou dans le le plan.
			
	\end{itemize}
\end{myobj}

%\vspace*{-0.3cm}

\begin{mycomp}
	\textbf{Représenter, Calculer, Raisonner}
%	\begin{itemize}
%		\item \textbf{Représenter}% (Re2)} :  produire et utiliser plusieurs représentations d’un nombre;
%		\item \textbf{Calculer}% (Ca1)} :  calculer avec des nombres rationnels, de manière exacte ou approchée en combinant astucieusement le calcul mental, le calcul posé et le calcul instrumenté ;
%		\item \textbf{Raisonner}% (Ra1)} :  résoudre des problèmes impliquant des grandeurs variées : mobiliser les connaissances nécessaires, analyser et exploiter ses erreurs, mettre à l’essai plusieurs solutions.		
%	\end{itemize}
\end{mycomp}


