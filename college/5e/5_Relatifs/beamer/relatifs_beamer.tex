\documentclass[xcolor={dvipsnames}]{beamer}
%\usepackage[utf8]{inputenc}
\usetheme{Madrid}
%\usetheme{Malmoe}
\usecolortheme{beaver}
%\usecolortheme{rose}

\input{../../../../utils_maths_beamer}


\usepackage{../../../../pas-math}
\usepackage{../../../../moncours_beamer}

\usepackage{amssymb,amsmath}


\newcommand{\myitem}{\item[\textbullet]}

\graphicspath{{../img/}}

\title{Séquence 5 : Nombres relatifs}
%\author{O. FINOT}\institute{Collège S$^t$ Bernard}
\date{}

%
\AtBeginSection[]
{
	\begin{frame}
		\frametitle{}
		\tableofcontents[currentsection, hideallsubsections]
	\end{frame} 

}
%
%
%\AtBeginSubsection[]
%{
%	\begin{frame}
%		\frametitle{Sommaire}
%		\tableofcontents[currentsection, currentsubsection]
%	\end{frame} 
%}

\begin{document}



\begin{frame}
  \titlepage 
\end{frame}


	

\begin{frame}
	\begin{block}{Objectifs}
		\begin{itemize}
			
		\item Savoir ce qu’est un nombre relatif et connaître le vocabulaire associé.
		\item Savoir comparer des nombres relatifs.
		\item Savoir additionner et soustraire des nombres relatifs.
		\item Savoir sur repérer sur un axe ou dans le le plan.
			
			\end{itemize}
	\end{block}
\end{frame}

\begin{frame}
	\begin{block}{Compétences travaillées}
		\begin{itemize}
			\item \kword{Représenter (Re2)} :  produire et utiliser plusieurs représentations d’un nombre;
			\item \kword{Calculer (Ca1)} :  calculer avec des nombres rationnels, de manière exacte ou approchée en combinant astucieusement le calcul mental, le calcul posé et le calcul instrumenté ;
			\item \kword{Raisonner (Ra1)} :  résoudre des problèmes impliquant des grandeurs variées : mobiliser les connaissances nécessaires, analyser et exploiter ses erreurs, mettre à l’essai plusieurs solutions.		
		\end{itemize}
	\end{block}
\end{frame}



\section{Définitions}




\begin{frame}
	\begin{mydefs}
		\begin{itemize}
			\item Un nombre supérieur à 0 est un \kword{nombre positif},\pause un nombre inférieur à 0 est un \kword{nombre négatif}.
			
			\begin{center}
				\includegraphics[scale=0.5]{relatifs}\pause
			\end{center}
			
			\item Les nombres positifs et négatifs forment l'ensembles des \kword{nombres relatifs}.\pause
			
			\item Un nombre relatif est composé d'\kword{un signe} (+ ou -) et d'une \kword{distance à zéro}.\pause
			
			\item Deux \kword{nombres opposés} ont la \kword{même distance à zéro} et des \kword{signes différents}. 
			
			
		\end{itemize}
	\end{mydefs}
	
\end{frame}


\begin{frame}
	\begin{myexs}
		\begin{itemize}
			\item $+7$ est un nombre \pause positif, sa distance à zéro est \pause $7$; 
			\begin{center}
				\includegraphics[scale=0.6]{ex1}\pause
			\end{center}
			\item $\num{-4}$ est un nombre \pause négatif, sa distance à zéro est \pause $\num{4}$;
			\begin{center}
				\includegraphics[scale=0.6]{ex2}\pause
			\end{center}
			
			\item $0$ est \pause à la fois un nombre positif et négatif.%, sa distance à zéro est 0.
			\item $-10$ et $+10$ sont \pause opposés.
			\begin{center}
				\includegraphics[scale=0.5]{opposes}
			\end{center}
		\end{itemize}
	\end{myexs}
\end{frame}
\end{document}