\documentclass[xcolor={dvipsnames}]{beamer}
%\usepackage[utf8]{inputenc}
\usetheme{Madrid}
%\usetheme{Malmoe}
\usecolortheme{beaver}
%\usecolortheme{rose}

\input{../../../../utils_maths_beamer}


\usepackage{../../../../pas-math}
\usepackage{../../../../moncours_beamer}

\usepackage{amssymb,amsmath}


\newcommand{\myitem}{\item[\textbullet]}

\graphicspath{{../img/}}

\title{Séquence 3 : Nombres relatifs}
%\author{O. FINOT}\institute{Collège S$^t$ Bernard}
\date{18 Novembre 2020}

%
\AtBeginSection[]
{
	\begin{frame}
		\frametitle{}
		\tableofcontents[currentsection, hideallsubsections]
	\end{frame} 

}
%
%
\AtBeginSubsection[]
{
	\begin{frame}
		\frametitle{Sommaire}
		\tableofcontents[currentsection, currentsubsection]
	\end{frame} 
}

\begin{document}



\begin{frame}
  \titlepage 
\end{frame}


	

\begin{frame}
	\begin{block}{Objectifs}
		\begin{itemize}
			
		\item Savoir ce qu’est un nombre relatif et connaître le vocabulaire associé.
		\item Savoir comparer des nombres relatifs.
		\item Savoir additionner et soustraire des nombres relatifs.
		\item Savoir sur repérer sur un axe ou dans le le plan.
			
			\end{itemize}
	\end{block}
\end{frame}

\begin{frame}
	\begin{block}{Compétences travaillées}
		\begin{itemize}
			\item \kword{Représenter}% (Re2)} :  produire et utiliser plusieurs représentations d’un nombre;
			\item \kword{Calculer}% (Ca1)} :  calculer avec des nombres rationnels, de manière exacte ou approchée en combinant astucieusement le calcul mental, le calcul posé et le calcul instrumenté ;
			\item \kword{Raisonner}% (Ra1)} :  résoudre des problèmes impliquant des grandeurs variées : mobiliser les connaissances nécessaires, analyser et exploiter ses erreurs, mettre à l’essai plusieurs solutions.		
		\end{itemize}
	\end{block}
\end{frame}



\section{Définitions}




\begin{frame}
	\begin{mydefs}
		\begin{itemize}
			\item Un nombre supérieur à 0 est \pause un \kword{nombre positif},\pause un nombre inférieur à 0 est un \kword{nombre négatif}.
			
			\begin{center}
				\includegraphics[scale=0.5]{relatifs}\pause
			\end{center}
			
			\item Les nombres positifs et négatifs forment l'ensembles des \kword{nombres relatifs}.\pause
			
			\item Un nombre relatif est composé d'\kword{un signe} (+ ou -) et d'une \kword{distance à zéro}.\pause
			
			\item Deux \kword{nombres opposés} ont la \kword{même distance à zéro} et des \kword{signes différents}. 
			
			
		\end{itemize}
	\end{mydefs}
	
\end{frame}


\begin{frame}
	\begin{myexs}
		\begin{itemize}
			\item $+7$ est un nombre \pause positif, sa distance à zéro est \pause $7$; 
			\begin{center}
				\includegraphics[scale=0.6]{ex1}\pause
			\end{center}
			\item $\num{-4}$ est un nombre \pause négatif, sa distance à zéro est \pause $\num{4}$;
			\begin{center}
				\includegraphics[scale=0.6]{ex2}\pause
			\end{center}
			
			\item $0$ est \pause à la fois un nombre positif et négatif.%, sa distance à zéro est 0.
			\item $-10$ et $+10$ sont \pause des nombres opposés.
			\begin{center}
				\includegraphics[scale=0.5]{opposes}
			\end{center}
		\end{itemize}
	\end{myexs}
\end{frame}

\section{Des nombres pour se repérer et à comparer}

%\input{./reperage_v2}

\section{Addition et soustraction de deux nombres relatifs}

\subsection{Soustraire deux nombres relatifs}

\begin{frame}
	\begin{myprop}
		Si deux nombres relatifs ont \kword{le même signe}, alors leur somme a :\pause
		\begin{itemize}
			\item \kword{le même signe};\pause
			\item pour distance à zéro, \pause \kword{la somme} de leurs distances à zéro.\pause
		\end{itemize}
	\end{myprop}


	\begin{myex}
		On veut calculer (+\num{2.4}) + (+\num{5.2}) : \pause
		
		Les deux nombres sont positifs :\pause
		\begin{itemize}
			\item leur somme est positive;\pause
			\item on ajoute les distances à zéro \\ $\num{2.4} + \num{5.2} = \num{7.6}$ \pause
			\item[$\Rightarrow$] (+\num{2.4}) + (+\num{5.2}) = \pause (+\num{7.6})
		\end{itemize}
	\end{myex}
\end{frame}


\begin{frame}
	\begin{myprop}
		Si deux nombres relatifs ont \kword{le même signe}, alors leur somme a :
		\begin{itemize}
			\item \kword{le même signe};
			\item pour distance à zéro,  \kword{la somme} de leurs distances à zéro. 
		\end{itemize}
	\end{myprop}
	
	
	\begin{myex}
		On veut calculer (-\num{4.6}) + (-\num{3.7}) :
		
		Les deux nombres sont négatifs :\pause
		\begin{itemize}
			\item leur somme est négative;\pause
			\item on ajoute les distances à zéro \\ $\num{4.6} + \num{3.7} = \num{8.3}$ \pause
			\item[$\Rightarrow$] (-\num{4.6}) + (-\num{3.7}) = \pause (-\num{8.3})
		\end{itemize}
	\end{myex}
\end{frame}


\begin{frame}
	\begin{myprop}
		Si deux nombres relatifs ont \kword{des signes différents}, alors leur somme a :\pause
		\begin{itemize}
			\item le signe du nombre qui à \kword{la plus grande distance à zéro}; \pause
			\item pour distance à zéro, \kword{la différence} de leurs distances à zéro. \pause
		\end{itemize}
	\end{myprop}


	\begin{myex}
		On veut calculer (-\num{2.4}) + (+\num{5.2}) : \pause
		
		Les deux nombres sont de signe différents : \pause
		\begin{itemize}
			\item (+ \num{5.2}) a la plus grande distance à zéro, \pause leur somme est positive; \pause
			\item on soustrait les distances à zéro  \pause \\ $\num{5.2} - \num{2.4} = \num{2.8}$ \pause
			\item[$\Rightarrow$] (-\num{2.4}) + (+\num{5.2}) = \pause (+\num{2.8})
		\end{itemize}
	\end{myex}
\end{frame}


\begin{frame}
	\begin{myprop}
		Si deux nombres relatifs ont \kword{des signes différents}, alors leur somme a :\pause
		\begin{itemize}
			\item le signe du nombre qui à \kword{la plus grande distance à zéro}; \pause
			\item pour distance à zéro, \kword{la différence} de leurs distances à zéro. \pause
		\end{itemize}
	\end{myprop}
	
	
	\begin{myex}
		On veut calculer (-\num{4.6}) + (+\num{3.7}) : \pause
		
		Les deux nombres sont de signe différents : \pause
		\begin{itemize}
			\item (- \num{4.6}) a la plus grande distance à zéro, leur somme est négative; \pause
			\item on soustrait les distances à zéro \pause \\ $\num{4.6} - \num{3.7} = \num{0.9}$ \pause
			\item[$\Rightarrow$] (-\num{4.6}) + (-\num{3.7}) = \pause (-\num{0.9})
		\end{itemize}
	\end{myex}
\end{frame}


\begin{frame}
	\begin{myprop}
		La \kword{somme de deux nombres opposés} est égale à 0.
		
	\end{myprop}
	
	\begin{myexs}
		
		
			\begin{equation*}
			(+ 25) + (-25) = 0
			\end{equation*}
			
			\begin{equation*}
			(- \num{4.2}) + (+ \num{4.2}) = 0
			\end{equation*}
		
		
		
	\end{myexs}
\end{frame}




\subsection{Soustraire deux nombres relatifs}


\begin{frame}
	\begin{myprop}
		Pour soustraire un nombre relatif, on \pause \kword{ajoute son opposé}.\pause
	\end{myprop}
	
	\begin{myex}
		
		On veut calculer $A = (-\num{5}) - (+2)$ : \pause
		
		Pour soustraire $(+2)$, \pause on ajoute son opposé : $(-2)$ : \pause
		\begin{eqnarray*}
			A &=& (-5) - (+2) \\ \pause
			A &=& (-5) + (-2) \\ \pause
			A &=& (- 7) \pause
		\end{eqnarray*} 
	\end{myex}	
\end{frame}
	

\begin{frame}
	\begin{myprop}
		Pour soustraire un nombre relatif, on \kword{ajoute son opposé}.
	\end{myprop}
	
	\begin{myex}
		
		On veut calculer $B = (+\num{3}) - (-\num{6.2})$ :\pause
		
		Pour soustraire $(-\num{6.2})$,  \pause on ajoute son opposé : $(+\num{6.2})$ : \pause
		\begin{eqnarray*}
			B &=& (+\num{3}) - (-\num{6.2}) \\ \pause
			B &=& (+\num{3}) + (+\num{6.2}) \\ \pause
			B &=& (+ \num{9.2})
		\end{eqnarray*} 
	\end{myex}	
\end{frame}
	
	
\end{document}