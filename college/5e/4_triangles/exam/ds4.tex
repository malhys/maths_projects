\documentclass[a4paper,11pt]{exam}
\usepackage{helvet}
\printanswers % pour imprimer les réponses (corrigé)
%\noprintanswers % Pour ne pas imprimer les réponses (énoncé)
\addpoints % Pour compter les points
% \noaddpoints % pour ne pas compter les points
%\qformat{\textbf{\thequestion ) } }
\qformat{\textbf{\thequestion )} (\thepoints) \\} % Pour définir le style des questions (facultatif)
\usepackage{color} % définit une nouvelle couleur
\shadedsolutions % définit le style des réponses
% \framedsolutions % définit le style des réponses
\definecolor{SolutionColor}{rgb}{0.8,0.9,1} % bleu ciel
\renewcommand{\solutiontitle}{\noindent\textbf{Solution:}\par\noindent} % Définit le titre des solutions




\makeatletter

\def\maketitle{{\centering%
	\par{\huge\textbf{\@title}}%
	\par{\@date}%
	\par}}

\makeatother

\lhead{NOM Pr\'enom :}
\rhead{\textbf{Les r\'eponses doivent \^etre justifi\'ees}}
\cfoot{\thepage / \pageref{LastPage}}


%\usepackage{../../pas-math}
%\usepackage{../../moncours}


%\usepackage{pas-cours}
%-------------------------------------------------------------------------------
%          -Packages nécessaires pour écrire en Français et en UTF8-
%-------------------------------------------------------------------------------
\usepackage[utf8]{inputenc}
\usepackage[frenchb]{babel}
\usepackage[T1]{fontenc}
\usepackage{lmodern}
%-------------------------------------------------------------------------------

%-------------------------------------------------------------------------------
%                          -Outils de mise en forme-
%-------------------------------------------------------------------------------
\usepackage{hyperref}
\hypersetup{pdfstartview=XYZ}
\usepackage{enumerate}
\usepackage{graphicx}
\usepackage{multicol}

\usepackage{anysize} %%pour pouvoir mettre les marges qu'on veut
%\marginsize{2.5cm}{2.5cm}{2.5cm}{2.5cm}

\usepackage{indentfirst} %%pour que les premier paragraphes soient aussi indentés
%-------------------------------------------------------------------------------


%-------------------------------------------------------------------------------
%                  -Nécessaires pour écrire des mathématiques-
%-------------------------------------------------------------------------------
\usepackage{amsfonts}
\usepackage{amssymb}
\usepackage{amsmath}
\usepackage{amsthm}
\usepackage{tikz}
%-------------------------------------------------------------------------------

%-------------------------------------------------------------------------------
%                     -Mise en forme d'exercices-
%-------------------------------------------------------------------------------
\newtheoremstyle{exostyle}
{\topsep}% espace avant
{\topsep}% espace apres
{}% Police utilisee par le style de thm
{}% Indentation (vide = aucune, \parindent = indentation paragraphe)
{\bfseries}% Police du titre de thm
{.}% Signe de ponctuation apres le titre du thm
{ }% Espace apres le titre du thm (\newline = linebreak)
{\thmname{#1}\thmnumber{ #2}\thmnote{. \normalfont{\textit{#3}}}}% composants du titre du thm : \thmname = nom du thm, \thmnumber = numéro du thm, \thmnote = sous-titre du thm

\theoremstyle{exostyle}
\newtheorem{exercice}{Exercice}

\newenvironment{questions}{
\begin{enumerate}[\hspace{12pt}\bfseries\itshape a.]}{\end{enumerate}
} %mettre un 1 à la place du a si on veut des numéros au lieu de lettres pour les questions 
%-------------------------------------------------------------------------------



%-------------------------------------------------------------------------------
%                    - Racourcis d'écriture -
%-------------------------------------------------------------------------------

% Angles orientés (couples de vecteurs)
\newcommand{\aopp}[2]{(\vec{#1}, \vec{#2})} %Les deuc vecteurs sont positifs
\newcommand{\aopn}[2]{(\vec{#1}, -\vec{#2})} %Le second vecteur est négatif
\newcommand{\aonp}[2]{(-\vec{#1}, \vec{#2})} %Le premier vecteur est négatif
\newcommand{\aonn}[2]{(-\vec{#1}, -\vec{#2})} %Les deux vecteurs sont négatifs

%Ensembles mathématiques
\newcommand{\naturels}{\mathbb{N}} %Nombres naturels
\newcommand{\relatifs}{\mathbb{Z}} %Nombres relatifs
\newcommand{\rationnels}{\mathbb{Q}} %Nombres rationnels
\newcommand{\reels}{\mathbb{R}} %Nombres réels
\newcommand{\complexes}{\mathbb{C}} %Nombres complexes
%-------------------------------------------------------------------------------




%\usepackage{fullpage}
\author{\ }
\date{12 Février 2020}
\title{$5^e G$ : DS num\'ero 4}


\usepackage{setspace}
% \singlespacing
\onehalfspacing
% \doublespacing



\begin{document}
%	\usepackage{fancyhdr}
%	
%	\pagestyle{fancy}
%	\fancyhf{}
	%\rhead{Share\LaTeX}

	\maketitle
	
\begin{center}
	\textbf{Calculatrice interdite}
\end{center}

%
%{ (Ch2)} :   (sur une feuille de papier, avec des objets, à l’aide de logiciels), chercher des exemples ou des contre-exemples;
%\item \kw{Raisonner (Ra3)} :   ;
%\item \kw{Communiquer (Co2)} :  expliquer à l’oral ou à l’écrit sa démarche ou son raisonnement; 
\begin{small}
	\begin{center}
		\begin{tabular}{|@{\ }l@{\ }|@{\ }c@{\ }|@{\ }c@{\ }|@{\ }c@{\ }|@{\ }c@{\ }|}
			\hline
			\textbf{Compétence} & \textbf{MI} & \textbf{MF} & \textbf{MS} & \textbf{TBM} \\
			\hline
			\textbf{Chercher} (observer, questionner, manipuler, expérimenter) &  \ \ & \ \ & \ \ & \ \  \\
			\hline	
			\textbf{Raisonner} (utiliser un raisonnement logique pour parvenir à une conclusion) & \ \ & \ \ &  \ \  & \ \ \\
			\hline
			\textbf{Communiquer} (Expliquer sa démarche, son raisonnement ) &  \ \ & \ \ & \ \ & \ \  \\
			\hline
		\end{tabular}
	\end{center}
\end{small}	

	
	
\section{Week-end entre amies (4 points)}

Trois amies vivent dans trois villes différentes. Elles souhaitent passer un week-end ensemble. Elles veulent parcourir la même distance <<à vol d'oiseau>>. Elles habitent à Lille, Strassbours et Bayonne.

\begin{center}
	\includegraphics[scale=1.55]{img/france}
\end{center}

\begin{questions}
	\question[4] Trouver \textbf{\underline{sur cette carte}} l'endroit idéal pour leur week-end. Expliquer la démarche, laisser apparents tous les traits de construction et coder la figure.
\end{questions}

	




\section{Construction}

\begin{questions}
	\question Construire un triangle $ABC$, tel que $AB$=\num{4.3} cm, $BC$ = \num{6.5} cm et $AC$=\num{8.3} cm.
	
	\question Tracer la hauteur issue de $B$, son pied est le point $E$. Coder la figure.
	
	\question Tracer la médiatrice de $[AC]$, elle coupe $(AC)$ en $D$ et $(BC)$ en F. Coder la figure
	
	\question Tracer les segments $[BD]$ et $[EF]$. 
\end{questions}
  

\section{Angle droit ou pas ? }

%\begin{multicols}{2}
	\begin{center}
		\includegraphics[scale=0.15]{img/fig}
	\end{center}
	
	Les points $H$, $A$  et $P$ sont alignés.
	
	\begin{questions}
		\question \`A partir des informations codées sur la figure, dire si la triangle $CAT$ est rectangle en A.
	\end{questions}
	\begin{solution}
		Calcul de la mesure de l'angle \monAngle{CAH} :
		
		Dans le triangle CAH, on a \monAngle{C} + \monAngle{A} + \monAngle{H} = 180\degree.
		
			\begin{eqnarray*}
				\widehat{A}  &=& 180 - (\widehat{C} + \widehat{H}) \\
				\widehat{A}  &=& 180 - (45 + 90) \\
				\widehat{A}  &=& 45
			\end{eqnarray*}
		L'angle \monAngle{CAH} mesure 45\degree.\\
		
		
		
		Calcul de la mesure de l'angle \monAngle{TAP} :
		
		Dans le triangle TAP, on a \monAngle{T} + \monAngle{A} + \monAngle{P} = 180\degree.
		
		\begin{eqnarray*}
			\widehat{A}  &=& 180 - (\widehat{T} + \widehat{P}) \\
			\widehat{A}  &=& 180 - (43 + 90) \\
			\widehat{A}  &=& 47
		\end{eqnarray*}
		L'angle \monAngle{TAP} mesure 47\degree.\\
		
		Calcul de la mesure de l'angle \monAngle{CAT} :
		Je sais que les points $H$, $A$ et $P$ sont alignés donc \monAngle{HAP} mesure 180\degree.
		
		On a donc :
		
		\begin{eqnarray*}
			\widehat{CAT}  &=& 180 - (\widehat{CAH} + \widehat{TAP}) \\
			\widehat{CAT}  &=& 180 - (45 + 47) \\
			\widehat{CAT}  &=& 88
		\end{eqnarray*}
	
		L'angle \monAngle{CAT} mesure 88\degree et non 90\degree, donc le triangle $CAT$ n'est pas rectangle en $A$.
	\end{solution}



	
%\end{multicols}

\section{Argumenter}

Le professeur Mathétic demande à ses élèves de  de construire un triangle ABC respectant les conditions suivantes :

\begin{itemize}
	\item Un périmètre égal à 13 cm;
	\item $AB$ = 4 cm.
\end{itemize}

3 élèves font les propositions suivantes :
\begin{itemize}
	\item Florie : $AC = 5$ cm et $BC= \num{4}$cm.
	\item Jeanne :  $AC = \num{6.5}$ cm et $BC= \num{2.5}$ cm.
	\item Jayan : $BC$ = 7cm.
\end{itemize}

\begin{questions}
	\question Pour chacun de ces 3 élèves, expliquer, en justifiant ce que l'on peut penser de leur proposition.
\end{questions}

\newpage

\section*{Bonus : Triangle dans un cercle (3 points)}

$A$ et $B$ sont deux points d'un cercle de centre $O$, tel que $\widehat{AOB}=54\degree$

\begin{center}
	\includegraphics[scale=0.15]{img/cercle}
\end{center}

\begin{questions}
	\question[3] Calculer la mesure de l'angle $\widehat{OAB}$. Expliquer la démarche et justifier.
\end{questions}
\begin{solution}
	$A$ et $B$ sont deux points d'un cercle de centre O. $[OA]$ et $[OB]$ sont des rayons de ce cercles, ils ont la même longueur. Donc le triangle $OAB$ est isocèle en $O$.
	
	Dans le triangle $OAB$ on a \monAngle{O} + \monAngle{A} + \monAngle{B} = 180\degree, et \monAngle{A} = \monAngle{B}.
	
	Dpnc : 
	
	\begin{eqnarray*}
		\widehat{A} &=& (180 - 54) \div 2\\
		\widehat{A} &=& 126 \div 2 \\
		\widehat{A} &=& 63 \\
	\end{eqnarray*}

Donc l'angle $\widehat{OAB}$ mesure 63\degree.
\end{solution}
\label{LastPage}

\end{document}