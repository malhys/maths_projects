\section{Argumenter}

Le professeur Mathétic demande à ses élèves de  de construire un triangle ABC respectant les conditions suivantes :

\begin{itemize}
	\item Un périmètre égal à 13 cm;
	\item $AB$ = 4 cm.
\end{itemize}

3 élèves font les propositions suivantes :
\begin{itemize}
	\item Daphné : $AC = 5$ cm et $BC= \num{4}$cm.
	\item Ophélie :  $AC = \num{6.5}$ cm et $BC= \num{2.5}$ cm.
	\item Nino : $BC$ = 7cm.
\end{itemize}

\begin{questions}
	\question Pour chacun de ces 3 élèves, expliquer, en justifiant ce que l'on peut penser de leur proposition.
	
	\begin{solution}
		\begin{enumerate}
			\item Dans le triangle proposé par Daphné, le plus grand coté est $AC$, avec 5 cm. On a $AC < AB + BC$, donc le triangle est constructible.
			
			\item Dans le triangle proposé par Ophélie, le plus grand coté est $AC$, avec \num{6.5} cm. On a $AC = AB + BC$, donc le triangle est plat mais constructible.
			
			\item Le périmètre du triangle ABC est de 13 cm, Nino propose $BC$=7cm, on a donc $AC$ = 2 cm ($13 - (7 + 4)$).
			
			Dans ce triangle la plus grand coté est $BC$, et $BC > AB + AC$. Donc il n'est pas constructible.
		\end{enumerate}
	\end{solution}
\end{questions}