\iftoggle{eleve}{%
	\begin{myprop}
		\hrulefill 
		
		\vspace*{0.2cm}
		\hrulefill 
	\end{myprop}
	
	
	\begin{myexs}
		\begin{multicols}{2}
%			\vspace*{0.5cm}
			Dans le triangle $ABC$, on a 
			
			\vspace*{0.2cm}
			\hrulefill
			\vspace*{1cm}
			
			
			Dans un triangle isocèle, \hrulefill 
			
			\vspace*{0.2cm}
			\hrulefill 
			
			\vspace*{0.2cm}
			\hrulefill 
			
			\vspace*{1.5cm}
			
			
			Dans un triangle équilatéral, \hrulefill 
			
			\vspace*{0.2cm}
			\hrulefill 
		
			
			\vspace*{0.2cm}
			\hrulefill 
			
				\vspace*{1.5cm}
			
			Dans un triangle rectangle, \hrulefill 
			
			\vspace*{0.2cm}
			\hrulefill 
			
			\vspace*{0.2cm}
			\hrulefill 
			
			
			\begin{center}	
				\includegraphics[scale=0.16]{quelconque2}	
				\includegraphics[scale=0.16]{isocele2}	
				\includegraphics[scale=0.16]{equilateral2}
				\includegraphics[scale=0.16]{rectangle2}
			\end{center}
		\end{multicols}
	\end{myexs}
	
}{%
	\begin{myprop}
		La \kw{somme des mesures} des angles d'un triangle vaut 180\degree.
	\end{myprop}
	
	
	\begin{myexs}
		\begin{multicols}{2}
			\vspace*{1cm}
			Dans le triangle $ABC$, on a \\ $\hat{A} + \hat{B} + \hat{C} = 180\degree$.
			\vspace*{1cm}
			
			
			Dans un triangle isocèle, les deux angles à la base sont égaux (ici 30\degree).
			\vspace*{2.5cm}
			
			
			Dans un triangle équilatéral, tous les angles sont égaux et mesurent 60\degree.
			\vspace*{2.5cm}
			
			Dans un triangle rectangle, la somme des mesures des angles non droits vaut 90\degree.
			
			
			\begin{center}	
				\includegraphics[scale=0.18]{quelconque}	
				\includegraphics[scale=0.18]{isocele}	
				\includegraphics[scale=0.18]{equilateral}
				\includegraphics[scale=0.18]{rectangle}
			\end{center}
		\end{multicols}
	\end{myexs}
}

