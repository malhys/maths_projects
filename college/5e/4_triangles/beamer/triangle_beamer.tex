\documentclass[xcolor={dvipsnames}]{beamer}
%\usepackage[utf8]{inputenc}
\usetheme{Madrid}
%\usetheme{Malmoe}
\usecolortheme{beaver}
%\usecolortheme{rose}

\input{../../../../utils_maths_beamer}


\usepackage{../../../../pas-math}
\usepackage{../../../../moncours_beamer}

\usepackage{amssymb,amsmath}


\newcommand{\myitem}{\item[\textbullet]}

\graphicspath{{../img/}}

\title{Séquence 4 : Géométrie du triangle}
%\author{O. FINOT}\institute{Collège S$^t$ Bernard}

%
\AtBeginSection[]
{
	\begin{frame}
		\frametitle{}
		\tableofcontents[currentsection, hideallsubsections]
	\end{frame} 

}
%
%
%\AtBeginSubsection[]
%{
%	\begin{frame}
%		\frametitle{Sommaire}
%		\tableofcontents[currentsection, currentsubsection]
%	\end{frame} 
%}

\begin{document}



\begin{frame}
  \titlepage 
\end{frame}


	
\begin{frame}
	\begin{myobj}
	\begin{itemize}
		\item Reconnaître un segment, une demie-droite, une droite et savoir les tracer;
		\item Tracer avec l’équerre la droite perpendiculaire à une droite donnée passant par un point donné;
		\item Tracer avec la règle et l’équerre la droite parallèle à une droite donnée passant par un point donné;
		\item Déterminer la distance entre deux points, entre un point et une droite;
		\item Savoir coder et lire une figure.
	\end{itemize}
\end{myobj}

\begin{mycomp}
	\begin{itemize}
		\item \kw{Modéliser} 
		\item \kw{Représenter} 
		\item \kw{Raisonner} 
		\item \kw{Communiquer}
		
	\end{itemize}
\end{mycomp}
\end{frame}

\section{Inégalité triangulaire}


\begin{frame}
	\begin{myprop}
		
			Dans un triangle la longueur d'un coté est inférieure à la somme des longueurs des deux autres côtés.\pause
			

		
	\end{myprop}
	
	\begin{block}{Méthode}
		Pour vérifier qu'un \kword{triangle est constructible}, \pause on vérifie que la longueur du plus grand côté  est inférieure à la somme des deux autres.
	\end{block}
\end{frame}


\begin{frame}
	\begin{myexs}
		
		\begin{itemize}
			\item Dans le triangle ABC ci-contre on a $AB < AC + CB.$
				\begin{center}
					\includegraphics[scale=0.25]{triangle1}\pause
					
					%\includegraphics[scale=0.2]{triangle2}
				\end{center}
			
			\item 	Un triangle de cotés 8 cm, 5 cm et 6 cm est constructible (8 < 11).
		\end{itemize}
		
		
		
	\end{myexs}
\end{frame}

\begin{frame}
	\begin{myexs}
		
		\begin{itemize}
			\item Le triangle $DEF$, tel que $DE = 7$ cm, $DF = 3$ cm et $FE = 4$ cm est plat, les points sont alignés ($4 + 3 = 7$).
			\begin{center}
				%\includegraphics[scale=0.25]{triangle1}\pause
				
				\includegraphics[scale=0.2]{triangle2}\pause
			\end{center}
			
			\item 	Un triangle de coté 10 cm, 4 cm et 5 cm n'est pas constructible ($10 > 4 + 5$).
		\end{itemize}
		
		
		
	\end{myexs}
\end{frame}
\end{document}