\documentclass[xcolor={dvipsnames}]{beamer}
%\usepackage[utf8]{inputenc}
%\usetheme{Madrid}
%\usetheme{CambridgeUS}
\usetheme{Malmoe}
%\usecolortheme{beaver}
\usecolortheme{seahorse}

\input{../../../../utils_maths_beamer}


\usepackage{../../../../pas-math}
\usepackage{../../../../moncours_beamer}

\usepackage{amssymb,amsmath}


\newcommand{\myitem}{\item[\textbullet]}

\graphicspath{{../img/}}

\title{Séquence 9 : Angles et parallélisme}
\subtitle{Correction des exercices}
%\author{O. FINOT}\institute{Collège S$^t$ Bernard}

%
%\AtBeginSection[]
%{
%	\begin{frame}
%		\frametitle{}
%		\tableofcontents[currentsection, hideallsubsections]
%	\end{frame} 
%
%}
%
%
%\AtBeginSubsection[]
%{
%	\begin{frame}
%		\frametitle{Sommaire}
%		\tableofcontents[currentsection, currentsubsection]
%	\end{frame} 
%}

\begin{document}



%\begin{frame}
%  \titlepage 
%\end{frame}


	

\begin{frame}
	\frametitle{Exercice 16}
	\begin{columns}
		\begin{column}{0.4\textwidth}
			\begin{center}
				\includegraphics[scale=0.5]{16}\pause
			\end{center}
		\end{column}
		\begin{column}{0.6\textwidth}
			\begin{enumerate}
				\item Les points $A$, $B$ et $C$ sont alignés donc l'angle $\widehat{ABC}$ mesure 180\degree . 
				
				\begin{align*}
				\widehat{xBy} &= \widehat{ABC} - (\widehat{ABx}+ \widehat{yBC}) \\
				\widehat{xBy} &= 180 \degree - (59 \degree + 62 \degree) \\
				\widehat{xBy} &= 59\degree 
				\end{align*}\pause
				\item Les angles $\widehat{ABx}$ et $\widehat{xBy}$ ont la même mesure, donc la demi-droite $[Bx)$ est la bissectrice de l'angle $\widehat{ABy} $.
				%		\item 
			\end{enumerate}
		\end{column}
	\end{columns}
	
	
	
	
	
\end{frame}

\begin{frame}
	\frametitle{Exercice 17}
	
	\begin{columns}
		\begin{column}{0.4\textwidth}
			\begin{center}
				\includegraphics[scale=0.5]{17}\pause
			\end{center}
		\end{column}
		\begin{column}{0.6\textwidth}
			D'après le codage, les angles $\widehat{uOx}$ et $\widehat{xOv}$ ont la même mesure. \pause
			
			\begin{align*}
				\widehat{zOv} &= \widehat{zOx} + \widehat{xOv} \\
				\widehat{zOv} &= 90\degree + 24\degree \\
				\widehat{zOv} &= 104\degree
			\end{align*}
			
			L'angle $\widehat{zOv}$ mesure 104\degree.\pause
			
			De la même manière, l'angle $\widehat{tOu}$ mesure 104\degree.
		\end{column}
	\end{columns}
	
	
	
\end{frame}



\begin{frame}
	\frametitle{Exercice 18}
	
	\begin{columns}
		\begin{column}{0.4\textwidth}
			\begin{center}
				\includegraphics[scale=0.5]{18}\pause
			\end{center}
		\end{column}
		\begin{column}{0.6\textwidth}
			\begin{enumerate}[a.]
				\item L'angle opposé par le sommet à l'angle $\widehat{xOv}$ est l'angle $\widehat{tOz}$ \pause , il mesure 38\degree. \pause
				\item L'angle opposé par le sommet à l'angle $\widehat{xOz}$ est l'angle $\widehat{tOv}$ \pause , il mesure 142\degree. \pause
				\item L'angle opposé par le sommet à l'angle $\widehat{uOz}$ est l'angle $\widehat{yOv}$ \pause , il mesure 138\degree. \pause\\
				\item L'angle opposé par le sommet à l'angle $\widehat{tOu}$ est l'angle $\widehat{xOy}$ \pause , il mesure 100\degree. \pause
			\end{enumerate}
		\end{column}
	\end{columns}
	
	
	
\end{frame}


\begin{frame}
	\frametitle{Exercice 19}
	
	\begin{columns}
		\begin{column}{0.4\textwidth}
			\begin{center}
				\includegraphics[scale=0.5]{19}\pause
			\end{center}
		\end{column}
		\begin{column}{0.6\textwidth}
			\begin{enumerate}[a.]
				\item L'angle  $\widehat{vIt}$ mesure 30\degree (90\degree - 60\degree) \pause
				\item L'angle  $\widehat{xIz}$ mesure 90\degree . \pause
				\item L'angle  $\widehat{zIu}$ est opposé par le sommet à l'angle $\widehat{vIt}$ il mesure 30\degree  \pause
				\item L'angle  $\widehat{uIy}$ est opposé par le sommet à l'angle $\widehat{xIv}$ il mesure 60\degree.
				
			\end{enumerate}
		\end{column}
	\end{columns}
	
	
	
\end{frame}
\end{document}