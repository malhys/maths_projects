\documentclass[12pt,a4paper]{article}

\usepackage[in, plain]{fullpage}
\usepackage{array}
\usepackage{../../../pas-math}
\usepackage{../../../moncours}


%\usepackage{pas-cours}
%-------------------------------------------------------------------------------
%          -Packages nécessaires pour écrire en Français et en UTF8-
%-------------------------------------------------------------------------------
\usepackage[utf8]{inputenc}
\usepackage[frenchb]{babel}
\usepackage[T1]{fontenc}
\usepackage{lmodern}
%-------------------------------------------------------------------------------

%-------------------------------------------------------------------------------
%                          -Outils de mise en forme-
%-------------------------------------------------------------------------------
\usepackage{hyperref}
\hypersetup{pdfstartview=XYZ}
\usepackage{enumerate}
\usepackage{graphicx}
\usepackage{multicol}

\usepackage{anysize} %%pour pouvoir mettre les marges qu'on veut
%\marginsize{2.5cm}{2.5cm}{2.5cm}{2.5cm}

\usepackage{indentfirst} %%pour que les premier paragraphes soient aussi indentés
%-------------------------------------------------------------------------------


%-------------------------------------------------------------------------------
%                  -Nécessaires pour écrire des mathématiques-
%-------------------------------------------------------------------------------
\usepackage{amsfonts}
\usepackage{amssymb}
\usepackage{amsmath}
\usepackage{amsthm}
\usepackage{tikz}
%-------------------------------------------------------------------------------

%-------------------------------------------------------------------------------
%                     -Mise en forme d'exercices-
%-------------------------------------------------------------------------------
\newtheoremstyle{exostyle}
{\topsep}% espace avant
{\topsep}% espace apres
{}% Police utilisee par le style de thm
{}% Indentation (vide = aucune, \parindent = indentation paragraphe)
{\bfseries}% Police du titre de thm
{.}% Signe de ponctuation apres le titre du thm
{ }% Espace apres le titre du thm (\newline = linebreak)
{\thmname{#1}\thmnumber{ #2}\thmnote{. \normalfont{\textit{#3}}}}% composants du titre du thm : \thmname = nom du thm, \thmnumber = numéro du thm, \thmnote = sous-titre du thm

\theoremstyle{exostyle}
\newtheorem{exercice}{Exercice}

\newenvironment{questions}{
\begin{enumerate}[\hspace{12pt}\bfseries\itshape a.]}{\end{enumerate}
} %mettre un 1 à la place du a si on veut des numéros au lieu de lettres pour les questions 
%-------------------------------------------------------------------------------



%-------------------------------------------------------------------------------
%                    - Racourcis d'écriture -
%-------------------------------------------------------------------------------

% Angles orientés (couples de vecteurs)
\newcommand{\aopp}[2]{(\vec{#1}, \vec{#2})} %Les deuc vecteurs sont positifs
\newcommand{\aopn}[2]{(\vec{#1}, -\vec{#2})} %Le second vecteur est négatif
\newcommand{\aonp}[2]{(-\vec{#1}, \vec{#2})} %Le premier vecteur est négatif
\newcommand{\aonn}[2]{(-\vec{#1}, -\vec{#2})} %Les deux vecteurs sont négatifs

%Ensembles mathématiques
\newcommand{\naturels}{\mathbb{N}} %Nombres naturels
\newcommand{\relatifs}{\mathbb{Z}} %Nombres relatifs
\newcommand{\rationnels}{\mathbb{Q}} %Nombres rationnels
\newcommand{\reels}{\mathbb{R}} %Nombres réels
\newcommand{\complexes}{\mathbb{C}} %Nombres complexes
%-------------------------------------------------------------------------------




%\makeatletter
%\renewcommand*{\@seccntformat}[1]{\csname the#1\endcsname\hspace{0.1cm}}
%\makeatother


%\author{Olivier FINOT}
\date{}
\title{}

\graphicspath{{./img/}}

\lhead{Seq 3: Fractions}
\rhead{O. FINOT}
%
%\rfoot{Page \thepage}
\begin{document}
%\maketitle




\chap[num=8, color=red]{Angles et parallélisme}{}

\begin{myobj}
	\begin{itemize}
		\item Reconnaître un segment, une demie-droite, une droite et savoir les tracer;
		\item Tracer avec l’équerre la droite perpendiculaire à une droite donnée passant par un point donné;
		\item Tracer avec la règle et l’équerre la droite parallèle à une droite donnée passant par un point donné;
		\item Déterminer la distance entre deux points, entre un point et une droite;
		\item Savoir coder et lire une figure.
	\end{itemize}
\end{myobj}

\begin{mycomp}
	\begin{itemize}
		\item \kw{Modéliser} 
		\item \kw{Représenter} 
		\item \kw{Raisonner} 
		\item \kw{Communiquer}
		
	\end{itemize}
\end{mycomp}


\section{Angles opposés par le sommet}


\begin{mydef}
	Deux angles ayant le même sommet et sont dans le prolongement l'un de l'autre, alors ils sont \kw{opposés par le sommet}
\end{mydef}

\begin{myprop}
	Deux angles opposés par le sommet ont la \kw{même mesure}.
\end{myprop}


\begin{myex}
	
	\begin{multicols}{2}
		les droites $(AB)$ et $(CD)$ sécantes en $O$ forment deux paires d'angles opposés par le sommet.
		
		On a :
		\begin{itemize}
			\item $\widehat{AOC} = \widehat{BOD}$
			\item $\widehat{AOD} = \widehat{BOC}$
		\end{itemize}	
	
		\includegraphics[scale=0.15]{opposes}
	\end{multicols}
	
\end{myex}
\section{Angles alternes-internes}


\begin{mydef}
	
	
		Soit deux droites $(d_1)$ et $(d_2)$, une troisième droite $(s)$, les coupe en $A$ et $B$.
		Dans angles formés par ces 3 droites sont alternes-internes si et seulement si :
		
		\begin{multicols}{2}
		\begin{itemize}
			\item ils ont pour sommet $A$ et $B$;
			\item ils sont de part et d'autre de la droite $(s)$;
			\item ils sont entre les droites $(d_1)$ et $(d_2)$.
		\end{itemize}
	
	
		\begin{center}
			\includegraphics[scale=0.13]{alt_int}
		\end{center}
	\end{multicols}

\end{mydef}


\begin{myprop}
	Si deux droites coupées par une sécante sont \kw{parallèles}, alors les angles alternes-internes sont \kw{égaux}.
\end{myprop}

\begin{myex}
	
	
	\begin{multicols}{2}
		Les droites $(AE)$ et $(BC)$ sont parallèles et les angles $\widehat{EAB}$ et $\widehat{ABC}$ sont alternes-internes. \\
		
		Donc les angles $\widehat{EAB}$ et $\widehat{ABC}$ ont la même mesure.
		
		\begin{center}
			\includegraphics[scale=0.15]{alt_int2}
		\end{center}
	\end{multicols}
\end{myex}


\begin{myprop}
	Si deux angles alternes-internes sont \kw{égaux}, alors les droites coupées par la sécante sont \kw{parallèles}.
\end{myprop}

\begin{myex}
	\begin{multicols}{2}
		
		Les angles $\widehat{AEC}$ et $\widehat{BCE}$ sont alternes-internes et de même mesure. \\
		
		Donc les droites $(AE)$ et $(BC)$ sont parallèles.
		
		\begin{center}
			\includegraphics[scale=0.15]{alt_int3}
		\end{center}
	\end{multicols}
\end{myex}
\end{document}

