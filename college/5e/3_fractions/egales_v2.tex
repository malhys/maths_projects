\begin{myprop}
	Une fraction ne change pas quand on \hspace*{6cm} le numérateur \kw{et} le dénominateur par un même nombre non nul.
	
	\begin{multicols}{3}
		\begin{equation*}
		\dfrac{a}{b} = %\dfrac{a \times k}{b \times k} 
		\end{equation*}
		
		 \begin{center}
		 	ou
		 \end{center}
		
		\begin{equation*}
		\dfrac{a}{b} = %\dfrac{a \div k}{b \div k} 
		\end{equation*}	
	\end{multicols}	
	
\end{myprop}

\begin{myexs}
	
	\begin{multicols}{2}
		
	\begin{equation*}
		\dfrac{\num{3.2}}{\num{1.5}} = % \dfrac{\num{3.2} \times 10 }{\num{1.5} \times 10 } = \dfrac{\num{32}}{\num{15}}
	\end{equation*}
	
	
	\begin{equation*}
		\dfrac{\num{12}}{\num{27}} =  %\dfrac{\num{12} \div 3 }{\num{27} \div 3 } = \dfrac{\num{4}}{\num{9}} 
	\end{equation*}
	\end{multicols}
\end{myexs}

\begin{mydef}
	Simplifier une fraction, c'est trouver une autre fraction \hspace*{6cm} avec le numérateur et le dénominateur %\kw{les plus petits possibles}.
\end{mydef}

\begin{myexs}
	\begin{multicols}{2}
		\begin{equation*}
			\dfrac{\num{27}}{\num{72}} =  %\dfrac{\num{27} \div 9 }{\num{72} \div 9 } = \dfrac{\num{3}}{\num{8}} 
		\end{equation*}	
	
		\begin{equation*}
			\dfrac{\num{25}}{\num{100}} = %\dfrac{\num{25} \div 25 }{\num{100} \div 25 } = \dfrac{\num{1}}{\num{4}} 
		\end{equation*}	
	\end{multicols}
\end{myexs}
	