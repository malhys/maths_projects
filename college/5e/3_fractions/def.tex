\begin{mydef}
	
	\iftoggle{eleve}{%
		\vspace*{0.20cm}
		\hrulefill
		
		\vspace*{0.20cm}
		\hrulefill
		
		\vspace*{0.20cm}
		\hrulefill
	}{%
		$a$ et $b$ sont deux nombres ($b$ $\neq$ 0).\pause Le \kw{quotient} de $a$ par $b$ se note $a \div b$ ou $\dfrac{a}{b}$, en écriture fractionnaire.\pause
	}
	
\end{mydef}

\begin{myex}
	%\begin{itemize}
		%\item 
		
		\iftoggle{eleve}{%
			Le quotient de 5 par 4 est \hrulefill
			
			\vspace*{0.20cm}
			\hrulefill
			
			\vspace*{0.20cm}
			\hrulefill
		}{%
			Le quotient de 5 par 4 est $\dfrac{5}{4}$, c'est le nombre qui multiplié par 4 donne 5. \pause
			\begin{equation*}
				\dfrac{5}{4} \times 4 = 5
			\end{equation*}
		}
		

		%\item Le quotient de 2 par 3 est $\dfrac{2}{3}$, c'est le nombre qui multiplié par 3 donne 2. $\dfrac{2}{3} \times 3 = 2 $.
	%\end{itemize}
\end{myex}

\begin{mydef}
	\iftoggle{eleve}{%
		\vspace*{0.20cm}
		\hrulefill
		
		\vspace*{0.20cm}
		\hrulefill
		
		\vspace*{0.20cm}
		\hrulefill
	}{%
		Si $a$ et $b$ sont entiers, alors $\dfrac{a}{b}$ est une \kw{fraction}.\pause $a$ est le\pause \kw{numérateur} et $b$ est le\pause \kw{dénominateur}.	
	}
	
	
\end{mydef}

\begin{center}
	\iftoggle{eleve}{%
		\vspace*{0.3cm}
		\includegraphics*[scale=0.5]{def_2}
	}{%
		\includegraphics*[scale=0.5]{def}
	}
	
\end{center}

\begin{myex}
	\iftoggle{eleve}{%
		$\dfrac{\num{4.2}}{\num{2}}$, $\dfrac{\num{5}}{\num{2.4}}$, $\dfrac{\num{1.3}}{\num{3.7}}$ et $\dfrac{\num{2}}{\num{3}}$ sont \hrulefill
		
		\vspace*{0.50cm}
		\hrulefill
	}{%
		$\dfrac{\num{4.2}}{\num{2}}$, $\dfrac{\num{5}}{\num{2.4}}$, $\dfrac{\num{1.3}}{\num{3.7}}$ et $\dfrac{\num{2}}{\num{3}}$ sont toutes des écritures fractionnaires, mais seule $\dfrac{\num{2}}{\num{3}}$ est une fraction.
	}
	
\end{myex}