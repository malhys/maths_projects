\section{Une partie de Uno (6 points)}

Quatres copains jouent au Uno. Ce jeu comporte :

\begin{itemize}
	\item des cartes numérotées bleues, rouges, jaunes et vertes (19 de chaque couleur);
	\item 36 cartes Action.
\end{itemize}

\begin{questions}
	\question[2] Quelle proportion du nombre total de cartes représentent les cartes numérotées rouges ?
	\begin{solution}
		$19 \times 4 + 36 = 112$\\
		
		En tout il y a 112 cartes dans le jeu.
		
		Les cartes numérotées rouges représentent $\frac{19}{112}$ du jeu.
	\end{solution}

	
	
	\question[2] Quelle proportion du nombre total de cartes représentent les cartes action ?
	\begin{solution}
		\begin{eqnarray*}
			\dfrac{36}{112} &=& \dfrac{9 \times 4}{28 \times 4} \\
			\dfrac{36}{112} &=& \dfrac{9}{28} \\
		\end{eqnarray*}
	
	Les cartes actions représentent $\frac{9}{28}$ du jeu.
	\end{solution}
	
	\question[2] Au début de la partie, on distribue 7 cartes à chaque joueur et on place le reste dans un paquet au centre de la table. Quelle proportion de cartes n'a pas été distribuée ? Donner le résultat sous forme d'une fraction simplifiée, puis en pourcentage.
	
	\begin{solution}
		Il a quatre joueurs, donc au début de la partie 28 ($7 \times 4$) cartes sont distribuées, il en reste 84.
		
		\begin{eqnarray*}
			\dfrac{84}{112} = \dfrac{3 \times 28}{4 \times 28} \\
			\dfrac{84}{112} = \dfrac{3}{4} \\
		\end{eqnarray*}
	
		Donc $\frac{3}{4}$ des cartes n'ont pas été distribuées, soit 75 \%.
	\end{solution}
\end{questions}