\section{Voyage (4 points)}

Quatre amis font un voyage en trois jours. Le premier jour, ils parcourent 40 \% du trajet total ; le deuxième jour, un quart et le dernier jour, $\dfrac{7}{20}$ du trajet total.

\begin{questions}
	\question[2] Quel jour ont-ils parcouru la plus grande distance ?
	
	\begin{solution}
		Je convertis toutes les fractions en pourcentages :
		
		\begin{eqnarray*}
			\dfrac{1}{4} &=& \dfrac{1 \times 25}{4 \times 25}\\
			\dfrac{1}{4} &=& \dfrac{25}{100}\\
		\end{eqnarray*}
	
		\begin{eqnarray*}
			\dfrac{7}{20} &=& \dfrac{7 \times 5}{20 \times 5}\\
			\dfrac{7}{20} &=& \dfrac{35}{100}\\
		\end{eqnarray*}
	
		Ils ont donc parcouru 40 \% du trajet le premier jour, puis 25 \% le deuxième et 35 \% le dernier. C'est donc le premier jour qu'ils ont parcouru la plus grande distance.
	\end{solution}
	
	\question[2] Peut-on calculer la distance parcourue chaque jour ?
	
	\begin{solution}
		Sans connaître la distance globale du voyage ou la distance effectuée un des trois jours on ne peut pas calculer la distance parcourue chaque jour.
	\end{solution}
\end{questions}