\begin{myprop}
	Un nombre $a$ est \kw{divisible} par un nombre $b$ si le reste de la division de $a$ par $b$ vaut 0. 
\end{myprop}

\begin{myexs}
	\begin{itemize}
		\item $ 5 \times 3 + 0 = 15$, donc 15 est divisible par 3 et 5.
		\item $ 5 \times 3 + 2 = 17$, donc 17 n'est pas divisible par 3 et 5.
	\end{itemize}
\end{myexs}

\begin{myprops}
	\begin{itemize}
		\item Un nombre est divisible par 2 s'il est pair (son chiffre des unités est 0, 2, 4, 6 ou 8).
		\item Un nombre est divisible par 3 si la somme de ses chiffres est divisible par 3.
		\item Un nombre est divisible par 5 si son chiffre des unités est 0 ou 5.
		\item Un nombre est divisible par 9 si la somme de ses chiffres est divisible par 9. 
	\end{itemize}
\end{myprops}

\begin{myexs}
	\begin{itemize}
		\item 20 est divisible par 2 et 5;
		\item 45 est divisible par 3, 5 et 9 (4 + 5 = 9);
		\item 2520 est divisible par 2, 3, 5 et 9 (2 + 5 + 2 =9 ).
	\end{itemize}
\end{myexs}

\begin{myprops}
	\begin{itemize}
		\item Un \kw{nombre premier} est un nombre qui est divisible uniquement par 1 et lui-même.	
		
		\item Les nombres premiers inférieurs à 30 sont : 1; 2; 3; 5; 7; 11; 13; 17; 19; 23 et 29. 
	\end{itemize}
	
\end{myprops}

\begin{myexs}
	\begin{itemize}
		\item 15 est divisible par 3 et 5, il n'est pas premier.
		\item 21 est divisible par 3 et 7, il n'est pas premier.
	\end{itemize}
\end{myexs}