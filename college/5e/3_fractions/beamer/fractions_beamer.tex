\documentclass[xcolor={dvipsnames}]{beamer}
%\usepackage[utf8]{inputenc}
\usetheme{Madrid}
%\usetheme{Malmoe}
\usecolortheme{beaver}
%\usecolortheme{rose}

%-------------------------------------------------------------------------------
%          -Packages nécessaires pour écrire en Français et en UTF8-
%-------------------------------------------------------------------------------
\usepackage[utf8]{inputenc}
\usepackage[frenchb]{babel}
\usepackage[T1]{fontenc}
\usepackage{lmodern}
\usepackage{textcomp}

%-------------------------------------------------------------------------------

%-------------------------------------------------------------------------------
%                          -Outils de mise en forme-
%-------------------------------------------------------------------------------
\usepackage{hyperref}
\hypersetup{pdfstartview=XYZ}
\usepackage{enumerate}
\usepackage{graphicx}
%\usepackage{multicol}
%\usepackage{tabularx}

%\usepackage{anysize} %%pour pouvoir mettre les marges qu'on veut
%\marginsize{2.5cm}{2.5cm}{2.5cm}{2.5cm}

\usepackage{indentfirst} %%pour que les premier paragraphes soient aussi indentés
\usepackage{verbatim}
%\usepackage[table]{xcolor}  
%\usepackage{multirow}
\usepackage{ulem}
%-------------------------------------------------------------------------------


%-------------------------------------------------------------------------------
%                  -Nécessaires pour écrire des mathématiques-
%-------------------------------------------------------------------------------
\usepackage{amsfonts}
\usepackage{amssymb}
\usepackage{amsmath}
\usepackage{amsthm}
\usepackage{tikz}
\usepackage{xlop}
\usepackage[output-decimal-marker={,}]{siunitx}
%-------------------------------------------------------------------------------


%-------------------------------------------------------------------------------
%                    - Mise en forme 
%-------------------------------------------------------------------------------

\newcommand{\bu}[1]{\underline{\textbf{#1}}}


\usepackage{ifthen}


\newcommand{\ifTrue}[2]{\ifthenelse{\equal{#1}{true}}{#2}{$\qquad \qquad$}}

\newcommand{\kword}[1]{\textcolor{red}{\underline{#1}}}


%-------------------------------------------------------------------------------



%-------------------------------------------------------------------------------
%                    - Racourcis d'écriture -
%-------------------------------------------------------------------------------

% Angles orientés (couples de vecteurs)
\newcommand{\aopp}[2]{(\vec{#1}, \vec{#2})} %Les deuc vecteurs sont positifs
\newcommand{\aopn}[2]{(\vec{#1}, -\vec{#2})} %Le second vecteur est négatif
\newcommand{\aonp}[2]{(-\vec{#1}, \vec{#2})} %Le premier vecteur est négatif
\newcommand{\aonn}[2]{(-\vec{#1}, -\vec{#2})} %Les deux vecteurs sont négatifs

%Ensembles mathématiques
\newcommand{\naturels}{\mathbb{N}} %Nombres naturels
\newcommand{\relatifs}{\mathbb{Z}} %Nombres relatifs
\newcommand{\rationnels}{\mathbb{Q}} %Nombres rationnels
\newcommand{\reels}{\mathbb{R}} %Nombres réels
\newcommand{\complexes}{\mathbb{C}} %Nombres complexes


%Intégration des parenthèses aux cosinus
\newcommand{\cosP}[1]{\cos\left(#1\right)}
\newcommand{\sinP}[1]{\sin\left(#1\right)}

%Fractions
\newcommand{\myfrac}[2]{{\LARGE $\frac{#1}{#2}$}}

%Vocabulaire courrant
\newcommand{\cad}{c'est-à-dire}

%Droites
\newcommand{\dte}[1]{droite $(#1)$}
\newcommand{\fig}[1]{figure $#1$}
\newcommand{\sym}{symétrique}
\newcommand{\syms}{symétriques}
\newcommand{\asym}{axe de symétrie}
\newcommand{\asyms}{axes de symétrie}
\newcommand{\seg}[1]{$[#1]$}
\newcommand{\monAngle}[1]{$\widehat{#1}$}
\newcommand{\bissec}{bissectrice}
\newcommand{\mediat}{médiatrice}
\newcommand{\ddte}[1]{$[#1)$}

%Figures
\newcommand{\para}{parallélogramme}
\newcommand{\paras}{parallélogrammes}
\newcommand{\myquad}{quadrilatère}
\newcommand{\myquads}{quadrilatères}
\newcommand{\co}{côtés opposés}
\newcommand{\diag}{diagonale}
\newcommand{\diags}{diagonales}
\newcommand{\supp}{supplémentaires}
\newcommand{\car}{carré}
\newcommand{\cars}{carrés}
\newcommand{\rect}{rectangle}
\newcommand{\rects}{rectangles}
\newcommand{\los}{losange}
\newcommand{\loss}{losanges}


%----------------------------------------------------


\usepackage{../../../../pas-math}
\usepackage{../../../../moncours_beamer}

\usepackage{amssymb,amsmath}


\newcommand{\myitem}{\item[\textbullet]}

\graphicspath{{../img/}}

\title{Séquence 3 : Fractions}
%\author{O. FINOT}\institute{Collège S$^t$ Bernard}

%
\AtBeginSection[]
{
	\begin{frame}
		\frametitle{}
		\tableofcontents[currentsection, hideallsubsections]
	\end{frame} 
	
}
%
%
%\AtBeginSubsection[]
%{
%	\begin{frame}
%		\frametitle{Sommaire}
%		\tableofcontents[currentsection, currentsubsection]
%	\end{frame} 
%}

\begin{document}
	
	
	
	\begin{frame}
		\titlepage 
	\end{frame}
	
	
	%	
	%
	%\begin{frame}
	%	\begin{block}{Objectifs}
	%		\begin{itemize}
	%			
	%			\item Savoir si deux fractions sont égales
	%			\item Savoir si un nombre est divisible par un autre
	%			\item Identifier un nombre premier
	%			\item Décomposer un nombre en produit de facteurs premiers
	%			\item Simplifier une fraction
	%			\item Comparer des fractions
	%			\item Additionner et soustraire des fractions dont les dénominateurs sont des multiples l’un de l’autre
	%			
	%			\end{itemize}
	%	\end{block}
	%\end{frame}
	%
	%\begin{frame}
	%	\begin{block}{Compétences travaillées}
	%		\begin{itemize}
	%			\item \kw{Représenter (Re2)} :  produire et utiliser plusieurs représentations d’un nombre;
	%			\item \kw{Calculer (Ca1)} :  calculer avec des nombres rationnels, de manière exacte ou approchée en combinant astucieusement le calcul mental, le calcul posé et le calcul instrumenté ;
	%			\item \kw{Raisonner (Ra1)} :  résoudre des problèmes impliquant des grandeurs variées : mobiliser les connaissances nécessaires, analyser et exploiter ses erreurs, mettre à l’essai plusieurs solutions.		
	%		\end{itemize}
	%	\end{block}
	%\end{frame}
	
	
	
	\section{Quotients et fractions}
	
	
	
	
	\begin{frame}
		\begin{mydef}
			$a$ et $b$ sont deux nombres ($b$ $\neq$ 0).\pause Le \kword{quotient} de $a$ par $b$ se note $a \div b$ ou $\dfrac{a}{b}$, en écriture fractionnaire.\pause
		\end{mydef}
		
		\begin{myex}
			%\begin{itemize}
			%\item 
			Le quotient de 5 par 4 est $\dfrac{5}{4}$, c'est le nombre qui multiplié par 4 donne 5. \pause
			\begin{equation*}
				\dfrac{5}{4} \times 4 = 5
			\end{equation*}
			
			%\item Le quotient de 2 par 3 est $\dfrac{2}{3}$, c'est le nombre qui multiplié par 3 donne 2. $\dfrac{2}{3} \times 3 = 2 $.
			%\end{itemize}
		\end{myex}
		
	\end{frame}
	
	
	\begin{frame}
		\begin{mydef}
			Si $a$ et $b$ sont entiers, alors $\dfrac{a}{b}$ est une \kword{fraction}.\pause $a$ est le \pause \kword{numérateur} et $b$ est le \pause \kword{dénominateur}.\pause	
			
		\end{mydef}
		
		\begin{center}
			\includegraphics<5>[scale=0.4]{def_2}
			\includegraphics<6>[scale=0.4]{def}
		\end{center}
	\end{frame}
	
	%\section{Divisibilité et nombres premiers}
	%
	%
	%\begin{frame}
	%	\begin{myprop}
	%		Un nombre $a$ est \kword{divisible} par un nombre $b$ si le reste de la division de $a$ par $b$ vaut 0. \pause
	%	\end{myprop}
	%	
	%	\begin{myexs}
	%		\begin{itemize}
	%			\item $ 5 \times 3 + 0 = 15$, donc \pause 15 est divisible par 3 et 5.\pause
	%			\item $ 5 \times 3 + 2 = 17$, donc \pause 17 n'est pas divisible par 3 et 5.
	%		\end{itemize}
	%	\end{myexs}
	%\end{frame}
	%
	%\begin{frame}
	%	\begin{myprops}
	%		\begin{itemize}
	%			\item Un nombre est divisible par 2 \pause s'il est pair (son chiffre des unités est 0, 2, 4, 6 ou 8).\pause
	%			\item Un nombre est divisible par 3 \pause si la somme de ses chiffres est divisible par 3.\pause
	%			\item Un nombre est divisible par 5  \pause si son chiffre des unités est 0 ou 5.
	%			\item Un nombre est divisible par 9 \pause si la somme de ses chiffres est divisible par 9. \pause
	%		\end{itemize}
	%	\end{myprops}
	%	
	%	\begin{myexs}
	%		\begin{itemize}
	%			\item 20 est divisible par \pause 2 et 5; \pause
	%			\item 45 est divisible par \pause 3, 5 et 9 (4 + 5 = 9);\pause
	%			\item 2520 est divisible par \pause 2, 3, 5 et 9 (2 + 5 + 2 =9 ).
	%		\end{itemize}
	%	\end{myexs}
	%\end{frame}
	%
	%\begin{frame}
	%	\begin{myprops}
	%		\begin{itemize}
	%			\item Un \kword{nombre premier} est un nombre qui est divisible uniquement par 1 et lui-même.	\pause
	%			
	%			\item Les nombres premiers inférieurs à 30 sont \pause: 1; 2; 3; 5; 7; 11; 13; 17; 19; 23 et 29. \pause
	%		\end{itemize}
	%		
	%	\end{myprops}
	%	
	%	\begin{myexs}
	%		\begin{itemize}
	%			\item 15 est divisible par 3 et 5, il n'est pas premier. \pause
	%			\item 21 est divisible par 3 et 7, il n'est pas premier.
	%		\end{itemize}
	%	\end{myexs}
	%\end{frame}
	
	\section{Fractions égales et simplifications}
	
	
	\begin{frame}
		\begin{myprop}
			Une fraction ne change pas quand on \kword{multiplie (ou on divise)} le numérateur \kword{et} le dénominateur par un même nombre non nul.\pause
			
			
			\begin{equation*}
				\dfrac{a}{b} = \dfrac{a \times k}{b \times k} \pause
			\end{equation*}
			ou 
			
			\begin{equation*}
				\dfrac{a}{b} = \dfrac{a \div k}{b \div k} 
			\end{equation*}
			
		\end{myprop}
		
		
		\begin{myexs}
			
			%\begin{multicols}{2}
			
			\begin{equation*}
				\dfrac{\num{7}}{\num{5}} = \pause \dfrac{7 \times 10 }{5 \times 10 } = \pause \dfrac{\num{70}}{\num{50}} \pause
			\end{equation*}
			
			
			\begin{equation*}
				\dfrac{\num{12}}{\num{27}} = \pause \dfrac{\num{12} \div 3 }{\num{27} \div 3 } = \pause \dfrac{\num{4}}{\num{9}} 
			\end{equation*}
			%\end{multicols}
		\end{myexs}
	\end{frame}
	
	\begin{frame}
		\begin{mydef}
			Simplifier une fraction, c'est trouver une autre fraction \kword{égale à la première} avec le numérateur et le dénominateur \kword{les plus petits possibles}.\pause
		\end{mydef}
		
		\begin{myex}
			
			\begin{equation*}
				\dfrac{\num{27}}{\num{72}} = \pause \dfrac{\num{27} \div 9 }{\num{72} \div 9 } = \pause \dfrac{\num{3}}{\num{8}} \pause
			\end{equation*}	
			
			\begin{equation*}
				\dfrac{\num{25}}{\num{100}} = \pause \dfrac{\num{25} \div 25 }{\num{100} \div 25 } = \pause \dfrac{\num{1}}{\num{4}} 
			\end{equation*}	
			
		\end{myex}
	\end{frame}
	
	
	\begin{frame}
		\begin{block}{Méthode}
			Je veux simplifier la fraction $\dfrac{105}{60}$ \pause
			
			\begin{enumerate}
				\item Je cherche un diviseur commun au numérateur et au dénominateur : \pause
				105 et 60 sont divisibles par 5.
				\item Je calcule les divisions :
				\begin{equation*}
					\frac{105}{60} = \pause \frac{105 \div 5}{60 \div 5 } = \frac{21}{12}
				\end{equation*}
				
				\vspace*{-0.3cm}
				\item Je recommence si je peux, autant de fois que possible, le numérateur et le dénominateur sont divisibles par \pause 3.
				\begin{equation*}
					\frac{21}{12} = \pause \frac{21 \div 3}{12 \div 3} = \frac{7}{4}		
				\end{equation*}
				
				\vspace*{-0.3cm}
				\item Si je ne peux pas continuer, j'ai terminé:
				\begin{equation*}
					\frac{105}{60} = \frac{7}{4}		
				\end{equation*}
			\end{enumerate}
		\end{block}
	\end{frame}
	
	%\section{\'Egalité des produits en croix}
	%
	%\begin{frame}
	%	\begin{myprop}
	%		$a$, $b$, $c$ et $d$  sont des nombres entiers \pause avec $b \neq 0$ et $ d \neq 0$.\pause
	%		
	%		$\dfrac{a}{b} = \dfrac{c}{d}$ signifie que \pause $a \times d = b \times c$.\pause
	%		
	%		
	%	\end{myprop}
	%	
	%	
	%	\begin{myexs}
	%		\begin{itemize}
	%			\item $\dfrac{34}{51} = \dfrac{2}{3}$ \pause car $34 \times 3 = 51 \times 2 = 102$ \pause
	%			
	%			\item Je veux compléter $\dfrac{23}{15} = \dfrac{207}{?}$ \pause
	%			
	%			On a :\pause
	%			
	%			\vspace*{-1cm}
	%			\begin{eqnarray*}
	%				23 \times ... &=& \pause 15 \times 207 \\ \pause
	%				23 \times ... &=& \num{3105} \\ \pause
	%			\end{eqnarray*}
	%			\vspace*{-1cm}
	%			
	%			Je calcule $\num{3105} \div 23 = 135$ \pause
	%			
	%			Donc $\dfrac{23}{15} = \dfrac{207}{135}$ 
	%		\end{itemize}
	%		
	%	\end{myexs}
	%\end{frame}
	%
	%\section{Addition et soustraction de fractions}
	%
	%
	%\begin{frame}
	%	\begin{block}{Méthode}
	%		Pour additionner ou soustraire deux fractions :\pause
	%		
	%		\begin{enumerate}
	%			\item Je les écrit avec le \kw{même dénominateur};\pause
	%			\item Je fais la \kw{somme des numérateurs};\pause
	%			\item Je ne modifie pas le dénominateur;\pause
	%		\end{enumerate}
	%	\end{block}
	%	
	%	\begin{myexs}
	%		\begin{columns}
	%			
	%			\begin{column}{0.3\textwidth}
	%				\begin{eqnarray*}
	%					A &=& \frac{3}{5} + \frac{1}{5} \\ \pause
	%					A &=& \frac{3 + 1}{5}\\ \pause
	%					A &=& \frac{4}{5}\\ \pause
	%				\end{eqnarray*} 
	%				
	%			\end{column}
	%						
	%			\begin{column}{0.3\textwidth}
	%				\begin{eqnarray*}
	%					B &=& \frac{14}{3} - 2 \\ \pause
	%					B &=& \frac{14}{3} - \frac{2 \times 3}{3}\\ \pause
	%					B &=& \frac{14}{3} - \frac{6}{3}\\ \pause
	%					B &=& \frac{14 - 6}{3}\\ \pause
	%					B &=& \frac{8}{3}\\ \pause
	%				\end{eqnarray*}
	%			\end{column}
	%			
	%			
	%			\begin{column}{0.3\textwidth}
	%				\begin{eqnarray*}
	%					C &=& \frac{2}{3} + \frac{4}{9} \\ \pause
	%					C &=& \frac{2 \times 3}{3 \times 3} + \frac{4}{9}\\ \pause
	%					C &=& \frac{6}{9} + \frac{4}{9}\\ \pause
	%					C &=& \frac{6 + 4}{9}\\ \pause
	%					C &=& \frac{10}{9}\\
	%				\end{eqnarray*}
	%			\end{column}
	%			
	%			
	%		\end{columns}
	%	\end{myexs}
	%\end{frame}
\end{document}