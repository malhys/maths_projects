\documentclass[xcolor={dvipsnames}]{beamer}
%\usepackage[utf8]{inputenc}
\usetheme{Madrid}
%\usetheme{Malmoe}
\usecolortheme{beaver}
%\usecolortheme{rose}

\input{../../../../utils_maths_beamer}


\usepackage{../../../../pas-math}
\usepackage{../../../../moncours_beamer}

\usepackage{amssymb,amsmath}


\newcommand{\myitem}{\item[\textbullet]}

\graphicspath{{../img/}}

\title{Séquence 3 : Fractions}
%\author{O. FINOT}\institute{Collège S$^t$ Bernard}

%
\AtBeginSection[]
{
	\begin{frame}
		\frametitle{}
		\tableofcontents[currentsection, hideallsubsections]
	\end{frame} 

}
%
%
%\AtBeginSubsection[]
%{
%	\begin{frame}
%		\frametitle{Sommaire}
%		\tableofcontents[currentsection, currentsubsection]
%	\end{frame} 
%}

\begin{document}



\begin{frame}
  \titlepage 
\end{frame}


	

\begin{frame}
	\begin{block}{Objectifs}
		\begin{itemize}
			
			\item Savoir si deux fractions sont égales
			\item Savoir si un nombre est divisible par un autre
			\item Identifier un nombre premier
			\item Décomposer un nombre en produit de facteurs premiers
			\item Simplifier une fraction
			\item Comparer des fractions
			\item Additionner et soustraire des fractions dont les dénominateurs sont des multiples l’un de l’autre
			
			\end{itemize}
	\end{block}
\end{frame}

\begin{frame}
	\begin{block}{Compétences travaillées}
		\begin{itemize}
			\item \kw{Représenter (Re2)} :  produire et utiliser plusieurs représentations d’un nombre;
			\item \kw{Calculer (Ca1)} :  calculer avec des nombres rationnels, de manière exacte ou approchée en combinant astucieusement le calcul mental, le calcul posé et le calcul instrumenté ;
			\item \kw{Raisonner (Ra1)} :  résoudre des problèmes impliquant des grandeurs variées : mobiliser les connaissances nécessaires, analyser et exploiter ses erreurs, mettre à l’essai plusieurs solutions.		
		\end{itemize}
	\end{block}
\end{frame}



\section{Quotients et fractions}




\begin{frame}
	\begin{mydef}
		$a$ et $b$ sont deux nombres ($b$ $\neq$ 0).\pause Le \kword{quotient} de $a$ par $b$ se note $a \div b$ ou $\dfrac{a}{b}$, en écriture fractionnaire.\pause
	\end{mydef}
	
	\begin{myex}
		%\begin{itemize}
		%\item 
		Le quotient de 5 par 4 est $\dfrac{5}{4}$, c'est le nombre qui multiplié par 4 donne 5. \pause
		\begin{equation*}
		\dfrac{5}{4} \times 4 = 5
		\end{equation*}
		
		%\item Le quotient de 2 par 3 est $\dfrac{2}{3}$, c'est le nombre qui multiplié par 3 donne 2. $\dfrac{2}{3} \times 3 = 2 $.
		%\end{itemize}
	\end{myex}
	
\end{frame}


\begin{frame}
	\begin{mydef}
		Si $a$ et $b$ sont entiers, alors $\dfrac{a}{b}$ est une \kword{fraction}.\pause $a$ est le \pause \kword{numérateur} et $b$ est le \pause \kword{dénominateur}.\pause	
		
	\end{mydef}
	
	\begin{center}
		\includegraphics<5>[scale=0.4]{def_2}
		\includegraphics<6>[scale=0.4]{def}
	\end{center}
\end{frame}

\section{Divisibilité et nombres premiers}


\begin{frame}
	\begin{myprop}
		Un nombre $a$ est \kword{divisible} par un nombre $b$ si le reste de la division de $a$ par $b$ vaut 0. \pause
	\end{myprop}
	
	\begin{myexs}
		\begin{itemize}
			\item $ 5 \times 3 + 0 = 15$, donc \pause 15 est divisible par 3 et 5.\pause
			\item $ 5 \times 3 + 2 = 17$, donc \pause 17 n'est pas divisible par 3 et 5.
		\end{itemize}
	\end{myexs}
\end{frame}

\begin{frame}
	\begin{myprops}
		\begin{itemize}
			\item Un nombre est divisible par 2 \pause s'il est pair (son chiffre des unités est 0, 2, 4, 6 ou 8).\pause
			\item Un nombre est divisible par 3 \pause si la somme de ses chiffres est divisible par 3.\pause
			\item Un nombre est divisible par 5  \pause si son chiffre des unités est 0 ou 5.
			\item Un nombre est divisible par 9 \pause si la somme de ses chiffres est divisible par 9. \pause
		\end{itemize}
	\end{myprops}
	
	\begin{myexs}
		\begin{itemize}
			\item 20 est divisible par \pause 2 et 5; \pause
			\item 45 est divisible par \pause 3, 5 et 9 (4 + 5 = 9);\pause
			\item 2520 est divisible par \pause 2, 3, 5 et 9 (2 + 5 + 2 =9 ).
		\end{itemize}
	\end{myexs}
\end{frame}

\begin{frame}
	\begin{myprops}
		\begin{itemize}
			\item Un \kword{nombre premier} est un nombre qui est divisible uniquement par 1 et lui-même.	\pause
			
			\item Les nombres premiers inférieurs à 30 sont \pause: 1; 2; 3; 5; 7; 11; 13; 17; 19; 23 et 29. \pause
		\end{itemize}
		
	\end{myprops}
	
	\begin{myexs}
		\begin{itemize}
			\item 15 est divisible par 3 et 5, il n'est pas premier. \pause
			\item 21 est divisible par 3 et 7, il n'est pas premier.
		\end{itemize}
	\end{myexs}
\end{frame}
\end{document}