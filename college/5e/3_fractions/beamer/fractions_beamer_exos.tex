\documentclass[xcolor={dvipsnames}]{beamer}
%\usepackage[utf8]{inputenc}
\usetheme{Madrid}
%\usetheme{Malmoe}
\usecolortheme{beaver}
%\usecolortheme{rose}

%-------------------------------------------------------------------------------
%          -Packages nécessaires pour écrire en Français et en UTF8-
%-------------------------------------------------------------------------------
\usepackage[utf8]{inputenc}
\usepackage[frenchb]{babel}
\usepackage[T1]{fontenc}
\usepackage{lmodern}
\usepackage{textcomp}

%-------------------------------------------------------------------------------

%-------------------------------------------------------------------------------
%                          -Outils de mise en forme-
%-------------------------------------------------------------------------------
\usepackage{hyperref}
\hypersetup{pdfstartview=XYZ}
\usepackage{enumerate}
\usepackage{graphicx}
%\usepackage{multicol}
%\usepackage{tabularx}

%\usepackage{anysize} %%pour pouvoir mettre les marges qu'on veut
%\marginsize{2.5cm}{2.5cm}{2.5cm}{2.5cm}

\usepackage{indentfirst} %%pour que les premier paragraphes soient aussi indentés
\usepackage{verbatim}
%\usepackage[table]{xcolor}  
%\usepackage{multirow}
\usepackage{ulem}
%-------------------------------------------------------------------------------


%-------------------------------------------------------------------------------
%                  -Nécessaires pour écrire des mathématiques-
%-------------------------------------------------------------------------------
\usepackage{amsfonts}
\usepackage{amssymb}
\usepackage{amsmath}
\usepackage{amsthm}
\usepackage{tikz}
\usepackage{xlop}
\usepackage[output-decimal-marker={,}]{siunitx}
%-------------------------------------------------------------------------------


%-------------------------------------------------------------------------------
%                    - Mise en forme 
%-------------------------------------------------------------------------------

\newcommand{\bu}[1]{\underline{\textbf{#1}}}


\usepackage{ifthen}


\newcommand{\ifTrue}[2]{\ifthenelse{\equal{#1}{true}}{#2}{$\qquad \qquad$}}

\newcommand{\kword}[1]{\textcolor{red}{\underline{#1}}}


%-------------------------------------------------------------------------------



%-------------------------------------------------------------------------------
%                    - Racourcis d'écriture -
%-------------------------------------------------------------------------------

% Angles orientés (couples de vecteurs)
\newcommand{\aopp}[2]{(\vec{#1}, \vec{#2})} %Les deuc vecteurs sont positifs
\newcommand{\aopn}[2]{(\vec{#1}, -\vec{#2})} %Le second vecteur est négatif
\newcommand{\aonp}[2]{(-\vec{#1}, \vec{#2})} %Le premier vecteur est négatif
\newcommand{\aonn}[2]{(-\vec{#1}, -\vec{#2})} %Les deux vecteurs sont négatifs

%Ensembles mathématiques
\newcommand{\naturels}{\mathbb{N}} %Nombres naturels
\newcommand{\relatifs}{\mathbb{Z}} %Nombres relatifs
\newcommand{\rationnels}{\mathbb{Q}} %Nombres rationnels
\newcommand{\reels}{\mathbb{R}} %Nombres réels
\newcommand{\complexes}{\mathbb{C}} %Nombres complexes


%Intégration des parenthèses aux cosinus
\newcommand{\cosP}[1]{\cos\left(#1\right)}
\newcommand{\sinP}[1]{\sin\left(#1\right)}

%Fractions
\newcommand{\myfrac}[2]{{\LARGE $\frac{#1}{#2}$}}

%Vocabulaire courrant
\newcommand{\cad}{c'est-à-dire}

%Droites
\newcommand{\dte}[1]{droite $(#1)$}
\newcommand{\fig}[1]{figure $#1$}
\newcommand{\sym}{symétrique}
\newcommand{\syms}{symétriques}
\newcommand{\asym}{axe de symétrie}
\newcommand{\asyms}{axes de symétrie}
\newcommand{\seg}[1]{$[#1]$}
\newcommand{\monAngle}[1]{$\widehat{#1}$}
\newcommand{\bissec}{bissectrice}
\newcommand{\mediat}{médiatrice}
\newcommand{\ddte}[1]{$[#1)$}

%Figures
\newcommand{\para}{parallélogramme}
\newcommand{\paras}{parallélogrammes}
\newcommand{\myquad}{quadrilatère}
\newcommand{\myquads}{quadrilatères}
\newcommand{\co}{côtés opposés}
\newcommand{\diag}{diagonale}
\newcommand{\diags}{diagonales}
\newcommand{\supp}{supplémentaires}
\newcommand{\car}{carré}
\newcommand{\cars}{carrés}
\newcommand{\rect}{rectangle}
\newcommand{\rects}{rectangles}
\newcommand{\los}{losange}
\newcommand{\loss}{losanges}


%----------------------------------------------------


\usepackage{../../../../pas-math}
\usepackage{../../../../moncours_beamer}

\usepackage{amssymb,amsmath}


\newcommand{\myitem}{\item[\textbullet]}

\graphicspath{{../img/}}

\title{Séquence 3 : Fractions}
%\author{O. FINOT}\institute{Collège S$^t$ Bernard}

%
\AtBeginSection[]
{
	\begin{frame}
		\frametitle{}
		\tableofcontents[currentsection, hideallsubsections]
	\end{frame} 

}
%
%
%\AtBeginSubsection[]
%{
%	\begin{frame}
%		\frametitle{Sommaire}
%		\tableofcontents[currentsection, currentsubsection]
%	\end{frame} 
%}

\begin{document}



%\begin{frame}
%  \titlepage 
%\end{frame}


	

\begin{frame}
	\frametitle{Exercice 19 page 84}
	
	\begin{columns}
		\begin{column}{0.5\textwidth}
			\begin{huge}
				\begin{itemize}
					\setlength\itemsep{1em}
					\item $\frac{15}{25}$ = \pause $\frac{5 \times 3}{5 \times 5}$ = \pause $\frac{3}{5}$\pause
					
					\item $\frac{32}{24}$ = \pause $\frac{8 \times 4}{8 \times 3}$ = \pause $\frac{4}{3}$\pause
					
					\item $\frac{111}{74}$ = \pause $\frac{37 \times 3}{37 \times 2}$ = \pause $\frac{3}{2}$\pause
					
					\item $\frac{4}{22}$ = \pause $\frac{2 \times 2}{11 \times 2}$ = \pause $\frac{2}{11}$\pause
				\end{itemize}	
			\end{huge}
					
		\end{column}
	
	
		\begin{column}{0.5\textwidth}
			\begin{huge}
				\begin{itemize}
					\setlength\itemsep{1em}
					\item $\frac{18}{27}$ = \pause $\frac{9 \times 2}{9 \times 3}$ = \pause $\frac{2}{3}$\pause
					
					\item $\frac{42}{35}$ = \pause $\frac{6 \times 7}{5 \times 7}$ = \pause $\frac{6}{5}$\pause
					
					\item $\frac{14}{24}$ = \pause $\frac{7 \times 2}{12 \times 2}$ = \pause $\frac{7}{12}$\pause
					
					\item $\frac{50}{45}$ = \pause $\frac{10 \times 5}{9 \times 5}$ = \pause $\frac{10}{9}$
				\end{itemize}	
			\end{huge}
			
		\end{column}
	\end{columns}		
	
	
\end{frame}



\begin{frame}
	\frametitle{Exercice 21 page 84}
	
	\begin{columns}
		\begin{column}{0.5\textwidth}
			\begin{huge}
				\begin{itemize}
					\setlength\itemsep{1em}
					\item $\frac{10}{15}$ = \pause $\frac{5 \times 2}{5 \times 3}$ = \pause $\frac{2}{3}$\pause
					
					\item $\frac{5}{7}$ = \pause $\frac{5}{7}$\pause
					
					\item $\frac{10}{14}$ = \pause $\frac{5 \times 2}{7 \times 2}$ = \pause $\frac{5}{7}$\pause
					
					\item $\frac{6}{9}$ = \pause $\frac{2 \times 3}{3 \times 3}$ = \pause $\frac{2}{3}$\pause
					
					\item $\frac{14}{21}$ = \pause $\frac{2 \times 7}{3 \times 7}$ = \pause $\frac{2}{3}$\pause
				\end{itemize}	
			\end{huge}
			
		\end{column}
		
		
		\begin{column}{0.5\textwidth}
			\begin{huge}
				\begin{itemize}
					\setlength\itemsep{1em}
					\item $\frac{30}{42}$ = \pause $\frac{6 \times 5}{6 \times 7}$ = \pause $\frac{5}{7}$\pause
					
					\item $\frac{2}{3}$ = \pause $\frac{2}{3}$\pause
					
					\item $\frac{40}{60}$ = \pause $\frac{20 \times 2}{20 \times 3}$ = \pause $\frac{2}{3}$\pause
					
					\item $\frac{70}{98}$ = \pause $\frac{7 \times 5 \times 2}{7 \times 7 \times 2}$ = \pause $\frac{5}{7}$
				\end{itemize}	
			\end{huge}
			
		\end{column}
	\end{columns}		
	
	
\end{frame}


\begin{frame}
	%\frametitle{Exercice 31 page 65}
	
	\begin{block}{Exercice 31 page 65}
		
	\begin{columns}
		\begin{column}{0.25\textwidth}
			\begin{huge}
				\begin{itemize}
%					\setlength\itemsep{1em}
					\item $\frac{14}{21}$ = \pause $\frac{2}{3}$ \pause
					
%					\item $\frac{9}{7}$ = \pause $\frac{81}{63}$\pause
					
					
					
				\end{itemize}	
			\end{huge}
			
		\end{column}
		
		\begin{column}{0.25\textwidth}
			\begin{huge}
				\begin{itemize}
					
					\item $\frac{14}{21}$ = \pause $\frac{2}{3}$ \pause
					
%					\item $\frac{9}{7}$ = \pause $\frac{81}{63}$\pause
					
					
					
				\end{itemize}	
			\end{huge}
			
		\end{column}
	
		\begin{column}{0.25\textwidth}
			\begin{huge}
				\begin{itemize}
					%\setlength\itemsep{1em}
					\item $\frac{5}{9}$ = \pause $\frac{35}{63}$\pause
					
					%\item $\frac{4}{3}$ = \pause $\frac{44}{33}$\pause					
				\end{itemize}	
			\end{huge}
			
		\end{column}
	
		\begin{column}{0.25\textwidth}
			\begin{huge}
				\begin{itemize}
					%\setlength\itemsep{1em}
					%\item $\frac{5}{9}$ = \pause $\frac{35}{63}$\pause
					
					\item $\frac{4}{3}$ = \pause $\frac{44}{33}$\pause					
				\end{itemize}	
			\end{huge}
			
		\end{column}
	\end{columns}		
	
	\end{block}	


	\begin{block}{Exercice 32 page 65}
		\begin{columns}
			\begin{column}{0.25\textwidth}
				\begin{LARGE}
					\begin{itemize}
						\setlength\itemsep{1em}
						\item $\frac{16}{32}$ =  $\frac{63}{126}$ \pause
						
						\item $\frac{32}{16}$ =  $\frac{126}{63}$ \pause
						
						
					\end{itemize}	
				\end{LARGE}
				
			\end{column}
			
			\begin{column}{0.25\textwidth}
				\begin{LARGE}
					\begin{itemize}
						\setlength\itemsep{1em}
						\item $\frac{18}{1}$ = $\frac{2016}{112}$ 
						
						\item $\frac{1}{18}$ = $\frac{112}{2016}$ \pause
						
						
						
					\end{itemize}	
				\end{LARGE}
				
			\end{column}
			
			\begin{column}{0.25\textwidth}
				\begin{LARGE}
					\begin{itemize}
						\setlength\itemsep{1em}
						\item $\frac{8}{84}$ = $\frac{24}{252}$
						
						\item $\frac{84}{8}$ = $\frac{252}{24}$\pause			
					\end{itemize}	
				\end{LARGE}
				
			\end{column}
			
			\begin{column}{0.25\textwidth}
				\begin{LARGE}
					\begin{itemize}
						%\setlength\itemsep{1em}
						\item $\frac{2}{36}$ = $\frac{56}{1008}$					
						
						\item $\frac{36}{2}$ = $\frac{1008}{56}$
					\end{itemize}	
				\end{LARGE}
				
			\end{column}
		\end{columns}
	\end{block}
	
\end{frame}

\begin{frame}
	\frametitle{Exercice 33 page 65}
	
	\begin{LARGE}
		\begin{itemize}
			\setlength\itemsep{1em}
			\item $3 \times a$ = $8 \times b$ $\Leftrightarrow$ $\dfrac{3}{8} = \dfrac{b}{a}$   \kword{FAUX} \pause
			
			\item $8 \times a$ = $3 \times b$ $\Leftrightarrow$ $\dfrac{3}{8} = \dfrac{a}{b}$    \kword{VRAI} \pause
			
			\item<3> $a \times b$ = $8$  \pause
			
			\item<4> $a \times b$ = $8 \times 1$ $\Leftrightarrow$ $\dfrac{a}{8} = \dfrac{1}{b}$    \kword{FAUX} \pause
		\end{itemize}
	\end{LARGE}
	
\end{frame}


\begin{frame}
	\frametitle{Exercice 37 page 66}
	
	\begin{Large}
		\begin{enumerate}
			
			\item On doit résoudre $\dfrac{\num{41.6}}{?} $ = $\dfrac{16}{9}$. \pause %\\
			
			On a donc $\num{41.6} \times 9 \div 16 $ \pause = \num{23.4} %\\
			
			Si la largeur de l'écran est \num{41.6} cm, alors sa hauteur sera \num{23.4} cm.
			
			\vspace*{1cm}
			
			\item On doit résoudre $\dfrac{\num{52}}{?} $ = $\dfrac{16}{9}$. %\pause %\\
			
			On a donc $\num{52} \times 9 \div 16 $  = \num{29.25} %\\
			
			Si la largeur de l'écran est \num{52} cm, alors sa hauteur sera \num{29.25} cm.
			
			
		\end{enumerate}
	\end{Large}
	
\end{frame}


\begin{frame}
	\frametitle{Exercice 37 page 66 (suite)}
	
	\begin{Large}
		\begin{enumerate}
			

			
			\item On doit résoudre $\dfrac{?}{\num{34.2}} $ = $\dfrac{16}{9}$. \pause %\\
			
			On a donc $\num{34.2} \times 16 \div 9 $ \pause = \num{60.8} %\\
			
			Si la hauteur de l'écran est \num{34.2} cm, alors sa largeur sera \num{60.8} cm.
			
			\vspace*{1cm}
			
			\item On doit résoudre $\dfrac{?}{\num{39.6}} $ = $\dfrac{16}{9}$. %\pause %\\
			
			On a donc $\num{39.6} \times 16 \div 9 $ = \num{70.4} %\\
			
			Si la hauteur de l'écran est \num{39.6} cm, alors sa largeur sera \num{70.4} cm.
		\end{enumerate}
	\end{Large}
	
\end{frame}
\end{document}