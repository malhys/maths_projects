\section{Le partage}

\begin{questions}
	\question Le père Donatien souhaite donner $\frac{1}{2}$ de ses 17 vaches à son fils ainé, $\frac{1}{3}$ au cadet et $\frac{1}{9}$ au benjamin.
	
	17 n'est divisible ni par 2, ni par 3 ou 9, et on ne peut couper de vache en deux donc ce partage n'est pas possible.
	
	\question 
	
		\begin{eqnarray*}
			\dfrac{1}{2} + \dfrac{1}{3} + \dfrac{1}{9} &=& \dfrac{1 \times 9}{2 \times 9} + \dfrac{1 \times 6}{3 \times 6} + \dfrac{1 \times 2}{9 \times 2}\\
			\dfrac{1}{2} + \dfrac{1}{3} + \dfrac{1}{9} &=& \dfrac{9}{18} + \dfrac{6}{18} + \dfrac{2}{18} \\
			\dfrac{1}{2} + \dfrac{1}{3} + \dfrac{1}{9} &=& \dfrac{17}{18}
		\end{eqnarray*}
	
		Avec 18 vaches on peut donc en donner 9 à l'ainé, 6 au cadet et 2 au  benjamin. Les 17 vaches sont réparties et la dernière peut être rendue au cousin.
\end{questions}