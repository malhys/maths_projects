\section{Les océans}

\begin{questions}
	\question Je sais que les océans représentent \num{71.1} \% de la surface terrestre.
	\begin{equation*}
		\num{30.9} + \num{13.8} + \num{3.8} + \num{2.6} = \num{51.1}
	\end{equation*}
	
	Les autres océans représentent \num{51.1} \% de la surface terrestre.
	
	\begin{equation*}
		 \num{71.1} - \num{51.1} = \num{20}
	\end{equation*}
	
	Donc l'océan Atlantique représente 20 \% de la surface terrestre.
	
	
	\question L'océan pacifique représente \num{30.9} \% de la surface terrestre, $\num{30.9} \div 100 = \num{0.309}$.
	$1 \div 3 \approx 0.333$, donc l'océan pacifique représente environ $\dfrac{1}{3}$ de la surface de la planète.
	
	
	\question \begin{parts}
		\part 
		Je simplifie la fraction $\dfrac{\num{361200000}}{\num{510000000}}$ :
			\begin{eqnarray*}
				\dfrac{\num{361200000}}{\num{510000000}} &=& \dfrac{\num{3612}}{\num{5100}}\\
				\dfrac{\num{361200000}}{\num{510000000}} &=& \dfrac{\num{301} \times 12}{\num{425} \times 12}\\
				\dfrac{\num{361200000}}{\num{510000000}} &=& \dfrac{\num{301}}{\num{425}}\\
			\end{eqnarray*}
		
		La forme simplifiée de cette fraction est $\dfrac{\num{301}}{\num{425}}$.
		
		\part 
		\begin{multicols}{2}
			\begin{equation*}
				301 \div 425 \approx \num{0.708}
			\end{equation*}
		
		\begin{equation*}
			\num{71.1} \div 100 = \num{0.711}
		\end{equation*}
		\end{multicols}
	
		La valeur obtenue est donc bien proche de \num{71.1} \%.
	\end{parts}
\end{questions}